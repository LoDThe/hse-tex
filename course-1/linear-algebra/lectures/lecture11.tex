\section{Лекция 21.11.2019}

Напомним, если $V$ -- векторное пространство над полем $F$, то при $S \subseteq V$, линейная оболочка 

$\langle S \rangle = \{\text{все линейные комбинации конечных наборов векторов из } S\}$

\begin{example}~
    \begin{enumerate}
    \setcounter{enumi}{3}
    \item
        $V = F^n$, $S = \{e_1, \dots, e_n\}$, где

        \begin{equation*}
            e_1 = \begin{pmatrix} 1 \\ 0 \\ \dots \\ 0 \end{pmatrix}, \
            e_2 = \begin{pmatrix} 0 \\ 1 \\ \dots \\ 0 \end{pmatrix}, \
            \dots \ , \
            e_n = \begin{pmatrix} 0 \\ 0 \\ \dots \\ 1 \end{pmatrix}
        .\end{equation*}

        Тогда $\langle S \rangle = \langle e_1, \dots, e_n \rangle = F^n$.

        Так как для любого $x \in F^n \implies x = \begin{pmatrix} x_1 \\ x_2 \\ \dots \\ x_n \end{pmatrix} = x_1 \begin{pmatrix} 1 \\ 0 \\ \dots \\ 0 \end{pmatrix} + x_2 \begin{pmatrix} 0 \\ 1 \\ \dots \\ 0 \end{pmatrix} + \dots + x_n \begin{pmatrix} 0 \\ 0 \\ \dots \\ 1 \end{pmatrix} = x_1 e_1 + x_2 e_2 + \dots + x_n e_n$.
    \end{enumerate}
\end{example}


\subsection{Утверждение о том, что линейная оболочка системы векторов является подпространством объемлющего векторного пространства}

Пусть $V$ -- векторное пространство, $S \subseteq V$.

\begin{proposal}
    $\langle S \rangle$ является подпространством в $V$.
\end{proposal}

\begin{proof}~
    \begin{enumerate}
    \item 
        Два случая:

        $S = \varnothing \implies \langle \varnothing \rangle = \{\overrightarrow{0}\} \implies \overrightarrow{0} \in \langle S \rangle$.

        $S \neq \varnothing \implies \exists V \in S \implies \underbracket{0 V}_{\in \langle S \rangle} = \overrightarrow{0} \implies \overrightarrow{0} \in \langle S \rangle$.

    \item 
        Пусть $v, w \in \langle S \rangle$:

        $v = \alpha_1 v_1 + \dots + \alpha_m v_m$,

        $w = \beta_1 w_1 + \dots + \beta_n w_n$, где $v_i, w_i \in S$, $\alpha_i, \beta_i \in F$.

        Тогда, $v + w = \alpha_1 v_1 + \dots + \alpha_m v_m + \beta_1 w_1 + \dots + \beta_n w_n \in \langle S \rangle$.

        (если $v_i = w_j$, то $\alpha_i v_i + \beta_j w_j = (\alpha_i + \beta_j) w_j$)

    \item
        $v \in \langle S \rangle$, $\alpha \in F \implies v = \alpha_1 v_1 + \dots + \alpha_m v_m$

        $\implies \alpha v = (\alpha \alpha_1) v_1 + \dots + (\alpha \alpha_m) v_m \in \langle S \rangle$. \qedhere
    \end{enumerate}
\end{proof}


\subsection{Линейно зависимые и линейно независимые системы векторов}

\begin{definition}
    Линейная комбинация $\alpha_1 v_1 + \dots + \alpha_n v_n$ называется \textit{тривиальной}, если $\alpha_1 = \dots = \alpha_n = 0$ и \textit{нетривиальной} иначе (то есть $\exists i : a_i \neq 0$ или $(\alpha_1, \dots, \alpha_n) \neq (0, \dots, 0)$).
\end{definition}

\begin{example}
    $v + (-v)$ -- нетривиальная линейная комбинация векторов $v$ и $-v$.
\end{example}

\begin{definition}~
    \begin{enumerate}
    \item 
        Векторы $v_1, \dots, v_n \in V $ называются \textit{линейно зависимыми} если существует их нетривиальная линейная комбинация, равная $\overrightarrow{0}$ (то есть $\exists (\alpha_1, \dots, \alpha_n) \neq (0, \dots, 0)$, такие что $\alpha_1 v_1 + \dots + \alpha_n v_n = \overrightarrow{0}$) и \textit{линейно независимыми} иначе (то есть из условия $\alpha_1 v_1 + \dots\alpha_n v_n = \overrightarrow{0}$ следует $\alpha_1 = \dots = \alpha_n = 0$).

    \item
        Множество $S \subseteq V$ (возможно бесконечное, возможно с повторяющимися элементами) называется \textit{линейно зависимым} если существует конечное линейно зависимое подмножество, и \textit{линейно независимым} если любое конечное подмножество линейно независимо.
    \end{enumerate}
\end{definition}

\begin{convention}
    \textit{Система векторов} -- множество векторов, в котором возможны повторения.
\end{convention}

\begin{example}~
    \begin{enumerate}
        \item $S = \{\overrightarrow{0}\} \quad$ $1 \cdot \overrightarrow{0}$ -- нетривиальная линейная комбинация$\implies \overrightarrow{0}$ линейно зависимо.
        \item $S = \{v\}$, $v \neq \overrightarrow{0}$ -- линейно независимо.

            Пусть $\lambda v = \overrightarrow{0} \implies \overrightarrow{0} = \lambda^{-1} \overrightarrow{0} = \lambda^{-1}(\lambda v) = (\lambda^{-1} \lambda) v = 1v = v$ -- противоречие.

        \item 
            $S = \{v_1, v_2\} \implies S$ линейно зависимо тогда и только тогда, когда $v_1$ и $v_2$ пропорциональны (то есть либо $v_2 = \lambda_1 v_1$, $\lambda_1 \in F$, либо $v_1 = \lambda_2 v_2$, $\lambda_2 \in F$).
            \begin{proof}~
                \begin{description}
                \item[$(\implies)$] 
                    $\mu_1 v_1 + \mu_2 v_2 = \overrightarrow{0}$, $(\mu_1, \mu_2) \neq (0, 0)$.

                    Если $\mu_1 \neq 0$, то $v_1 = -\frac{\mu_2}{\mu_1} v_2$. 

                    Аналогично для $\mu_2 \neq 0$.

                \item[$(\impliedby)$]
                    $v_2 = \lambda_1 v_1 \implies \lambda_1 v_1 + (-1) v_2 = \overrightarrow{0} \implies v_1, v_2$ линейно зависимы.

                    Аналогично для $v_1 = \lambda_2 v_2$. \qedhere
                \end{description}
            \end{proof}

        \item
            $V = F^n$, $S = \{e_1, \dots, e_n\} \implies S$ линейно независимо.

            \begin{equation*}
                \alpha_1 e_1 + \dots + \alpha_n e_n = \overrightarrow{0} \iff \begin{pmatrix} \alpha_1 \\ \dots \\ \alpha_n \end{pmatrix} = \begin{pmatrix} 0 \\ \dots \\ 0 \end{pmatrix} \iff \alpha_1 = \dots = \alpha_n = 0
            .\end{equation*}
    \end{enumerate}
\end{example}


\subsection{Критерий линейной зависимости конечного набора векторов}

\begin{proposal}
    \label{lec11:proposal_a_i}
    Пусть $v_1, \dots, v_n \in V$, $i \in \{1, \dots, n\}$, тогда следующие условия эквивалентны:
    \begin{enumerate}
        \item $\exists (\alpha_1, \dots, \alpha_n) \in F^n$, такой что $\alpha_1 v_1 + \dots + \alpha_n v_n = \overrightarrow{0} (\star) $ и $\alpha_i \neq 0$. 
        \item $v_i \in \langle v_1, \dots, v_{i - 1}, v_{i + 1}, \dots, v_n \rangle$.
    \end{enumerate}
\end{proposal}

\begin{proof}~
    \begin{description}
        \item[$(1) \implies (2)$] $\alpha_i \neq 0$ в $(\star) \implies v_i = -\dfrac{\alpha_1}{\alpha_i} v_1 - \dots - \dfrac{\alpha_{i - 1}}{\alpha_i} v_{i - 1} - \dfrac{\alpha_{i + 1}}{\alpha_i} v_{i + 1} - \dots - \dfrac{\alpha_n}{\alpha_i} v_n \in \langle v_1, \dots v_{i - 1}, v_{i + 1}, \dots, v_n \rangle$.
        \item[$(2) \implies (1)$] 
            $v_i = \beta_1 v_1 + \dots + \beta_{i - 1} v_{i - 1} + \beta_{i + 1} v_{i + 1} + \dots + \beta_n v_n \implies $
            
            \begin{equation*}
                \beta_1 v_1 + \dots + \beta_{i - 1} v_{i - 1} + \underbrace{(-1)}_{\neq 0} v_i + \beta_{i + 1} v_{i + 1} + \dots + \beta_n v_n = \overrightarrow{0}
            .\end{equation*}

            (нетривиальная линейная комбинация с $i$-м скаляром $\neq 0$). \qedhere
    \end{description}
\end{proof}

\begin{corollary}
    Векторы $v_1, \dots, v_n$ линейно зависимы тогда и только тогда, когда $\exists i \in \{1, \dots, n\}$, такое что $v_i \in \langle v_1, \dots, v_{i - 1}, v_{i + 1}, \dots, v_n \rangle$.
\end{corollary}


\subsection{Основная лемма о линейной зависимости}

\begin{lemma}
    \label{lec11:osnovnaya_lemma_o_lin_zavisimosti}
    Пусть есть две системы векторов $v_1, \dots, v_m$ и $w_1, \dots, w_n$, причем $m < n$ и $w_i \in \langle v_1, \dots, v_m \rangle \quad \forall i = 1, \dots, n$.

    Тогда векторы $w_1, \dots, w_n$ линейно зависимы.
\end{lemma}

\begin{proof}
    \begin{align*}
        w_1 &= a_{11} v_1 + a_{21} v_2 + \dots + a_{m1} v_m = (v_1, \dots, v_m) \begin{pmatrix} a_{11} \\ a_{21} \\ \dots \\ a_{m1} \end{pmatrix} \\
        \dots\\
        w_n &= a_{1n} v_1 + a_{2n} v_2 + \dots + a_{mn} v_m = (v_1, \dots, v_m) \begin{pmatrix} a_{1n} \\ a_{2n} \\ \dots \\ a_{mn} \end{pmatrix}
    .\end{align*}

    \begin{equation*}
        \tag{$\star$}
        \label{v_times_A}
        \implies (w_1, \dots, w_n) = (v_1, \dots, v_m) A
    ,\end{equation*}
    где $A = (a_{ij}) \in \text{Mat}_{m \times n} (F)$.

    Так как $m < n$, то ОСЛУ $Ax = \overrightarrow{0}$ имеет ненулевое решение $z = \begin{pmatrix} z_1 \\ \dots \\ z_n \end{pmatrix} \in F^n$.

    Тогда умножим \eqref{v_times_A} справа на $z$:
    \begin{equation*}
        (w_1, \dots, w_n) \cdot z = (v_1, \dots, v_m) \cdot \underbrace{A \cdot z}_{= \overrightarrow{0}} = (v_1, \dots, v_m) \begin{pmatrix} 0 \\ \dots \\ 0 \end{pmatrix} = \overrightarrow{0}
    .\end{equation*}

    \begin{equation*}
        \implies (w_1, \dots, w_n) \begin{pmatrix} z_1 \\ \dots \\ z_n \end{pmatrix} = \overrightarrow{0} \implies z_1 w_1 + \dots z_n w_n = \overrightarrow{0}
    .\end{equation*}

    Это нетривиальная линейная комбинация, так как $z \neq 0$.

    Следовательно, $w_1, \dots, w_n$ линейно зависимы.
\end{proof}

\begin{example}
    Любые $n + 1$ векторов в $F^n$ линейно зависимы, так как $F^n = \langle e_1, \dots, e_n \rangle$.
\end{example}


\subsection{Базис векторного пространства}

\begin{definition}
    Подмножество $S \subseteq V$ называется \textit{базисом} пространства $V$, если
    \begin{enumerate}[nosep]
    \item $S$ линейно независимо,
    \item $\langle S \rangle = V$.
    \end{enumerate}
\end{definition}

\begin{example}
    $e_1, \dots, e_n$ -- это базис в $F^n$. Он называется \textit{стандартным базисом} в $F^n$.
\end{example}

\begin{comment}
    Всякая линейно независимая система векторов является базисом своей линейной оболочки.
\end{comment}


\subsection{Конечномерные и бесконечномерные векторные пространства}

\begin{definition}
    Векторное пространство $V$ называется \textit{конечномерным}, если в нем есть конечный базис, и \textit{бесконечномерным} иначе.
\end{definition}


\subsection{Независимость числа элементов в базисе векторного пространства от выбора базиса}

\begin{proposal}
    $V$ -- конечномерное векторное пространство. Тогда, все базисы в $V$ содержат одно и то же количество элементов.
\end{proposal}

\begin{proof}
    $V$ конечномерно, тогда существует конечный базис $e_1, \dots, e_n$.

    Пусть $S \subseteq V$ -- другой базис. 
    Так как $\langle e_1, \dots, e_n \rangle = V$, то $\forall v \in S \implies v \in \langle e_1, \dots, e_n \rangle$. 
    Тогда любые $n + 1$ векторов в $S$ линейно зависимы по основной лемме о линейной зависимости.  Но $S$ линейно независимо, значит $|S| \leq n$.

    Пусть $S = \{e'_1, \dots, e'_m\}$, где $m \leq n$. Тогда $\forall i = 1, \dots, n \quad e_i \in \langle e'_1, \dots, e'_m \rangle$, по основной лемме о линейной зависимости получаем $n \leq m$.
    
    То есть $m = n$.
\end{proof}


\subsection{Размерность конечномерного векторного пространства}

\begin{definition}
    \textit{Размерностью} конечномерного векторного пространства называется число элементов в (любом) его базисе.

    Обозначение: $\dim V$.
\end{definition}

\begin{example}~
    \begin{enumerate}
    \item $\dim F^n = n$,
    \item $V = \{\overrightarrow{0}\} \implies \dim V = 0$ так как базисом $V$ будет $\varnothing$.
    \end{enumerate}
\end{example}
