\section{Пять следствий из теоремы Лагранжа}

\begin{theorem}[Теорема лагранжа]
    Пусть $G$ --- конечная группа и $H \subseteq G$ --- подгруппа. Тогда
    \begin{equation*}
        |G| = |H| \cdot \left[G : H\right]
    .\end{equation*}
\end{theorem}

Рассмотрим некоторые следствия из теоремы Лагранжа.

\begin{corollary}
    Пусть $G$ --- конечная группа и $H \subseteq G$ --- подгруппа. Тогда $|H|$ делит $|G|$.
\end{corollary}

\begin{corollary}
    Пусть $G$ --- конечная группа и $g \in G$. Тогда $\ord(g)$ делит $|G|$.
\end{corollary}

\begin{proof}
    Вытекает из следствия 1 и факта, что $\ord(g) = |\left< g \right>|$.
\end{proof}

\begin{corollary}
    Пусть $G$ --- конечная группа и $g \in G$. Тогда $g^{|G|} = e$.
\end{corollary}

\begin{proof}
    Согласно следствию 2, мы имеем $|G| = \ord(g) \cdot s$, откуда $g^{|G|} = \left(g^{\ord(g)}\right)^{s} = e^{s} = e$.
\end{proof}

\begin{corollary}[малая теорема Ферма]
    Пусть $\overline{a}$ --- ненулевой вычет по простому модулю $p$. Тогда $\overline{a}^{p - 1} = 1$.
\end{corollary}

\begin{proof}
    Применим следствие 3 к группе $(\ZZ_p \setminus \{0\}, \times)$.
\end{proof}

\begin{corollary}
    Пусть $G$ --- группа. Предположим, что $|G|$ --- простое число. Тогда $G$ --- циклическая группа, порождаемая любым своим неединичным элементов.
\end{corollary}

\begin{proof}
    Пусть $g \in G$ --- произвольный неединичный элемент. Тогда циклическая подгруппа $\left< g \right>$ содержит более одного элемента и $\left|\left< g \right>\right|$ делит $|G|$ по следствию 1. Значит, $\left|\left< g \right>\right| = |G|$, откуда $G = \left< g \right>$.
\end{proof}
