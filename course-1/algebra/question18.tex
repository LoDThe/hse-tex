\section{Лексикографический порядок на множестве одночленов от нескольких переменных. Лемма о конечности убывающих цепочек одночленов}
        
Пусть $K$ --- поле, $R = K[x_1, \dots, x_n]$.

$M := \{ax_1^{k_1} \cdot \ldots \cdot x_n^{k_n} \mid a \in K \setminus \{0\}, k_i \in \NN \cup \{0\} \}$ --- все одночлены от $x_1, \dots, x_n$.

\begin{definition}[Лексикографический порядок на $M$]
    \begin{equation*}
        a x_{1}^{i_1} \cdot \ldots \cdot x_n^{i_n} \succ b x_1^{j_1} \cdot \ldots \cdot x_n^{j_n} \quad \iff \quad 
        \begin{aligned}
            &i_1 = j_1 \\
            &i_2 = j_2 \\
                &\dots \\
                &i_{k - 1} = j_{k - 1} \\
                &i_k > j_k.
        \end{aligned}
    \end{equation*}
\end{definition}

\begin{example}
    $x_1^2 x_2 \succ x_1^2 x_3^{228}$.
\end{example}

\begin{comment}~
    \begin{enumerate}
    \item $m_1, m_2, m_3 \in M$, $m_1 \prec m_2 \implies m_1 m_3 \prec m_2 m_3$;
    \item $m_1, m_2, m_3 \in M$, $m_1 \prec m_2$, $m_2 \prec m_3 \implies m_1 \prec m_3$.
    \end{enumerate}
\end{comment}

\bigskip
$g \in R \leadsto M(g) := \{\text{все одночлены входящие в $g$}\}$.

\begin{lemma}
    Не существует бесконечных убывающих цепочек $m_1 \succ m_2 \succ m_3 \succ \dots$, где $m_i \in M \ \forall i$.
\end{lemma}

\begin{proof}
    От противного.

    Пусть $m_1 \succ m_2 \succ m_3 \succ \dots$ --- бесконечная убывающая цепочка. Пусть $m_i = a_i x_1^{k_1(i)} \cdot \ldots \cdot x_n^{k_n(i)} \ \forall i \in \NN$.

    Имеем
    \begin{equation*}
        \begin{array}{ccccccccc}
            k_1(1) & \geq & k_1(2) & \geq & k_1(3) & \geq & \dots & \implies & \exists i_1 \in \NN : k_1(i) = k_1(i_1) \ \forall i \geq i_1 \\
            k_2(i_1) & \geq & k_2(i_1 + 1) & \geq & k_2(i_1 + 2) & \geq & \dots & \implies & \exists i_2 \geq i_1 : k_2(i) = k_2(i_2) \ \forall i \geq i_2 \\
            \dots & \dots & \dots & \dots & \dots & \dots & \dots & \dots & \\
            \dots & \dots & \dots & \dots & \dots & \dots & \dots & \implies & \exists i_n \geq i_{n - 1} : k_n(i) = k_n(i_n) \ \forall i \geq i_n
        \end{array}
    \end{equation*}

    \bigskip
    Итог: при $i \geq i_n$ все $m_i$ имеют одинаковые наборы степеней --- противоречие.
\end{proof}
