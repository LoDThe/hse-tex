\section{Старший член многочлена от нескольких переменных. Элементарная редукция многочлена относительно другого многочлена. Лемма о конечности цепочек элементарных редукций относительно системы многочленов}

\begin{definition}
    $f \in R \setminus \{0\} \implies $ \textit{старший член} $L(f)$ --- это наибольший в лексикографическом порядке одночлен, присутствующий в $f$.
\end{definition}

\begin{lemma}[Лемма о старшем члене]
    Пусть $f, g \in R \setminus \{0\}$. Тогда, $L(f, g) = L(f) \cdot L(g)$.
\end{lemma}

\begin{proof}
    $u \in M(f), \ v \in M(g) \implies u \preceq L(f), \ v \preceq L(g)$.

    $uv \preceq L(f) \cdot v \preceq L(f) \cdot L(g) \implies uv \preceq L(f) \cdot L(g)$, причем равенство достигается при
    \begin{math}
        \begin{cases}
            u = L(f),\\
            v = L(g).
        \end{cases}
    \end{math}

    Значит, $L(f) \cdot L(g)$ больше любого другого одночлена в $fg \implies L(f) \cdot L(g) = L(f \cdot g)$.
\end{proof}

Пусть $f, g \in R \setminus \{0\}$, $g$ содержит одночлен $m$, такой что $m \divby L(f)$. Тогда $m = L(f) \cdot m'$, где $m' \in M$.

Элементарная редукция: $g \xrightarrow[]{f} g' := g - m' \cdot f$.

В $g$ одночлен $m$ заменяется суммой нескольких меньших одночленов.

\medskip
Пусть $F \subseteq R \setminus \{0\}$.

\begin{definition}
    $g$ \textit{редуцируем} к $g'$ при помощи $F$, если существует конечная цепочка элементарных редукций
    \begin{equation*}
        g \xrightarrow[]{f_1} g_1 \xrightarrow[]{f_2} g_2 \xrightarrow[]{f_3} \dots \xrightarrow[]{f_k} g_k = g' \text{, где } f_i \in F
    .\end{equation*}

    Обозначение: $g \overset{F}{\rightsquigarrow} g'$.
\end{definition}

$g$ \textit{нередуцируем} относитльно $F$, если $\forall m \in M(g) \ \forall f \in F \quad m \!\!\not\;\divby L(f)$.


\begin{lemma}[Конечность цепочек элементарных редукций]
    $F \subseteq R \setminus \{0\} \implies $ всякая последовательность элементарных редукций относительно $F$ за коненчое число шагов приводит к нередуцируемому многочлену.
\end{lemma}

Обозначение: $L_k(g)$ --- $k$-й по старшинству одночлен в $g \in R$.

\begin{proof}
    От противного.

    Пусть существует бесконечная цепочка элементарных редукций $g_1 \xrightarrow[]{f_1} g_2 \xrightarrow[]{f_2} g_3 \xrightarrow[]{f_3} \dots$

    В силу леммы о конечности убывающих цепочек одночленов имеем
    \begin{equation*}
        \begin{array}{ccccccccc}
            L(g_1) & \succeq & L(g_2) & \succeq & L(g_3) & \succeq & \dots & \implies & \exists i_1 \in \NN : L(g_i) = L(g_{i_1}) \ \forall i \geq i_1 \\
            L_2(g_{i_1)} & \succeq & L_2(g_{i_1} + 1) & \succeq & L_2(g_{i_1} + 2) & \succeq & \dots & \implies & \exists i_2 \geq i_1 : L_2(g_i) = L_2(g_{i_2}) \ \forall i \geq i_2 \\
            \dots & \dots & \dots & \dots & \dots & \dots & \dots
        \end{array}
    \end{equation*}

    Итог: $L(g_{i_1}) = L(g_{i_2}) \succ L_2(g_{i_2}) = L_2(g_{i_3}) \succ L_3(g_{i_3}) = L_3(g_{i_4}) \succ \dots \implies L(g_{i_1}) \succ L_2(g_{i_2}) \succ L_3(g_{i_3}) \succ \dots$ --- бесконечно убывающая цепочка одночленов --- противоречие.
\end{proof}
