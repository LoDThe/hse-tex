\section{Гомоморфизмы групп. Простейшие свойства гомоморфизмов. Изоморфизмы групп. Ядро и образ гомоморфизма групп, их свойства}

\begin{definition}
    Пусть $(G, \circ)$ и $(F, \cdot)$ --- две группы.

    Отображение $\phi \colon G \to F$ называется \textit{гомоморфизмом}, если
    \begin{equation*}
        \phi(g_1 \circ g_2) = \phi(g_1) \cdot \phi(g_2), \quad \forall g_1, g_2 \in G
    .\end{equation*}
\end{definition}

\begin{comment}
    Пусть $\phi \colon G \to F$ --- гомоморфизм групп, и пусть $e_G$ и $e_F$ --- нейтральные элементы группы $G$ и $F$ соответственно. Тогда:
    \begin{enumerate}[nosep]
        \item $\phi(e_G) = e_F$.
        \item $\phi(a^{-1}) = \phi(a)^{-1}$ для любого $a \in G$.
    \end{enumerate}
\end{comment}

\begin{proof}~
    \begin{enumerate}
        \item Имеем $\phi(e_G) = \phi(e_G e_G) = \phi(e_G) \phi(e_G)$. 

            Теперь умножая крайние части этого равенства на $\phi(e_G)^{-1}$, получим $e_F = \phi(e_G)$.

        \item $\phi(g \cdot g^{-1}) = e_F = \phi(g) \phi(g^{-1})$. Умножив обе части на $\phi(g)^{-1}$ получаем необходимое.
            \qedhere
    \end{enumerate}
\end{proof}

\begin{definition}
    Гомоморфизм групп $\phi \colon G \to F$ называется \textit{изоморфизмом}, если отображение $\phi$ биективно.
\end{definition}

\begin{definition}
    Группы $G$ и $F$ называет \textit{изоморфными}, если между ними существует изоморфизм.

    Обозначение: $G \simeq F$.
\end{definition}

В алгебре рассматривают с точностью до изоморфизма: изоморфные группы считаются <<одинаковыми>>.

\begin{definition}
    С каждым гомоморфизмом групп $\phi \colon G \to F$ связаны его \textit{ядро}
    \begin{equation*}
        \ker \phi = \{g \in G \mid \phi(g) = e_f\}
    ,\end{equation*}
    и \textit{образ}
    \begin{equation*}
        \Im \phi = \phi(G) = \{a \in F \mid \exists g \in G : \phi(g) = a\}
    .\end{equation*}
\end{definition}

Ясно, что $\ker \phi \subseteq G$ и $\Im \phi \subseteq F$ --- подгруппы.

\begin{lemma}
    Гомоморфизм групп $\phi \colon G \to F$ инъективен тогда и только тогда, когда $\ker \phi = \{e_G\}$.
\end{lemma}

\begin{proof}
    Ясно, что если $\phi$ инъективен то $\ker \phi = \{e_G\}$.

    Обратно, пусть $g_1, g_2 \in G$ и $\phi(g_1) = \phi(g_2)$. Тогда $g_1^{-1} g_2 \in \ker \phi$, поскольку $\phi(g_1^{-1} g_2) = \phi(g_1^{-1}) \phi(g_2) = \phi(g_1)^{-1} \phi(g_2) = e_F$. Отсюда $g_1^{-1} g_2 = e_G$ и $g_1 = g_2$.
\end{proof}

\begin{corollary}
    Гомоморфизм групп $\phi \colon G \to F$ является изоморфизмом тогда и только тогда, когда $\ker \phi = \{e_G\}$ и $\Im \phi = F$.
\end{corollary}

\begin{proposal}
    Пусть $\phi \colon G \to F$ --- гомоморфизм групп. Тогда подгруппа $\ker \phi$ нормальна в $G$.
\end{proposal}

\begin{proof}
    Достаточно проверить, что $g^{-1} h g \in \ker \phi$ для любых $g \in G$ и $h \in \ker \phi$. Это следует из цепочки равенств
    \begin{equation*}
        \phi(g^{-1} h g) = \phi(g^{-1}) \phi(h) \phi(g) = \phi(g^{-1}) e_F \phi(g) = \phi(g^{-1}) \phi(g) = \phi(g)^{-1} \phi(g) = e_F
    .\qedhere\end{equation*}
\end{proof}
