\section{Подгруппы. Циклические подгруппы. Циклические группы. Порядок элемента. Связь между порядком элемента и порядком порождаемой им циклической подгруппы}

Пусть $G$ --- группа, $g \in G$ и $n \in \ZZ$. Определим степень следующим образом:

\begin{equation*}
    g^n = \begin{cases}
        \underbrace{g \cdots g}_n, &n > 0, \\
        e, &n = 0 \\
        \underbrace{g^{-1} \cdots g^{-1}}_n, &n < 0.
    \end{cases}
\end{equation*}

Свойства:
\begin{enumerate}
\item $g^{m} \cdot g^{n} = g^{m + n}$, $\forall n, m \in \ZZ$;
\item $\left(g^{k}\right)^{-1} = g^{-k}$, $\forall k \in \ZZ$;
\item $\left(g^{n}\right)^{m} = g^{nm}$, $\forall n, m \in \ZZ$.
\end{enumerate}

\begin{definition}
    Пусть $G$ --- группа и $g \in G$. \textit{Циклической подгруппой}, порожденной элементом $g$, называется подмножество $ \{g^{n} \mid n \in \ZZ\}$ в $G$.

    Циклическая подгруппа, порождённая элементом $g$, обозначается $\left< g \right>$. Элемент $g$ называется \textit{порождающим} или \textit{образующим} для подгруппы $\left< g \right>$.
\end{definition}

Например, подгруппа $2\ZZ$ в $(\ZZ, +)$ является циклической, и в качестве порождающего элемента в ней можно взять $g = 2$ или $g = -2$. Другими словами, $2\ZZ = \left< 2 \right> = \left< -2 \right>$.

\begin{definition}
    Группа $G$ называется \textit{циклической}, если найдется такой элемент $g \in G$, что $G = \left< g \right>$.
\end{definition}

\begin{definition}
    Пусть $G$ --- группа и $g \in G$. \textit{Порядком} элемента $g$ называется такое наименьшее натуральное число $m$, что $g^{m} = e$. Если такого натурального числа $m$ не существует, говорят, что порядок элемента $g$ равен бесконечности.
\end{definition}

Порядок элемента обозначается $\ord(g)$. Заметим, что $\ord(g) = 1$ тогда и только тогда, когда $g = e$.

\begin{proposal}
    Пусть $G$ --- группа и $g \in G$. Тогда $\ord(g) = \left|\left< g \right>\right|$.
\end{proposal}

\begin{proof}
    Заметим, что если $g^{k} = g^{s}$, то $g^{k - s} = e$. Поэтому если элемент $g$ имеет бесконечный порядок, то все элементы $g^{n}$, $n \in \ZZ$, попарно различны, и подгруппа $\left< g \right>$ содержит бесконечно много элементов. Если же порядок элемента $g$ равен $m$, то из минимальности числа $m$ следует, что элементы $e = g^0, g = g^{1}, g^{2}, \dots, g^{m - 1}$ попарно различны. Далее, для всякого $n \in \ZZ$ мы имеем $n = mq + r$, где $0 \leq r \leq m - 1$, и 
    \begin{equation*}
        g^{n} = g^{mq + r} = \left(g^{m}\right)^{q} g^{r} = e^{q} g^{r} = g^{r}
    .\end{equation*}
    Следовательно, $\left< g \right> = \{e, g, g^2, \dots, g^{m - 1}\}$ и $\left|\left< g \right>\right| = m$.
\end{proof}

Ясно, что всякая циклическая группа коммутативна и не более чем счётна. Примерами циклических группа являются группы $(\ZZ, +)$ и $(\ZZ_n, +)$, $n \geq 1$.
