\section{Порядок конечного поля. Автоморфизм Фробениуса}

K -- конечное поле $\chara{K}=p>0$ -- простое число.\\
Пусть $\langle 1\rangle \subseteq K$ -- подгруппа по сложению, порождаемая 1.\\
Заметим, что $\langle 1\rangle$ -- подкольцо, изоморфное $\Z_p\Rightarrow \langle 1\rangle$ -- поле, изоморфное $\Z_p$.\\
\textbf{Теорема.} $|K|=p^n$, где $n=\dim_{\Z_p}K$
\begin{proof}
    $K\supseteq \Z_p\Rightarrow K$ -- векторное пространство над $\Z_p$.\\
    Пусть $n=\dim_{\Z_p}K$. Выберем базис $e_1,...,e_n$ в $K$ над $\Z_p$.\\
    Тогда $K=\{ a_1e_1+...+a_ne_n \ | \ a_i\in \Z_p\}$\\
    $\forall \ a_i$ есть ровно $p$ вариантов $\Rightarrow |K|=p^n$
\end{proof}
\noindent \textbf{Общая конструкция конечных полей.}\\
Выбираем неприводимый многочлен $h\in \Z_p[x], \ \deg{h}=n$. Тогда $F:=\Z_p[x]/(h)$ -- поле, векторное пространство над $\Z_p$ размерности $n\Rightarrow |F|=p^n$.\\\\
\textbf{Автоморфизм Фробениуса.}\\
$a,b\in K\Rightarrow $\\
$(a+b)^p=a^p+C_p^1a^{p-1}b+C_p^2a^{p-2}b^2+...+C_p^{p-1}ab^{p-1}+b^p=a^p+b^p$, так как $C_p^k\ \vdots\ p$ при $1\leqslant k\leqslant p-1$\\
Рассмотрим отображение $\varphi: K\to K, a\to a^p$. Имеем:\\
$\varphi(a+b)=(a+b)^p=a^p+b^p=\varphi(a)+\varphi(b)$,\\
$\varphi(ab)=(ab)^p=a^pb^p=\varphi(a)\cdot\varphi(b)$\\
$\Rightarrow \varphi$ -- гомоморфизм колец.\\
$\ker\varphi$ -- идеал в $K$, но в поле нет собственных идеалов $\Rightarrow $ либо $\ker\varphi=K$, либо $\ker\varphi=\{0\}$. Так как $\varphi(1)=1$, то $\ker\varphi\neq K\Rightarrow \ker\varphi=\{0\}\Rightarrow\varphi$ инъективно.\\
Если $|K|<\infty$, то $\varphi$ -- биекция. В этом случае $\varphi$ называется \textbf{автоморфизмом Фробениуса}. (''автоморфизм''=''изоморфизм в себя'')
