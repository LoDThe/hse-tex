\section{Критерий того, что факторкольцо $\KK[x]/(h)$ является полем. Базис и размерность факторкольца $\KK[x]/(h)$ как векторного пространства над полем $\KK$}

Пусть $h = a_n x^n + \dots + a_1 x + a_0 \in K[x]$ --- многочлен, $\deg h = n > 0$.

Тогда, рассмотрим $F := K[x] / (h) \quad f \in K[x] \leadsto \overline{f} := f + (h) \in F$.

\begin{comment}
    $\overline{f} = \overline{0} \iff f \divby h$.
\end{comment}

\begin{proposal}
    $F$ --- поле $\iff h$ неприводим.
\end{proposal}

\begin{proof}~
    \begin{description}
        \item[$\implies$]  Если $h = h_1 \cdot h_2$ и $\deg h_i < n$, то $\overline{h} = \overline{h_1} \cdot \overline{h_2}$.

            Так как $\overline{h} = 0$, то $\overline{h_1} \cdot \overline{h_2} = 0$. 
            Значит в $F$ есть делители нуля $ \implies F$ --- не поле --- противоречие.

        \item[$\impliedby$] $f \in K[x]$, $\overline{f} \neq \overline{0} \implies f \!\!\not\;\divby h \implies \gcd(f, h) = 1$.

            Значит, $\exists u, v \in K[x]$, такие что $1 = uf + vh$. Отсюда $\overline{1} = \overline{u} \cdot \overline{f} + \overline{v} \cdot \overline{h} = \overline{u} \cdot \overline{f}$.

            Получили, что $\overline{f}$ обратим $\implies F$ --- поле.
            \qedhere
    \end{description}
\end{proof}

\begin{example}~
    \begin{enumerate}
        \item $\RR[x] / (x^2 + 1)$ --- поле $(\simeq \CC)$.
        \item $\RR[x] / (x^2)$ --- не поле, $\overline{x}$ --- нильпотент, $\overline{x}^2 = \overline{0}$.
    \end{enumerate}
\end{example}

\bigskip
Рассмотрим отображение $K \to F, \quad \lambda \mapsto \overline{\lambda} = \lambda + (h)$, оно инъективно. Тогда $K$ отождествляется с подполем в $F$, значит $F$ становится векторным пространством над $K$.

\begin{proposal}
    Элементы $\overline{1}$, $\overline{x}$, $\dots$, $\overline{x}^{n - 1}$ образуют базис в $F$ над $K$. В частности $\dim_K F = n$.
\end{proposal}

\begin{proof}
    Пусть $\overline{f} \in F$, $f \in K[x]$. Поделим $f$ на $h$ с остатком:

    $f = q \cdot h + r$, где $r = 0$ или $\deg r < n$. Тогда, $\overline{f} = \underbrace{\overline{q} \cdot \overline{h}}_{= \overline{0}} + \overline{r} = \overline{r} \in \left< \overline{1}, \overline{x}, \dots, \overline{x}^{n - 1} \right>$.

    Если $b_0 \overline{1} + b_1 \overline{x} + \dots + b_{n - 1} \overline{x}^{n - 1} = \overline{0}$ для некоторых $b_i \in K$, то $b_0 + b_1 x + \dots + b_{n - 1} x^{n - 1} \divby h \implies b_0 = b_1 = \dots = b_{n - 1} = 0$.
\end{proof}
