\section{Сходимость степенного метода для диагонализуемых матриц.}

Дано: $A$ --- диагонализуемая и $\lambda^{(1)}$ --- простое ($|\lambda^{(1)}| >
    |\lambda^{(2)} \geq \dots \geq |\lambda^{(n)}$).

Доказать: отношение Релея $R(A x_k, x_k) = \dfrac{(A x_k, x_k)}{(x_k, x_k)}
    = (A x_k, x_k)$ сходится к $\lambda^{(1)}$ --- старшему собственному значению.

\begin{proof}
    Поскольку $A$ диагонализуемая, то существует $n$ линейно независимых собственных
    векторов $v^{(i)}$, далее считаем, что $\|v^{(i)}\| = 1$.
    В таком случае $x_0 = \alpha_1 v^{(1)} + \dots +
        \alpha_n v^{(n)}$. \textbf{Ключевой момент:} считаем, что $\alpha_1 \neq 0$. Мы используем
    это в доказательстве, а в противоположном случае метод вообще сойдётся не пойми
    куда, но точно не туда, куда надо было.

    \begin{flalign*}
        x_k & = \dfrac{A x_{k - 1}}{\norm[2]{A x_{k - 1}}} = \dfrac{A^k x_0}{\norm[2]{A^k x_0}} =
        \dfrac{\alpha_1 (\lambda^{(1)})^k v^{(1)} + \dots + \alpha_n (\lambda^{(n)})^k v^{(n)}}
        {\norm[2]{\alpha_1 (\lambda^{(1)})^k v^{(1)} + \dots + \alpha_n (\lambda^{(n)})^k v^{(n)}}}                           \\
            & = \left(\dfrac{\lambda^{(1)}}{|\lambda^{(1)}|}\right)^k \cdot \left(\dfrac{\alpha_1}{|\alpha_1|}\right)^k
        \cdot \dfrac{v^{(1)} +
            \dfrac{\alpha_2}{\alpha_1} \left(\dfrac{\lambda^{(2)}}{\lambda^{(1)}}\right)^k v^{(2)} +
            \dots + \dfrac{\alpha_n}{\alpha_1} \left(\dfrac{\lambda^{(n)}}{\lambda^{(1)}}\right)^k v^{(n)}}
        {\norm[2]{v^{(1)} + \dfrac{\alpha_2}{\alpha_1} \left(\dfrac{\lambda^{(2)}}{\lambda^{(1)}}\right)^k v^{(2)} +
        \dots + \dfrac{\alpha_n}{\alpha_1} \left(\dfrac{\lambda^{(n)}}{\lambda^{(1)}}\right)^k v^{(n)}}}                      \\
            & = e^{i\varphi} \cdot \dfrac{v^{(1)} + \text{O}\left(\left(\dfrac{\lambda^{(2)}}{\lambda^{(1)}}\right)^k\right)}
        {\norm[2]{v^{(1)} + \text{O}\left(\left(\dfrac{\lambda^{(2)}}{\lambda^{(1)}}\right)^k\right)}}               \oeq     \\
    \end{flalign*}
    Знаменатель равен $1 + \text{O}\left(\left(\dfrac{\lambda^{(2)}}{\lambda^{(1)}}\right)^k\right)$,
    разложим $\dfrac{1}{1 + \text{O}\left(\left(\dfrac{\lambda^{(2)}}{\lambda^{(1)}}\right)^k\right)}$ по Тейлору
    и получим $1 + \text{O}\left(\left(\dfrac{\lambda^{(2)}}{\lambda^{(1)}}\right)^k\right)$, откуда:
    \begin{flalign*}
         & \oeq e^{i\varphi} \cdot (v^{(1)} + \text{O}\left(\left(\dfrac{\lambda^{(2)}}{\lambda^{(1)}}\right)^k\right))
        (1 + \text{O}\left(\left(\dfrac{\lambda^{(2)}}{\lambda^{(1)}}\right)^k\right))                                  \\
         & = e^{i\varphi} \cdot (v^{(1)} + \text{O}\left(\left(\dfrac{\lambda^{(2)}}{\lambda^{(1)}}\right)^k\right))    \\
    \end{flalign*}

    Отсюда имеем:
    \begin{flalign*}
        \text{R}(x_k)
         & = (Ax_k, x_k) = (A \cdot (e^{i\varphi} \cdot
            (v^{(1)} + \text{O}\left(\left(\dfrac{\lambda^{(2)}}{\lambda^{(1)}}\right)^k\right))))^T
        \cdot (e^{i\varphi} \cdot
        (v^{(1)} + \text{O}\left(\left(\dfrac{\lambda^{(2)}}{\lambda^{(1)}}\right)^k\right)))                   \\
         & =(e^{i\varphi} \cdot
        (v^{(1)} + \text{O}\left(\left(\dfrac{\lambda^{(2)}}{\lambda^{(1)}}\right)^k\right)))^T
        A (e^{i\varphi} \cdot
        (v^{(1)} + \text{O}\left(\left(\dfrac{\lambda^{(2)}}{\lambda^{(1)}}\right)^k\right)))                   \\
         & =
        (e^{i\varphi} \cdot
        (v^{(1)} + \text{O}\left(\left(\dfrac{\lambda^{(2)}}{\lambda^{(1)}}\right)^k\right)))^T
        (e^{i\varphi} \cdot
        (\lambda^{(1)}v^{(1)} + \text{O}\left(\left(\dfrac{\lambda^{(2)}}{\lambda^{(1)}}\right)^k\right)))      \\
         & =
        \lambda^{(1)} \cdot \norm[2]{e^{i\varphi} \cdot
        (v^{(1)} + \text{O}\left(\left(\dfrac{\lambda^{(2)}}{\lambda^{(1)}}\right)^k\right))}                   \\
         & = \lambda^{(1)} \cdot (1 + \text{O}\left(\left(\dfrac{\lambda^{(2)}}{\lambda^{(1)}}\right)^k\right))
    \end{flalign*}
\end{proof}
