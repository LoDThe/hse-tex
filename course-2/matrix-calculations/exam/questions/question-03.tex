\section{Существование и единственность LU и LDL разложений.}
\begin{definition*}
    Пусть есть матрица $A \in \RR^{n\times n}$. Разложение $A = LU$ называется $LU$-разложением, если матрица $L$ нижнетреугольная с единицами на диагонали, а матрица $U$ верхнетреугольная.
\end{definition*}
\begin{definition*}
    Матрица $A$ называется \textbf{строго регулярной}, если все её ведущие подматрицы невырождены. (ведущие подматрицы~--- верхние левые $k\times k$ блоки).
\end{definition*}
\begin{theorem*}[Существование $LU$-разложения]
    Пусть $\det(A) \neq 0$. Тогда $A$ имеет $LU$-разложение $\iff$ $A$ строго регулярна.
\end{theorem*}
\begin{proof}~

    $\Longrightarrow$ 
    
    $A$ имеет $LU$-разложение, то есть $A = LU$.
    
    Мы знаем, что матрица невырождена, то есть 
    \begin{equation*}
        0 \neq \det(A) = \underbracket{\det(L)}_1 \det(U) = u_{11} \cdot \ldots \cdot u_{nn}.
    \end{equation*}
    Следовательно, $u_{kk} \neq 0$ для любого $k \in \{1, \dots, n\}$. Дальше нам надо убедиться, что матрица $A$ строго регулярна. То есть, надо проверить что ведущие подматрицы тоже невырождены. Запишем для ведущих подматриц:
    \begin{equation*}
        A = \begin{pmatrix}
            L_k & 0 \\
            * & *
        \end{pmatrix} \begin{pmatrix}
            U_k & * \\
            0 & *
        \end{pmatrix} = \begin{pmatrix}
            L_k U_k & * \\
            * & *
        \end{pmatrix}.
    \end{equation*}
    Обозначим $A_k := L_k U_k$. Тогда $\det(A_k) = \underbracket{\det(L_k)}_1 \det(U_k) = u_{11} \cdot \ldots \cdot u_{kk} \neq 0$, аналогично случаю с полной матрицей.

    $\Longleftarrow$
    
    Доказываем по индукции. Мы считаем, что пусть для $n-1$ уже доказано утверждение. Докажем для $n$.

    Пусть матрица $A = \begin{pmatrix}
        a & c^t \\
        b & D
    \end{pmatrix}$. Здесь $a$~--- число, не равное нулю в силу строгой регулярности, $D$~--- матрица $n - 1 \times n - 1$. Мы считаем, что для $D$ мы уже умеем искать $LU$~--- разложение. Давайте попробуем преобразовать нашу матрицу, чтобы она привелась к блочно-верхнетреугольному виду:
    \begin{equation*}
        \begin{pmatrix}
            1 & 0 \\[2pt]
            -\frac{1}{a}b & I
        \end{pmatrix} \begin{pmatrix}
            a & c^t \\
            b & D
        \end{pmatrix} = \begin{pmatrix}
            a & c^T \\
            0 & D - \frac{1}{a}bc^T
        \end{pmatrix} =: A'.
    \end{equation*}
    Блок $D - \frac{1}{a}bc^T$ называется дополнением по Шуру матрицы $A$. Обозначим $A_1 := D - \frac{1}{a}bc^T$. 
    
    Докажем, что $A_1$ строго регулярна (было в ДЗ). $A'$ получилась из $A$ с помощью $n-1$-го элементарного преобразования первого типа (вычесть из строки другую, домноженную на коэффициент). Помним, что такие элементарные преобразования не меняют определитель матрицы, поэтому $0 \neq \Delta_k(A) = \Delta_k(A')$ для любого $k \in \{1, \dots, n\}$. (здесь $\Delta_k(A)$~--- главный угловой минор матрицы $A$). Но в матрице $A'$ мы видим угол нулей, поэтому $0 \neq \Delta_k(A') = a \cdot \Delta_{k-1}(A_1)$ для любого $k \in \{1, \dots, n \}$. Следовательно, $\Delta_{i}(A_1) \neq 0$ для любого $i \in \{1, \dots, n-1\} \implies A_1$ строго регулярна. 

    Продолжим доказательство теоремы. По предположению индукции тогда считаем, что $A_1$ имеет $LU$-разложение: \\[2pt] $D - \frac{1}{a}bc^T = A_1 = L_1U_1$. Этого уже достаточно для того, чтобы построить $LU$-разложение самой матрицы $A$:
    \begin{equation*}
        \begin{pmatrix}
            1 & 0 \\
            \frac{1}{a}b & L1
        \end{pmatrix} \begin{pmatrix}
            a & c^T \\
            0 & U_1
        \end{pmatrix} = \begin{pmatrix}
            a & c^T \\ 
            b & \frac{1}{a}bc^T + L_1U_1
        \end{pmatrix} = \begin{pmatrix}
            a & c^T \\ 
            b & D
        \end{pmatrix}.
    \end{equation*}

\end{proof}
\begin{proposition*}
    $LU$-разложение определяется единственным образом.
\end{proposition*}
\begin{proof}~

    Предположим, что есть два разложения:
    \begin{equation*}
        A = L_1 U_1 = L_2 U_2.
    \end{equation*}
    Преобразуем равенство:
    \begin{equation*}
        L_2^{-1}L_1 = U_2U_1^{-1}.
    \end{equation*}
    Обратная к нижнетреугольной матрице~--- нижнетреугольная матрица, и произведение нижнетреугольных~--- тоже нижнетреугольная. Для верхнетреугольных то же самое. Значит, $L_2^{-1}L_1$~--- диагональная матрица. Более того, это единичная матрица (в силу того, что на диагонали матриц $L_2$ и $L_1$ стоят $1$). Значит, 
    \begin{align*}
        L_1 &= L_2; \\
        U_1 &= U_2.
    \end{align*}
\end{proof}
\begin{corollary}[$LDL$-разложение]
    Пусть $A \in \CC^{n \times n}$ является строго регулярной и $A = A^*$. Тогда $\exists L$~--- нижнетреугольная и $D$~--- диагональная, такие что
    \begin{equation*}
        A = LDL^*.
    \end{equation*}
\end{corollary}
\begin{proof}~

    $A$~--- строго регулярна, следовательно,
    \begin{equation*}
        A = LU = L\underbracket{D}_{\mathclap{\diag(u_{11}, \dots, u_{nn})}}D^{-1}U = A^* = (U^* D^{-*})(D^* L^*)
    \end{equation*}
    Но $LU$-разложение единственно, следовательно, $L = U^*D^{-*}$. Значит, $U = DL^*$. Доказали.

\end{proof}