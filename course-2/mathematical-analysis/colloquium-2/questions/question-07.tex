% Здесь НЕ НУЖНО делать begin document, включать какие-то пакеты..
% Все уже подрубается в головном файле
% Хедер обыкновенный хсе-теха, все его команды будут здесь работать
% Пожалуйста, проверяйте корректность теха перед пушем

% Здесь формулировка билета
\subsection{Сформулируйте и докажите основные свойства внешней меры Жордана: монотонность и полуаддитивность} 

\textbf{\underline{Св-во:} } Монотонность внешней меры означает, что при $A \subseteq B$ выполняется $\overline{\mu}(A) \leq \overline{\mu}(B)$. \\
\textbf{\underline{Док-во:} } Обозначим через $\mathcal{E}_A$ класс простых множеств, покрывающих заданное ограниченное множество $A$. Так как $A \subseteq B$, то класс $\mathcal{E}_A$ шире чем $\mathcal{E}_B$, а значит в нем найдется простое множество которое не больше любого из $\mathcal{E}_B$, а отсюда из определения внешней меры следует, что $\overline{\mu}(A) \leq \overline{\mu}(B)$
\begin{flushright}
$\blacksquare$
\end{flushright}
\textbf{\underline{Св-во:} } Полуаддитивностью внешней меры называется 
$$\overline{\mu}(A\sqcup B) \leq \overline{\mu}(A) + \overline{\mu}(B),$$
где $A$ и $B$ - произвольные ограниченные множества. \\
\textbf{\underline{Док-во:} } Для любых $E_A$ и $E_B$ покрывающих $A$ и $B$ соответственно, верно что $E_A\cup E_B$ - покрывает $A\cup B$. По свойствам меры верно
$$\mu(E_A\cup E_B) \leq \mu(E_A) + \mu(E_B)$$
Далее по определению точной нижней грани, для любого $\varepsilon > 0$ найдутся такие $E_A$ и $E_B$, что 
$$\mu(E_A) \leq \overline{\mu}(A) + \varepsilon, \ \ \ \ \mu(E_B) \leq \overline{\mu}(B) + \varepsilon$$
Отсюда имеем
$$\overline{\mu}(A\cup B) \leq \mu(E_A\cup E_B) \leq \overline{\mu}(A) + \overline{\mu}(B) + 2\varepsilon$$
Переходя к пределу $\varepsilon \rightarrow 0$ имеем
$$\overline{\mu}(A\cup B) \leq \overline{\mu}(A) + \overline{\mu}(B)$$
(Искомое свойство выполняется как частный случай)
\begin{flushright}
$\blacksquare$
\end{flushright}
