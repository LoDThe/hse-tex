% Здесь НЕ НУЖНО делать begin document, включать какие-то пакеты..
% Все уже подрубается в головном файле
% Хедер обыкновенный хсе-теха, все его команды будут здесь работать
% Пожалуйста, проверяйте корректность теха перед пушем

% Здесь формулировка билета
\subsection{Что такое разбиение Жорданова множества? Как вводится произведение разбиений? В каком случае говорят, что одно разбиение является измельчением другого?}

Пусть $E \supseteq D$ - простое множество покрывающее $D$. Пусть $E = \sqcup Q_i$, где $Q_i$ - $m$-мерные полуинтервалы составляющие простое множество $E$. \\
\textbf{\underline{Опр.:} } \textit{Разбиением $\tau$} множества $D$, соответствующим данному простому множеству $E$, назовем представление $D$ в виде 
\[D = \sqcup(D\cap Q_i) = \sqcup D_i, \ \ \ \ \ D_i = D\cap Q_i\]
\textbf{\underline{Опр.:} } \textit{Произведение разбиений} $\tau = \{D_i \ | \ i = 1, ..., n\}$ и $\tau' = \{D_j' \ | \ j = 1, ..., k\}$ называется система множеств
\[\tau\cdot\tau' = \{D_i\cap D_j'\ | \ i = 1, ..., n, \ j = 1, ..., k\}\]
\textbf{\underline{Опр.:} } Разбиение $\tau$ называется \textit{измельчением} разбиения $\tau'$ (пишется $\tau \leq \tau'$), если для любого $D_j' \in \tau'$ найдутся такие $D_1, ..., D_m \in \tau$, что 
\[D_j' = D_1\sqcup ... \sqcup D_m\]

