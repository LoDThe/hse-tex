\subsection{Что такое исчерпание $\{D_n\}$ множества $D \subseteq \RR^m$? Что можно утверждать в случае, когда $D$ -- жорданово множество?}

Пусть множество $D \subseteq R^m$ таково, что существует последовательность жордановых множеств $D_n \subseteq D$ такая, что
\[ D_1 \subseteq D_2 \subseteq \dots \text{, а также } D_1 \cup D_2 \cup \dots = D \]

Тогда последовательность $\{ D_n\}$ называется \textit{исчерпанием} множества $D$, а само множество $D$ называется \textit{пределом} возрастающей
последовательности $\{D_n\}$.

\begin{theorem*}
    Если $D$ -- жорданово, то $\lim_{n \to +\infty} \mu(D_n) = \mu(D)$
\end{theorem*}

\begin{proof}
    Последовательность жордановых множеств $A_n = D \setminus D_n$ убывает и сходится к пустому множеству.
\end{proof}