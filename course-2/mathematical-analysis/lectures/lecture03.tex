\section{Лекция 3 - 15.09.2020 - Знакопеременные ряды}
\subsection{Абсолютная и условная сходимость}
\begin{definition}
    $\sum_{n=1}^{\infty} a_n, a_n \in \RR$

    Если $a_n \cdot a_{n+1} < 0$, то ряд называется знакочередующимся.
\end{definition}

Пусть $\sum  a_n$ сходится

\begin{definition}
    Рассмотрим дополнительный ряд $\sum\left|a_n\right|$ (*)

    Если (*) сходится, то $\sum a_n$ называется сходящимся абсолютно

    Если (*) расходится, то $\sum a_n$ называется сходящимся условно
\end{definition}


\begin{definition}
    Введём $a_n^{+} = \begin{cases}
        a_n, a_n > 0 \\
        0
    \end{cases}$ 
    $a_n^{+} = \begin{cases}
        |a_n|, a_n < 0 \\
        0
    \end{cases}$ 

    Ряды $\sum a_n^+$, $\sum a_n^-$ называются положительной и отрицательной частью исходного ряда $\sum a_n$
\end{definition}


$S_N^+ = \sum_{n=1}^{N} a_n^+$, $S_N^- = \sum_{n=1}^{N} a_n^-$

$a_n = a_n^+ - a_n^-$, $\left|a_n\right| = a_n^+ + a_n^-$

$\sum_{n=1}^{\infty}a_n = S_N^+ - S_N^-$, $\sum_{n=1}^{\infty}a_n = S_N^+ + S_N^-$

\begin{comment}
    Ряд $\sum a_n$ сходится абсолютно $\iff$ оба ряда $\sum a_n^+$, $\sum a_n^-$ сходятся
    Ряд $\sum a_n$ сходится условно $\implies$ оба ряда $\sum a_n^+$, $\sum a_n^-$ расходятся
\end{comment}

\subsection{Мажорантный признак Вейерштрасса}

\begin{theorem}
    Если $|a_n| \leq b_n$ при $n > n_0$ и положительный ряд $\sum b_n$ сходится,
    то $\sum a_n$ сходится, причём абсолютно.
\end{theorem}

\begin{example}
$\sum_{n=1}^{\infty} \dfrac{\sin(nx)}{n^p}$, $p > 0$

$|sin(nx)| \leq 1 \implies \left|\dfrac{sin(nx)}{n^P}\right| \leq \dfrac{1}{n^p}$

$\sum \dfrac{1}{n^p} $ сходится $(p > 1) \implies \sum_{n=1}^{\infty} \dfrac{\sin(nx)}{n^p}$ сходится абсолютно.
\end{example}

\subsection{Группировка членов ряда}

Говорят, что ряд $\sum b_k$ получен из $\sum a_n$ группировкой членов, если $\exists n_1 < n_2 < \dots$:

$b_1 = a_1 + a_2 + \dots + a_{n_1}$

$b_2 = a_{n_1 + 1} + a_{n_1 + 2} + \dots + a_{n_2}$

$\dots$

\begin{comment}
    Если $\sum a_n$ сходится, то ряд $\sum b_k$ сходится к той же сумме.
\end{comment}

\begin{proof}
$\sum_{k=1}^{m} b_k = \sum_{n=1}^{n_m} a_n$
\end{proof}

\textit{Обратное утверждение неверно:} $(1 - 1) + (1 - 1) + \dots$

Знакопеременный ряд при помощи группировки сводится к знакочередующемуся:

$a_1 \leq 0$, $\dots$, $a_{n_1} \leq 0$; $b_1 = \sum_{i=1}^{n_1} a_i \leq 0$

$a_{n_1+1} \geq 0$, $\dots$, $a_{n_2} \geq 0$; $b_1 = \sum_{i={n_1 + 1}}^{n_2} a_i \leq 0$

При такой группировке сходимость исходного ряда $\iff$ сходимость $\sum b_n$

\begin{example}
    $\sum_{n=1}^{\infty} \dfrac{(-1)^{[\ln n]}}{n}$

    $\sum_{k=0}^{\infty} b_k$, где $b_k = (-1)^k$

    $|b_k| = \sum_{n=[e^k] + 1}^{[e^{k+1}]} \dfrac{1}{n} \leq \dfrac{1}{[e^k] + 1} \cdot ([e^{k+1}]-[e^k]) \approx \dfrac{e^{k+1} - e^k}{e^k} \to e - 1 > 0$
\end{example}

\subsection{Знакочередующиеся ряды, пр-к Лейбница}

$\sum_{n=1}^{\infty} a_n$, где $a_n = (-1)^n \cdot u_n$, $u_n > 0$

\begin{theorem}
Признак Лейбница. Если $u_n \downarrow 0$, то ряд сходится, причём $|r_n| \leq u_{n+1}$
\end{theorem}

\begin{example}
    $\sum_{n=1}^{\infty} \dfrac{(-1)^{n}}{n^p}$, $p > 0$

    $\dfrac{1}{n^p} \downarrow 0 \implies $ ряд сходится (при $\forall p > 0$)
\end{example}

При этом $\sum_{n=1}^{\infty} \left|\dfrac{(-1)^{n}}{n^p}\right| = \sum_{n=1}^{\infty} \dfrac{1}{n^p}$ -- сходится при $p > 1$
и расходится при $p \leq 1$

$\sum_{n=1}^{\infty} \dfrac{(-1)^{n}}{n^p}$: $p \in (0;1]$ -- сходится условно, $p \in (1; +\infty)$ -- абсолютно

\subsection{О неприменимости эквивалентности}

Рассмотрим 2 ряда:

$\sum_{n=1}^{\infty} \dfrac{(-1)^n}{\sqrt{n} - (-1)^n}$ $\sum_{n=1}^{\infty} \dfrac{(-1)^n}{\sqrt{n}}$

$\dfrac{(-1)^n}{\sqrt{n} - (-1)^n} \approx \dfrac{(-1)^n}{\sqrt{n}}$

При этом правый ряд сходится по признаку Лейбница, а левый -- расходится:

$\dfrac{(-1)^n}{\sqrt{n} - (-1)^n} - \dfrac{(-1)^n}{\sqrt{n}} = \dfrac{1}{\sqrt{n}(\sqrt{n} - (-1)^n)} \approx \dfrac{1}{n}$

$\sum_{n=1}^N \dfrac{(-1)^n}{\sqrt{n} - (-1)^n} = \sum_{n=1}^N \dfrac{(-1)^n}{\sqrt{n}} +\sum_{n=1}^N \dfrac{1}{\sqrt{n}(\sqrt{n} - (-1)^n)} \to \infty$

\subsection{Признаки Дирихле и Абеля}

$\sum_{n=1}^{\infty}a_n \cdot b_n$

\begin{theorem}
    Признак Дирихле. Если $a_n \downarrow 0$, а частичные суммы $\left| \sum_{n=1}^N b_n \right| \leq C$ ограничены,
    то $\sum_{n=1}^{\infty}a_n \cdot b_n$ сходится.
\end{theorem}

\begin{theorem}
    Признак Абеля. Если $a_n$ монотонна и ограничена, а ряд $\sum_{n=1}^{\infty}b_n$ сходится,
    то $\sum_{n=1}^{\infty}a_n \cdot b_n$ сходится.
\end{theorem}

$a_n \to a$, $a_n = a +- \alpha_n$, $\alpha_n \downarrow 0$; $\sum_{n=1}^{\infty}a_n \cdot b_n = a \sum_{n=1}^{\infty}b_n +- \sum_{n=1}^{\infty}\alpha_n \cdot b_n$

\begin{example}
    $\sum_{n=1}^{\infty} \dfrac{\sin(nx)}{n^p}$, $p > 0$

    $a_n = \dfrac{1}{n^p} \downarrow 0$, $b_n = \sin nx$

    $b_1 + b_2 + b_3 + \dots + b_N = \sin x+ \sin 2x + \dots + \sin Nx = \dfrac{\cos \dfrac{x}{2} - \cos\left((N + 1/2)x\right)}{2 \sin \dfrac{x}{2}}$; $\left|\sum_{n=1}^{N}b_n\right| \leq \dfrac{2}{2\sin{\dfrac{x}{2}}} = \dfrac{1}{\sin{\dfrac{x}{2}}}$

    Ряд сходится по признаку Дирихле
\end{example}

\subsection{Влияние перестановки членов ряда на его сумму}

Пусть $f: \NN \to \NN$ -- биекция

Говорят, что ряд $\sum b_n$ получен из $\sum a_n$ перестановкой членов, если $b_n = a_{f(n)}$

Если ряд $\sum a_n$ сходится абсолютно, то $\forall$ ряд, полученный из него перестановкой членов, сходится абсолютно к той же сумме.

\begin{theorem}
(Римана) Если ряд $\sum a_n$ сходится условно, то для $\forall S \in [-\infty; +\infty]$ то $\exists$ перестановка $f$ такая, что $\sum a_{f(n)} = S$
\end{theorem}