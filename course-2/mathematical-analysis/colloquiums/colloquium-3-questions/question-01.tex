\subsection{Собственный интеграл, зависящий от параметра. Теорема о непрерывности по параметру. Теорема о дифференцировании по параметру под знаком интеграла. Теорема об интегрировании по параметру под знаком интеграла.}

\subsubsection{Собственный интеграл, зависящий от параметра.}

\begin{definition*}
    Собственным интегралом, зависящем от параметра, будем называть интеграл вида
    \begin{equation*}
        F(y) = \int_{\alpha(y)}^{\beta(y)} f(x, y) \dd x,
    \end{equation*}
    где $\alpha$ и $\beta$ это некие функции, определенные для $y$ из некоторого отрезка $[c; d]$. 

    Часто $\alpha$ и $\beta$ являются константами и интеграл принимает следующий вид:
    \begin{equation*}
        F(y) = \int_{\alpha}^{\beta} f(x, y) \dd x.
    \end{equation*}
\end{definition*}

\subsubsection{Теорема о непрерывности по параметру}

\begin{theorem*}
    Рассмотрим $G = [a; b] \times [c; d]$ и пусть функция $f\colon G \to \RR$ --- непрерывна на ограниченном замкнутом множестве, откуда следует, что она равномерно непрерывна.

    Пусть $\alpha$, $\beta$ непрерывны на отрезке $[c; d]$, тогда функция
    \begin{equation*}
        F(y) = \int_{\alpha(y)}^{\beta(y)} f(x, y) \dd x
    \end{equation*}
    равномерно непрерывна на $[c; d]$.
\end{theorem*}

\begin{proof}
    Докажем непрерывность. 
    
    Пусть функция $f$ ограничена каким-то числом $M$.

    В силу непрерывности $\alpha$ и $\beta$ для любого $\eps > 0$ существует $\delta > 0$, что из условия $|y - y_0| < \delta$ следует $|\alpha(y) - \alpha(y_0)| < \eps$ и $|\beta(y) - \beta(y_0)| < \eps$.

    В силу равномерной непрерывности $f$ для любого $\eps > 0$ существует $\delta > 0$, что из условия $|y - y_0| < \delta$ следует $|f(x, y) - f(x, y_0)| < \eps$.

    Воспользуемся этим:
    \begingroup
    \allowdisplaybreaks
    \begin{align*}
        |F(y) - F(y_0)|
        &= \Big|\int_{\alpha(y)}^{\beta(y)} f(x, y) \dd x - \int_{\alpha(y_0)}^{\beta(y_0)} f(x, y_0) \dd x\Big| \\
        &\left[\text{ --- прибавим и вычтем член }\int_{\alpha(y_0)}^{\beta(y_0)} f(x, y) \dd x\right] \\
        &= \Big|\int_{\alpha(y)}^{\beta(y)} f(x, y) \dd x - \int_{\alpha(y_0)}^{\beta(y_0)} f(x, y) \dd x + \int_{\alpha(y_0)}^{\beta(y_0)} f(x, y) \dd x - \int_{\alpha(y_0)}^{\beta(y_0)} f(x, y_0) \dd x\Big| \\
        &\leq \Big|\int_{\alpha(y)}^{\beta(y)} f(x, y) \dd x - \int_{\alpha(y_0)}^{\beta(y_0)} f(x, y) \dd x\Big| + \Big|\int_{\alpha(y_0)}^{\beta(y_0)} f(x, y) \dd x - \int_{\alpha(y_0)}^{\beta(y_0)} f(x, y_0) \dd x\Big| \\
        &\left[\text{ --- оценим слагаемое с модулем интеграла как интеграл модуля }\right] \\
        &\leq \Big|\int_{\alpha(y)}^{\beta(y)} f(x, y) \dd x - \int_{\alpha(y_0)}^{\beta(y_0)} f(x, y) \dd x\Big| + \int_{\alpha(y_0)}^{\beta(y_0)} \Big|f(x, y) - f(x, y_0)\Big| \dd x \\
        &\left[\text{ --- раскроем первое  слагаемое; для понимания представьте, что $\alpha(y) < \alpha(y_0) < \beta(y_0) < \beta(y)$ }\right] \\
        &= \Big|\int_{\alpha(y)}^{\alpha(y_0)} f(x, y) \dd x + \int_{\beta(y_0)}^{\beta(y)} f(x, y) \dd x\Big| + \int_{\alpha(y_0)}^{\beta(y_0)} \Big|f(x, y) - f(x, y_0)\Big| \dd x \\
        &\leq \int_{\alpha(y)}^{\alpha(y_0)} \underbrace{|f(x, y)|}_{\leq M} \dd x + \int_{\beta(y_0)}^{\beta(y)} \underbrace{|f(x, y)|}_{\leq M} \dd x + \int_{\alpha(y_0)}^{\beta(y_0)} \underbrace{\Big|f(x, y) - f(x, y_0)\Big|}_{\leq \eps} \dd x \\
        &\leq (\alpha(y_0) - \alpha(y)) \cdot M + (\beta(y) - \beta(y_0)) \cdot M + (\beta(y_0) - \alpha(y_0)) \cdot \eps \\
        &= 2 \cdot \eps \cdot M + (\beta(y) - \alpha(y)) \cdot \eps = \eps',
    \end{align*}
    то есть выбирая $\delta > 0$ мы можем сделать так, что $|F(y) - F(y_0)| < \eps'$ для любого $\eps' > 0$.
    \endgroup
\end{proof}

Теперь немного о том, зачем нам эта теорема. Если вместо отрезка $[c; d]$ рассмотреть $[c; +\infty)$, то утверждение из теоремы остается верным и из равномерной непрерывности $f(x, y)$ на $[a; b] \times [c; +\infty)$ следует
\begin{equation*}
    \exists \lim_{y \to +\infty} F(y) = \lim_{y \to +\infty} \int_{\alpha(y)}^{\beta(y)} f(x, y) \dd x = \int_{\alpha(y)}^{\beta(y)} \lim_{y \to +\infty} f(x, y) \dd x.
\end{equation*}

\subsubsection{Теорема о дифференцировании по параметру под знаком интеграла.}

Для простоты изложения будем рассматривать $a = \alpha(y)$ и $b = \beta(y)$. Тогда
\begin{equation*}
    F(y) = \int_a^b f(x, y) \dd x.
\end{equation*}

\begin{theorem*}
    Если $f$ непрерывна на $G = [a; b] \times [c; d]$, а также производная $\dfrac{\partial f}{\partial y}$ существует и непрерывна на $G$, то $F$ непрерывно дифференцируема на $[c; d]$.

    Причем эта производная может быть вычислена:
    \begin{equation*}
        F'(y) = \int_a^b \dfrac{\partial f}{\partial y} (x, y) \dd x.
    \end{equation*}
\end{theorem*}

\begin{proof}
    Необходимо доказать, что отношение стремится в пределе к интегралу:
    \begin{align*}
        D = \dfrac{F(y) - F(y_0)}{y - y_0} - \int_a^b \dfrac{\partial f}{\partial y} (x, y_0) \dd x
        &= \int_a^b \dfrac{f(x, y) - f(x, y_0)}{y - y_0} \dd x - \int_a^b \dfrac{\partial f}{\partial y} (x, y_0) \dd x.
    \end{align*}

    По теореме о среднем (теорема Лагранжа, 1 курс) на отрезке $[y_0; y]$ найдется точка $y^*$ такая, что
    \begin{equation*}
        f(x, y) - f(x, y_0) = \dfrac{\partial f}{\partial y} (x, y^*) \cdot (y - y_0).
    \end{equation*}

    Подставим в нашу разность:
    \begin{align*}
        |D| = \dots
        &= \Big|\int_a^b \dfrac{f(x, y) - f(x, y_0)}{y - y_0} \dd x - \int_a^b \dfrac{\partial f}{\partial y} (x, y_0) \dd x \Big| 
        = \Big|\int_a^b \dfrac{\partial f}{\partial y}(x, y^*) \dd x - \int_a^b \dfrac{\partial f}{\partial y} (x, y_0) \dd x \Big|\\
        &= \Big|\int_a^b \left(\dfrac{\partial f}{\partial y}(x, y^*) - \dfrac{\partial f}{\partial y} (x, y_0) \right) \dd x \Big|
        \leq \int_a^b \underbrace{\Big|\dfrac{\partial f}{\partial y}(x, y^*) - \dfrac{\partial f}{\partial y} (x, y_0) \Big|}_{\leq \eps} \dd x
        \leq (b - a) \cdot \eps.
    \end{align*}

    Последний переход получается в силу равномерной непрерывности $\dfrac{\partial f}{\partial y}$ на $G$ и того, что $|y^* - y^*| \leq |y - y_0| < \eps$.

    То есть мы доказали, что $\dfrac{F(y) - F(y_0)}{y - y_0}$ равномерно стремится к числу $\int_a^b \dfrac{\partial f}{\partial y} (x, y_0) \dd x$, то есть существует предел, который мы и называем производной $F'(y)$.

    Непрерывность производной получается как следствие предыдущей теоремы (о непрерывности по параметру), где в роли непрерывной функции выступает $\dfrac{\partial f}{\partial y}$.
\end{proof}

Иногда мы не можем взять какой-то интеграл, но с помощью этой теоремы мы можем взять производную интеграла, а зная производную потом найти сам интеграл.

\subsubsection{Теорема об интегрировании по параметру под знаком интеграла.}

Пусть $F(y) = \int_a^b f(x, y) \dd x$. Мы хотим эту функцию проинтегрировать, то есть найти $\int_c^d F(y) \dd y$. Возникает вопрос, можно ли переставить интегралы.

\begin{theorem*}
    Если $f$ непрерывна на множестве $G = [a; b] \times [c; d]$ (то есть она интегрируема на $G$), и выполняются следующие два пункта:
    \begin{itemize}
    \item 
        при любом значении $y \in [c; d]$ функция $f(x, y)$ интегрируема по $x$, то есть существует $\int_a^b f(x, y) \dd x$;

    \item 
        при любом значении $x \in [a; b]$ функция $f(x, y)$ интегрируема по $y$, то есть существует $\int_c^d f(x, y) \dd y$;
    \end{itemize}

    то эти интегралы равны друг другу, то есть
    \begin{equation*}
        \int_a^b \int_c^d f(x, y) \dd x \dd y
        = \int_c^d \int_a^b f(x, y) \dd y \dd x.
    \end{equation*}
\end{theorem*}

\begin{proof}
    Доказательство следует из теоремы Фубини о том, что повторные интегралы равны двойном интегралу по прямоугольнику:
    \begin{equation*}
        \int_a^b \int_c^d f(x, y) \dd x \dd y
        = \iint_G f(x, y) \dd x \dd y
        = \int_c^d \int_a^b f(x, y) \dd y \dd x.
    \end{equation*}
\end{proof}

Интегрирование по параметру также иногда позволяет вычислить интеграл, который по-другому вычислить невозможно.