\subsection{Кусочно-гладкая кривая и её длина. Элемент длины для параметрически заданной кривой. Криволинейный интеграл I-го рода.}

\subsubsection{Кусочно-гладкая кривая и её длина.}

\begin{definition*}
    \textbf{Область} --- это открытое связное (грубо говоря, любые две точки множества могут быть соединены ломаной) множество в $\mathbb{R}^k$.
\end{definition*}

\begin{definition*}
    \textbf{Замкнутая область} --- это замыкание некоторой области.
\end{definition*}

\begin{definition*}
    \textbf{Жорданова область} --- это ограниченная область, измеримая по Жордану.
\end{definition*}

Пусть $G \subset \mathbb{R}^k$ (где $k \leq m$) --- замкнутая жорданова область, и $\phi: G \rightarrow \mathbb{R}^m$ --- непрерывно дифференцируемая \textit{инъективная} функция. Рассмотрим $x \in \RR^m$ и  $u \in G$ и определим $x$ следующим образом:
\begin{equation*}
    x := \phi(u) \iff x_i = \phi_i(u_1, ..., u_k),
\end{equation*}
причем матрица якоби
\begin{equation*}
    \dfrac{\partial x}{\partial u} 
    = 
    \begin{pmatrix}
        \frac{\partial x_1}{\partial u_1} & ... & \frac{\partial x_1}{\partial u_k} \\
        \vdots & ... & \vdots \\
        \frac{\partial x_m}{\partial u_1} & ... & \frac{\partial x_m}{\partial u_k} \\
    \end{pmatrix}
\end{equation*}
имеет максимальный ранг в каждой точке $u \in G$.

\begin{definition*}
    Образ $\phi(G)$ называется \textbf{гладкой кривой} при $k = 1$.
\end{definition*}

\begin{definition*}
    Если $L_i$ это гладкая кривая, то их объединение $L = \bigsqcup_{i=1}^{n} L_i$ называется \textbf{кусочно-гладкой кривой}.   
\end{definition*}

\subsubsection{Элемент длины для параметрически заданной кривой.}

Пусть даны отрезок $G = [a; b]$, непрерывно дифференцируемое отображение $\phi: G \to \RR^m$ и гладкая кривая $L$, которая задается через $G$ и $\phi$.

\begin{definition*}
    \textbf{Длина кривой} $L$ определяется следующим образом
    \begin{equation*}
        \mu(L) := 
        \underset{G}{\int} \sqrt{\det \left( \left( \dfrac{\partial x}{\partial u} \right)^T \cdot \left( \dfrac{\partial x}{\partial u} \right) \right)} \dd u
        = \int_a^b \left| \dfrac{\partial x}{\partial u} \right| \dd u.
    \end{equation*} 
\end{definition*}

\subsubsection{Криволинейный интеграл I-го рода.}

\href{https://youtu.be/h_VmHDiSPJI?list=PLEwK9wdS5g0qV-430pfXzTawd6pI_VUgq&t=3109}{Ссылка на лекцию.}

Пусть $L$ --- гладкая кривая к $\RR^m$, то есть $L = \phi([a; b])$, и рассмотрим некую функцию $f: L \to \RR$, которая задана на точках кривой.

\begin{definition*}
    Криволинейный интеграл первого рода определяется следующим образом:
    \begin{equation*}
        \int_L f(x) \dd l
        = \int_a^b f(\phi(u)) \cdot \left| \dfrac{\partial x}{\partial u} \right| \dd u.
    \end{equation*}

    $\dd l := \left| \dfrac{\partial x}{\partial u} \right| \dd u$ --- элемент длины. 
\end{definition*}
