\subsection{Вычет голоморфной функции в однозначной особой точке. Теорема Коши о вычетах. Вычет как коэффициент $c_{-1}$ ряда Лорана. Вычисления вычета в полюсе.}

\begin{definition*}
	Пусть функция $f(z)$ голоморфна в $0 < |z - z_0| < \delta$, тогда вычет функции $f$ в точке $z_0(res_{z_0}f)$ это величина, равная $\dfrac{1}{2\pi i} \oint_{|z - z_0| = \varepsilon} f(z)dz$, где $0 < \varepsilon < \delta$.
\end{definition*}

\begin{theorem*}
	Теорема Коши о вычетах.
	
	Пусть $f$ голоморфна в области $D$ всюду, за исключением конечного числа однозначных особых точек $z_1, \dots, z_n$, тогда
	
	$$\oint_{\partial D} f(z)dz = 2\pi i \sum_{k = 1}^{n} res_{z_k}f$$
\end{theorem*}

\begin{proof}
	Окружим каждую точку маленьким кругом, которые не пересекаются и не вылезают за предел множества. Каждая точка - $z_i$, а её круг - $U_i$.
	
	Рассмотрим множество $D' = D \setminus (U_1 \cup U_2 \cup \dots \cup U_n)$, тогда по теореме Коши: 
	
	$$\oint_{\partial D} f(z)dz = 0 \implies \oint_{\partial D} f(z)dz = \sum_{k=1}^{n} \oint_{\partial U_k} f(z)dz = 2\pi i \sum_{k = 1}^{n} res_{z_k}f$$ $$(\oint_{\partial U_k} f(z)dz = 2\pi i res_{z_k}f )$$
\end{proof}

\begin{theorem*}
	Вычет как коэффициент $c_{-1}$ ряда Лорана: $res_{z_0}f = c_{-1}$.
\end{theorem*}

\begin{proof}
	Пусть $f(z) = \sum_{k = -\inf}^{\inf} c_k(z-z_0)^k$ в некоторой проколотой окрестности $0 < |z - z_0| < \delta$. Так как этот ряд
	сходится, то мы можем его почленно проинтегрировать: $res_{z_0}f = \dfrac{1}{2\pi i} \oint_{\varepsilon} f(z) dz$. Возьмём замкнутое множество $\delta < |z - z_0| < r - \delta$, тогда на этом множестве ряд будет сходиться равномерно, а значит мы можем почленно применить этот интеграл к каждому слагаемому ряда: $res_{z_0}f = \dfrac{1}{2\pi i} \oint_{\varepsilon} f(z)dz = \dfrac{1}{2\pi i} \sum_{k = -\inf}^{\inf} c_k \oint_{|z - z_0| = \varepsilon} (z - z_0)^k dz = \dfrac{1}{2\pi i} \cdot c_{-1} \cdot 2 \pi i = c_{-1}$. Здесь мы заметили, что интеграл внутри суммы обращается в $2\pi i$ при $k+1=0$, и в $0$ в обратном случае.
\end{proof}

\begin{proposal}
	Пусть $z_0$ - полюс порядка $n$. $f(z) = \dfrac{c_{-n}}{(z - z_0)^n} + \dots + \dfrac{c_{-n}}{(z - z_0)} + c_0 + c_1(z - z_0) + \dots$. Домножим на $(z - z_0)^n$. Получим $f(z)(z-z_0)^n = c_{-n} + \dots + c_{-1}(z-z_0)^{n-1} + c_0(z-z_0)^n + \dots$. Сделав разложение по Тейлору получим $c_{-1} = \dfrac{1}{(n-1)!}(f(z)(z-z_0)^n)^{(n-1)} |_{z=z_0}$. Если $n=1$, то $c_{-1} = (f(z)(z - z_0))|_{z=z_0}$, на самом деле так как у $f$ есть неприятность в точке $z_0$, то как правило необходимо считать предел.
\end{proposal}

