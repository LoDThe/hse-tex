\subsection{Формула Стокса. Внешний дифференциал 3-мерной 1-формы, ротор векторного поля и краткая запись формулы Стокса.}

\subsubsection{Формула Стокса}
\begin{theorem*}
     Пусть $S$ -- ориентированная кусочно гладкая поверхность с краем $L = \partial S$, лежащая в области $D$.\\
     $\omega = Pdx + Qdy + Rdz$ -- непрерывно дифференцируема в D.\\ 
    Тогда\\
        \begin{center}
            $\oint_{\partial S} \omega = \iint_S d\omega$ -- краткая запись\\
        \end{center}
        $$\oint_L Pdx + Qdy + Rdz = 
        \iint_S (\frac{\partial R}{\partial y} - \frac{\partial Q}{\partial z})dy \wedge dz + (\frac{\partial P}{\partial z} - \frac{\partial R}{\partial x})dz \wedge dx + (\frac{\partial Q}{\partial x} - \frac{\partial P}{\partial y})dx \wedge dy$$
\end{theorem*}
\begin{proof}
    \begin{enumerate}
        \item{Пусть $S = \phi(G)$, где $G$ -- прямоугольник на плоскости параметров $(u_1, u_2)$.
            Тогда\\
                \begin{center}
                    $\oint_{\partial S}\omega = \{$ крив. инт. на пл-ти $\} = \iint_G d(\phi^{*}
                        \omega) = \{$Формула Грина$\} \iint_G \phi^{*}(d\omega) = \iint_S d\omega$
                \end{center}
            }
        \item{В общем случае поверхность разбивается на прямоугольники и интегралы по ним суммируются.}
    \end{enumerate}
\end{proof}

\begin{definition*}
    Внешний дифференциал 3-мерной 1-формы $\omega = Pdx + Qdy + Rdz $ -- это 
    $d\omega = (\frac{\partial R}{\partial y} - \frac{\partial Q}{\partial z})dy \wedge dz 
    + (\frac{\partial P}{\partial z} - \frac{\partial R}{\partial x})dz \wedge dx
    + (\frac{\partial Q}{\partial x} - \frac{\partial P}{\partial y})dx \wedge dy$ -- ротор
\end{definition*}

