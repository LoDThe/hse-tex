\subsection{Формула Грина и её приложение к вычислению площади клоской фигуры. Внешний дифференциал 2-мерный 1-формы и краткая запись формулы Грина.}
    \begin{definition*}
        Область $D \subset \mathbb{R}^2$ называется \textbf{$y$-проектором} (то есть проектируемой вдоль оси $y$), если она задается неравенствами:\\
        $$D:  a \leq x \leq b, h_1\left(x\right)\leq y \leq h_2\left(x\right),$$\\
        $h_1, h_2$ - непрерывные функции, $h_1\left(x\right) \leq h_2\left(x\right)$\\
        Аналогично вводится \textbf{$x$-проектор}.\\
    \end{definition*}
    \begin{definition*}
        Область называется \textbf{проектируемой}, если она является $x$- и $y$-проектируемой.\\
    \end{definition*}
    \begin{definition*}
        Область $D$ называется \textbf{простой}, если она есть объединение конечного числа проектируемых областей.\\
    \end{definition*}
\subsubsection{Формула Грина}
    \begin{theorem*}
        Пусть $D$ - простая область с кусочно гладкой границей $L = \partial D$, ориентация которой соответствует ориентации области $D$. Пусть $P, Q$ непрерывно дифференцируемы в $D$. Тогда:\\
        $$\oint_{\partial D} \left(P dx + Q dy\right) = \iint_D \left(\frac{\partial Q}{\partial x} - \frac{\partial P}{\partial y}\right)dx dy$$\\
    \end{theorem*}
    \begin{proof}
    \begin{enumerate}
        \item 
        Рассмотрим $y$-проектируемую область $D: a \leq x \leq b, h_1\left(x\right) \leq y \leq h_x\left(x\right)$\\
        $$-\iint_D \frac{\partial P}{\partial y} dxdy = 
        -\int_a^b dx \int_{h_1\left(x\right)}^{h_2\left(x\right)}\frac{\partial P}{\partial y}dy =
        -\int_a^b dx \left(P\left(x, h_2\left(x\right)\right) - P\left(x, h_1\left(x\right)\right)\right) = $$
        $$= -\int_a^bP\left(x, h_2\left(x\right)\right)dx + \int_a^b P\left(x, h_1\left(x\right)\right)dx + \int_a^a Pdx + \int_b^b Pdx = $$
        $$= \int_b^aP\left(x, h_2\left(x\right)\right)dx + \int_a^b P\left(x, h_1\left(x\right)\right)dx + \int_a^a Pdx + \int_b^b Pdx = 
        \oint_{\partial D} Pdx$$\\
        \item
        Аналогично для $x$-проектируемой области $D$ получим\\
        $$\iint_D\frac{\partial Q}{\partial x}dxdy = \oint_{\partial D} Q dy$$\\
        \item
        $D$ - $x$- и $y$-проектируемая область, то \\
        $$\oint_{\partial D}\left(Pdx + Qdy\right) = \iint_D \left(\frac{\partial Q}{\partial x} - \frac{\partial P}{\partial y}dxdy\right)$$
        \item (Нет четкой формулировки, записано со слов лектора)\\
        $D$ - простая область. В соседних областях по границе будем интегрировать в правильном направлении. Двойные интегралы (правые части) частей простой области будут складываться по аддитивности. Криволинейные интегралы (при разбиении в суммы) дадут интегралы по внешним границам и интегралы по внутренним границам. Интегралы по всем внутренним кусочкам границы будут взаимно уничтожаться (так как при обходе в противоположных направлениях будут давать разные знаки).
    \end{enumerate}
    
    \end{proof}
    
\subsubsection{Внешний дифференциал 2-мерной 1-формы}
    \begin{definition*}    
        $\omega\left(\overline{x}, d\overline{x}\right) = a_1\left(\overline{x}\right)dx_1 + ... + a_n\left(\overline{x}\right)dx_x$\\
        $a_1, ..., a_n$ - непрерывно дифференцируемы\\
        Внешним дифференциалом формы $\omega$ называется:\\
        $$d\omega = da_1 \wedge dx_1 + ... + da_n \wedge dx_n,$$
        где $da_1 = \frac{\partial a_1}{\partial x_1} dx_1 + ... + \frac{\partial a_1}{\partial x_n}dx_n$ - обычный дифференциал.\\
        Операция $\wedge$ линейна и кососимметрична:\\
        $$dx_1 \wedge dx_2 = - dx_2 \wedge dx_1 \Rightarrow dx_1 \wedge dx_1 = 0$$\\
    \end{definition*}
    
\begin{example}    
    $d\left(Pdx + Qdy\right) = dP \wedge dx + dQ \wedge dy = \left(\frac{\partial P}{\partial x}dx + \frac{\partial P}{\partial y}dy\right) \wedge dx + \left(\frac{\partial Q}{\partial x}dx + \frac{\partial Q}{\partial y}dy\right) \wedge dy = \left(\frac{\partial Q}{\partial x} - \frac{\partial P}{\partial y}\right)dx \wedge dy$\\
\end{example}
    
\subsubsection{Краткая запись формулы Грина}
    $$\iint_D d\left(x, y\right)dx \wedge dy = \iint_D f\left(x, y\right) dxdy$$
    $$\Rightarrow \oint_{\partial D} \omega = \iint_D d\omega, \omega = Pdx + Qdy$$