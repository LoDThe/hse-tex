\subsection{Кусочно-гладкая поверхность и её площадь. Элемент площади для параметрически заданной поверхности. Поверхностный интеграл I-го рода.}
\subsubsection{Кусочно-гладкая поверхность}

\begin{definition*}
    \textbf{Область} --- это открытое связное (грубо говоря, любые две точки множества могут быть соединены ломаной) множество в $\mathbb{R}^k$.
\end{definition*}

\begin{definition*}
    \textbf{Замкнутая область} --- это замыкание некоторой области.
\end{definition*}

\begin{definition*}
    \textbf{Жорданова область} --- это ограниченная область, измеримая по Жордану.
\end{definition*}

Пусть $G \subset \mathbb{R}^k$ (где $k < m$) --- замкнутая жорданова область, и $\phi: G \rightarrow \mathbb{R}^m$ --- непрерывно дифференцируемая \textit{инъективная} функция. Рассмотрим $x \in \RR^m$ и  $u \in G$ и определим $x$ следующим образом:
\begin{equation*}
    x := \phi(u) \iff x_i = \phi_i(u_1, ..., u_k),
\end{equation*}
причем матрица якоби
\begin{equation*}
    \dfrac{\partial x}{\partial u} 
    = 
    \begin{pmatrix}
        \frac{\partial x_1}{\partial u_1} & ... & \frac{\partial x_1}{\partial u_k} \\
        \vdots & ... & \vdots \\
        \frac{\partial x_m}{\partial u_1} & ... & \frac{\partial x_m}{\partial u_k} \\
    \end{pmatrix}
\end{equation*}
имеет максимальный ранг в каждой точке $u \in G$.

\begin{definition*}
    Образ $\phi(G)$ называется \textbf{гладкой $k$-мерной поверхностью} (при $k > 1$).
\end{definition*}

\begin{definition*}
    Если $S_i$ это гладкая поверхность, то их объединение $S = \bigsqcup_{i=1}^{n} S_i$ называется \textbf{кусочно-гладкой поверхностью}.   
\end{definition*}

\subsubsection{Элемент площади для параметрически заданной поверхности}

Пусть $G \subset \RR^2$ --- это замкнутая жорданова область, $\phi: G \to \RR^m$ --- параметризующее отображение.

\begin{definition*}
    \textbf{Площадь поверхности} $S$ определяется как
    \begin{equation*}
        \mu(S) := \iint_G \sqrt{
            \det 
            \begingroup
                \renewcommand*{\arraystretch}{2.5}
                \begin{pmatrix}
                    \left|\frac{\partial x}{\partial u}\right|^2 & \left\langle \frac{\partial x}{\partial u}, \frac{\partial x}{\partial v}\right\rangle \\
                    \left\langle \frac{\partial x}{\partial u}, \frac{\partial x}{\partial v}\right\rangle  & \left|\frac{\partial x}{\partial v}\right|^2\\
                \end{pmatrix}
            \endgroup
        } \dd u \dd v 
        = \iint_G \sqrt{
            \left|\frac{\partial x}{\partial u}\right|^2 \cdot \left|\frac{\partial x}{\partial v}\right|^2 - \left\langle \frac{\partial x}{\partial u}, \frac{\partial x}{\partial v}\right\rangle ^2
        } \dd u \dd v.
    \end{equation*}
\end{definition*}

\subsubsection{Поверхностный интеграл I-го рода}

\href{https://youtu.be/h_VmHDiSPJI?list=PLEwK9wdS5g0qV-430pfXzTawd6pI_VUgq&t=4348}{Ссылка на лекцию.}

Рассмотрим поверхность $S := \phi(G)$, заданную через замкнутую жорданову область $G \subset \RR^2$ и параметризующее отображение $\phi: G \to \RR^m$.

Задана некая функция $f: S \to \RR$, которая с одной стороны может принимать $x \in \RR^m$, а с другой стороны $u \in G$:
\begin{equation*}
    f(x) = f(x_1, x_2, x_3) = f(\phi_1(u, v), \phi_2(u, v), \phi_3(u, v)) = f(\phi(u, v)).
\end{equation*}

\begin{definition*}
    Элемент площади определяется как
    \begin{equation*}
        \dd s := \sqrt{
        \left|\frac{\partial x}{\partial u}\right|^2 \cdot \left|\frac{\partial x}{\partial v}\right|^2 - \left\langle \frac{\partial x}{\partial u}, \frac{\partial x}{\partial v}\right\rangle ^2
        } \dd u \dd v.
    \end{equation*}

    Поверхностный интеграл первого рода определяется следующим образом:
    \begin{equation*}
        \int_S f(x) \dd s 
        = 
        \iint_G f(\phi(u, v)) \cdot \sqrt{\left|\frac{\partial x}{\partial u}\right|^2 \cdot \left|\frac{\partial x}{\partial v}\right|^2 - \left\langle \frac{\partial x}{\partial u}, \frac{\partial x}{\partial v}\right\rangle ^2} \dd u \dd v.
    \end{equation*}
\end{definition*}
