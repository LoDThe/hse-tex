\subsection{Кусочно-гладкая поверхность и её площадь. Элемент площади для параметрически заданной поверхности. Поверхностный интеграл I-го рода.}
\subsubsection{Кусочно-гладкая поверхность}
    \begin{definition*}
        \textbf{Область} - открытое связное множество в $\mathbb{R}^k$\\
    \end{definition*}
    \begin{definition*}
        Замкнутая область - это замыкание некоторой области\\
    \end{definition*}
    \begin{definition*}
        Жорданова область - ограниченная область, измеримая по Жордану\\
    \end{definition*}
    \begin{definition*}
        Пусть $G \subset \mathbb{R}^k, k<m$ - замкнутая жорданова область и $\varphi: G \rightarrow \mathbb{R}^m$ - непрерывно дифференцируемая инъективная функция.\\
        $x= \varphi(u), x_i = \phi_i(u_1, ..., u_k)$\\
        $x \in \mathbb{R}^m, u \in G$, причем \\
        $$\frac{\partial x}{\partial u} = 
        \begin{pmatrix}
        \frac{\partial x_1}{\partial u_1} & ... & \frac{\partial x_1}{\partial }u_k \\
        \vdots & ... & \vdots \\
        \frac{\partial x_m}{\partial u_1} & ... & \frac{\partial x_m}{\partial u_k} \\
        \end{pmatrix}$$\\
        имеет в любом $u \in G$ максимальный ранг $k$.\\
        Тогда образ $\varphi(G)$ называется гладкой $k$-мерной поверхностью $S$ (при $k \geq 2$).\\
        $S = \bigsqcup_{i=1}^{n} S_i$ - кусочно гладкая поверхность\\
    \end{definition*}
\subsubsection{Элемент площади для параметрически заданной поверхности}
    \begin{theorem*}
    $G \subset \mathbb{R}^2$ - замкнутая жорданова область\\
    $\varphi: G \rightarrow \mathbb{R}^m$ - параметризующее отображение\\
    $(u, v) \in G$ - параметры поверхности\\
    $$\mu(S) = \iint_G \sqrt{
    \begin{vmatrix}
    \left|\frac{\partial x}{\partial u}\right|^2 & \left\langle \frac{\partial x}{\partial u}, \frac{\partial x}{\partial v}\right\rangle \\
    \left\langle \frac{\partial x}{\partial u}, \frac{\partial x}{\partial v}\right\rangle  & \left|\frac{\partial x}{\partial v}\right|^2\\
    \end{vmatrix}
    }\ dudv = \iint_G \sqrt{\left|\frac{\partial x}{\partial u}\right|^2 \cdot \left|\frac{\partial x}{\partial v}\right|^2 - \left\langle \frac{\partial x}{\partial u}, \frac{\partial x}{\partial v}\right\rangle ^2}\ dudv$$
    \end{theorem*}
\subsubsection{Поверхностный интеграл I-го рода}
    \begin{theorem*}
        Пусть:

        $G \subset \mathbb{R}^2$\\
        $\varphi: G \rightarrow \mathbb{R}^m$\\
        $S = \varphi(G)$\\
        $f: S \rightarrow \mathbb{R}$\\
        $f(x) = f(\varphi)$\\
        $f(x_1, x_2, x_3) = f(\varphi_1(u, v), \varphi_2(u, v), \varphi_3(u, v))$\\
        $ds = \sqrt{\left|\frac{\partial x}{\partial u}\right|^2 \cdot \left|\frac{\partial x}{\partial v}\right|^2 - \left\langle \frac{\partial x}{\partial u}, \frac{\partial x}{\partial v}\right\rangle ^2}\ dudv$, тогда:
        
        $$\int_S f(x) ds = \iint_G f(\phi(u, v)) \cdot \sqrt{\left|\frac{\partial x}{\partial u}\right|^2 \cdot \left|\frac{\partial x}{\partial v}\right|^2 - \left\langle \frac{\partial x}{\partial u}, \frac{\partial x}{\partial v}\right\rangle ^2}\ dudv$$        
    \end{theorem*}
