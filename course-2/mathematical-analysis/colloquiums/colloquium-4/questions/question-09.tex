\subsection{Дифференциал, дифференцируемость и производная комплексной функции. Условия Коши--Римана и голоморфность. Интеграл от голоморфной функции по кусочно-гладкой кривой. Теорема Коши. Интегральная формула Коши.}

\subsubsection{Дифференциал, дифференцируемость и производная комплексной функции.}

\begin{definition*}
	Функция $f: D \to \mathbb{R}^2, D \subseteq  \mathbb{R}^2$  называется \textit{дифференцируемой} (в общем случае) в точке $(x_0, y_0)$, если 
	$$
	\Delta f:= \begin{pmatrix}
	\Delta u \\ \Delta v
	\end{pmatrix} = P \cdot \begin{pmatrix}
	\Delta x \\ \Delta y
	\end{pmatrix}
	 + \overline{o}\sqrt{(\Delta x^2 + \Delta y^2)}
	 $$  
	 Тогда $P$ -- матрица Якоби, т.е. $P = \begin{pmatrix}
	 \frac{\partial u}{\partial x} & \frac{\partial u}{\partial y} \\
	 \frac{\partial v}{\partial x} & \frac{\partial v}{\partial y}
	 \end{pmatrix}\Biggl|_{(x_0, y_0)}$.
	
\end{definition*}

 В комплексном случае: $
\Delta f := \Delta u + i \Delta v = p \Delta z + q \Delta \overline{z} + \overline{o}(|\Delta z|) \implies $ 
\begin{align*}
\implies  p = \frac{\partial f}{\partial x} \cdot \frac 12 + \frac{\partial f}{\partial y} \cdot \frac{1}{2i} \left(= \frac{\partial f}{\partial z}\right) \qquad  
q = \frac{\partial f}{\partial x} \cdot \frac 12 + \frac{\partial f}{\partial y} \cdot \left( -\frac{1}{2i} \right) \left(= \frac{\partial f}{\partial \overline{z}}\right) 
\end{align*}

\begin{definition*}
	Тогда $p \Delta z + q \Delta \overline{z} $ -- это \textit{дифференциал} функции $f$.	 
\end{definition*}

\begin{definition*}
	$\mathbb{R}$--дифференцируемая функция называется \textit{$\mathbb{C}$--дифференцируемой}, если ее дифференциал $df =  \frac{\partial f}{\partial z}\, dz +  \frac{\partial f}{\partial \overline{z}}\, d\overline{z}$ является $\mathbb{C}$ -- линейным, т.е. $df =  \frac{\partial f}{\partial z}\, dz$. 
	
	Следовательно, получим следующее равенство:
	\begin{align*}
		\frac{\partial f}{\partial \overline{z}} =  \frac{\partial f}{\partial x} \cdot \frac 12 + \frac{\partial f}{\partial y} \cdot \left( -\frac{1}{2i} \right)  = 0 \iff 
		\begin{cases}
		\frac{\partial u}{\partial x} = \frac{\partial v}{\partial y}  \\
		\frac{\partial u}{\partial y} = -\frac{\partial v}{\partial x} 
		\end{cases} \text{-- условия Коши--Римана}
	\end{align*}
\end{definition*}

\begin{definition*}	
	Если существует предел
	\begin{align*}
	 \lim\limits_{z \to z_0} \frac{f(z) - f(z_0)}{z - z_0} = \lim\limits_{(x, y) \to( x_0, y_0)} \frac{(u(x,y) - u(x_0, y_0) + i(v(x,y) - v(x_0, y_0))}{(x - x_0) + i (y - y_0)},
	\end{align*}
	то он называется \textit{производной} $f$ по $z$ и обозначается $\dfrac{\partial f}{\partial z}$.
\end{definition*}

\begin{definition*} 
	(эквивалентное определение $\mathbb{C}$--дифференцируемости) 
	
	Пусть существует предел из определения выше (длинный такой). 
	
	Пусть $y = y_0$, тогда $\dfrac{\partial f}{\partial z} =\lim\limits_{x\to( x_0} \dfrac{(u(x,y) - u(x_0, y_0) + i(v(x,y) - v(x_0, y_0))}{x - x_0}  = \dfrac{\partial u}{\partial x} + i \dfrac{\partial v}{\partial x} $.
	
	Аналогично при $x = x_0$ получим, что $\dfrac{\partial f}{\partial z} = \frac 1i \left( \dfrac{\partial u}{\partial y} + i \dfrac{\partial v}{\partial y}\right)$. 
	
	Тогда получим, что $\dfrac{\partial u}{\partial x} + i \dfrac{\partial v}{\partial x} =\frac 1i \left( \dfrac{\partial u}{\partial y} + i \dfrac{\partial v}{\partial y}\right) \implies
	\begin{cases}
	\dfrac{\partial u}{\partial x} = \dfrac{\partial v}{\partial y} \\
	\dfrac{\partial u}{\partial y} =-\dfrac{\partial v}{\partial x} 
	\end{cases} \implies
	\dfrac{\partial f}{\partial \overline{z}}  = 0,
	$ 
	
	поэтому в этом случае $\dfrac{\partial f}{\partial z}$ логично обозначать как $\dfrac{df}{dz}$. Тогда\textit{ $\mathbb{C}$ -- дифференцируемость} равносильна существованию и конечности производной $\dfrac{df}{dz}$ (доказательство было в курсе МА--1).
\end{definition*}

\subsubsection{Условия Коши--Римана и голоморфность.}

\begin{definition*}
	Пусть $D \subseteq \RR^2$ -- область определения и $f: D \to \RR^2$ -- непрерывно дифференцируема.
	
	$f$ называется \textit{голоморфной} в $D$, если она удовлетворяет условиям Коши-Римана:
	
	\begin{equation}
	\begin{cases}
	\dfrac{\partial u}{\partial x} = \dfrac{\partial v}{\partial y} \\
	\dfrac{\partial v}{\partial x} = -\dfrac{\partial u}{\partial y}
	\end{cases}
	\end{equation}
\end{definition*}

\subsubsection{Интеграл от голоморфной функции по кусочно-гладкой кривой.}

\begin{definition*}
	Пусть $L$ -- кусочно--гладкая ориентированная (по умолчанию положительная ориентированность) кривая в области $D$, тогда \textit{интеграл по кривой на комплексной плоскости}  равен сумме двух криволинейных интегралов второго рода: 
	
	\begin{align*}
	\int_{L} f(z)\,dz = \int_{L} (u(x, y) + iv(x,y)) \cdot (dx + i\,dy) :=  \int_{L}  (u\,dx -v\,dy) + i  \int_{L} (v\,dx + u\,dy) 
	\end{align*}
\end{definition*}

\subsubsection{Теорема Коши.}
\begin{theorem*}
	Если функция $f$ голоморфна в замыкании $\overline{D}$ жордановой области $D$, то $\oint_{\partial D} f(z)\,dz = 0.$
\end{theorem*}
\begin{proof}
	Распишем интеграл по определению:
	\begin{align*}
		\oint_{\partial D} f(z)\,dz &= 	\oint_{\partial D} (u\,dx - v\,dy) + i 	\oint_{\partial D} (v\,dx + u\, dy) = \text{ по формуле Грина } = \\		
		&= \iint_{D} \left(-\frac{\partial u}{\partial y} - \frac{\partial v}{\partial x}\right)\, dx \wedge dy + i \iint_{D} \left(-\frac{\partial v}{\partial y} + \frac{\partial u}{\partial x}\right)\, dx \wedge dy  = 0
	\end{align*}

	Заметим, что так как функция голоморфна, то есть непрерывно дифференцируема на $D$ и выполнены условия Коши--Римана. Тогда в силу условий Коши--Римана оба подыинтегральных выражения равны нулю, и весь интеграл тоже равен нулю.
\end{proof}	

\subsubsection{Интегральная формула Коши.}
	Если функция $f$ голоморфна в замыкании $\overline{D}$ жордановой области $D$, то $f(z_0) = \dfrac{1}{2\pi i} \oint_{\partial D} \dfrac{f(z)}{z - z_0}\, dz$.