% Здесь НЕ НУЖНО делать begin document, включать какие-то пакеты..
% Все уже подрубается в головном файле
% Хедер обыкновенный хсе-теха, все его команды будут здесь работать
% Пожалуйста, проверяйте корректность теха перед пушем

% Здесь формулировка билета
% Timecode: lect 1/12/2020 1:03:29
\subsection{Дайте определение понятиям: гладкая $k$-мерная поверхность в $\mathbb{R}^{m}$, параметризация поверхности, координатные линии на поверхности}

\begin{definition}
    Пусть $G \subset \mathbb{R}^{k}$ --- замкнутое связное жорданово множество, $\varphi \colon G \to \mathbb{R}^{m}$ --- непрерывно дифференцируемое отображение, $\varphi(G) = S \subset \mathbb{R}^{m}$.
    Также, $\varphi \colon G \to S$ --- биекция, и соответствующая ей матрица Якоби
    \[
        \pdv{x}{u} = \pdv{(x_{1}, \ldots, x_{m})}{(u_{1}, \ldots, u_{k})} = \begin{pmatrix}
            \pdv{x_{1}}{u_{1}} & \pdv{x_{1}}{u_{2}} & \ldots & \pdv{x_{1}}{u_{k}} \\
            \vdots & & & \vdots \\
            \pdv{x_{m}}{u_{1}} & \pdv{x_{m}}{u_{2}} & \ldots & \pdv{x_{m}}{u_{k}}
        \end{pmatrix},
    \]
    где $\varphi(u) = x$, то есть координата $u_{i}$ принадлежит пространству $\RR^{k}$, а координата $x_{j}$ принадлежит пространству $\RR^{m}$.
    Обратим внимание, что матрица Якоби не квадратная, поскольку мы считаем $k < m$.
    Поскольку $\varphi$ непрерывно дифференцируема, то матрица Якоби составлена из непрерывных функций.
    Пусть $\rk\pqty{\pdv{x}{u}} = k$, тогда $S$ называется {\it гладкой $k$-мерной поверхностью в $\RR^{m}$}.
\end{definition}

\begin{definition}
    Функция $\varphi$ из прошлого определения называется {\it параметризацией} поверхности $S$.
\end{definition}

\begin{definition}
    Вектор $u = (u_{1}, \ldots, u_{k})$ называется {\it координатами на поверхности $S$}.
\end{definition}




