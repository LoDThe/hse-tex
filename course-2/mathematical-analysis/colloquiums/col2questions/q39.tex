% Здесь НЕ НУЖНО делать begin document, включать какие-то пакеты..
% Все уже подрубается в головном файле
% Хедер обыкновенный хсе-теха, все его команды будут здесь работать
% Пожалуйста, проверяйте корректность теха перед пушем

% Здесь формулировка билета
\subsection{Сформулируйте теорему (Фубини) о сведении интеграла по декартову произведению жордановых множеств к повторному интегралу.}
    \begin{theorem}[Фубини]
        $f : G \times H \to \mathbb{R} \; (G \subset \mathbb{R}^k, \, H \subset \mathbb{R}^l)$
        --- ограничена и интегрируема.
        
        Её интегральная сумма: $\sum\limits_{i, j} f(\xi_i, \eta_j) \cdot \mu(G_i \times H_j) \implies 
        \hspace{50 pt} (\xi_i \in G_i, \, \eta_j \in H_j) \\
        \underset{\Delta \to 0}{\implies} \int\limits_{G \times H} f(w)\, dw = \iint\limits_{G \times H}
        f(x, y)\, dxdy \hspace{50 pt} (w = (x, y),\, x \in G,\, y \in H)$
        
        Пусть $f(x, y)$ интегрируема по $x \; \forall y \in H$. Положим $g(y) = \int\limits_G f(x, y)\, dx$.
        
        Тогда $g$ интегрируема на $H$ и 
        \[ \iint\limits_{G \times H} f(x, y)\, dxdy = \int\limits_H g(y)\, dy
        = \int\limits_H dy \left( \int\limits_G f(x, y)\, dx \right) \]
    \end{theorem}
