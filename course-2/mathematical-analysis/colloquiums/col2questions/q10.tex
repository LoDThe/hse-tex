% Здесь НЕ НУЖНО делать begin document, включать какие-то пакеты..
% Все уже подрубается в головном файле
% Хедер обыкновенный хсе-теха, все его команды будут здесь работать
% Пожалуйста, проверяйте корректность теха перед пушем

% Здесь формулировка билета
\textcolor{red}{\subsection{Докажите, что множество измеримо по Жордану ровно тогда, когда его граница имеет Жорданову меру нуль}}

***МУТНАЯ ТЕМА, ЕСТЬ ВОПРОСЫ***\\
\textbf{\underline{Теор.:} } Пусть $A \subseteq \mathbb{R}^m$ - произвольное множество, тогда 
множество измеримо тогда и только тогда, когда $\overline{\mu}(\partial A) = 0$, где $\partial A$ - граница множества $A$. \\
\textbf{\underline{Док-во:} } \\
$\Rightarrow$ Пусть множество $A$ - измеримо по Жордану. Пусть $E_1 \subseteq A$ - простое множество, такое что $\mu(E_1) = \underline{\mu}(A)$, а также $E_2 \supseteq A$ - такое, что $\mu(E_2) = \overline{\mu}(A)$ \\
По определению границы знаем, что $\partial A \subseteq E_2 \backslash E_1$. Можно заметить, что так как $E_1 \subseteq A \subseteq E_2$, то $E_2 \backslash E_1 = (E_2 \backslash A) \cup (A \backslash E_1)$


