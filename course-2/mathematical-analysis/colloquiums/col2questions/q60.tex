% Здесь НЕ НУЖНО делать begin document, включать какие-то пакеты..
% Все уже подрубается в головном файле
% Хедер обыкновенный хсе-теха, все его команды будут здесь работать
% Пожалуйста, проверяйте корректность теха перед пушем

% Здесь формулировка билета
\subsection{Выведите выражение для площади бесконечно малого координатного параллелограмма в плоскости, касательной к поверхности в данной точке, через матрицу Якоби параметризации поверхности}

Найдем площадь $\mu(P)$ параллелограмма $P$, построенного в касательной плоскости к поверхности $S$ в точке $u^{0} = \pqty{u_{1}^{0}, \ldots, u_{k}^{0}}$, $x^{0} = \varphi\pqty{u^{0}}$ на векторах: $\pdv{x}{u_{1}} \cdot \dd u_{1}, \ldots, \pdv{x}{u_{k}} \cdot \dd u_{k}$.
Запись $\pdv{x}{u_{i}} \cdot \dd u_{i}$ означает, что мы берем касательный вектор соответствующей координатной линии $u_{i}$ в точке $u^{0}$, который мы умножили на $\dd u_{i}$, то есть берем его (вектор) длины $\norm{\pdv{x}{u_{i}}} \cdot \dd u_{i}$:
\[
    \mu(P)^{2} = \begin{vmatrix}
        \langle \pdv{x}{u_{1}} \cdot \dd u_{1}, \pdv{x}{u_{1}} \cdot \dd u_{1} \rangle & \ldots & \langle \pdv{x}{u_{1}} \cdot \dd u_{1}, \pdv{x}{u_{k}} \cdot \dd u_{k} \rangle \\
        \vdots & \ddots & \vdots \\
        \langle \pdv{x}{u_{k}} \cdot \dd u_{k}, \pdv{x}{u_{1}} \cdot \dd u_{1} \rangle & \ldots & \langle \pdv{x}{u_{k}} \cdot \dd u_{k}, \pdv{x}{u_{k}} \cdot \dd u_{k} \rangle
    \end{vmatrix} = \begin{vmatrix}
        \dd u_{1} \dd u_{1} \langle \pdv{x}{u_{1}}, \pdv{x}{u_{1}} \rangle & \ldots & \dd u_{1} \dd u_{k} \langle \pdv{x}{u_{1}}, \pdv{x}{u_{k}} \rangle \\
        \vdots & \ddots & \vdots \\
        \dd u_{k} \dd u_{1} \langle \pdv{x}{u_{k}}, \pdv{x}{u_{1}} \rangle & \ldots & \dd u_{k} \dd u_{k} \langle \pdv{x}{u_{k}}, \pdv{x}{u_{k}} \rangle
    \end{vmatrix},
\]
заметим, что $\dd u_{i}$ встречается в каждом элементе $i$-ой строки и в каждом элементе $i$-ого столбца, поэтому мы можем вынести $\pqty{\dd u_{i}}^{2}$ за определитель:
\[
    \mu(P)^{2} = \pqty{\dd u_{1} \dd u_{2} \ldots \dd u_{k}}^{2} \begin{vmatrix}
        \langle \pdv{x}{u_{1}}, \pdv{x}{u_{1}} \rangle & \ldots & \langle \pdv{x}{u_{1}}, \pdv{x}{u_{k}} \rangle \\
        \vdots & \ddots & \vdots \\
        \langle \pdv{x}{u_{k}}, \pdv{x}{u_{1}} \rangle & \ldots & \langle \pdv{x}{u_{k}}, \pdv{x}{u_{k}} \rangle
    \end{vmatrix} = \pqty{\dd u_{1} \dd u_{2} \ldots \dd u_{k}}^{2} \cdot \det G\pqty{\pdv{x}{u_{1}}, \pdv{x}{u_{2}}, \ldots, \pdv{x}{u_{k}}}.
\]
Заметим, что матрица $\pqty{\pdv{x}{u}} = \begin{pmatrix}
    \pdv{x}{u_{1}} & \pdv{x}{u_{2}} & \ldots & \pdv{x}{u_{k}}
\end{pmatrix}$ является матрицей Якоби системы векторов $\Bqty{\pdv{x}{u_{i}}}$, поэтому
\[
    \mu(P)^{2} = \pqty{\dd u_{1} \dd u_{2} \ldots \dd u_{k}}^{2} \cdot \det G\pqty{\pdv{x}{u_{1}}, \pdv{x}{u_{2}}, \ldots, \pdv{x}{u_{k}}} = \pqty{\dd u_{1} \dd u_{2} \ldots \dd u_{k}}^{2} \cdot \det \pqty{\pqty{\pdv{x}{u}}^{T} \times \pqty{\pdv{x}{u}}}.
\]
Отсюда,
\[
    \mu(P) = \dd u_{1} \dd u_{2} \ldots \dd u_{k} \cdot \sqrt{\det \pqty{\pqty{\pdv{x}{u}}^{T} \times \pqty{\pdv{x}{u}}}}.
\]

