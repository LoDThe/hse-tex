\subsection{Выведите формулу для площади гладкой поверхности вращения в $\RR^3$, заданной в цилиндрических координатах $(r, \varphi, z)$ уравнением $z = \rho(z)$, где $\rho$ -- непрерывно дифференцируемая функция.}

Поверхность $D$ называется поверхностью вращения, если она может быть задана в цилиндрических координатах уравнением
\[ r = \rho(z) \]

Параметризация поверхности вращения имеет вид
\[ x = \rho(z) \cos \varphi,\ y = \rho(z) \sin \varphi,\ z \in [a; b],\ \varphi \in [0; 2\pi) \]

Получим формулу для площади поверхности вращения. Вычислислим матрицу Якоби:
\[ \dfrac{\partial(x, y, z)}{\partial(z, \varphi)} =
\begin{pmatrix}
    \rho'(z) \cos \varphi & -\rho(z) \sin \varphi\\
    \rho'(z) \sin \varphi & \rho(z) \cos \varphi\\
    1                     & 0
\end{pmatrix}
\]

Найдём матрицу Грама:
\[ \left(\dfrac{\partial(x, y, z)}{\partial(z, \varphi)}\right)^T \cdot \left(\dfrac{\partial(x, y, z)}{\partial(z, \varphi)}\right) = 
\begin{pmatrix}
    (\rho'(z))^2 + 1 & 0\\
    0                 & \rho^2(z)
\end{pmatrix}
\]

и её определитель:
\[ ((\rho'(z))^2 + 1) \rho^2(z) \]

Получаем площадь поверхности вращения
\[ \mu(D) = \iint_{G} \sqrt{(\rho'(z))^2 + 1}\:\rho(z)\ dzd\varphi = \int_{a}^{b} d\varphi \int_{0}^{2\pi} \sqrt{(\rho'(z))^2 + 1}\:\rho(z)\ dz =
2\pi \int_{a}^{b} \sqrt{(\rho'(z))^2 + 1}\:\rho(z)\ dz\]