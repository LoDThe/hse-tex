% Здесь НЕ НУЖНО делать begin document, включать какие-то пакеты..
% Все уже подрубается в головном файле
% Хедер обыкновенный хсе-теха, все его команды будут здесь работать
% Пожалуйста, проверяйте корректность теха перед пушем

% Здесь формулировка билета
\subsection{Что такое матрица Грама системы $k$ векторов пространства $\RR^{m}$ и как она выражается через матрицу, составленную из координат векторов? Почему определитель Грама неотрицателен? Каков геометрический смысл определителя Грама?}

Пусть у нас зафиксировано пространство $\RR^{m}$, и у нас есть $k$-мерный параллелограмм $S$, построенный на векторах $l_{1}, \ldots, l_{k}$, тогда
\[
    \mu^{2}(S) = \begin{vmatrix}
        \langle l_{1}, l_{1} \rangle & \langle l_{1}, l_{2} \rangle & \ldots & \langle l_{1}, l_{k} \rangle \\
        \langle l_{2}, l_{1} \rangle & \langle l_{2}, l_{2} \rangle & \ldots & \langle l_{2}, l_{k} \rangle \\
        \vdots & & \ddots & \vdots \\
        \langle l_{k}, l_{1} \rangle & \ldots & \ldots & \langle l_{k}, l_{k} \rangle
    \end{vmatrix} = \det G(l_{1}, \ldots, l_{k}),
\]
где $G(l_{1}, \ldots, l_{k})$ --- матрица Грама системы векторов $\{l_{1}, \ldots, l_{k}\}$.
Пусть векторы $l_{1}, \ldots, l_{k}$ записаны в столбцы матрицы $L$, тогда
\[
    L = \begin{pmatrix}
        l_{1} & l_{2} & \ldots & l_{k}
    \end{pmatrix} \implies G = L^{T} \times L.
\]

\paragraph*{Неотрицательность определителя Грама}
Если система $\{l_{1}, \ldots, l_{k}\}$ линейно зависима, то определитель равен нулю.
Иначе, мы можем применить процесс ортогонализации $\{l_{1}, \ldots, l_{k}\} \leadsto \{w_{1}, \ldots, w_{k}\}$, где $\langle w_{i}, w_{j} \rangle = 0$, если $i \neq j$.
Но тогда $\det G(l_{1}, \ldots, l_{k}) = \det G(w_{1}, \ldots, w_{k}) = \norm{w_{1}}^{2} \cdot \norm{w_{2}}^{2} \cdot \ldots \cdot \norm{w_{k}}^{2} > 0$.

\paragraph{Геометрический смысл}
Определитель Грама системы векторов $\{l_{i}\}$ равен объему $k$-мерного параллелепипеда, порожденного системой векторов $\{ l_{i} \}$.