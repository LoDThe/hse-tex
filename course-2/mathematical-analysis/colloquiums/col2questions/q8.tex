% Здесь НЕ НУЖНО делать begin document, включать какие-то пакеты..
% Все уже подрубается в головном файле
% Хедер обыкновенный хсе-теха, все его команды будут здесь работать
% Пожалуйста, проверяйте корректность теха перед пушем

% Здесь формулировка билета
\subsection{Что такое измеримое по Жордану (= жорданово) множество? Как определяестя его мера? Приведите примеры измеримого и неизмеримого множества}

\textbf{\underline{Опр.:} } Ограниченное множество $A \subset \mathbb{R}^m$ называется \textit{измеримым по Жордану}, если 
\[\forall \varepsilon > 0 \ \ \ \exists E, E \supseteq A \ \ \ \ \overline{\mu}(E\backslash A) < \varepsilon \] \\
***ПРОВЕРИТЬ, НАДО ЛИ ЧТО-ТО ДОБАВИТЬ***\\
Заметим, что так как измеримые множества образуют кольцо, а также внешняя мера на кольце измеримых множеств обладает свойством аддитивности, то выполняются все свойства меры, а значит можно дать следующее определение \\
\textbf{\underline{Опр.:} } \textit{Мерой Жордана} измеримого множества называется его внешняя мера Жордана. \\
\textbf{\underline{Пример:} } Любое просто множество является измеримым по Жордану. \\
\textbf{\underline{Пример:} } Пусть $Q = \{q_1, q_2, ...\}$ - множество всех рациональных чисел отрезка $[0;1]$ и $A_n = [0;1] \backslash \{q_1, ..., q_n\}$. Множество $\bigcap\limits_{n\in\mathbb{N}} A_n = [0; 1] \backslash Q$ не является измеримым


