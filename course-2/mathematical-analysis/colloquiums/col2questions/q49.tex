% Здесь НЕ НУЖНО делать begin document, включать какие-то пакеты..
% Все уже подрубается в головном файле
% Хедер обыкновенный хсе-теха, все его команды будут здесь работать
% Пожалуйста, проверяйте корректность теха перед пушем

% Здесь формулировка билета
\subsection{Докажите, что композиция диффеоморфизма и меры Жордана является мерой. Чему равна её плотность?}

Расмотрим функцию множества: $\nu = \mu \circ \varphi$, то есть $\nu(A) = \mu(\varphi(A))$.

\begin{theorem*}
    Функция $\nu$ является мерой на кольце жордановых подмножеств множества $U$.
\end{theorem*}
\begin{proof}
    Просто проверим определение меры.
    \begin{enumerate}
        \item Понятно, что $\varphi(\varnothing) = \varnothing \implies \nu(\varnothing) = \mu(\varnothing) = 0$. (вроде этого нет в определениях меры, но Маевский зачем-то сказал на лекции)
        \item Понятно, что $\nu \geq 0$, так как $\mu \geq 0$.
        \item Проверим свойство аддитивности.
        Пусть $A_1, A_2 \subseteq U$~--- дизъюнктные жордановы множества. Тогда $B_1 = \varphi(A_1)$, $ B_2 = \varphi(A_2)$ дизъюнктны (так как $\varphi$ ~--- биекция) и жордановы (по прошлой теореме). Тогда:
        \begin{equation*}
            \nu(A_1 \sqcup A_2) = \mu(B_1 \sqcup B_2) = \mu(B_1) + \mu(B_2) = \nu(A_1) + \nu(A_2)
        \end{equation*}
    \end{enumerate}
    Значит, $\nu$~--- мера.

    
\end{proof}