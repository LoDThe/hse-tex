% Здесь НЕ НУЖНО делать begin document, включать какие-то пакеты..
% Все уже подрубается в головном файле
% Хедер обыкновенный хсе-теха, все его команды будут здесь работать
% Пожалуйста, проверяйте корректность теха перед пушем

% Здесь формулировка билета
\subsection{Докажите, что композиция диффеоморфизма и меры Жордана является мерой. Чему равна её плотность?}

Расмотрим функцию множества: $\nu = \mu \circ \varphi$, то есть $\nu(A) = \mu(\varphi(A))$.

\begin{theorem*}
    Функция $\nu$ является мерой на кольце жордановых подмножеств множества $U$.
\end{theorem*}
\begin{proof}
    Просто проверим определение меры.
    \begin{enumerate}
        \item Понятно, что $\varphi(\varnothing) = \varnothing \implies \nu(\varnothing) = \mu(\varnothing) = 0$. (вроде этого нет в определениях меры, но Маевский зачем-то сказал на лекции)
        \item Понятно, что $\nu \geq 0$, так как $\mu \geq 0$.
        \item Проверим свойство аддитивности.
        Пусть $A_1, A_2 \subseteq U$~--- дизъюнктные жордановы множества. Тогда $B_1 = \varphi(A_1)$, $ B_2 = \varphi(A_2)$ дизъюнктны (так как $\varphi$ ~--- биекция) и жордановы (по прошлой теореме). Тогда:
        \begin{equation*}
            \nu(A_1 \sqcup A_2) = \mu(B_1 \sqcup B_2) = \mu(B_1) + \mu(B_2) = \nu(A_1) + \nu(A_2)
        \end{equation*}
    \end{enumerate}
    Значит, $\nu$~--- мера.
\end{proof}
    
Пусть $U, X \subset \RR^m$, $\varphi:U \to X$~--- диффеоморфизм. Пусть $A$~--- жорданово, $A \subseteq U$. Рассмотрим меру $\nu(A) = \mu(\varphi(A))$. 

\begin{proposition*}
    $\nu$ имеет плотность, равную $\lvert J_\varphi \rvert$.
\end{proposition*}

\begin{proof}
    Воспользуемся тем, что $\varphi$ непрерывно дифференцируемо, то есть имеет непрерывные частные производные и напишем разложение по формуле Тейлора (первого порядка) в окрестности любой точки $u^0$:
    \begin{equation*}
        x-x^0 = \frac{\partial x}{\partial u}\Bigg|_{u^0} \cdot (u - u^0) + o(\lvert du \rvert), \hspace{2em} du = u-u^0,\ x^0 = \varphi(u^0),\ x = \varphi(u)
    \end{equation*}

    Посмотрим, как действует линейное отображение (предполагаем, что $o(|du|)$ мало) на простые множества~--- объединения дизъюнктных параллелепипедов. Пусть $P$~--- прямоугольный параллелепипед $\subset U$ и $P = p_1 \times p_2 \times \dots \times p_m$.

    Под действием преобразования $U \mapsto \frac{\partial x}{\partial u} \Bigg|_{u^0} \cdot (u - u^0)$ $P$ перейдёт в (необязательно прямоугольный) параллелепипед объёма (пользуемся геометрическим смыслом определителя):
    \begin{equation*}
        \left\lvert \det \frac{\partial x}{ \partial u} \Bigg|_{u^0} \right\rvert \cdot p_1 \cdot \dots \cdot p_m = \left\lvert J_\varphi(u^0) \right\rvert \cdot \mu(P)
    \end{equation*}

    Теперь разберёмся с $o(|du|)$. По определению:
    \begin{equation*}
        \forall \varepsilon > 0 \exists \delta > 0 : \lvert o(\lvert du \rvert) \leq \varepsilon \cdot \lvert du \rvert \text{ если только } \lvert u - u^0 \rvert < \delta
    \end{equation*}
    Если $P$ находится в $\delta$-окрестности точки $u^0$, то
    \begin{equation*}
        (1 - \varepsilon)^m \cdot \lvert J_\varphi(u^0) \rvert \cdot \mu(P) \leq \nu(P) \leq (1 + \varepsilon)^m \cdot \lvert J_\varphi(u^0) \rvert \cdot \mu(P)
    \end{equation*}
    Простое множество~--- это $\sqcup P_i = F \subset \delta$-окрестности точки $u^0$.
    \begin{align*}
        &(1 - \varepsilon)^m \cdot \lvert J_\varphi(u^0) \rvert \cdot \mu(P_i) \leq \nu(P_i) \leq (1 + \varepsilon)^m \cdot \lvert J_\varphi(u^0) \rvert \cdot \mu(P_i) \hspace{2em} \Bigg| \sum_i\\[1em]
        &(1 - \varepsilon)^m \cdot \lvert J_\varphi(u^0) \rvert \cdot \mu(F) \leq \nu(F) \leq (1 + \varepsilon)^m \cdot \lvert J_\varphi(u^0) \rvert \cdot \mu(F)
    \end{align*}
    Измеримое множество $G \subset \delta$-окрестности точки $u^0$ сколь угодно точно приближается простыми, то есть
    \begin{align*}
        &\exists F: (1 - \varepsilon) \cdot \mu(F) \leq \mu(G) \leq (1 + \varepsilon) \cdot \mu(F) \implies\\
        &\implies (1 - \varepsilon)^{m + 1} \cdot \lvert J_\varphi(u^0) \rvert \cdot \mu(G) \leq \nu(G) \leq (1 + \varepsilon)^{m + 1} \cdot \lvert J_\varphi(u^0) \rvert \cdot \mu(G)
    \end{align*}
    Поэтому, если измеримое множество $A$ стягивается к точке $u^0$, то 
    \begin{equation*}
        \frac{\nu(A)}{\mu(A)} \to \lvert J_\varphi(u^0) \rvert
    \end{equation*}
\end{proof}