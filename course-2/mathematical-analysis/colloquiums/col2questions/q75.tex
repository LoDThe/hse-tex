% Здесь НЕ НУЖНО делать begin document, включать какие-то пакеты..
% Все уже подрубается в головном файле
% Хедер обыкновенный хсе-теха, все его команды будут здесь работать
% Пожалуйста, проверяйте корректность теха перед пушем

% Здесь формулировка билета
\subsection{При каких значениях параметра $p > 0$ сходятся несобственные интегралы: \\ $\iiint_{x^2 + y^2 + z^2 \leqslant 1}\frac{dx~dy~dz}{(x^2+y^2+z^2)^p}, ~ \iiint_{x^2+y^2+z^2 \geqslant 1} \frac{dx~dy~dz}{(x^2+y^2+z^2)^p}$}
\begin{enumerate}
    \item $\iiint_{x^2 + y^2 + z^2 \leqslant 1}\frac{dx~dy~dz}{(x^2+y^2+z^2)^p} = 
    \begin{bmatrix}
        x &=& r\cos \varphi \cos \psi \\
        y &=& r \sin \varphi \cos \psi \\
        z &=& r \sin \psi
    \end{bmatrix} = 
    \int_0^{2\pi} d\varphi \int_{-\frac{\pi}{2}}^{\frac{\pi}{2}}\cos \psi d\psi
    \int_0^1 \frac{dr}{r^{2p-2}}$

    Очевидно, что $\int_0^1\frac{dr}{r^{2p-2}}$ сходится при $p < \frac{3}{2}$ и расходится при $p \geqslant \frac{3}{2}$.

    \item $\iiint_{x^2+y^2+z^2 \geqslant 1} \frac{dx~dy~dz}{(x^2+y^2+z^2)^p} =
    [\text{та же замена, что и в предыдущем}] =
    \int_0^{2\pi} d\varphi \int_{-\frac{\pi}{2}}^{\frac{\pi}{2}}\cos \psi d\psi \int_1^{+\infty}\frac{dr}{r^{2p-2}} =
    4\pi \int_1^{+\infty}\frac{dr}{r^{2p - 2}}$

    Эта штука сходится при $p > \frac{3}{2}$ и расходится при $p \leqslant \frac{3}{2}$.
\end{enumerate}
