\subsection{Как определяется плотность заряда в точке? Приведите пример заряда, не имеющего плотности.}

\begin{definition*}
    замыканием $\overline{A}$ множества $A$ называется пересечение всех замкнутых множеств, содержащих $A$.
\end{definition*}

Удобно представлять, что $\overline{A} = A \sqcup (\text{ граница } A)$.

\begin{definition*}
    Число $\rho$ называется плотностью заряда $\nu$ в точке $a$, если для любого $\eps > 0$ существует $\delta > 0$, что из выполнения следующих условий:
    \begin{enumerate}[label=\arabic*)]
    \item 
        жорданово множество $A \subseteq D$;
        
    \item 
        точка $a$ лежит в $\overline{A}$;

    \item 
        $\mu(A) > 0$;

    \item 
        диаметр $\operatorname{diam} A = \sup\limits_{x, y \in A} |x - y| < \delta$;
    \end{enumerate}
    следует $\left| \dfrac{\nu(A)}{\mu(A)} - \rho \right| < \eps$. То есть мы устремляем диаметр множества $A$ к нулю таким образом, чтобы точка $a$ лежала в замыкании этого множества.

    Эквивалентная запись через математические символы:
    \begin{equation*}
        \forall \eps > 0 \ \exists \delta > 0: \left[
            \text{жорданово } A \subseteq D, 
            a \in \overline{A},
            \mu(A) > 0,
            \operatorname{diam} A = \sup\limits_{x, y \in A} |x - y| < \delta)
        \right] \implies \left| \dfrac{\nu(A)}{\mu(A)} - \rho \right| < \eps.
    \end{equation*}
\end{definition*}

Можно рассмотреть аналогию с плотностью в физике. Если мы рассматриваем множество малого объема, окружающего точку $a$, то мы должны взять соответствующую массу этого множества и разделить на величину объема этого множества. И если эта дробь при бесконечном измельчении множества (диаметр стремится к $0$) стремится к некоторому число, то это число $\rho$ называется плотностью в этой точке.

\begin{theorem*}
    Если функция $f(x)$ непрерывна на жордановом множестве $D$ и $\nu(A) = \int_A f(x) dx$, то плотность $\nu$ в точке $a$ есть $f(a)$.
\end{theorem*}

\begin{example}
    Возьмем конечный набор точек множества $D$ и каждой из них сопоставим некоторое число --- заряд точки.

    Каждому подмножеству $A$ множества $D$ поставим в соответствие заряд, равный сумме зарядов тех выделенных точек, которые попали в множество $A$. Если не попало ни одной выделенной точки, тогда заряд равен $0$. 
    
    Легко видеть, что таким образом определенный заряд в самом деле является зарядом (удовлетворяет определению). С другой стороны, он не имеет плотности ни в одной из выделенных точек, следовательно, он не имеет плотности на множестве $D$.
\end{example}

