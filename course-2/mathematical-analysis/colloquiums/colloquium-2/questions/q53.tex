% Здесь НЕ НУЖНО делать begin document, включать какие-то пакеты..
% Все уже подрубается в головном файле
% Хедер обыкновенный хсе-теха, все его команды будут здесь работать
% Пожалуйста, проверяйте корректность теха перед пушем

% Здесь формулировка билета
\subsection{Что такое внешнее произведение векторов? Как определяется линейная дифференциальная форма?}

\textbf{Внешнее произведение векторов}~--- это некая абстрактная (то есть, ``пощупать'' её мы не можем) линейная кососимметричная операция $(\wedge)$.
\begin{itemize}
    \item Линейность:
    \begin{equation*}
        (\alpha\vec{a} + \beta\vec{b}) \wedge \vec{c} = \alpha \cdot \vec{a} \wedge \vec{c} + \beta \cdot \vec{b} \wedge \vec{c}
    \end{equation*}
    \item Кососимметричность:
    \begin{equation*}
        \vec{a} \wedge \vec{b} = -\vec{b} \wedge \vec{a} (\implies \vec{a} \wedge \vec{a} = 0)
    \end{equation*}       
\end{itemize}

\textbf{Линейная дифференциальная форма} от переменных $u, v$~--- формальная линейная комбинация их дифференциалов.

Например,
\begin{equation*}
    dx = d\varphi(u, v) = \varphi'_udu + \varphi'_vdv
\end{equation*}
~--- линейная дифференциальная форма от $du$, $dv$. 

Можем заметить, что дифференциал какой-то функции является линейной дифференциальной формой, но обратное неверно.

\begin{example}
    Перемножим две линейные дифференциальные формы и посмотрим, что получится.
    \begin{equation*}
        (\alpha du + \beta dv) \wedge (\gamma du + \delta dv) = \alpha \gamma \underbrace{du \wedge du}_0 + \alpha \delta du \wedge dv + \beta \gamma dv \wedge du + \beta \delta \underbrace{dv \wedge dv}_0 = (\alpha \delta - \beta \gamma) \cdot du \wedge dv = \begin{vmatrix}
            \alpha & \beta\\
            \gamma & \delta
        \end{vmatrix} \cdot du \wedge dv
    \end{equation*}
\end{example}