% Здесь НЕ НУЖНО делать begin document, включать какие-то пакеты..
% Все уже подрубается в головном файле
% Хедер обыкновенный хсе-теха, все его команды будут здесь работать
% Пожалуйста, проверяйте корректность теха перед пушем

% Здесь формулировка билета
\subsection{Что понимается под элементом объёма? Как следует понимать произведение дифференциалов в элементе объёма?}

Поскольку 
\begin{equation*}
    \int_D dx = \mu(D),
\end{equation*}
то дифференциал $dx = dx_1 dx_2 \dots dx_m$ называют \textbf{элементом объёма}.

Обсудим вопрос, в каком смысле следует понимать произведение дифференциалов $dx_1 dx_2 \dots dx_m$ в элементе объёма под знаком кратного интеграла 
\begin{equation*}
    \int_D f(x) dx = \idotsint_D f(x_1, x_2,\dots, x_m) dx_1 dx_2 \dots dx_m
\end{equation*}

Рассмотрим двумерный случай, для $m$-мерного аналогично.

Пусть $G, D \subset \RR^2$ и у нас есть интеграл $\iint_D f(x, y) dx dy$.

Мы хотим сделать замену координат, отображающую некое $G \subset \RR^2$ в наше $D$: 
\begin{equation*}
    \begin{dcases}
        x = \varphi(u, v)\\
        y = \psi(u, v)
    \end{dcases}
\end{equation*}
Тогда знаем, что
\begin{align*}
    dx &= \varphi'_u du + \varphi'_v dv,\\
    dy &= \psi'_u du + \psi'_v dv
\end{align*}
Если мы просто перемножим эти две формулы, то не получим правильной замены переменных. Как быть? Ответ: использовать внешнее произведение дифференциалов. (здесь рекомендую прочитать 53-й вопрос, после него будет проще)

Перемножим дифференциалы $dx$ и $dy$ внешним образом:
\begin{equation*}
    dx \wedge dy = (\varphi'_u du + \varphi'_v dv) \wedge (\psi'_u du + \psi'_v dv) = \begin{vmatrix}
        \varphi'_u & \varphi'_v \\
        \psi'_u & \psi'_v
    \end{vmatrix} \cdot du \wedge dv = J \cdot du \wedge dv
\end{equation*}
Почему нет модуля? Это из-за того, что мы пока рассматриваем ориентированный интеграл (ориентированное пространство), и у нас $dxdy = -dydx$. Но обычно мы рассматриваем неориентированный интеграл, и тогда 
\begin{equation*}
    dxdy = |dx \wedge dy| = |J| \cdot |du \wedge dv|.
\end{equation*}
Для $m$-мерного случая:
\begin{equation*}
    dx_1 \wedge \dots \wedge dx_m = J_\varphi(u) \cdot du_1 \wedge \dots \wedge du_m.
\end{equation*}
