\subsection{Дайте определение сферических координат (формулы, область задания, координатные линии, матрица Якоби перехода, якобиан)}

Сферические координаты $(r, \theta, \varphi)$ в пространстве $(x, y, z)$ вводятся формулами
\[ x = r \sin \theta \cos \varphi \]
\[ y = r \sin \theta \sin \varphi \]
\[ z = r \cos \theta \]
При этом $U = (0; +\infty) \times (0; \pi) \times [0; 2\pi)$ и $X = \RR^3 \setminus \{(0, 0, z) | z \in \RR \}$
Выколотая ось $z$ при этом называется полярной осью. Угол $\theta$ называется полярным углом, а угол $\varphi$ называется азимутальным углом.

Координатные линии $r$ -- лучи, выходящие из начала координат. Координатные линии $\theta$ -- полуокружности с центром в начале координат,
и концами, расположенными на полярной оси. Координатные линии $\varphi$ -- окружности с центром на полярной оси, расположенные в плоскостях, 
перпендикулярных полярной оси.

Матрица Якоби перехода имеет вид:

$
\begin{pmatrix}
    \sin \theta \cos \varphi & r \cos \theta \cos \varphi & -r \sin \theta \sin \varphi\\
    \sin \theta \sin \varphi & r \cos \theta \sin \varphi & r \sin \theta \cos \varphi \\
    \cos \theta              & -r \sin \theta             & 0
\end{pmatrix}
$

Якобиан равен $r^2 \sin \theta$.