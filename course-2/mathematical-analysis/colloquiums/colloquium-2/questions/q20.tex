% Здесь НЕ НУЖНО делать begin document, включать какие-то пакеты..
% Все уже подрубается в головном файле
% Хедер обыкновенный хсе-теха, все его команды будут здесь работать
% Пожалуйста, проверяйте корректность теха перед пушем

% Здесь формулировка билета
\subsection{Сформулируйте и докажите основные свойства сумм Дарбу}

***ПРОВЕРИТЬ ВСЕ ЛИ НУЖНЫЕ СВОЙСТВА ТУТ***\\
\textbf{\underline{Св-во:} } При измельчении разбиения $\tau \leq \tau'$ нижняя сумма Дарбу не уменьшается $s_D(f, \tau) \geq s_D(f, \tau')$ \\
\textbf{\underline{Док-во:} } Рассмотрим $D_j' = D_{j1}\sqcup ... \sqcup D_{jk}$. Тогда $\forall i, \ m_j' \leq m_{ji}$ и в силу аддитивности меры $\mu(D_j') = \mu(D_{j1}) + ... + \mu(D_{jk}) $ \\
Из этого следует, что 
\[m_j'\mu(D_j') \leq m_{j1}\mu(D_{j1}) + ... + m_{jk}\mu(D_{jk})\]
Данное неравенство верно при всех $j$, из чего как и раз и следует искомое. \begin{flushright}
$\blacksquare$
\end{flushright}
\textbf{\underline{Св-во:} } При измельчении разбиения $\tau \leq \tau'$ верхняя сумма Дарбу не увеличивается $S_D(f, \tau) \leq S_D(f, \tau')$ \\
\textbf{\underline{Док-во:} } Аналогично предыдущему пункту.
\begin{flushright}
$\blacksquare$
\end{flushright}
\textbf{\underline{Св-во:} } Для любых разбиений $\tau$ и $\tau'$ выполняется $s_D(f, \tau) \leq S_D(f, \tau')$\\
\textbf{\underline{Док-во:} } Рассмотрим измельчение $\tau'' = \tau\cdot\tau'$ \\
Из двух предыдущих пунктов имеем 
\[s_D(f, \tau) \leq s_D(f, \tau'')\]
\[S_D(f, \tau') \geq S_D(f, \tau'')\]
так как $s_D \leq S_D$ при каком либо фиксированном разбиении, а также из этих двух неравенств имеем
\[s_D(f, \tau) \leq s_D(f, \tau'') \leq S_D(f, \tau'') \leq S_D(f, \tau')\]
\begin{flushright}
$\blacksquare$
\end{flushright}

