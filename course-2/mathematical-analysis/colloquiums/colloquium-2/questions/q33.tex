\subsection{Как вводится понятие заряда на кольце множеств? Покажите, что для заряда справедлива формула включения-исключения.}

\begin{definition*}
    Функция $\nu$, определенная на некотором кольце множеств, называется зарядом, если
    \begin{enumerate}[label=\alph*)]
    \item 
        $\nu(\varnothing) = 0$;

    \item 
        $\nu(A \sqcup B) = \nu(A) + \nu(B)$ (аддитивность).
    \end{enumerate}
\end{definition*}

Таким образом, мера --- это неотрицательный заряд.

\begin{theorem*}
    Для заряда справедлива формула включений-исключений:
    \begin{equation*}
        \nu(A \cup B) = \nu(A) + \nu(B) - \nu(A \cap B).
    \end{equation*}
\end{theorem*}

\begin{proof}
    Заметим, что $A \cup B = A \sqcup (B \setminus A)$ и $B = (B \setminus A) \sqcup (A \cap B)$.

    \begin{itemize}
    \item 
        С одной стороны имеем
        \begin{equation*}
            \nu(A \cup B) = \nu(A \sqcup (B \setminus A)) = \nu(A) + \nu(B \setminus A).
        \end{equation*}

    \item 
        С другой стороны имеем
        \begin{equation*}
            \nu(A) + \nu(B) - \nu(A \cap B) 
            = \nu(A) + \nu((B \setminus A) \sqcup (A \cap B)) - \nu(A \cap B)
            = \nu(A) + \nu(B \setminus A) + \nu(A \cap B) - \nu(A \cap B)
            = \nu(A) + \nu(B \setminus A).
        \end{equation*}
    \end{itemize}

    То есть оба выражения равны $\nu(A) + \nu(B \setminus A)$, из чего делаем вывод:
    \begin{equation*}
        \nu(A \cup B) = \nu(A) + \nu(B \setminus A) = \nu(A) + \nu(B) - \nu(A \cap B).
    \end{equation*}
\end{proof}