\subsection{Выведите формулу, выражающую внешнее произведение дифференциалов зависимых переменных через внешнее произведение дифференциалов независимых переменных.}

Мы хотим сделать замену координат, отображающую некое $G \subset \RR^2$ в наше $D$: 
\begin{equation*}
    \begin{dcases}
        x = \varphi(u, v)\\
        y = \psi(u, v)
    \end{dcases}
\end{equation*}
Тогда знаем, что
\begin{align*}
    dx &= \varphi'_u du + \varphi'_v dv,\\
    dy &= \psi'_u du + \psi'_v dv
\end{align*}

Если мы просто перемножим эти две формулы, то не получим правильной замены переменных. Как быть? Ответ: использовать внешнее произведение дифференциалов. (здесь рекомендую прочитать 53-й вопрос, после него будет проще)

Перемножим дифференциалы $dx$ и $dy$ внешним образом:
\begin{equation*}
    dx \wedge dy = (\varphi'_u du + \varphi'_v dv) \wedge (\psi'_u du + \psi'_v dv) = \begin{vmatrix}
        \varphi'_u & \varphi'_v \\
        \psi'_u & \psi'_v
    \end{vmatrix} \cdot du \wedge dv = J \cdot du \wedge dv
\end{equation*}

Почему нет модуля? Это из-за того, что мы пока рассматриваем ориентированный интеграл (ориентированное пространство), и у нас $dxdy = -dydx$. Но обычно мы рассматриваем неориентированный интеграл, и тогда 
\begin{equation*}
    dxdy = |dx \wedge dy| = |J| \cdot |du \wedge dv|.
\end{equation*}

Для $m$-мерного случая:
\begin{equation*}
    dx_1 \wedge \dots \wedge dx_m = J_\varphi(u) \cdot du_1 \wedge \dots \wedge du_m.
\end{equation*}
