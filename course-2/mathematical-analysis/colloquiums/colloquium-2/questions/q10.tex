% Здесь НЕ НУЖНО делать begin document, включать какие-то пакеты..
% Все уже подрубается в головном файле
% Хедер обыкновенный хсе-теха, все его команды будут здесь работать
% Пожалуйста, проверяйте корректность теха перед пушем

% Здесь формулировка билета
\subsection{Докажите, что множество измеримо по Жордану ровно тогда, когда его граница имеет Жорданову меру нуль}
\begin{theorem}
    Ограниченное множество $A$ измеримо тогда и только тогда, когда мера Жордана его границы равна нулю:
    $\mu(\partial A) = 0$.
\end{theorem}
\begin{proof}
    
    Необходимость.

    В силу другого определения измеримости по Жордану для любого положительного $\varepsilon$ найдутся простые
    множества $E_1$ и $E_2$ такие, что $E_1 \subset A \subset E_2$ и $|E_2 \backslash E_1|$. При этом поскольку объём
    внутренности $\text{int}E$ совпадает с замыканием $\overline{E}$ простого множества, поэтому можем считать, что
    $E_1$ открыто, а $E_2$ --- замкнуто. Но тогда $\partial A = \overline A \backslash \text{int}A \subset E_2 \backslash E_1$,
    откуда $\overline{\mu}(\partial A) < \varepsilon$. Отсюда в силу произвольности $\varepsilon > 0$ получаем
    $\overline{\mu}(\partial A) = 0$

    Достаточность.

    $\overline{\mu}(A) \leqslant \overline{\mu}(\overline(A)) = \mu(\overline(A)) = \mu(\text{int}A) + \mu(\partial A) =
    \mu(\text{int} A)$. Учитывая, что внутренность --- открытое множество, отсюда имеем $\underline{\mu}(A) \geqslant
    \underline{\mu}(\text{int} A) = \mu(\text{int} A) \geqslant \overline \mu(A)$.
\end{proof}


