% Здесь НЕ НУЖНО делать begin document, включать какие-то пакеты..
% Все уже подрубается в головном файле
% Хедер обыкновенный хсе-теха, все его команды будут здесь работать
% Пожалуйста, проверяйте корректность теха перед пушем

% Здесь формулировка билета
\subsection{Дайте определения понятиям: несобственный интеграл от функции f по множеству D; сходящийся несобственный интеграл; расходящийся несобственный интеграл; функция, интегрируемая на D в несобственном смысле.}
Пусть $f: D \rightarrow \mathbb{R}$. Исчерпание $\{D_n\}$ множества $D$ называем допустимым для функции $f$, если $\forall n$ $f$ ограничена и интегрируема на $D_n$. Рассмотрим последовательность $\int_{D_n} f(x) dx$. Если эта последовательность сходится и её предел не зависит от выбора допустимого исчерпания, то несобственный интеграл $\int_{D} f(x) dx = \lim_{n \to \infty} \int_{D_n} f(x) dx \in \mathbb{R}$ называется сходящимся, а функцию $f$ называем интегрируемой на $D$ в несобственном смысле. Если предел бесконечен или для различных допустимых исчерпаний получаются разные значения предела, то несобственный интеграл называется расходящимся. 
