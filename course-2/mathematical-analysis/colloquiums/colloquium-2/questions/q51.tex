% Здесь НЕ НУЖНО делать begin document, включать какие-то пакеты..
% Все уже подрубается в головном файле
% Хедер обыкновенный хсе-теха, все его команды будут здесь работать
% Пожалуйста, проверяйте корректность теха перед пушем

% Здесь формулировка билета
\subsection{Сформулируйте и докажите теорему о замене переменных в кратном интеграле}

\begin{theorem*}[о замене переменных в кратном интеграле]
    Пусть функция $f$ ограничена и интегрируема на замкнутом связном жордановом множестве $D$; $\varphi$~--- диффеоморфизм, $\varphi: G \to D$, $\varphi(G) = D$. Тогда $f(\varphi(u)) \cdot |J_\varphi(u)|$ интегрируема на $G$, причём 
    \begin{equation*}
        \int_D f(x)dx = \int_G f(\varphi(u)) \cdot |J_\varphi(u)| \cdot du
    \end{equation*}
\end{theorem*}
\begin{proof}
    Рассмотрим разбиение множества $G = \bigsqcup G_i$, $D_i = \varphi(G_i)$, $D = \bigsqcup D_i$. 

    По теореме о среднем:
    \begin{equation*}
        \exists \eta_i \in \overline{G_i}: \nu(G_i) = |J_\varphi(\eta_i)| \cdot \mu(G_i)
    \end{equation*}
    Обозначим $\xi_i = \varphi(\eta_i) \in \overline{D_i}$.
    \begin{equation*}
        \sum f(\xi_i) \cdot \mu(D_i) = \sum f(\xi_i) \cdot |J_\varphi(\eta_i)| \cdot \mu(G_i) = \sum f(\varphi(\eta_i)) \cdot |J_\varphi(\eta_i)| \cdot \mu(G_i).
    \end{equation*}
    Так как $f \in \mathcal{R}(D)$, то 
    \begin{equation*}
        \sum f(\xi_i) \cdot \mu(D_i) \to \int_D f(x) dx
    \end{equation*}
    $\sum f(\varphi(\eta_i)) \cdot |J_\varphi(\eta_i)| \cdot \mu(G_i)$~--- в точности интегральная сумма для функции $f(\varphi(u)) \cdot |J_\varphi(u)|$, следовательно
    \begin{equation*}
        \sum f(\varphi(\eta_i)) \cdot |J_\varphi(\eta_i)| \cdot \mu(G_i) \to \int_G f(\varphi(u)) \cdot |J_\varphi(u)| du
    \end{equation*}
    Но $\sum f(\xi_i) \cdot \mu(D_i) = \sum f(\varphi(\eta_i)) \cdot |J_\varphi(\eta_i)| \cdot \mu(G_i)$, а значит,
    \begin{equation*}
        \int_D f(x) dx = \int_G f(\varphi(u)) \cdot |J_\varphi(u)| du
    \end{equation*}
\end{proof}