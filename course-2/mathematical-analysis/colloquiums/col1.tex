\documentclass[a4paper]{article}
\usepackage{header}

% Use \begin{theorem*} instead of \begin{theorem}.
% Use \iff instead of \Leftrightarrow.

% Команды для ToC'a
\newcommand\enumtocitem[3]{\item\textbf{#1}\addtocounter{#2}{1}\addcontentsline{toc}{#2}{\protect{\numberline{#3}} #1}}
\newcommand\defitem[1]{\enumtocitem{#1}{subsection}{\thesubsection}}
\newcommand\proofitem[1]{\enumtocitem{#1}{subsubsection}{\thesubsubsection}}

\newlist{colloq}{enumerate}{1}
\setlist[colloq]{label=\textbf{\arabic*.}}

\title{\Huge Математический Анализ - 2 - Коллоквиум 1}
\author{
    Серёжа Рахманов | \href{https://t.me/virg1n}{telegram}, \href{http://shoraii.github.io}{website}
    \\
    Денис Болонин | \href{https://t.me/ultrakekul}{telegram}
    \\
    Максим Николаев | \href{https://t.me/MaximND}{telegram}

}
\date{Версия от {\ddmmyyyydate\today} \currenttime}

\begin{document}
    \maketitle

    \tableofcontents

    \newpage

    \section{Вопросы}

    \begin{colloq}

    \defitem{Дайте определения: числовой ряд, частичная сумма ряда, сумма ряда, сходящийся ряд, расходящийся ряд. Рассмотрим ряд с общим членом $a_n$. Докажите, что если ряд сходится, то $a_n \to 0$.}
        \begin{definition*}
        Пусть $a_{n}$ -- последовательность, т.е. $\NN \to \RR$. Формальная бесконечная сумма: $a_1 + a_2 + a_3 + \dots = \sum_{n=1}^{\infty} a_n$ называется рядом.
        $S_N = \sum_{n = 1}^{N} a_n$ -- частичная сумма, сумма ряда: $S = \lim_{N \to \infty} S_N$
        \end{definition*}
        
        Возможны 3 случая:
        \begin{enumerate}
            \item $\exists S \in \RR$
            \item $\exists S = \infty$
            \item $\nexists S$
        \end{enumerate}
        
        В первом случае говорят, что ряд сходится, иначе -- что ряд расходится.

        \begin{comment}~
            Если ряд сходится, то $a_n \to 0$
        \end{comment}
        \begin{proof}
            $a_n = S_n - S_{n - 1} \to 0$, т.к. $S_n \to S$ и $S_{n - 1} \to S$
        \end{proof}
    
    \defitem{Сформулируйте критерий Коши сходимости числовой последовательности. Сформулируйте и докажите критерий Коши сходимости числового ряда.}
        \begin{definition*}
            ${S_n}$ называется фундаментальной, если $\forall \epsilon > 0$  $\exists N: \forall n > m > N, |S_n - S_m| < \epsilon$
        \end{definition*}
        \begin{theorem*}
            ${S_n}$ -- сходится $\iff {S_n}$ -- фундаментальная
        \end{theorem*}
        \begin{proof}
            Сходимость числового ряда -- это сходимость последовательности $\{S_n\}$, его частичных сумм, а для сходимости последовательности $\{S_n\}$ необходимо и достаточно, чтобы она была фундаментальной.
        \end{proof}

    \defitem{Сформулируйте и докажите признак сравнения положительных числовых рядов, основанный на неравенстве $a_n \leq b_n$.}
        
        $0 \leq a_n \leq b_n$ при всех $n \geq n_0$
	
	    Ряд $\sum b_n$ сходится $\implies$ ряд $\sum a_n$ сходится
	
        Ряд $\sum a_n$ расходится $\implies$ ряд $\sum b_n$ расходится
        
        \begin{proof}
        	На основании того, что отбрасывание конечного числа элементов ряда не отражается на его поведении, мы можем считать, что $0 \leq a_n \leq b_n$ при всех $n = 1, 2, 3, \dots$ Обозначив частные суммы через $A$ и $B$ соответственно, имеем $A_n \leq B_n$. Пусть ряд $\sum b_n$ сходится, тогда $B_n$ ограничена, $B_n \leq S, S = const, \forall n$. В таком случае $A_n$ также меньше либо равна некоторому $S$, что даёт нам ограниченность $\sum a_n$.
        \end{proof}
    
    \defitem{Сформулируйте и докажите признак сравнения положительных числовых рядов, основанный на неравенстве $\frac{a_{n+1}}{a_n} \leq \frac{b_{n+1}}{b_n}$.}
        
        Ряд $\sum b_n$ сходится $\implies$ ряд $\sum a_n$ сходится
	
        Ряд $\sum a_n$ расходится $\implies$ ряд $\sum b_n$ расходится
        
        \begin{proof}~
        
        $a_{n_0+1} \leq \frac{a_{n_0}}{b_{n_0}}\cdot b_{n_0 + 1}$
        
        $a_{n_0+2} \leq \frac{a_{n_0 + 1}}{b_{n_0 + 1}}\cdot b_{n_0 + 2} \leq \frac{a_{n_0}}{b_{n_0}}\cdot b_{n_0 + 2}$
        
        $\vdots$
        
        $a_{n_0+k} \leq \frac{a_{n_0}}{b_{n_0}}\cdot b_{n_0 + k} \implies \sum_{n=n_0}^{N} a_n \leq \frac{a_{n_0}}{b_{n_0}}\cdot \sum_{n=n_0}^{N} b_n$
        \end{proof}

    \defitem{Сформулируйте и докажите признак сравнения положительных числовых рядов, основанный на пределе $\lim \frac{a_n}{b_n}$.}
        
        $\lim_{n \to \infty} \frac{a_n}{b_n} \in (0; +\infty) \implies$ сходимость $\sum a_n \iff$ сходимость $\sum b_n$
        
        \begin{proof}~
            
        $c = \lim_{n \to \infty} \frac{a_n}{b_n} > 0$
        
        $\forall \epsilon\ \exists n_0:\ c - \epsilon \leq \frac{a_n}{b_n} \leq c + \epsilon$, при $n \geq n_0$
        
        Возьмём $c - \epsilon > 0 \implies (c - \epsilon)\cdot b_n \leq a_n \leq (c + \epsilon)\cdot b_n$
        
        Сходимость следует из правой части неравенства, а расходимость из левой. 
        \end{proof}

    \defitem{Пусть последовательности $\{a_n\}$, $\{A_n\}$ таковы, что $a_n - (A_n - A_{n - 1}) = c_n$ и ряд $\sum c_n$ сходится.
        Докажите, что существует $C$ такое, что $a_1 + a_2 + \dots + a_n = A_n + C + o(1)$.}
        
        \begin{proof}~
        	
        $\sum_{n=1}^{N} c_n = \sum_{n=1}^{N} a_n - \sum_{n=1}^{N} (A_n - A_{n-1}) = \sum_{n=1}^{N} a_n - A_N + A_{0} \implies a_1 + a_2 + \dots + a_n = A_n + \left(- A_{0} + \sum_{n=1}^{N} c_n \right)$.
        
        Получим требуемое, если возьмём $C = \lim_{N\to \infty} - A_{0} + \sum_{n=1}^{N} c_n$.
		\end{proof}
	
    \defitem{Сформулируйте и докажите признак Лобачевского-Коши.}
        \begin{proposal}
            Пусть $a_n > 0$ и $a_n \downarrow$
        
            Тогда ряды $\sum a_n$ и $\sum 2^n \cdot a_{2^n}$ ведут себя одинаково
        \end{proposal}
        \begin{proof}
            $a_1 + (a_2) + (a_3 + a_4) + (a_5 + \dots + a_8) + \dots$
        
            $a_2 \leq a_1$
            
            $a_2 \geq a_2$
        
            $a_3 + a_4 \leq 2a_2$
            
            $a_3 + a_4 \geq 2a_4$
        
            $a_5 + \dots + a_8 \leq 4a_4$
        
            $a_5 + \dots + a_8 \geq 4a_8$
        
        
            $\dots$
        
            $a_1 + \dfrac{1}{2}\sum_{n=1}^{m} 2^n a_{2^n} \leq \sum_{n = 1}^{2^m} a_n \leq a_1 + \sum_{n=0}^{m-1} 2^n a_{2^n}$
        
        \end{proof}

    \defitem{Сформулируйте теорему Штольца о пределе последовательности. Покажите на примере, как с помощью теоремы Штольца можно уточнить асимптотическую оценку для частичной суммы ряда. $\frac{p_n}{q_n}$, $p_n$, $q_n \to 0$.}

        \begin{theorem*}
            (Штольца.) Если $p_n, q_n \to 0, q_n \downarrow$ и $\exists \lim \dfrac{p_{n + 1} - p_n}{q_{n + 1} - q_n}$, то
            $\lim \dfrac{p_n}{q_n} = \lim \dfrac{p_{n + 1} - p_n}{q_{n + 1} - q_n}$
        \end{theorem*}

        \begin{example}
            Дан ряд $\sum_{n = 1}^{\infty} \dfrac{1}{n^2}$. Пусть $S$ --- сумма соответствующего ряда. Необходимо доказать, что 
            \begin{equation*}
                S_N = S - \dfrac{1}{N} + \operatorname{o} \left( \dfrac{1}{N} \right).
            \end{equation*}

            Обозначим $x_n = S - S_n \to 0$ и $y_n = \dfrac{1}{n} \to 0$. Рассмотрим предел отношения разностей:
            \begin{equation*}
                \lim_{n \to \infty} 
                = \dfrac{S - S_n - (S - S_{n - 1})}{\frac{1}{n} - \frac{1}{n - 1}}
                = \dfrac{1}{n^2} \cdot n \cdot (n - 1) \to 1.
            \end{equation*}

            По теореме Штольца $\lim_{n \to \infty} \dfrac{x_n}{y_n} = 1$. То есть 
            \begin{equation*}
                \dfrac{x_n}{y_n} = 1 + \operatorname{o}(1) \implies x_n = y_n + \operatorname{o}(y_n).
            \end{equation*}

            Отсюда и получаем то, что было в условии:
            \begin{equation*}
                S_n = S - \dfrac{1}{n} + \operatorname{o} \left( \dfrac{1}{n} \right).
            \end{equation*}
        \end{example}

    \defitem{Пусть $\sum a_n$, $\sum a_n'$ - сходящиеся положительные ряды. Говорят, что ряд $\sum a_n'$ сходится быстрее ряда $\sum a_n$, если $a_n' = o(a_n)$. Докажите, что в этом случае также $r_n' = o(r_n)$, где $r_n$, $r_n'$ - остатки соответствующих рядов.}
        
        Рассмотрим остатки каждого из рядов. $r_n = S - S_N$, где $S_N$ - частичная сумма ряда $\sum a_n$ и $S_N \to S$ при $N \to \infty$. Для $\sum a_n'$ аналогично $r_n' = S' - S_N'$, где $S_N'$ - частичная сумма ряда $\sum a_n'$ и $S_N' \to S'$ при $N \to \infty$. Идёт речь о том, что ряд $a_n'$ сходится быстрее ряда $a_n$, т.е. оба ряда сходятся и $S = S'$. Но, поскольку члены рядов находятся в отношении $a_n' = o(a_n)$, то мы можем сделать выводы о частичных суммах $S_N$ и $S_N'$. $\forall N, S_N' = o(S_N)$, что указывает нам в результате на отношение между остатками $r_n' = o(r_n)$.

    \defitem{Пусть $\sum a_n$, $\sum a_n'$ - расходящиеся положительные ряды. Говорят, что ряд $\sum a_n'$ расходится медленнее ряда $\sum a_n$, если $a_n' = o(a_n)$. Докажите, что в этом случае также $S_n' = o(S_n)$, где $S_n$, $S_n'$ - частичные суммы соответствующих рядов.}
        
        Оба ряда расходятся, тогда $S_n \to \infty$ и $S_n' \to \infty$ при $n \to \infty$. Мы понимаем, что $S_n = \sum_{n = 1}^{N} a_n$, $S_n' = \sum_{n = 1}^{N} a_n'$. Это значит, что для некоторого $n_1$ мы имеем следующее: $S_{n_1} = \sum_{n = 1}^{n_1} a_n$, $S_{n_1}' = \sum_{n = 1}^{n_1} a_n'$, где для любого $n = 1, 2, 3, \dots, n_1$ выполняется отношение $a_n' = o(a_n)$. В таком случае для частичных сумм справедливо отношение $S_{n_1}' = o(S_{n_1})$. А так как и для всех последующих $a_n$ и $a_n'$ также справедливо отношение $a_n' = o(a_n)$, то мы можем сказать, что $S_n' = o(S_n)$.

    \defitem{Пусть положительный ряд $\sum a_n$ сходится и $r_n$ --- его остаток. Докажите, что ряд $\sum (\sqrt{r_{n}} - \sqrt{r_{n+1}})$ также сходится, причём медленнее, чем ряд $\sum a_{n+1}$.}
        
        Вспомним, что $r_n = S - S_n$.

        Докажем сходимость:
        \begin{align*}
            \sum_{n=0}^{N} (\sqrt{r_{n}} - \sqrt{r_{n+1}}) 
            &= \sqrt{r_0} - \sqrt{r_1} + \sqrt{r_1} - \sqrt{r_2} + \dots + \sqrt{r_{N}} - \sqrt{r_{N+1}} \\
            &= \sqrt{r_0} - \sqrt{r_{N+1}} \\
            &= \sqrt{S} - \sqrt{r_{N+1}} \\
            &\to \sqrt{S} \text{ (так как } r_{N+1} \to 0)
        \end{align*}
        	

        Докажем, что ряд сходится медленнее:
        \begin{equation*}
            \frac{\sqrt{r_{n}} - \sqrt{r_{n+1}}}{a_{n+1}} = \frac{\sqrt{r_{n}} - \sqrt{r_{n+1}}}{r_{n} - r_{n+1}} = \frac{1}{\sqrt{r_{n}} + \sqrt{r_{n+1}}} \to \infty,
        \end{equation*}
        так как $\sqrt{r_{n}} \to 0$ и $\sqrt{r_{n+1}} \to 0$.
        
    \defitem{Пусть положительный ряд $\sum a_n$ расходится и $S_n$ его частичная сумма. Докажите, что ряд $\sum (\sqrt{S_{n+1}} - \sqrt{S_n})$ также расходится, причём медленнее, чем ряд $\sum a_{n+1}$.}

        Докажем расходимость:
        \begin{align*}
            \sum_{n=0}^{N} (\sqrt{S_{n+1}} - \sqrt{S_n}) 
            &= \sqrt{S_1} - \sqrt{S_0} + \sqrt{S_2} - \sqrt{S_1} + \dots + \sqrt{S_{N+1}} - \sqrt{S_{N}} \\
            &= \sqrt{S_{N+1}} - \sqrt{S_0} \\
            &= \sqrt{S_{N+1}} \to \sqrt{S}.
        \end{align*}

        Перейдем ко второй части вопроса:
        \begin{align*}
            \dfrac{\sqrt{S_{n+1}} - \sqrt{S_n}}{a_{n+1}} 
            &= \dfrac{\sqrt{S_{n+1}} - \sqrt{S_n}}{S_{n+1} - S_n} \\
            &= \dfrac{1}{\sqrt{S_{n+1}} + \sqrt{S_n}},
        \end{align*}
        где $\sqrt{S_{n+1}} + \sqrt{S_n} \to \infty$. Это значит, что $\dfrac{1}{\sqrt{S_{n+1}} + \sqrt{S_n}}$ стремится к $0$. Тогда ряд $\sum (\sqrt{S_{n+1}} - \sqrt{S_n})$ расходится, причём медленнее, чем ряд $\sum a_{n+1}$.
        
    \defitem{Сформулируйте признак Даламбера для положительного ряда}
        \begin{theorem*}
            Признак Даламбера. Пусть $a_n > 0$.
            
            $\overline{\lim} \dfrac{a_{n+1}}{a_n} < 1 \implies $ ряд $\sum a_n$ сходится.
            
            $\underline{\lim} \dfrac{a_{n+1}}{a_n} > 1 \implies $ ряд $\sum a_n$ расходится.
        \end{theorem*}

    \defitem{Сформулируйте радикальный признак Коши для положительного ряда.}
        \begin{theorem*}
            Радикальный признак Коши. Пусть $a_n \geq 0$.
            
            $\overline{\lim} \sqrt[n]{a_n} < 1 \implies$ ряд $\sum a_n$ сходится.
             
            $\underline{\lim} \sqrt[n]{a_n} > 1 \implies$ ряд $\sum a_n$ расходится.
        \end{theorem*}

    \defitem{Докажите, что всякий раз, когда признак Даламбера даёт ответ на вопрос о сходимости ряда, то радикальный признак Коши даёт (тот же) ответ на этот вопрос.}
        
        Пусть $a_n > 0$. Тогда заметим, что

        \begin{equation*}
            \underline{\lim} \dfrac{a_{n+1}}{a_n} \leq \underline{\lim}{\sqrt[n]{a_n}} \leq \overline{\lim}{\sqrt[n]{a_n}} \leq \overline{\lim}\dfrac{a_{n+1}}{a_n}.
        \end{equation*}

        Докажем правую часто неравенства, левая доказывается аналогично.

        \begin{proof}
            Пусть $q = \overline{\lim} \sqrt[n]{a_n}$ и $p = \overline{\lim} \dfrac{a_{n + 1}}{a_n}$. Тогда необходимо доказать, что $q \leq p$.

            Докажем от противного. Предположим, что $p < q$. 

            Так как мы берем верхний предел, то для любого $\eps > 0$ существует $\{n_k\}$, что $\sqrt[n_k]{a_{n_k}} \geq q - \eps \iff a_{n_k} \geq (q - \eps)^{n_k}$.

            Из определения $p$ следует, что для любого $\eps > 0$ существует $n_0$, что $\dfrac{a_{n + 1}}{a_n} \leq p + \eps$ для любых $n \geq n_0$, что равносильно $a_{n_0 + m} \leq a_{n_0} \cdot (p + \eps)^m$.

            Теперь взяв $a_{n_k}$ можем получить следующее:
            \begin{equation*}
                (q - \eps)^{n_k} \leq a_{n_k} \leq a_{n_0} \cdot (p + \eps)^{n_k - n_0} = a_{n_0} \cdot \dfrac{(p + \eps)^{n_k}}{(p + \eps)^{n_0}}.
            \end{equation*}

            Отсюда получаем, что $\dfrac{a_{n_0}}{(p + \eps)^{n_0}} \geq \left( \dfrac{q - \eps}{p + \eps} \right)^{n_k}$ при всех $k = 1, 2, \dots$. Но при малом $\eps$ мы имеем
            \begin{equation*}
                \dfrac{q - \eps}{p + \eps} > 1.
            \end{equation*}

            Тогда мы пришли к противоречию, так как в $\dfrac{a_{n_0}}{(p + \eps)^{n_0}} \geq \left( \dfrac{q - \eps}{p + \eps} \right)^{n_k}$ слева записано конечное число, а справа будет бесконечность.
        \end{proof}

        Если $\overline{\lim}\dfrac{a_{n+1}}{a_n} < 1 \implies \overline{\lim}{\sqrt[n]{a_n}} < 1$

        Если $\underline{\lim}\dfrac{a_{n+1}}{a_n} > 1 \implies \underline{\lim} \sqrt[n]{a_n} > 1$

        Если $\exists \lim \frac{a_{n+1}}{a_n}$, то $\overline{\lim} \frac{a_{n+1}}{a_n} = \underline{\lim} \frac{a_{n+1}}{a_n} \implies \exists \lim \sqrt[n]{a_n} = \lim \frac{a_{n+1}}{a_n}$

    \defitem{Докажите, что если для положительного ряда $\sum a_n$ существует $\lim \frac{a_{n + 1}}{a_n} = q$, то существует и $\lim \sqrt[n]{a_n} = q$.}

    \defitem{Приведите пример положительного ряда, вопрос о поведении которого не может быть решен с помощью признака Даламбера, но может быть решен с помощью радикального призанка Коши (с обоснованием).}

    \defitem{Приведите пример ряда, который сходится медленнее любого ряда геометрической прогрессии, но быстрее любого обобщённого гармонического ряда (с обоснованием).}
        
        Докажем, что ряд $\sum_{n = 1}^{\infty} e^{-\sqrt{n}}$ подходит.
        
        \begin{itemize}
        \item 
            $\sum q^n$ --- ряд геометрической прогрессии, $0 < q < 1$; $q^n = e^{n * \ln q}$, где $\ln q < 0$.

        \item 
            $\sum \dfrac{1}{n^p}$ --- обобщённый гармонический ряд. $\dfrac{1}{n^p} = e^{-p \ln n}$, $p > 1$.
        \end{itemize}

        Заметим, что при любых $p$, $q$ и при любом $n \geq n_0$ выполняется
        \begin{equation*}
            p \ln n < \sqrt{n} < n \ln \dfrac{1}{q}.
        \end{equation*}

        Перейдем к доказательствам.
        \begin{itemize}
        \item 
            Докажем, что выбранный ряд сходится медленнее геометрической прогрессии:
            \begin{equation*}
                \dfrac{e^{-\sqrt{n}}}{q^n} = e^{-\sqrt{n} + n \ln \frac{1}{q}} \to +\infty,
            \end{equation*}
            так как $-\sqrt{n} + n \ln \dfrac{1}{q} \to + \infty$.

        \item 
            Докажем, что выбранный ряд сходится быстрее гармонического ряда:
            \begin{equation*}
                \dfrac{e^{-\sqrt{n}}}{1 / n^p} = e^{-\sqrt{n} + p \ln n} \to 0,
            \end{equation*}
            так как $-\sqrt{n} + p \ln n \to -\infty$.
        \end{itemize}
        
    \defitem{Сформулируйте признак Гаусса для положительного ряда. Приведите пример применения признака Гаусса.}

        Если существует $\delta > 0$ и $p$ такие, что $\dfrac{a_{n+1}}{a_n} = 1 - \dfrac{p}{n} + \operatorname{O}\left(\dfrac{1}{n^{1 + \delta}}\right) $
        то справедливо следующее:
        \begin{itemize}
        \item 
            $p > 1 \implies$ ряд $\sum a_n$ сходится.

        \item 
            $p \leq 1 \implies$ ряд $\sum a_n$ расходится.
        \end{itemize}

        \begin{example}
            Исследуем на сходимость следующий ряд:
            \begin{equation*}
                \sum_{n = 1}^{\infty} \left( \dfrac{(2n - 1)!!}{(2n)!!} \right)^2.
            \end{equation*}
    
            Выполним некоторые преобразования:
            \begin{align*}
                \dfrac{a_n}{a_{n + 1}} 
                &= \left( \dfrac{(2n - 1)!!}{(2n!!)} \cdot \dfrac{(2n + 2)!!}{(2n + 1)!!} \right)^2
                = \left( \dfrac{2n + 2}{2n + 1} \right)^2
                = \left( \dfrac{1 + \frac{1}{n}}{1 + \frac{1}{2n}} \right)^2 \\
                &\sim \left[ \dfrac{1}{1 + x} \sim 1 - x \right] 
                \sim \left( 1 + \dfrac{1}{n} \right)^2 \cdot \left( 1 - \dfrac{1}{2n} \right)^2 \\
                &\sim \left( 1 + \dfrac{2}{n} \right) \cdot \left( 1 - \dfrac{1}{n} \right)
                = 1 + \dfrac{1}{n} + \dfrac{2}{n^2}
            \end{align*}

            Получили $p = 1$, $\gamma = 1$. Тогда ряд расходится.
        \end{example}

    \defitem{Приведите пример положительного ряда, вопрос о поведении которого не может быть решен с помощью признака Гаусса (с обоснованием).}

    \defitem{Выведите двустороннюю оценку для частичной суммы ряда через определённый интеграл. Сформулируйте и докажите интегральный признак Коши-Маклорена}

        Рассмотрим $f(x) \downarrow$ при $x \geq n_0 - 1$ и ряд $\sum_{n=n_0}^{\infty} a_n$, где $a_n = f(n)$:
        \[\begin{gathered}
            f(n + t) \leq a_n \leq f(n - 1 + t), t \in [0; 1].
        \end{gathered}\]

        Отсюда следует, что
        \begin{equation*}
            \int_{0}^{1} dt : \ \ \ \ \int_{n}^{n+1} f(x)dx \leq a_n \leq \int_{n-1}^{n} f(x)dx.
        \end{equation*}

        Просуммируем полученное:
        \begin{equation*}
            \int_{n_0}^{N+1} f(x)dx \leq \sum_{n=n_0}^{N} a_n \leq \int_{n_0-1}^{N} f(x)dx
        \end{equation*}

        Тогда $\sum a_n$ ведёт себя так же, как и несобственный интеграл $\int^{\infty}f(x)dx$.

    \defitem{Что такое улучшение сходимости положительного ряда? Покажите на примере как можно улучшить сходимость ряда.}

        Пусть у нас есть некоторый ряд $\sum a_n$ и он сходится медленно. В таких случаях для расчёта суммы ряда с необходимой точностью потребуется взять больше членов, что неудобно. Мы можем преобразовать наш ряд для улучшения сходимости, т.е. получить некоторый ряд $\sum a_n'$, который будет сходиться быстрее, чем исходный $\sum a_n$.
       	\begin{example}
            Пусть у нас есть ряд $S = \sum_{n = 1}^{\infty} \dfrac{1}{n^2 + 2} \approx \sum_{n = 1}^{\infty} \dfrac{1}{n^2}$. Воспользуемся  методом Куммера. Для улучшения сходимости будем брать ряды вида $\sum_{n = 1}^{\infty} \dfrac{1}{n(n+1)} = 1, \sum_{n = 1}^{\infty} \dfrac{1}{n(n+1)(n+2)} = \dfrac{1}{4}, \dots$. 
            
            В данном случае нам подойдёт первый ряд в этом списке, поскольку $\dfrac{1}{n^2} \sim \dfrac{1}{n(n+1)}$.
            
			$\sum_{n=1}^{\infty} \left(\frac{1}{n^2+2} - \frac{1}{n(n+1)}\right) = S - 1 \implies S = 1 + \sum_{n=1}^{\infty} \left(\frac{1}{n^2 + 2} - \frac{1}{n(n+1)}\right)$
			
			$\frac{1}{n^2+2} - \frac{1}{n(n+1)} = \frac{1}{n^2} \cdot \left(\frac{1}{1 + \frac{2}{n^2}} - \frac{1}{1 + \frac{1}{n}}\right) = \frac{1}{n^2} \cdot \left(1 - \frac{2}{n^2} + o\left(\frac{1}{n^2}\right) - \left(1 - \frac{1}{n} + \frac{1}{n^2} + o\left(\frac{1}{n^2}\right)\right)\right) = \frac{1}{n^3} + o\left(\frac{1}{n^3}\right)$.
         
            Получили ряд $\sum_{n = 1}^{\infty} \dfrac{1}{n^3}$, который сходится быстрее, $1 + \sum_{n = 1}^{\infty} \dfrac{1}{n^3} \approx \sum_{n = 1}^{\infty} \dfrac{1}{n^2 + 2}$.
        \end{example}
    
    \defitem{Дайте определения: знакопеременный ряд, знакочередующийся ряд, абсолютно сходящийся ряд, условно сходящийся ряд, положительная часть ряда, отрицательная часть ряда.}
    
    	\begin{definition*}
    		Пусть существует ряд $\sum a_n$. такой, что $\forall i$, $a_i$ может быть, как больше 0, так и меньше 0. В таком случае ряд $\sum a_n$ называется знакопеременным.
    	\end{definition*}
    		
    	\begin{definition*}
    		Пусть существует ряд $\sum a_n$. такой, что $\forall i$, $a_i \cdot a_{i+1} < 0$. В таком случае ряд $\sum a_n$ называется знакочередующимся.
    	\end{definition*}
    
    	\begin{definition*}
            Ряд $\sum\limits_{n = 1}^\infty {{a_n}}$ называется абсолютно сходящимся, если ряд $\sum\limits_{n = 1}^\infty {\left| {{a_n}} \right|}$ также сходится.
            
            Если ряд $\sum\limits_{n = 1}^\infty {{a_n}}$ сходится абсолютно, то он является сходящимся (в обычном смысле). Обратное утверждение неверно.
   		\end{definition*}
        
        \begin{definition*}
            Ряд $\sum\limits_{n = 1}^\infty {{a_n}}$ называется условно сходящимся, если сам он сходится, а ряд, составленный из модулей его членов, расходится.
        \end{definition*}
        
   		\begin{definition*}
   			Введем два ряда: $a_n^+ = \begin{cases}
   			a_n, a_n > 0 \\
   			0
   			\end{cases}$ 
   			и $a_n^- = \begin{cases}
   			|a_n|, a_n < 0 \\
   			0
   			\end{cases}$.
   			Тогда ряды $\sum a_n^+$ и $a_n^-$ соответственно называются положительной и отрицательной частью ряда $\sum a_n$.
   		\end{definition*}
   	
   		\defitem{Докажите, что ряд сходится абсолютно ровно в том случае, когда сходятся его положительная и отрицательная части.}
   			
   		\begin{proof}
   		
   			Рассмотрим ряд $\sum a_n$, дополнительный ряд $\sum |a_n|$, а также положительную и отрицательную части $\sum a_n^+$ и $\sum a_n^-$.
   			
   			1) Пусть ряд $\sum a_n$ сходится абсолютно. В таком случае ряд $\sum |a_n|$ сходится, а так как члены рядов $\sum a_n^+$ и $\sum a_n^-$ все содержатся в ряде $\sum |a_n|$, то для всех их частичных сумм справедливо следующее: $0 \leq P_k \leq A_n'$ и $0 \leq Q_m \leq A_n'$, где $P_k$ и $Q_m$ - частичные суммы положительной и отрицательной части соответственно, а $A_n'$ - частичная сумма дополнительного ряда и $A_n' = P_k + Q_m, n = m + k$. Это значит, что оба ряда $\sum a_n^+$ и $\sum a_n^-$ сходятся.
   			
   			2) Исходя из того, что $S_n = P_k - Q_m, n = m + k$ и положительных и отрицательных элементов в $\sum a_n$ бесконечное множество, мы получаем, что при $n \to \infty$ одновременно $m \to \infty$ и $k \to \infty$. Переходя к пределу получаем, что исходный ряд сходится абсолютно и его сумма будет равна $P - Q$.
   			
   			
   		\end{proof}
   	
   		\defitem{Докажите, что если ряд сходится условно, то его положительная и отрицательная части расходятся (имеют бесконечные суммы).}
   		
   		\begin{proof}
   			Рассмотрим ряд $\sum a_n$, дополнительный ряд $\sum |a_n|$, а также положительную и отрицательную части $\sum a_n^+$ и $\sum a_n^-$. Поскольку ряд $\sum a_n$ сходится условно, то $\sum |a_n|$ расходится. Рассмотри частичные суммы $\sum |a_n|$, $\sum a_n^+$ и $\sum a_n^-$ - $A_n', P_k, Q_m$ соответственно. Для любого $n = m + k$, $A_n' = P_k + Q_m$. При $n \to \infty$, $m \to \infty$ и $k \to \infty$. Так как ряд $\sum |a_n|$ расходится, то сумма $A_n' \to \infty$. Поскольку число положительных и отрицательных элементов бесконечно, то получаем $P_k \to \infty$ и $Q_m \to \infty$, а значит ряды $\sum a_n^+$ и $\sum a_n^-$ расходятся.
   		\end{proof}
           
    \defitem{Сформулируйте мажорантный признак Вейерштрасса. Приведите пример применения признака.}
        
        \begin{theorem*}
            Если $|a_n| \leq b_n$ при $n > n_0$ и положительный ряд $\sum b_n$ сходится,
            то $\sum a_n$ сходится, причём абсолютно.
        \end{theorem*}
        
        \begin{example}
        $\sum_{n=1}^{\infty} \dfrac{\sin(nx)}{n^p}$, $p > 0$
        
        $|sin(nx)| \leq 1 \implies \left|\dfrac{sin(nx)}{n^P}\right| \leq \dfrac{1}{n^p}$
        
        $\sum \dfrac{1}{n^p} $ сходится $(p > 1) \implies \sum_{n=1}^{\infty} \dfrac{\sin(nx)}{n^p}$ сходится абсолютно.
        \end{example}

    \defitem{Что такое группировка членов ряда? Докажите, что любой ряд, полученный из сходящегося ряда группировкой его членов, сходится и имеет ту же сумму.}
        
        Говорят, что ряд $\sum b_k$ получен из $\sum a_n$ группировкой членов, если $\exists n_1 < n_2 < \dots$:

        $b_1 = a_1 + a_2 + \dots + a_{n_1}$

        $b_2 = a_{n_1 + 1} + a_{n_1 + 2} + \dots + a_{n_2}$

        $\dots$

        \begin{comment}
            Если $\sum a_n$ сходится, то ряд $\sum b_k$ сходится к той же сумме.
        \end{comment}

        \begin{proof}
        $\sum_{k=1}^{m} b_k = \sum_{n=1}^{n_m} a_n$
        \end{proof}

        \textit{Обратное утверждение неверно:} $(1 - 1) + (1 - 1) + \dots$

    \defitem{Как с помощью группировки преобразовать знакопеременный ряд в знакочередующийся? Что можно утверждать о сходимости полученного знакочередующегося ряда?}
        
        Знакопеременный ряд при помощи группировки сводится к знакочередующемуся:

        $a_1 \leq 0$, $\dots$, $a_{n_1} \leq 0$; $b_1 = \sum_{i=1}^{n_1} a_i \leq 0$

        $a_{n_1+1} \geq 0$, $\dots$, $a_{n_2} \geq 0$; $b_2 = \sum_{i={n_1 + 1}}^{n_2} a_i \geq 0$

        При такой группировке сходимость исходного ряда $\iff$ сходимость $\sum b_n$

    \defitem{Приведите пример преобразования знакопеременного (но не знакочередующегося) ряда к знакочередующемуся.}

        \begin{example}
            $\sum_{n=1}^{\infty} \dfrac{(-1)^{[\ln n]}}{n}$

            $\sum_{k=0}^{\infty} b_k$, где $b_k = (-1)^k \cdot \sum_{e^k \leq n < e^{k+1}} \dfrac{1}{n}$

            $|b_k| = \sum_{n=[e^k] + 1}^{[e^{k+1}]} \dfrac{1}{n} \leq \dfrac{1}{[e^k] + 1} \cdot ([e^{k+1}]-[e^k]) \approx \dfrac{e^{k+1} - e^k}{e^k} \to e - 1 > 0$
        \end{example}

    \defitem{Для знакочередующегося ряда с убывающем по модулю общим членом сформулируйте оценку $n$-го остатка. Приведите пример применения этой оценки.}

    \defitem{Сформулируйте признак Лейбница для знакочередующегося ряда. Приведите пример применения признака Лейбница.}
        
        \begin{theorem*}
            Признак Лейбница. Если $u_n \downarrow 0$, то ряд сходится, причём $|r_n| \leq u_{n+1}$
            \end{theorem*}
            
            \begin{example}
                $\sum_{n=1}^{\infty} \dfrac{(-1)^{n}}{n^p}$, $p > 0$
            
                $\dfrac{1}{n^p} \downarrow 0 \implies $ ряд сходится (при $\forall p > 0$)
            \end{example}

    \defitem{Покажите на примере, что к знакопеременным рядам неприменим предельный признак сравнения.}

        Рассмотрим 2 ряда: $\sum_{n=1}^{\infty} \dfrac{(-1)^{n}}{\sqrt{n} - (-1)^{n}}$ и $\sum_{n=1}^{\infty} \dfrac{(-1)^{n}}{\sqrt{n}}$. Второй ряд сходится по признаку Лейбница.

        $\dfrac{(-1)^{n}}{\sqrt{n} - (-1)^{n}} \approx \dfrac{(-1)^{n}}{\sqrt{n}}$

        $\dfrac{(-1)^{n}}{\sqrt{n} - (-1)^{n}} - \dfrac{(-1)^{n}}{\sqrt{n}} = \dfrac{1}{\sqrt{n}(\sqrt{n} - (-1)^{n})} \approx \dfrac{1}{n}$ -- расходится

        $\sum_{n=1}^{N} \dfrac{(-1)^{n}}{\sqrt{n} - (-1)^{n}} = \sum_{n=1}^{N} \dfrac{(-1)^{n}}{\sqrt{n}} + \sum_{n=1}^{N} \dfrac{1}{\sqrt{n}(\sqrt{n} - (-1)^{n})}$ -- расходится как сумма сходящегося и расходящегося ряда.
        
    \defitem{Покажите, что для любых числовых последовательностей $\{a_n\}$, $\{B_n\}$ справедлива формула суммирования по частям: $\sum_{n=m+1}^{N} a_n(B_n - B_{n-1}) = (a_NB_N-a_mB_m) - \sum_{n=m+1}^{N} (a_n - a_{n - 1}) B_{n-1}$.}
        
        Суммируем равенство по индексу $n$: $\sum_{n=m+1}^{N}$. $\sum_{n=m+1}^{N} a_n(B_n - B_{n-1}) = \sum_{n=m+1}^{N} (a_nB_n - a_{n-1}B_{n-1}) - \sum_{n=m+1}^{N} (a_n - a_{n-1})B_{n-1}$. Получаем из первой скобки путём сокращения элементов $a_N B_N - a_m B_m$. В итоге получаем $\sum_{n=m+1}^{N} a_n(B_n - B_{n-1}) = (a_NB_N-a_mB_m) - \sum_{n=m+1}^{N} (a_n - a_{n - 1}) B_{n-1}$.
        
    \defitem{Сформулируйте признак Дирихле. Приведите пример его применения.}
        
        $\sum_{n=1}^{\infty}a_n \cdot b_n$

        \begin{theorem*}
            Признак Дирихле. Если $a_n \downarrow 0$, а частичные суммы $\left| \sum_{n=1}^N b_n \right| \leq C$ ограничены,
            то $\sum_{n=1}^{\infty}a_n \cdot b_n$ сходится.
        \end{theorem*}

        \begin{example}
            $\sum_{n=1}^{\infty} \dfrac{\sin(nx)}{n^p}$, $p > 0$
        
            $a_n = \dfrac{1}{n^p} \downarrow 0$, $b_n = \sin nx$
        
            $b_1 + b_2 + b_3 + \dots + b_N = \sin x+ \sin 2x + \dots + \sin Nx = \dfrac{\cos \dfrac{x}{2} - \cos\left((N + 1/2)x\right)}{2 \sin \dfrac{x}{2}}$; $\left|\sum_{n=1}^{N}b_n\right| \leq \dfrac{2}{2\sin{\dfrac{x}{2}}} = \dfrac{1}{\sin{\dfrac{x}{2}}}$
        
            Ряд сходится по признаку Дирихле
        \end{example}

    \defitem{Сформулируйте признак Абеля. Выведите утверждение признака Абеля из признака Дирихле.}
        
        \begin{theorem*}
            Признак Абеля. Если $a_n$ монотонна и ограничена, а ряд $\sum_{n=1}^{\infty}b_n$ сходится,
            то $\sum_{n=1}^{\infty}a_n \cdot b_n$ сходится.
        \end{theorem*}
        
        % $a_n \to a$, $a_n = a +- \alpha_n$, $\alpha_n \downarrow 0$; $\sum_{n=1}^{\infty}a_n \cdot b_n = a \sum_{n=1}^{\infty}b_n +- \sum_{n=1}^{\infty}\alpha_n \cdot b_n$

    \defitem{Что такое перестановка членов ряда? Приведите пример.}
        
        Пусть $f: \NN \to \NN$ -- биекция

        Говорят, что ряд $\sum b_n$ получен из $\sum a_n$ перестановкой членов, если $b_n = a_{f(n)}$

    \defitem{Сформулируйте свойство абсолютно сходящегося ряда, связанное с перестановкой членов.}

        \begin{theorem*}
            Если ряд $\sum a_n$ сходится абсолютно, то $\forall$ ряд, полученный из него перестановкой членов, сходится абсолютно к той же сумме.
        \end{theorem*}

    \defitem{Сформулируйте свойство условно сходящегося ряда, связанное с перестановкой членов (теорема Римана).}
        
        \begin{theorem*}
            (Римана) Если ряд $\sum a_n$ сходится условно, то для $\forall S \in [-\infty; +\infty]$ то $\exists$ перестановка $f$ такая, что $\sum a_{f(n)} = S$
        \end{theorem*}

    \defitem{Приведите пример условно сходящегося ряда и перестановки, меняющей его сумму (с обоснованием).}

    \defitem{Как определяется произведение рядов? Что можно утверждать о произведении абсолютно сходящихся рядов?}
        
        $\sum_{k=1}^{\infty} a_k$, $\sum_{m=1}^{\infty} b_m$

        $\left(\sum_{k=1}^{K} a_k\right) \cdot \left(\sum_{m=1}^{M} b_m \right) = \sum_{1 \leq k \leq K, 1 \leq m \leq M} a_k \cdot b_m$

        Если эта сумма имеет предел при $K, M \to \infty$, не зависящий от порядка суммирования, то говорят, что определено произведение рядов.

        \begin{theorem*}
        (Коши) Если $\sum a_k$, $\sum b_m$ сходятся абсолютно, то определено их произведение.

        $\left(\sum_{k=1}^{\infty} a_k\right) \cdot \left(\sum_{m=1}^{\infty} b_m \right) = \sum_{n=1}^{\infty} a_{k_n} \cdot b_{m_n}$
        \end{theorem*}

    \defitem{Что такое произведение рядов в форме Коши? Приведите пример вычисления такого произведения.}
        
        Произведение рядов по Коши:

        $c_2 = a_1 \cdot b_1$

        $c_3 = a_2 \cdot b_1 + a_1 \cdot b_2$

        $c_4 = a_3 \cdot b_1 + a_2 \cdot b_2 + a_1 \cdot b_3$

        $\dots$

        $$\left(\sum_{k=1}^{\infty} a_k\right) \cdot \left(\sum_{m=1}^{\infty} b_m \right) = \sum_{n=2}^{\infty} c_n$$

    \defitem{Дайте определения: бесконечное произведение, частичное произведение, сходящееся бесконечное произведение, расходящееся бесконечное произведение.}
        
        $\prod_{n=1}^{N} a_n = a_1 \cdot a_2 \cdot \dots \cdot a_N$ -- частичное произведение.

        Бесконечным произведением называют формальную запись $\prod_{n=1}^{\infty} a_n$

        Значением бесконечного произведения является предел частичного произведения:

        $\prod_{n=1}^{\infty} a_n = \lim_{N \to \infty} \prod_{n=1}^{N} a_n$

        Если предел существует и он конечен -- то бесконечное произведение сходится, иначе расходится.

    \defitem{Сформулируйте и докажите необходимое условие сходимости бесконечного произведения.}

        Если $P_N = \prod_{n=1}^N a_n$ сходится, то $a_n = \frac{P_n}{P_{n - 1}} \to 1$

    \defitem{Пусть последовательности $\{a_n\}$, $\{A_n\}$, $A_n \neq 0$ таковы, что $a_n = \frac{A_n}{A_{n - 1}} \cdot c_n$ и бесконечное произведение $\prod c_n$ сходится. Докажите, что существует число $C \neq 0$, что $\prod_{n = 1}^N a_n = A_N \cdot (C + \operatorname{o}(1))$.}

    \defitem{Как определяется соответствующий бесконечному произведению ряд? Сформулируйте и докажите утверждение об их взаимосвязи.}
        
        $\prod_{n=1}^{N} a_n = e^{\ln \prod_{n=1}^{N} a_n} = e^{\sum_{n=1}^{N} \ln a_n}$

        $\prod_{n=1}^{\infty} a_n = P \iff \sum_{n=1}^{\infty} \ln a_n = \ln P$ $(P \neq 0, a_n \to 1)$

    \defitem{В каком случае бесконечное произведение называется сходящимся абсолютно? Сформулируйте и докажите критерий абсолютной сходимости бесконечного произведения.}
        
        $\prod_{n=1}^{\infty} a_n$ называется абсолютно сходящимся, если абсолютно сходится соответствующий ему ряд $\sum_{n=1}^{\infty} \ln a_n$

        \begin{comment}
            $\prod_{n=1}^{\infty} a_n$ сходится абсолютно $\iff \sum_{n=1}^{\infty} (a_n - 1)$ сходится абсолютно.
        \end{comment}

    \defitem{Напишите произведение Валлиса и его значение. Вычисление каких интегралов приводит к этой формуле?}

        \begin{example}
            (Произведение Валлиса)
            $\prod_{n=1}^{\infty} \frac{4n^2}{4n^2 - 1} = \frac{\pi}{2}$ -- получается из анализа интегралов $\int_{0}^{\frac{\pi}{2}} \sin^n x dx$
            
            Прим. ред.: есть отличное \href{https://www.youtube.com/watch?v=8GPy_UMV-08}{видео} с интуитивно понятным доказательством.
        \end{example}

    \defitem{Дайте определение дзета-функции ($\zeta$-функция) Римана. Сформулируйте тождество Эйлера для $\zeta$-функции.}

        \begin{example}
            (Дзета-функция Римана) $\zeta(s) = \sum_{n=1}^{\infty} \frac{1}{n^s}, s > 1$
        
            Тождество Эйлера:
        
            $\zeta(s) = \dfrac{1}{\prod_{n=1}^{\infty}(1 - \frac{1}{p_n^s})}$, где $p_1 = 2, p_2 = 3, p_3 = 5, \dots$
        \end{example}
    
	    \defitem{Дайте определения: функциональная последовательность, точка сходимости функциональной последовательности, область (множество) сходимости функциональной последовательности, поточечная сходимость функциональной последовательности на данном множестве.}
	    
	    \begin{definition*}
	    	Функциональным рядом (последовательностью) называется такой ряд (последовательность), что его элементами являются не числа, а функции $f_n(x)$.
	   	\end{definition*}
   	
   		\begin{definition*}
   			Пусть $\forall n, n \in \mathbb{N}, f_n: D \rightarrow \mathbb{R}, D \subseteq \mathbb{R}$
   			
   			Говорят, что $a \in D$ - точка сходимости $\{f_n(x)\}$, если последовательность $\{f_n(a)\}$ сходится.
   		\end{definition*}
   	
   		\begin{definition*}
   			Множество всех точек сходимости называется множеством сходимости.
   		\end{definition*}
   	
   		\begin{definition*}
   			Говорят, что последовательность сходится на $D$ поточечно, если $D$ – множество сходимости.
   		\end{definition*}

		\defitem{Что такое равномерная норма? Покажите (исходя из определения нормы), что равномерная норма является нормой в соответствующем линейном пространстве (всех числовых функций, определённых на заданном множестве).}
		
		\begin{definition*}
			Рассмотрим множество всех функций $f: D \rightarrow \mathbb{R}$. $||f||=\sup_{x \in D} |f(x)|$ - равномерная норма в пространстве $D$. 
		\end{definition*}
		
		\defitem{Сформулируйте определения равномерной сходимости функциональной последовательности: в терминах нормы и на языке $\epsilon-\delta$.}
		
		\begin{definition*}
			1) $f_n \overset{D}{\rightrightarrows} f \iff ||f_n - f|| \rightarrow 0$.
			
			2) $\sum f_n(x) \rightrightarrows S(x) \iff \forall \epsilon > 0, \exists N(\epsilon): \forall n \geqslant N(\epsilon), |S_n(x) - S(x)| < \epsilon$.
		\end{definition*}
	
		\defitem{Докажите, что из равномерной сходимости следует поточечная сходимость на данном множестве.}
		
		\begin{proof}
			Рассмотрим определения поточечной сходимости и равномерной сходимости:
			
			$\forall x \in E, \forall \epsilon > 0 \exists N = N(\epsilon, x): \forall n \geqslant N, |f_n(x) - f(x)| < \epsilon$ - поточечная сходимость.
			
			
			$\forall \epsilon > 0 \exists N = N(\epsilon, x): \forall n \geqslant N, \forall x \in E ,|f_n(x) - f(x)| < \epsilon$ - равномерная сходимость.
			
			Заметим, что свойство равномерной сходимости не слабее, чем поточечной, а значит, из равномерной сходимости следует поточечная.
		\end{proof}
        
    \defitem{Приведите пример функциональной последовательности, сходящейся поточечно, но не сходящейся равномерно (с обоснованием).}
        
        Рассмотрим последовательность функций вида $f_n(x) = arctg(nx)$. Поточечная сходимость в данном случае обусловлена тем, что при $n \rightarrow \infty$ $arctg(nx) \rightarrow 0$, если $x = 0$ и $|arctg(nx)| \rightarrow \dfrac{\pi}{2}$, если $|x| > 0$. Но при этом, так как при $n \rightarrow \infty$ на точке $x_0 = 0$ происходит разрыв, то функциональная последовательность $f_n(x) = arctg(nx)$ не сходится равномерно.
        
    \defitem{-}

    \defitem{-}

    \defitem{-}

    \defitem{-}

    \defitem{-}

    \defitem{-}

    \defitem{-}

    \defitem{-}

    \defitem{Сформулируйте теорему Дини о монотонной сходимости. Приведите пример её применения для доказательства равномерной сходимости функциональной последовательности (с обоснованием).}
        
        \begin{theorem*}
            Пусть $f_n: [a, b] \to \RR, f_n(x)$ монотонна по $n$ при каждом $x \in [a, b]$, $f_n \to f$ на $[a, b]$

            Тогда $f_n \overset{D}{\rightrightarrows} f$
        \end{theorem*}

        \begin{example}
            См. №16-19 листка №4
        \end{example}

    \defitem{Сформулируйте и докажите теорему о почленном переходе к пределу в функциональной последовательности.}
        
        \begin{theorem*}
        	Если функциональная последовательность ${f_n(x)}$ сходится равномерно на множестве ${x}$ к предельной функции $f(x)$ и все элементы этой последовательности имеют предел в точке $х_0$, то и предельная функция  имеет предел в точке $х_0$, причём $\lim_{x \rightarrow x_0} f(x) = \lim_{x \rightarrow x_0}(\lim_{n \rightarrow \infty} f_n(x)) = \lim_{n \rightarrow \infty}(\lim_{x \rightarrow x_0} f_n(x))$, т.e. символ $\lim_{n \rightarrow \infty}$ предела последовательности и символ $\lim_{x \rightarrow x_0}$ предела функции можно переставлять местами (или, как говорят, к пределу при $x \rightarrow x_0$ можно переходить почленно).
        \end{theorem*}
    
    \defitem{Покажите на примере как доказать неравномерность сходимости функциональной последовательности с помощью локализации особенности (с обоснованием).}
        
        См. №20-25 листка №4

    \defitem{-}

    \defitem{-}

    \defitem{-}

    \defitem{Как определяются множества абсолютной и условной сходимости функционального ряда? Как они связаны с множеством сходимости?}
        
        \begin{definition*}
            Множество абсолютной сходимости -- множество всех тех значений $x$, при которых ряд сходится абсолютно.
        \end{definition*}

        \begin{definition*}
            Множество условной сходимости -- множество всех тех значений $x$, при которых ряд сходится условно.
        \end{definition*}

        Объединение множеств абсолютной сходимости и условной сходимости образует множество сходится.
        
    \defitem{Дайте определение равномерной сходимости функционального ряда.}

    \defitem{Сформулируйте и докажите необходимое условие равномерной сходимости функционального ряда.}
        
        \begin{theorem*}
            Если $\sum_{n=1}^{\infty} a_n(x)$ равномерно сходится к сумме $S(x)$, то $a_n \overset{D}{\rightrightarrows} 0$
        \end{theorem*}

        \begin{proof}
        $S_n(x) = a_1(x) + \dots + a_n(x)$, $a_n(x) = S_n(x) - S_{n-1}(x)$

        $S_n \overset{D}{\rightrightarrows} S \implies a_n \overset{D}{\rightrightarrows} (S - S) = 0$
        \end{proof}

    \defitem{Сформулируйте критерий Коши равномерной сходимости функционального ряда.}

        \begin{theorem*}
            Функциональный ряд $\sum_{n=1}^{\infty} a_n(x)$ сходится равномерно на $D \iff$ $\forall \varepsilon > 0$ $\exists N(\varepsilon)$, $\forall n \geq N$, $\forall m$: 
            $$||a_n + a_{n + 1} + \dots + a_{n + m}|| < \varepsilon$$
            
            Т.е. $|a_n(x) + a_{n + 1}(x) + \dots + a_{n + m}(x)| < \varepsilon$ $\forall x \in D$.
        \end{theorem*}

    \defitem{Сформулируйте следствие критерия Коши -- достаточное условие того, что функциональный ряд не является сходящимся равномерно.}
        
        \begin{corollary}
            (Отрицание критерия Коши) Если $\exists \{x_n\} \subset D$, $\exists \{m_n\} \in \NN$, $\exists \varepsilon_0$:
            $$|a_n(x_n) + a_{n + 1}(x_n) + \dots + a_{m_n}(x_n)| > \varepsilon_0$$            
        \end{corollary}

    \defitem{Приведите пример функционального ряда, сходящегося на некотором множестве поточечно, но не равномерно (с обоснованием).}

    \defitem{Сформулируйте мажорантный признак Вейерштрасса абсолютной и равномерной сходимости функционального ряда.}
        
        \begin{theorem*}
            (Признак Вейерштрасса) Если $|a_n(x)| \leq b_n$ при $\forall n \geq n_0$, $\forall x \in D$, а ряд $\sum b_n$ сходится, то $\sum a_n(x)$ сходится на $D$ абсолютно и равномерно.
        \end{theorem*}

    \defitem{Как применяются признаки Даламбера и Коши (радикальный) для исследования сходимости функционального ряда?}
        
        \begin{theorem*}
            (Признак Даламбера) Если $\exists q < 1$: $|a_{n+1}(x)| \leq q \cdot |a_n(x)|$ при $\forall n \geq n_0$, $\forall x \in D$, причём $a_{n_0}(x)$ -- ограничена на $D$, то $\sum a_n(x)$ сходится на $D$ абсолютно и равномерно.
        \end{theorem*}
            
        \begin{example}
            $\sum_{n = 0}^{\infty} \dfrac{x^n}{n!}$, $D=[-r; r]$, $r > 0$
            
            $\left|\dfrac{x^{n + 1}}{(n + 1)!}\right| \leq q \cdot \left|\dfrac{x^n}{n!}\right|$
            
            $\left|\dfrac{x}{n + 1}\right| \leq q$. Пусть $n_0: \dfrac{r}{n_0 + 1} < 1$, берём $q = \dfrac{r}{n_0 + 1}$. Значит, ряд абсолютно и равномерно сходится.
        \end{example}

    \defitem{Сформулируйте неравенство для остатка знакочередуегося функционального ряда (и условия его применимости).}

    \defitem{Сформулируйте признак Лейбница равномерной сходимости знакочередующегося функционального ряда.}
        
        \begin{theorem*}
            (Признак Лейбница) Если $u_n(x) \downarrow_{(n)}$ и $u_n \overset{D}{\rightrightarrows} 0$, то ряд сходится равномерно.
        \end{theorem*}

    \defitem{Сформулируйте признак Дирихле равномерной сходимости функционального ряда.}

        \begin{theorem*}
            (Признак Дирихле) Если $a_n(x) \downarrow_{(n)}$ и $a_n \overset{D}{\rightrightarrows} 0$, а $||b_1 + \dots + b_n|| \leq C$ $\forall n$, то ряд равномерно сходится на $D$.
        \end{theorem*}

    \defitem{Сформулируйте признак Абеля равномерной сходимости функционального ряда.}

        \begin{theorem*}
            (Признак Абеля) Если $a_n(x)$ монотонна по $n$ (при $\forall x \in D$), и $||a_n|| \leq C$ при всех $n$, а ряд $\sum b_n(x)$ сходится равномерно, то ряд $\sum_{n=1}^{\infty} a_n(x) \cdot b_n(x)$ сходится равномерно.
        \end{theorem*}

    \defitem{Сформулируйте теорему о почленном переходе к пределу в функциональном ряде.}

        \begin{theorem*}
            $-\infty \leq a < b \leq +\infty$, $D= (a; b)$, $D = [a; b]$

            Пусть функциональный ряд $\sum_{n=1}^{\infty} c_n(x)$ сходится равномерно на $D$, $x_0 \in D$, $\exists \lim_{x \to x_0} c_n(x) = y_n$ и $\exists \sum_{n=1}^{\infty} y_n = y$.

            Тогда $\lim_{x \to x_0} \sum_{n = 1}^{\infty} c_n(x) = \sum_{n=1}^{\infty} \lim_{x \to x_0} c_n(x) = \sum_{n=1}^{\infty} y_n = y$
        \end{theorem*}
        
    \defitem{Сформулируйте теорему о почленном дифференцировании функционального ряда.}

        \begin{theorem*}
            $-\infty \leq a < b \leq +\infty$, $D= (a; b)$, $D = [a; b]$

            Пусть $c_n(x)$ дифференцируемы на $D$ и $\sum_{n=1}^{\infty} c'_n(x)$ сходится равномерно на $D$.

            Тогда ряд $\sum_{n=1}^{\infty} c_n(x)$ сходится на $D$ (а если $D$ огр, то сходится равномерно), а его сумма будет дифференцируемой функцией на $D$ и $\left(\sum_{n=1}^{\infty} c_n(x)\right)' = \sum_{n=1}^{\infty} c'_n(x)$
        \end{theorem*}

    \defitem{Сформулируйте теорему о почленном интегрировании функционального ряда.}

        \begin{theorem*}
            $-\infty < a < b < +\infty$, $D= (a; b)$, $D = [a; b]$

            $\int_{a}^{x}\left(\sum_{n=1}^{\infty} c_n(t)\right) dt = \sum_{n=1}^{\infty} \int_{a}^{x} c_n(t) dt$ -- сходится равномерно на $D$.
        \end{theorem*}

    \defitem{Что такое степенной ряд? Как определяются радиус и интервал сходимости степенного ряда? Что можно утверждать о характере сходимости ряда на интервале сходимости?}

        \begin{definition*}
            Степенной ряд -- $\sum_{n=0}^{\infty} c_n \cdot(x - x_0)^n$
        \end{definition*}

        \begin{definition*}
            Пусть:

            $R_{cv} = \sup \{|x - x_0|: \text{ряд сходится}\}$

            $R_{dv} = \inf \{|x - x_0|: \text{ряд расходится}\} \text{ или} +\infty \text{, если ряд сходится всюду}$

            $\exists R = R_{cv} = R{dv}$ -- радиус сходимости.
        \end{definition*}

        \begin{theorem*}
            На интервале сходимости степенной ряд сходится абсолютно.
        \end{theorem*}

    \defitem{Что можно утверждать про равномерную сходимость степенного ряда?}

        \begin{theorem*}
            Если $R > 0$, то степенной ряд сходится равномерно при $|x - x_0| \leq r$, где $r < R$ (доказательство через признак Вейерштрасса).
        \end{theorem*}

    \defitem{Сформулируйте и докажите теорему Абеля о сходимости степенного ряда.}

        \begin{theorem*}
            (Абеля)
            \begin{enumerate}
                \item Если степенной ряд сходится в точке $x_1 \neq x_0$, то он сходится при всеx $x : |x - x_0| < |x_1 - x_0|$
                \item Если степенной ряд расходится в точке $x_2 \neq x_0$, то он расходится при всеx $x : |x - x_0| > |x_2 - x_0|$
            \end{enumerate}
        \end{theorem*}
            
        \begin{proof}
            \begin{enumerate}
                \item $\left|\sum_{n = m}^{N} c_n (x - x_0)^n\right| = \left|\sum_{n = m}^{N} c_n \cdot (x_1 - x_0)^n \cdot \left(\frac{x-x_0}{x_1 - x_0}\right)^n\right| \leq 
                \sum_{n = m}^{N} \left|c_n \cdot (x_1 - x_0)^n \right| \cdot \left|\left(\frac{x-x_0}{x_1 - x_0}\right)^n\right| \leq \varepsilon (q^m + \dots + q^N) \leq \varepsilon \cdot q^m \cdot \frac{1}{1-q} \to 0$
            \end{enumerate}
        \end{proof}

    \defitem{Докажите, что если степенной ряд $\sum c_n \cdot (x - x_n)^n$ расходится в точке $x_1$, то он расходится во всех точках $x$, для которых $|x - x_0| > |x_1 - x_0|$.}

    \defitem{Выведите формулу Коши-Адамара для радиуса сходимости степенного ряда.}

    \defitem{-}

    \defitem{-}

    \defitem{-}

    \defitem{-}

    \defitem{Что можно утверждать о радиусе сходимости степенного ряда, полученного почленным дифференцированием исходного ряда?}

        Радиус сходимости при дифференцировании степенного ряда не изменяется.

    \defitem{Что можно утверждать о радиусе сходимости степенного ряда, полученного почленным интегрировании исходного ряда?}

        Радиус сходимости при интегрировании степенного ряда не изменяется.

    \defitem{Сформулируйте и докажите теорему о почленном дифференцировании степенного ряда.}

        \begin{theorem*}
            $\left(\sum_{n=0}^{\infty}c_n\cdot (x - x_0)^n\right)' = \sum_{n=0}^{\infty} c_n \cdot n \cdot (x - x_0)^{n - 1} = \sum_{n=0}^{\infty} c_{n+1}(n+1)(x - x_0)^n$
        \end{theorem*}

    \defitem{Сформулируйте и докажите теорему о почленном интегрировании степенного ряда.}

        \begin{theorem*}
            $\int_{x_0}^{x}\left(\sum_{n=0}^{\infty}c_n(t - x_0)^n\right) dt = \sum_{n=0}^{\infty} \frac{c_n}{n+1} (x - x_0)^{n + 1}$
        \end{theorem*}

    \defitem{Запишите формулу Тейлора для бесконечно дифференцируемой функции с остаточным членом в формах Лагранжа и Коши.}

        Если функция $f(x)$ бесконечно дифференцируема в точке $x_0$, то функции $f(x)$ можно сопоставить её ряд Тейлора:

        $\sum_{n=0}^{\infty} \frac{f^{(n)}(x_0)}{n!}(x - x_0)^n$

        При этом $f(x) = \sum_{n=0}^{N} \frac{f^{(n)}(x_0)}{n!}(x - x_0)^n + r_N(x)$

        $r_N(x) = \frac{f^{(N+1)}(x_0+\theta)(x - x_0)}{(N+1)!}(x - x_0)^{N+1}$, $\theta \in (0; 1)$ -- формула Лагранжа

        $r_N(x) = \frac{f^{(N+1)}(x_0+\theta)(x - x_0)}{N!}(1-\theta)^N(x - x_0)^{N+1}$, $\theta \in (0; 1)$ -- формула Коши

    \defitem{Сформулируйте и докажите утверждение о единственности разложения функции в степенной ряд.}

    \defitem{Что такое функция, аналитическая в данной точке? Каково соотношение между понятиями бесконечно дифференцируемости и аналитичности?}

        Функция называется аналитической в т.$x_0$, если она представима в окрестности этой точки в виде степенного ряда. Не всякая бесконечно дифференцируемая функция будет аналитической.

    \defitem{Приведите пример бесконечно дифференцируемой функции, не являющейся аналитической.}

        \begin{example}

            $f(x) = \begin{cases}
                  e^{-\frac{1}{x^2}}, x \neq 0 \\
                  0, x = 0
            \end{cases}$
            
            $f(0) = f'(0) = f''(0) = \dots = 0$, ряд Тейлора при $x_0 = 0$ равен 0
        \end{example}

    \defitem{Запишите разложения в степенной ряд с центром в нуле для функций $e^x$, $\sin x$, $\cos x$. Каково множество сходимости ряда? На каком множестве сумма ряда представляет собой исходную функцию?}

        $e^x = \sum_{k=0}^{\infty} \dfrac{x^k}{k!}$

        $\sin x = \sum_{k=0}^{\infty} (-1)^{k} \dfrac{x^{2k+1}}{(2k+1)!}$

        $\cos x = \sum_{k=0}^{\infty} (-1)^{k} \dfrac{x^{2k}}{(2k)!}$

    \defitem{Запишите разложение в степенной ряд с центром в нуле для функции $(1+x)^p$. Каково множество сходимости ряда? На каком множестве сумма ряда представляет собой исходную функцию?}

        $(1+x)^p = \sum_{n=0}^{\infty} \dfrac{(p)_n}{n!}x^n$, где $(p)_n = p(p-1)\dots(p - n + 1), R = 1$

    \defitem{Получите разложение для $\ln (1 + x)$ интегрирование раложения для $\frac{1}{1 + x}$. Каково множество сходимости ряда? На каком множестве сумма ряда представляет собой исходную функцию? Обоснуйте ответ.}

    \end{colloq}

\end{document}