\documentclass[a4paper]{article}
\usepackage{header}

% Use \begin{theorem*} instead of \begin{theorem}.
% Use \iff instead of \Leftrightarrow.

\title{\Huge Математический Анализ - 2 - Коллоквиум 1}
\author{
    Серёжа Рахманов | \href{https://t.me/virg1n}{telegram}, \href{http://shoraii.github.io}{website}
    \\
    Денис Болонин | \href{https://t.me/ultrakekul}{telegram}

}
\date{Версия от {\ddmmyyyydate\today} \currenttime}

\begin{document}
    \maketitle

    \begin{enumerate}
        \item Дайте определения: числовой ряд, частичная сумма ряда, сумма ряда, сходящийся ряд, расходящийся ряд. Рассмотрим ряд с общим членом $a_n$. Докажите, что $a_n \to 0$.
        \begin{definition*}
        Пусть $a_{n}$ -- последовательность, т.е. $\NN \to \RR$. Формальная бесконечная сумма: $a_1 + a_2 + a_3 + \dots = \sum_{n=1}^{\infty} a_n$ называется рядом.
        $S_N = \sum_{n = 1}^{N} a_n$ -- частичная сумма, сумма ряда: $S = \lim_{N \to \infty} S_N$
        \end{definition*}
        
        Возможны 3 случая:
        \begin{enumerate}
            \item $\exists S \in \RR$
            \item $\exists S = \infty$
            \item $\nexists S$
        \end{enumerate}
        
        В первом случае говорят, что ряд сходится, иначе -- что ряд расходится.

        \begin{comment}~
            Если ряд сходится, то $a_n \to 0$
        \end{comment}
        \begin{proof}
            $a_n = S_n - S_{n - 1} \to 0$, т.к. $S_n \to S$ и $S_{n - 1} \to S$
        \end{proof}
        \item Сформулируйте критерий Коши сходимости числовой последовательности. Сформулируйте и докажите критерий Коши сходимости числового ряда.
        \begin{definition*}
            ${a_n}$ называется фундаментальной, если $\forall \epsilon > 0$  $\exists N: \forall n > m > N, |S_n - S_m| < \epsilon$
        \end{definition*}
        \begin{theorem*}
            ${S_n}$ -- сходится $\iff {S_n}$ -- фундаментальная
        \end{theorem*}
        \begin{proof}
        $S_n - S_m = \sum_{k = m + 1}^{n} a_{k}$
        Тогда $\sum a_n$ -- сходится $\iff$ $\forall \epsilon > 0$  $\exists N: \forall n > m > N$
        $|a_{m + 1} + a_{m + 2} + \dots + a_{n}| < \epsilon$
        \end{proof}
        \item Сформулируйте и докажите признак сравнения положительных числовых рядов, основанный на неравенстве $a_n \leq b_n$.
        
        $a_n \leq b_n$ при всех $n \geq n_0$
	
	    Ряд $\sum b_n$ cходится $\implies$ ряд $\sum a_n$ сходится
	
        Ряд $\sum a_n$ расходится $\implies$ ряд $\sum b_n$ расходится
        
        \begin{proof}
        	На основании того, что отбрасывание конечного числа элементов ряда не отражается на его поведении, мы можем считать, что $a_n \leq b_n$ при всех $n = 1, 2, 3, \dots$ Обозначив частные суммы через $A$ и $B$ соответственно, имеем $A_n \leq B_n$. Пусть ряд $\sum b_n$ сходится, тогда $B_n$ ограничена, $B_n \leq S (S = const, n = 1, 2, 3, \cdots)$. В таком случае $A_n$ также меньше либо равна некоторому $S$, что даёт нам ограниченность $\sum a_n$.
        \end{proof}
    
        \item Сформулируйте и докажите признак сравнения положительных числовых рядов, основанный на неравенстве $\frac{a_{n+1}}{a_n} \leq \frac{b_{n+1}}{b_n}$.
        
        Ряд $\sum b_n$ cходится $\implies$ ряд $\sum a_n$ сходится
	
        Ряд $\sum a_n$ расходится $\implies$ ряд $\sum b_n$ расходится
        
        \begin{proof}~
        
        $a_{n_0+1} \leq \frac{a_{n_0}}{b_{n_0}}\cdot b_{n_0 + 1}$
        
        $a_{n_0+2} \leq \frac{a_{n_0 + 1}}{b_{n_0 + 1}}\cdot b_{n_0 + 2} \leq \frac{a_{n_0}}{b_{n_0}}\cdot b_{n_0 + 2}$
        
        $\vdots$
        
        $a_{n_0+k} \leq \frac{a_{n_0}}{b_{n_0}}\cdot b_{n_0 + k} \implies \sum_{n=n_0}^{N} a_n \leq \frac{a_{n_0}}{b_{n_0}}\cdot \sum_{n=n_0}^{N} b_n$
        \end{proof}

        \item Сформулируйте и докажите признак сравнения положительных числовых рядов, основанный на пределе $\lim \frac{a_n}{b_n}$.
        
        $\lim_{n \to \infty} \frac{a_n}{b_n} \in (0; +\infty) \implies$ сходимость $\sum a_n \iff$ сходимость $\sum b_n$
        
        \begin{proof}~
            
        $c = \lim_{n \to \infty} \frac{a_n}{b_n} > 0$
        
        $\forall \epsilon\ \exists n_0:\ c - \epsilon \leq \frac{a_n}{b_n} \leq c + \epsilon$, при $n \geq n_0$
        
        Возьмём $c - \epsilon > 0 \implies (c - \epsilon)\cdot b_n \leq a_n \leq (c + \epsilon)\cdot b_n$
        
        Сходимость следует из правой части неравенства, а расходимость из левой. 
        \end{proof}
        \item Пусть последовательности $\{a_n\}$, $\{A_n\}$ таковы, что $a_n - (A_n - A_{n - 1}) = c_n$ и ряд $\sum c_n$ сходится.
        Докажите, что существует $C$ такое, что $a_1 + a_2 + \dots + a_n = A_n + C + o(1)$. 
        \item Сформулируйте и докажите признак Лобачевского-Коши.
        \begin{proposal}
            Пусть $a_n > 0$ и $a_n \downarrow$
        
            Тогда ряды $\sum a_n$ и $\sum 2^n \cdot a_{2^n}$ ведут себя одинаково
        \end{proposal}
        \begin{proof}
            $a_1 + (a_2) + (a_3 + a_4) + (a_5 + \dots + a_8) + \dots$
        
            $a_2 \leq a_1$
            
            $a_2 \leq a_2$
        
            
            $a_3 + a_4 \leq 2a_2$
            
            $a_3 + a_4 \geq 2a_4$
        
            $a_5 + \dots + a_8 \leq 4a_4$
        
            $a_5 + \dots + a_8 \geq 4a_8$
        
        
            $\dots$
        
            $a_1 + \sum_{n=0}^{m - 1} 2^n a_{2n} \leq \sum_{n = 1}^{2^m} a_n \leq a_1 + \dfrac{1}{2} \sum_{n=0}^{m} 2^n a_{2n}$
        
        \end{proof}
        \item Сформулируйте теорему Штольца о пределе последовательности. Покажите на примере, как с помощью теоремы Штольца можно уточнить асимптотическую оценку для частичной суммы ряда. $\frac{p_n}{q_n}$, $p_n$, $q_n \to 0$.
        \begin{theorem*}
            (Штольца.) Если $p_n, q_n \to 0, q_n \downarrow$ и $\exists lim \dfrac{p_{n + 1} - p_n}{q_{n + 1} - q_n}$, то
            $\lim \dfrac{p_n}{q_n} = \lim \dfrac{p_{n + 1} - p_n}{q_{n + 1} - q_n}$
        \end{theorem*}
        \item Пусть $\sum a_n$, $\sum a_n'$ - сходящиеся положительные ряды. Говорят, что ряд $\sum a_n'$ сходится быстрее ряда $\sum a_n$, если $a_n' = o(a_n)$. Докажите, что в этом случае также $r_n' = o(r_n)$, где $r_n$, $r_n'$ - остатки соответствующих рядов.
        
        Рассмотрим остатки каждого из рядов. $r_n = S - S_N$, где $S_N$ - частичная сумма ряда $\sum a_n$ и $S_N \to S$ при $N \to \infty$. Для $\sum a_n'$ аналогично $r_n' = S' - S_N'$, где $S_N'$ - частичная сумма ряда $\sum a_n'$ и $S_N' \to S'$ при $N \to \infty$. Идёт речь о том, что ряд $a_n'$ сходится быстрее ряда $a_n$, т.е. оба ряда сходятся и $S = S'$. Но, поскольку члены рядов находятся в отношении $a_n' = o(a_n)$, то мы можем сделать выводы о частичных суммах $S_N$ и $S_N'$. $\forall N, S_N' = o(S_N)$, что указывает нам в результате на отношение между остатками $r_n' = o(r_n)$.
        \item Пусть $\sum a_n$, $\sum a_n'$ - расходящиеся положительные ряды. Говорят, что ряд $\sum a_n'$ расходится медленнее ряда $\sum a_n$, если $a_n' = o(a_n)$. Докажите, что в этом случае также $S_n' = o(S_n)$, где $S_n$, $S_n'$ - частичные суммы соответствующих рядов.
        
        Оба ряда расходятся, тогда $S_n \to \infty$ и $S_n' \to \infty$ при $n \to \infty$. Мы понимаем, что $S_n = \sum_{n = 1}^{N} a_n$, $S_n' = \sum_{n = 1}^{N} a_n'$. Это значит, что для некоторого $n_1$ мы имеем следующее: $S_{n_1} = \sum_{n = 1}^{n_1} a_n$, $S_{n_1}' = \sum_{n = 1}^{n_1} a_n'$, где для любого $n = 1, 2, 3, \dots, n_1$ выполняется отношение $a_n' = o(a_n)$. В таком случае для частичных сумм справедливо отношение $S_{n_1}' = o(S_{n_1})$. А так как и для всех последующих $a_n$ и $a_n'$ также справедливо отношение $a_n' = o(a_n)$, то мы можем сказать, что $S_n' = o(S_n)$.
        \item Пусть положительный ряд $\sum a_n$ расходится и $S_n$ его частичная сумма. Докажите, что ряд $\sum (\sqrt{r_{n}} - \sqrt{r_{n+1}})$ также расходится, причём медленнее, чем ряд $\sum a_{n+1}$.
        
        $r_n = S - S_n$, тогда $\dfrac{\sqrt{r_{n}} - \sqrt{r_{n+1}}}{a_{n+1}} = \dfrac{\sqrt{S - S_n} - \sqrt{S - S_{n+1}}}{a_{n+1}} = \dfrac{S_{n+1} - S{n}}{(S_{n+1} - S_n)(\sqrt{S - S_n} + \sqrt{S - S_{n+1}})}$, где $(S_{n+1} - S_n)(\sqrt{S - S_n} + \sqrt{S - S_{n+1}}) \to \infty$, $\dfrac{S_{n+1} - S{n}}{(S_{n+1} - S_n)(\sqrt{S - S_n} + \sqrt{S - S_{n+1}})} \to 0$. Тогда ряд $\sum (\sqrt{r_{n}} - \sqrt{r_{n+1}})$ расходится, причём медленнее, чем ряд $\sum a_{n+1}$.
        
        \item Пусть положительный ряд $\sum a_n$ расходится и $S_n$ его частичная сумма. Докажите, что ряд $\sum (\sqrt{S_{n+1}} - \sqrt{S_n})$ также расходится, причём медленнее, чем ряд $\sum a_{n+1}$.
        
        $\dfrac{\sqrt{S_{n+1}} - \sqrt{S_n}}{a_{n+1}} = \dfrac{\sqrt{S_{n+1}} - \sqrt{S_n}}{S_{n+1} - S_n} = \dfrac{1}{\sqrt{S_{n+1}} + \sqrt{S_n}}$, где $\sqrt{S_{n+1}} + \sqrt{S_n} \to \infty$, а значит $\dfrac{1}{\sqrt{S_{n+1}} + \sqrt{S_n}} \to 0$. Тогда ряд $\sum (\sqrt{S_{n+1}} - \sqrt{S_n})$ расходится, причём медленнее, чем ряд $\sum a_{n+1}$.
        
        \item Сформулируйте признак Даламбера для положительного ряда
        \begin{theorem*}
            Признак Даламбера. Пусть $a_n > 0$.
            
            $\overline{\lim} \dfrac{a_{n+1}}{a_n} < 1 \implies $ ряд $\sum a_n$ сходится.
            
            $\underline{\lim} \dfrac{a_{n+1}}{a_n} > 1 \implies $ ряд $\sum a_n$ расходится.
        \end{theorem*}
        \item Сформулируйте радикальный признак Коши для положительного ряда.
        \begin{theorem*}
            Радикальный признак Коши. Пусть $a_n \geq 0$.
            
            $\overline{\lim} \sqrt[n]{a_n} < 1 \implies$ ряд $\sum a_n$ сходится.
             
            $\underline{\lim} \sqrt[n]{a_n} > 1 \implies$ ряд $\sum a_n$ расходится.
        \end{theorem*}
        \item Докажите, что всякий раз, когда признак Даламбера даёт ответ на вопрос о сходимости ряда, то радикальный признак Коши даёт (тот же) ответ на этот вопрос.
        
        Пусть $a_n > 0$. Тогда:

        $$ \underline{\lim} \dfrac{a_{n+1}}{a_n} \leq \underline{\lim}{\sqrt[n]{a_n}} \leq \overline{\lim}{\sqrt[n]{a_n}} \leq \overline{\lim}\dfrac{a_{n+1}}{a_n}$$

        Если $\overline{\lim}\dfrac{a_{n+1}}{a_n} < 1 \implies \overline{\lim}{\sqrt[n]{a_n}} < 1$

        Если $\underline{\lim}\dfrac{a_{n+1}}{a_n} > 1 \implies \underline{\lim} \sqrt[n]{a_n} > 1$

        Если $\exists \lim \frac{a_{n+1}}{a_n}$, то $\overline{\lim} \frac{a_{n+1}}{a_n} = \underline{\lim} \frac{a_{n+1}}{a_n} \implies \exists \lim \sqrt[n]{a_n} = \lim \frac{a_{n+1}}{a_n}$
        \item -
        \item -
        \item Приведите пример ряда, который сходится медленнее любого ряда геометрической прогрессии, но быстрее любого обобщённого гармонического яда (с обоснованием).
        
        Рассмотрим $e^{-\sqrt{n}}$
        
        $\sum q^n$ - ряд геометрической прогрессии, $0 < q < q$; $q^n = e^{n * \ln q}$, где $\ln q < 0$
        
        $\sum \dfrac{1}{n^p}$ - обобщённый гармонический ряд. $\dfrac{1}{n^p} = e^{-p \ln n}$, $p > 1$.
        
        $p \ln n < \sqrt{n} < n \ln \dfrac{1}{q}$, $\forall p, q$ при $n \geq n_0$. 
        
        $\dfrac{e^{-\sqrt{n}}}{q^n} = e^{-\sqrt{n} + n \ln \frac{1}{q}} \to +\infty$, где $-\sqrt{n} + n \ln \dfrac{1}{q} \to + \infty$ $\implies$ $\sum e^{-\sqrt{n}}$ сходится медленнее ряда геометрической прогрессии.
        
        $\dfrac{e^{-\sqrt{n}}}{1 / n^p} = e^{-\sqrt{n} + p \ln n} \to 0$, где $-\sqrt{n} + p \ln n \to -\infty$ $\implies$ $\sum e^{-\sqrt{n}}$ сходится быстрее гармонического ряда.
        
        \item Сформулируйте признак Гаусса для положительного ряда. Приведите пример применения признака Гаусса.

        Если $\exists \delta > 0,\; p$:$ \dfrac{a_{n+1}}{a_n} = 1 - \dfrac{p}{n} + O\left(\dfrac{1}{n^{1 + \delta}}\right) $
        то:

        $p > 1 \implies$ ряд $\sum a_n$ сходится.

        $p \leq 1 \implies$ ряд $\sum a_n$ расходится.
        \item -
        \item -
        \item Что такое улучшение сходимости положительного ряда? Покажите на примере как можно улучшить сходимость ряда.

        Пусть у нас есть некоторый ряд $\sum a_n$ и он сходится медленно. В таких случаях для расчёта суммы ряда с необходимой точностью потребуется взять больше членов, что неудобно. Мы можем преобразовать наш ряд для улучшения сходимости, т.е. получить некоторый ряд $\sum a_n'$, который будет сходиться быстрее, чем исходный $\sum a_n$.
       	\begin{example}
            Пусть у нас есть ряд $S = \sum_{n = 1}^{\infty} \dfrac{1}{n^2 + 2} \approx \sum_{n = 1}^{\infty} \dfrac{1}{n^2}$. Воспользуемся  методом Куммера. Для улучшения сходимости будем брать ряды вида $\sum_{n = 1}^{\infty} \dfrac{1}{n(n+1)} = 1, \sum_{n = 1}^{\infty} \dfrac{1}{n(n+1)(n+2)} = \dfrac{1}{4}, \dots$. 
            
            В данном случае нам подойдёт первый ряд в этом списке, поскольку $\dfrac{1}{n^2} \sim \dfrac{1}{n(n+1)}$.
            
			$\sum_{n=1}^{\infty} \left(\frac{1}{n^2+2} - \frac{1}{n(n+1)}\right) = S - 1 \implies S = 1 + \sum_{n=1}^{\infty} \left(\frac{1}{n^2 + 2} - \frac{1}{n(n+1)}\right)$
			
			$\frac{1}{n^2+2} - \frac{1}{n(n+1)} = \frac{1}{n^2} \cdot \left(\frac{1}{1 + \frac{2}{n^2}} - \frac{1}{1 + \frac{1}{n}}\right) = \frac{1}{n^2} \cdot \left(1 - \frac{2}{n^2} + o\left(\frac{1}{n^2}\right) - \left(1 - \frac{1}{n} + \frac{1}{n^2} + o\left(\frac{1}{n^2}\right)\right)\right) = \frac{1}{n^3} + o\left(\frac{1}{n^3}\right)$.
         
            Получили ряд $\sum_{n = 1}^{\infty} \dfrac{1}{n^3}$, который сходится быстрее, $1 + \sum_{n = 1}^{\infty} \dfrac{1}{n^3} \approx \sum_{n = 1}^{\infty} \dfrac{1}{n^2 + 2}$.
        \end{example}
    
   		\item Дайте определения: знакопеременный ряд, знакочередующийся ряд, абсолютно сходящийся ряд, условно сходящийся ряд, положительная часть ряда, отрицательная часть ряда.
    
    	\begin{definition*}
    		Пусть существует ряд $\sum a_n$. такой, что $\forall i$, $a_i$ может быть, как больше 0, так и меньше 0. В таком случае ряд $\sum a_n$ называется знакопеременным.
    	\end{definition*}
    		
    	\begin{definition*}
    		Пусть существует ряд $\sum a_n$. такой, что $\forall i$, $a_i \cdot a_{i+1} < 0$. В таком случае ряд $\sum a_n$ называется знакочередующимся.
    	\end{definition*}
    
    	\begin{definition*}
            Ряд $\sum\limits_{n = 1}^\infty {{a_n}}$ называется абсолютно сходящимся, если ряд $\sum\limits_{n = 1}^\infty {\left| {{a_n}} \right|}$ также сходится.
            
            Если ряд $\sum\limits_{n = 1}^\infty {{a_n}}$ сходится абсолютно, то он является сходящимся (в обычном смысле). Обратное утверждение неверно.
   		\end{definition*}
        
        \begin{definition*}
            Ряд $\sum\limits_{n = 1}^\infty {{a_n}}$ называется условно сходящимся, если сам он сходится, а ряд, составленный из модулей его членов, расходится.
        \end{definition*}
        
   		\begin{definition*}
   			Введем два ряда: $a_n^+ = \begin{cases}
   			a_n, a_n > 0 \\
   			0
   			\end{cases}$ 
   			и $a_n^- = \begin{cases}
   			|a_n|, a_n < 0 \\
   			0
   			\end{cases}$.
   			Тогда ряды $\sum a_n^+$ и $a_n^-$ соответственно называются положительной и отрицательной частью ряда $\sum a_n$.
   		\end{definition*}
   	
   		\item Докажите, что ряд сходится абсолютно ровно в том случае, когда сходятся его положительная и отрицательная части.
   			
   		\begin{proof}
   		
   			Рассмотрим ряд $\sum a_n$, дополнительный ряд $\sum |a_n|$, а также положительную и отрицательную части $\sum a_n^+$ и $\sum a_n^-$.
   			
   			1) Пусть ряд $\sum a_n$ сходится абсолютно. В таком случае ряд $\sum |a_n|$ сходится, а так как члены рядов $\sum a_n^+$ и $\sum a_n^-$ все содержатся в ряде $\sum |a_n|$, то для всех их частичных сумм справедливо следующее: $P_k \leq A_n'$ и $Q_m \leq A_n'$, где $P_k$ и $Q_m$ - частичные суммы положительной и отрицательной части соответственно, а $A_n'$ - частичная сумма дополнительного ряда и $A_n' = P_k + Q_m, n = m + k$. Это значит, что оба ряда $\sum a_n^+$ и $\sum a_n^-$ сходятся.
   			
   			2) Исходя из того, что $S_n = P_k - Q_m, n = m + k$ и положительных и отрицательных элементов в $\sum a_n$ бесконечное множество, мы получаем, что при $n \to \infty$ одновременно $m \to \infty$ и $k \to \infty$. Переходя к пределу получаем, что исходный ряд сходится абсолютно и его сумма будет равна $P - Q$.
   			
   			
   		\end{proof}
   	
   		\item Докажите, что если ряд сходится условно, то его положительная и отрицательная части расходятся (имеют бесконечные суммы).
   		
   		\begin{proof}
   			Рассмотрим ряд $\sum a_n$, дополнительный ряд $\sum |a_n|$, а также положительную и отрицательную части $\sum a_n^+$ и $\sum a_n^-$. Поскольку ряд $\sum a_n$ сходится условно, то $\sum |a_n|$ расходится. Рассмотри частичные суммы $\sum |a_n|$, $\sum a_n^+$ и $\sum a_n^-$ - $A_n', P_k, Q_m$ соответственно. Для любого $n = m + k$, $A_n' = P_k + Q_m$. При $n \to \infty$, $m \to \infty$ и $k \to \infty$. Так как ряд $\sum |a_n|$ расходится, то сумма $A_n' \to \infty$. Поскольку число положительных и отрицательных элементов бесконечно, то получаем $P_k \to \infty$ и $Q_m \to \infty$, а значит ряды $\sum a_n^+$ и $\sum a_n^-$ расходятся.
   		\end{proof}
           
        \item Сформулируйте мажорантный признак Вейерштрасса. Приведите пример применения признака
        
        \begin{theorem*}
            Если $|a_n| \leq b_n$ при $n > n_0$ и положительный ряд $\sum b_n$ сходится,
            то $\sum a_n$ сходится, причём абсолютно.
        \end{theorem*}
        
        \begin{example}
        $\sum_{n=1}^{\infty} \dfrac{\sin(nx)}{n^p}$, $p > 0$
        
        $|sin(nx)| \leq 1 \implies \left|\dfrac{sin(nx)}{n^P}\right| \leq \dfrac{1}{n^p}$
        
        $\sum \dfrac{1}{n^p} $ сходится $(p > 1) \implies \sum_{n=1}^{\infty} \dfrac{\sin(nx)}{n^p}$ сходится абсолютно.
        \end{example}

        \item Что такое группировка членов ряда? Докажите, что любой ряд, полученный из сходящегося ряда группировкой его членов, сходится и имеет ту же сумму
        
        Говорят, что ряд $\sum b_k$ получен из $\sum a_n$ группировкой членов, если $\exists n_1 < n_2 < \dots$:

        $b_1 = a_1 + a_2 + \dots + a_{n_1}$

        $b_2 = a_{n_1 + 1} + a_{n_1 + 2} + \dots + a_{n_2}$

        $\dots$

        \begin{comment}
            Если $\sum a_n$ сходится, то ряд $\sum b_k$ сходится к той же сумме.
        \end{comment}

        \begin{proof}
        $\sum_{k=1}^{m} b_k = \sum_{n=1}^{n_m} a_n$
        \end{proof}

        \textit{Обратное утверждение неверно:} $(1 - 1) + (1 - 1) + \dots$

        \item Как с помощью группировки преобразовать знакопеременный ряд в знакочередующийся? Что можно утверждать о сходимости полученного знакочередующегося ряда?
        
        Знакопеременный ряд при помощи группировки сводится к знакочередующемуся:

        $a_1 \leq 0$, $\dots$, $a_{n_1} \leq 0$; $b_1 = \sum_{i=1}^{n_1} a_i \leq 0$

        $a_{n_1+1} \geq 0$, $\dots$, $a_{n_2} \geq 0$; $b_1 = \sum_{i={n_1 + 1}}^{n_2} a_i \leq 0$

        При такой группировке сходимость исходного ряда $\iff$ сходимость $\sum b_n$

        \item Приведите пример преобразования знакопеременного (но не знакочередующегося) ряда к знакочередующемуся.

        \begin{example}
            $\sum_{n=1}^{\infty} \dfrac{(-1)^{[\ln n]}}{n}$

            $\sum_{k=0}^{\infty} b_k$, где $b_k = (-1)^k$

            $|b_k| = \sum_{n=[e^k] + 1}^{[e^{k+1}]} \dfrac{1}{n} \leq \dfrac{1}{[e^k] + 1} \cdot ([e^{k+1}]-[e^k]) \approx \dfrac{e^{k+1} - e^k}{e^k} \to e - 1 > 0$
        \end{example}

        \item -
        \item Сформулируйте признак Лейбница для знакочередующегося ряда. Приведите пример применения признака Лейбница.
        
        \begin{theorem*}
            Признак Лейбница. Если $u_n \downarrow 0$, то ряд сходится, причём $|r_n| \leq u_{n+1}$
            \end{theorem*}
            
            \begin{example}
                $\sum_{n=1}^{\infty} \dfrac{(-1)^{n}}{n^p}$, $p > 0$
            
                $\dfrac{1}{n^p} \downarrow 0 \implies $ ряд сходится (при $\forall p > 0$)
            \end{example}

        \item -
        
        \item Покажите, что для любых числовых последовательностей $\{a_n\}$, $\{B_n\}$ справедлива формула суммирования по частям: $\sum_{n=m+1}^{N} a_n(B_n - B_{n-1}) = (a_NB_N-a_mB_m) - \sum_{n=m+1}^{N} (a_n - a_{n - 1}) B_{n-1}$
        
        Суммируем равенство по индексу $n$: $\sum_{n=m+1}^{N}$. $\sum_{n=m+1}^{N} a_n(B_n - B_{n-1}) = \sum_{n=m+1}^{N} (a_nB_n - a_{n-1}B_{n-1}) - \sum_{n=m+1}^{N} (a_n - a_{n-1})B_{n-1}$. Получаем из первой скобки путём сокращения элементов $a_N B_N - a_m B_m$. В итоге получаем $\sum_{n=m+1}^{N} a_n(B_n - B_{n-1}) = (a_NB_N-a_mB_m) - \sum_{n=m+1}^{N} (a_n - a_{n - 1}) B_{n-1}$.
        
        \item Сформулируйте признак Дирихле. Приведите пример его применения.
        
        $\sum_{n=1}^{\infty}a_n \cdot b_n$

        \begin{theorem*}
            Признак Дирихле. Если $a_n \downarrow 0$, а частичные суммы $\left| \sum_{n=1}^N b_n \right| \leq C$ ограничены,
            то $\sum_{n=1}^{\infty}a_n \cdot b_n$ сходится.
        \end{theorem*}

        \begin{example}
            $\sum_{n=1}^{\infty} \dfrac{\sin(nx)}{n^p}$, $p > 0$
        
            $a_n = \dfrac{1}{n^p} \downarrow 0$, $b_n = \sin nx$
        
            $b_1 + b_2 + b_3 + \dots + b_N = \sin x+ \sin 2x + \dots + \sin Nx = \dfrac{\cos \dfrac{x}{2} - \cos\left((N + 1/2)x\right)}{2 \sin \dfrac{x}{2}}$; $\left|\sum_{n=1}^{N}b_n\right| \leq \dfrac{2}{2\sin{\dfrac{x}{2}}} = \dfrac{1}{\sin{\dfrac{x}{2}}}$
        
            Ряд сходится по признаку Дирихле
        \end{example}

        \item Сформулируйте признак Абеля. Выведите утверждение признака Абеля из признака Дирихле.
        
        \begin{theorem*}
            Признак Абеля. Если $a_n$ монотонна и ограничена, а ряд $\sum_{n=1}^{\infty}b_n$ сходится,
            то $\sum_{n=1}^{\infty}a_n \cdot b_n$ сходится.
        \end{theorem*}
        
        % $a_n \to a$, $a_n = a +- \alpha_n$, $\alpha_n \downarrow 0$; $\sum_{n=1}^{\infty}a_n \cdot b_n = a \sum_{n=1}^{\infty}b_n +- \sum_{n=1}^{\infty}\alpha_n \cdot b_n$

        \item Что такое перестановка членов ряда? Приведите пример.
        
        Пусть $f: \NN \to \NN$ -- биекция

        Говорят, что ряд $\sum b_n$ получен из $\sum a_n$ перестановкой членов, если $b_n = a_{f(n)}$

        \item -

        \item Сформулируйте свойство абсолютно сходящегося ряда, связанное с перестановкой членов. (теорема Римана)
        
        \begin{theorem*}
            (Римана) Если ряд $\sum a_n$ сходится условно, то для $\forall S \in [-\infty; +\infty]$ то $\exists$ перестановка $f$ такая, что $\sum a_{f(n)} = S$
        \end{theorem*}

        \item -
        \item Как определяется произведение рядов? Что можно утверждать о произведении абсолютно сходящихся рядов?
        
        $\sum_{k=1}^{\infty} a_k$, $\sum_{m=1}^{\infty} b_m$

        $\left(\sum_{k=1}^{K} a_k\right) \cdot \left(\sum_{m=1}^{M} b_m \right) = \sum_{1 \leq k \leq K, 1 \leq m \leq M} a_k \cdot b_m$

        Если эта сумма имеет предел при $K, M \to \infty$, не зависящий от порядка суммирования, то говорят, что определено произведение рядов.

        \begin{theorem*}
        (Коши) Если $\sum a_k$, $\sum b_m$ сходятся абсолютно, то определено их произведение.

        $\left(\sum_{k=1}^{\infty} a_k\right) \cdot \left(\sum_{m=1}^{\infty} b_m \right) = \sum_{n=1}^{\infty} a_{k_n} \cdot b_{m_n}$
        \end{theorem*}

        \item Что такое произведение рядов в форме Коши? Приведите пример вычисления такого произведения.
        
        Произведение рядов по Коши:

        $c_2 = a_1 \cdot b_1$

        $c_3 = a_2 \cdot b_1 + a_1 \cdot b_2$

        $c_4 = a_3 \cdot b_1 + a_2 \cdot b_2 + a_1 \cdot b_3$

        $\dots$

        $$\left(\sum_{k=1}^{\infty} a_k\right) \cdot \left(\sum_{m=1}^{\infty} b_m \right) = \sum_{n=2}^{\infty} c_n$$

        \item Дайте определения: бесконечное произведение, частичное произведение, сходящееся бесконечное произведение, расходящееся бесконечное произведение
        
        $\prod_{n=1}^{N} a_n = a_1 \cdot a_2 \cdot \dots \cdot a_N$ -- частичное произведение.

        Бесконечным произведением называют формальную запись $\prod_{n=1}^{\infty} a_n$

        Значением бесконечного произведения является предел частичного произведения:

        $\prod_{n=1}^{\infty} a_n = \lim_{N \to \infty} \prod_{n=1}^{N} a_n$

        Если предел существует и он конечен -- то бесконечное произведение сходится, иначе расходится.

        \item Сформулируйте и докажите необходимое условие сходимости бесконечного произведения.

        Если $P_N = \prod_{n=1}^N a_n$ сходится, то $a_n = \frac{P_n}{P_{n - 1}} \to 1$

        \item - 
        \item Как определяется соответствующий бесконечному произведению ряд? Сформулируйте и докажите утверждение об их взаимосвязи.
        
        $\prod_{n=1}^{N} a_n = e^{\ln \prod_{n=1}^{N} a_n} = e^{\sum_{n=1}^{N} \ln a_n}$

        $\prod_{n=1}^{\infty} a_n = P \iff \sum_{n=1}^{\infty} \ln a_n = \ln P$ $(P \neq 0, a_n \to 1)$
        \item В каком случае бесконечное произведение называется сходящимся абсолютно? Сформулируйте и докажите критерий абсолютной сходимости бесконечного произведения
        
        $\prod_{n=1}^{\infty} a_n$ называется абсолютно сходящимся, если абсолютно сходится соответствующий ему ряд $\sum_{n=1}^{\infty} \ln a_n$

        \begin{comment}
            $\prod_{n=1}^{\infty} a_n$ сходится абсолютно $\iff \sum_{n=1}^{\infty} (a_n - 1)$ сходится абсолютно.
        \end{comment}

        \item Напишите произведение Валлиса и его значение. Вычисление каких интегралов приводит к этой формуле?

        \begin{example}
            (Произведение Валлиса)
            $\prod_{n=1}^{\infty} \frac{4n^2}{4n^2 - 1} = \frac{\pi}{2}$ -- получается из анализа интегралов $\int_{0}^{\frac{\pi}{2}} \sin^n x dx$
            
            Прим. ред.: есть отличное \href{https://www.youtube.com/watch?v=8GPy_UMV-08}{видео} с интуитивно понятным доказательством.
        \end{example}

        \item Дайте определение дзета-функции ($\zeta$-функция) Римана. Сформулируйте тождество Эйлера для $\zeta$-функции.

        \begin{example}
            (Дзета-функция Римана) $\zeta(s) = \sum_{n=1}^{\infty} \frac{1}{n^s}, s > 1$
        
            Тождество Эйлера:
        
            $\zeta(s) = \dfrac{1}{\prod_{n=1}^{\infty}(1 - \frac{1}{p_n^s})}$, где $p_1 = 2, p_2 = 3, p_3 = 5, \dots$
        \end{example}
        
    \end{enumerate}

\end{document}