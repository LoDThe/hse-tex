\subsection{Свойства равномерно сходящегося несобственного интеграла.
Теорема о предельном переходе под знаком несобственного интеграла.
Монотонный предельный переход и теорема Дини и равномерной сходимости семейства функций.
Следствие из теоремы Дини о монотонном предельном переходе под знаком несобственного интеграла.
Теорема о непрерывности несобственного интеграла по параметру.}

\subsubsection{Теорема о предельном переходе под знаком несобственного интеграла.}
\begin{example}
    \[ \int\limits_{0}^{+\infty} f(x, n)\,dx \xrightarrow[n \to \infty]{} \int\limits_{0}^{+\infty} \varphi(x)\,dx \]
\end{example}

Покажем, что $f(x, n) \rightrightarrows \varphi(x)$ недостаточно:
\[ f(x, n) = \left\{\begin{array}{ll} \frac{n}{x^3}\,e^{-\frac{n}{2x^2}}, & x > 0; \\ 0, & x = 0. \end{array}\right. 
\ \ \ \forall n \in \NN \text{ непрерывна на } \; [0; +\infty) \]

Проверяем равномерную сходимость $f(x, n) \rightrightarrows \varphi(x) = 0$:
\[ \underset{x \ge 0}{\sup}\:\left| f(x, n) - \varphi(x) \right| = 
\underset{x > 0}{\sup}\:\frac{n}{x^3}\,e^{-\frac{n}{2x^2}} \underset{\underset{x = \sqrt{\frac{n}{3}}}{\uparrow}}= 
3 \sqrt{\frac3n}\,e^{-3/2} \xrightarrow[n \to \infty]{} 0 \;\Rightarrow\; \text{ сходимость равномерная} \]

Проверяем значение интеграла:
\[ \int\limits_{0}^{+\infty} \varphi(x)\,dx = 0 \]
\[ \int\limits_{0}^{+\infty} f(n, x)\,dx = \int\limits_{0}^{+\infty} \frac{n}{x^3}\,e^{-\frac{n}{2x^2}}\,dx = 
\int\limits_{-\infty}^{0} e^{z}\,dz = e^z \Bigm|_{-\infty}^0 = 1 \]
\[ z = -\frac{n}{2x^2}, \; dz = \frac{n}{x^3}\,dx \]
\[ \lim_{n \to \infty} \int\limits_{0}^{+\infty} f(x, n)\,dx \ne \int\limits_{0}^{+\infty} \lim_{n \to \infty} f(x, n)\,dx \]

Требуется равномерная сходимость несобственного интеграла.

\begin{theorem*} О предельном переходе под знаком несобственного интеграла.

    Рассмотрим $f(x, y)$, определенную на $[a; \omega) \times H$ ($\omega$ --- особая точка).
    
    Пусть \\
    \phantom{Пусть} $\forall y \in H \ \ f(x, y)$ несобственно интегрируема на $[a; \omega)$, \\
    \phantom{Пусть} причем $\int\limits_{a}^{\omega} f(x, y)\,dx$ сходится равномерно по $y \in H$, \\
    \phantom{Пусть} $\forall t \in [a; \omega) \ \ 
    f(x, y) \overset{x \in [a; t]}{\underset{y \to y_0}{\rightrightarrows}} \varphi(x), \; y_0 \in H$.
    
    Тогда $\varphi$ несобственно интегрируема на $[a; \omega)$, причем 
    \[ \lim_{y \to y_0} \int\limits_{a}^{\omega} f(x, y)\,dx = \int\limits_{a}^{\omega} \varphi(x)\,dx \]
\end{theorem*}
\begin{proof}
\begin{enumerate}
    \item Покажем, что $\int\limits_{a}^{\omega} \varphi(x)\,dx$ сходится: \\[5 pt]
    $a \le t_1 < t_2 < \omega$ \\[3 pt]
    По критерию Коши для равномерно сходящегося несобственного интеграла $\int\limits_{a}^{\omega} f(x, y)\,dx$
    \[ \left| \int\limits_{t_1}^{t_2} f(x, y)\,dx \right| < \varepsilon \ \ \ \forall t_1, t_2 \in U(\omega), \; \forall y \in H \]
    \[ y \to y_0: \ \ \ \lim_{y \to y_0} \int\limits_{t_1}^{t_2} f(x, y)\,dx = 
    \int\limits_{t_1}^{t_2} \left( \lim_{y \to y_0} f(x, y) \right) dx = \int\limits_{t_1}^{t_2} \varphi(x)\,dx \ \ \ 
    \text{(т.к. $f(x, y) \overset{x \in [a; t]}{\underset{y \to y_0}{\rightrightarrows}} \varphi(x)$)} \]
    \[ \Rightarrow\; \left| \int\limits_{t_1}^{t_2} \varphi(x)\,dx \right| \le \varepsilon \ \ \ \forall t_1, t_2 \in U(\omega) \]
    \[ \Rightarrow\; \varphi \text{ несобственно интегрируема} \]
    
    \item Покажем, что $\left| \int\limits_{a}^{\omega} f(x, y)\,dx - \int\limits_{a}^{\omega} \varphi(x)\,dx \right|
    \xrightarrow[y \to y_0]{} 0$:
    \[ \left| \int\limits_{a}^{\omega} f(x, y)\,dx - \int\limits_{a}^{\omega} \varphi(x)\,dx \right| \le 
    \underset{< \text{\large$\frac{\varepsilon}{3}$} \ \begin{array}{l} \forall t \in U(\omega) \\[-7 pt] 
    \forall y \in H \end{array}}{\underbrace{\left| \int\limits_{a}^{\omega} f(x, y)\,dx - \int\limits_{a}^{t} f(x, y)\,dx \right|}} + \]
    \[ + \underset{< \text{\large$\frac{\varepsilon}{3}$} \ \begin{array}{c} 
    \text{\footnotesize При фикс. } t \in U(\omega) \\[-7 pt] |y - y_0| < \delta \end{array}}
    {\underbrace{\left| \int\limits_{a}^{t} f(x, y)\,dx - \int\limits_{a}^{t} \varphi(x)\,dx \right|}} + 
    \underset{< \text{\large$\frac{\varepsilon}{3}$} \ \displaystyle\forall t \in U(\omega)}
    {\underbrace{\left| \int\limits_{a}^{t} \varphi(x)\,dx - \int\limits_{a}^{\omega} \varphi(x)\,dx \right|}} < \varepsilon \ \ 
    \forall \varepsilon > 0, \ \ \text{что и требовалось} \]
\end{enumerate}
\end{proof}
\subsubsection{Монотонный предельный переход и теорема Дини и равномерной сходимости семейства функций.}

\begin{theorem*} Теорема Дини

    Пусть \\
	\phantom{Пусть} $f(x, y) \ge 0$ и непрерывна по $x \in [a; \omega)$ $\forall y \in H$, \\
	\phantom{Пусть} $\forall x \in [a; \omega) \ \ f(x, y) \uparrow$ по $y$ и $f(x, y) \underset{y \to y_0}{\nearrow} \varphi(x)$, \\
	\phantom{Пусть} $\varphi(x)$ непр. на $[a; \omega)$.
    \[ \text{Тогда} \ \ f(x, y) \overset{x \in [a; t]}{\underset{y \to y_0}{\rightrightarrows}} \varphi(x) \ \ \forall t \in (a; \omega) \]
\end{theorem*}
\begin{proof}
    В силу монотонности по $y$:
    \[ (\varphi(x) - f(x, y)) \downarrow \text{ по } y, \ \varphi(x) - f(x, y) \ge 0 \]
    \[ \forall y_1 < y_2 \ \ \ \varphi(x) - f(x, y_1) \ge \varphi(x) - f(x, y_2) \;\Rightarrow \]
    \[ \Rightarrow\; \underset{x}{\sup}\:\left| \varphi(x) - f(x, y_1) \right| \ge 
    \underset{x}{\sup}\:\left| \varphi(x) - f(x, y_2) \right| \;\Rightarrow \]
    \[ \Rightarrow\; \psi(y) = \underset{x}{\sup}\:\left| \varphi(x) - f(x, y) \right| \ \text{--- убывает и } \ge 0 \]
    
    Докажем, что $\psi(y) \xrightarrow[y \to y_0]{} 0$. От противного: пусть $\psi(y) \ge \varepsilon > 0$.

    Тогда $\forall$ фиксированного ``$y$'' $\exists \{ x_n \} \colon \varphi(x_n) - f(x_n, y) \ge \varepsilon / 2$
    
    $\{ x_n \} \subset [a; t] \;\Rightarrow\;$ из нее можно выделить сходящуюся подпоследовательность $\{ x_{n_k} \}$ такую, что $x_{n_k} \xrightarrow[k \to \infty]{} c \in [a; t]$.
    \[ \varphi(x_{n_k}) - f(x_{n_k}, y) \ge \varepsilon / 2 \]
    \[ k \to \infty: \ \varphi(c) - f(c, y) \ge \varepsilon / 2 \ \ \ \forall y \in U(y_0) \textrm{ - малая окрестность точки } y_0 \]

    Пусть $c(y) := $ такая точка, что $\varphi(c(y)) - f(c(y), y) \geq \varepsilon / 2$. Рассмотрим $\{y_n\} \to y_0$ и соответствующую
    $\{c_n\} := \{c(y_n)\}$. Поскольку $\{c_n\} \subset [a; t] \implies$ из неё можно выделить сходящуюся
    подпоследовательность $\{c_{n_k}\} \xrightarrow[k\to\infty]{} z$. Тогда в точке $z$ не будет поточечной сходимости
    потому что $y_{n_k}$ к $y_0$ стремится, а разница к 0 --- нет, потому что $\varphi(c_{n_k}) - f(c_{n_k}, y_{n_k}) \geq \varepsilon/2$
    $\forall y_{n_k}$.
\end{proof}
\subsubsection{Следствие из теоремы Дини о монотонном предельном переходе под знаком несобственного интеграла.}
\begin{corollary}
	Пусть:
	\begin{itemize}
		\item выполняются условия теоремы Дини: \\
		$f(x, y) \ge 0$ и непрерывна по $x \in [a; \omega)$ $\forall y \in H$, \\
		$\forall x \in [a; \omega) \ \ f(x, y) \uparrow$ по $y$ и $f(x, y) \underset{y \to y_0}{\nearrow} \varphi(x)$, \\
		$\varphi(x)$ непр. на $[a; \omega)$,
		\item $\int\limits_a^{\omega} \varphi(x)\,dx$ --- сходится.
	\end{itemize}
	\[ \text{Тогда} \ \ \lim_{y \to y_0} \int\limits_a^{\omega} f(x, y)\,dx = \int\limits_a^{\omega} \varphi(x)\,dx \]
\end{corollary}
\begin{proof}
    По теореме Дини: $f(x, y) \overset{x \in [a; t]}{\underset{y \to y_0}{\rightrightarrows}} \varphi(x) \qquad
    \forall t \in (a; \omega)$
    
    $0 \le f(x, y) \le \varphi(x)$
    
    Т.к. $\int\limits_a^{\omega} \varphi(x)\,dx$ сходится, то $\int\limits_a^{\omega} f(x, y)\,dx$ сходится равномерно (по признаку Вейерштрасса),
    а поскольку выполнены все условия теоремы о предельном переходе под знаком несобственного интеграла (5.1),
    то $\lim_{y \to y_0} \int\limits_a^{\omega} f(x, y)\,dx = \int\limits_a^{\omega} \varphi(x)\,dx$.
\end{proof}

\begin{example}
    \[ \int\limits_0^{+\infty} e^{-x^2}\,dx = ? \]
    \[ \left( 1 + \frac{x^2}{n} \right)^n \underset{n \to \infty}{\nearrow} e^{x^2} \ \ \forall x \ge 0 \]
    Т.к. функции непрерывны, сходимость равномерная (по теореме Дини) \\
    $\Rightarrow$ равномерно сходится $f(x, n) = \left( 1 + \frac{x^2}{n} \right)^{-n} 
    \underset{n \to \infty}{\rightrightarrows} e^{-x^2} = \varphi(x)$
    
    Но последовательность $f(x, n)$ убывает --- рассмотрим разность
    \[ g(x, n) = f(x, 1) - f(x, n) = \left( 1 + x^2 \right)^{-1} - \left( 1 + \frac{x^2}{n} \right)^{-n} \ge 0 \ \ 
    \uparrow \text{ по } n \]
    \[ g(x, n) \rightrightarrows \psi(x) = (1  +x^2)^{-1} - e^{-x^2} \ge 0 \ \ \ \int\limits_0^{+\infty} \psi(x)\,dx \ \ 
    \text{ сходится} \]
    
    По следствию из теоремы Дини:
    \[\int\limits_0^{+\infty} g(x, n)\,dx \to \int\limits_0^{+\infty} \psi(x)\,dx \]
    \[ \int\limits_0^{+\infty} \left( \frac1{1 + x^2} - \frac1{\left( 1 + \frac{x^2}n \right)^n} \right) dx \to
    \int\limits_0^{+\infty} \left( \frac1{1 + x^2} - e^{-x^2} \right) dx \]
    \[ \int\limits_0^{+\infty} e^{-x^2}\,dx = 
    \lim_{n \to \infty} \int\limits_0^{+\infty} \frac{dx}{\left( 1 + \frac{x^2}n \right)^n} \] 
    Интегрируем правый по частям, постепенно понижая степень $n$ — появится рекуррента и придем к
    \[ \int\limits_0^{+\infty} \frac{dx}{\left( 1 + \frac{x^2}n \right)^n} = 
    \frac{(2n - 3)!!}{(2n - 2)!!} \cdot \sqrt{n} \cdot \frac{\pi}2 \]
    Формула Валлиса: $\ \prod\limits_{k = 1}^n \frac{4k^2}{4k^2 - 1} \to \frac{\pi}2$. Подставляем 
    формулу Валлиса в равенство выше, а дальше ``дел минут на двадцать''. Хорошо, что это пример, а не билет.
\end{example}

\subsubsection{Теорема о непрерывности несобственного интеграла по параметру.}
\[ f(x, y) \ \ \ [a; \omega) \times [c; d] \]
Пусть $\ f(x, y)$ непрерывна на $[a; \omega) \times [c; d]$ \\
\phantom{Пусть} $F(y) = \int\limits_a^{\omega} f(x, y)\,dx$ сходится равномерно на $[c; d]$

Тогда $F(y)$ непрерывна на $[c; d]$
\begin{proof}
    \[ g(t, y) = \int_a^t f(x, y)\,dx \text{ --- непрерывна по } y \in [c; d] \ \ \forall t \in (a; \omega) \]
    \[ \int_a^\omega f(x, y)\,dx \text{ --- сходится равномерно} \;\Leftrightarrow\; g(t, y) 
    \overset{y \in [c; d]}{\underset{t \to \omega}{\rightrightarrows}} F(y) \]
    Равномерно сходящееся семейство непрерывных функций сходится к непрерывной функции (3.2. Теорема о непрерывности по параметру). Далее просто краткое напоминание, почему это так:
    \[ |F(y) - F(y_0)| \le \underset{< \text{\large$\frac{\varepsilon}{3}$} \ 
    \begin{array}{l} \forall t \in U(\omega) \\[-7 pt] 
    \forall y \in H \end{array}}{\underbrace{\left| F(y) - g(t, y) \right|}} + 
    \underset{< \text{\large$\frac{\varepsilon}{3}$} \ \begin{array}{c} 
    \text{\footnotesize При фикс. } t \in U(\omega) \\[-7 pt] |y - y_0| < \delta \end{array}}
    {\underbrace{\left| g(t, y) - g(t, y_0) \right|}} + 
    \underset{< \text{\large$\frac{\varepsilon}{3}$} \ \begin{array}{l} \forall t \in U(\omega) \\[-7 pt] 
    \forall y \in H \end{array}}
    {\underbrace{\left| g(t, y_0) - F(y_0) \right|}} < \varepsilon \ \ 
    \forall \varepsilon > 0 \]
\end{proof}
