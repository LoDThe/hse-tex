% Здесь НЕ НУЖНО делать begin document, включать какие-то пакеты..
% Все уже подрубается в головном файле
% Хедер обыкновенный хсе-теха, все его команды будут здесь работать
% Пожалуйста, проверяйте корректность теха перед пушем

% Здесь формулировка билета
\subsection{Каково основное различие между общей теорией несобственного (кратного) интеграла и соответствующей теорией, принятой в одномерном случае? В чем причина этого различия?}
Важнейшим отличием данного понятия несобственного интеграла от его простейшего одномерного аналога состоит в том, что
несобственный интеграл, определяемый через исчерпания множества, не может оказаться сходящимся условно.

Пусть при некотором допустимом исчерпании $\{D_n\}$
\[
    \int_{D_n} |f(x)| dx \to \infty.
\]
Покажем, что 
\[
    \int_D f(x) dx
\]
не может быть сходящимся.

Введём положительную и отрицательную части функции $f$
\[f^+(x) = \frac{1}{2}(|f(x)| + f(x)) \geqslant 0, ~ f^-(x) = \frac{1}{2}(|f(x)| - f(x)) \geqslant 0\]
Тогда
\[f(x) = f^+(x) - f^-(x), ~ |f(x)| = f^+(x) + f^-(x)\]

