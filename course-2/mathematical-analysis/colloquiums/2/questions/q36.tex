\subsection{Что можно утверждать о зарядке, имеющем нулевую плотность?}

Рассмотрим заряд $\nu$, определенный на алгебре жордановых подмножеств $A \subseteq D$ жорданова множества $D$ и такой, что из $\mu(A) = 0$ следует $\nu(A) = 0$.

\begin{theorem*}
    Пусть $\rho(a) = 0$ для всех точек $a \in D$. Тогда $\nu(A) = 0$ для любого жорданова $A \subseteq D$.
\end{theorem*}

