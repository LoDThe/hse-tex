% Здесь НЕ НУЖНО делать begin document, включать какие-то пакеты..
% Все уже подрубается в головном файле
% Хедер обыкновенный хсе-теха, все его команды будут здесь работать
% Пожалуйста, проверяйте корректность теха перед пушем

% Здесь формулировка билета
\subsection{Что такое криволинейные координаты в области $X\subseteq\mathbb{R}^m$ Как определяется координатная линия и единичный координатный вектор}
\begin{definition*}
	При этом числа $(u_1, \ldots, u_m)$ называются \textbf{криволинейными координатами} точки $(x_1, \ldots, x_m)$
\end{definition*}
\begin{definition*}
    \text{ }
    
    $u^0 = (u_1^0,\ldots,u^0_m)$
    
    $x^0 = \phi(u^0)$
    
	Кривая $x = \phi(u_1, u_2^0,\ldots,u^0_m)$ называется \textbf{координатной линией} $u_1$ на множестве X
\end{definition*}
\begin{definition*}
	\textbf{Касательный вектор(координатный вектор)} к координатной линии $u_1$
	\[\frac{\partial x}{\partial u_1} = (\frac{\partial x_1}{\partial u_1}, \ldots, \frac{\partial x_m}{\partial u_1})\neq 0\]
	\textbf{Единичный касательный вектор(единичный координатный вектор)}
	\[e_1 = \frac{\frac{\partial x}{\partial u_1}}{|\frac{\partial x}{\partial u_1}|}\]
\end{definition*}

