% Здесь НЕ НУЖНО делать begin document, включать какие-то пакеты..
% Все уже подрубается в головном файле
% Хедер обыкновенный хсе-теха, все его команды будут здесь работать
% Пожалуйста, проверяйте корректность теха перед пушем

% Здесь формулировка билета
\subsection{Докажите, что простые множества в $\mathbb{R}^m$ образуют кольцо}

\textbf{\underline{Утв.:} } Класс всех простых множеств образует кольцо. \\
\textbf{\underline{Док-во:} } \\
\begin{enumerate}
    \item $\varnothing = [a; a)$ - пустой полуинтервал является простым множеством.
    \item $E_1 \cup E_2 = E$ - объединение простых множеств является простым множеством. Так как каждое из простых множеств представимо в виде объединения конечного количества полуинтервалов, то их объединение представимо в виде объединения всех полуинтервалов входящих в каждое из простых, а значит является простым множеством.
    \item $E_1 \cap E_2 = E$ - пересечение простых множеств является простым множеством. Пересечение представимо в виде объединения пересечений всех возможных пар из первого и второго множества. Так как пересечение полуинтервалов является полуинтервалом, то пересечение простых множеств, является простым множеством. 
    \item $E_1 \backslash E_2 = E$ - разность простых множеств является простым множеством. Пусть есть некоторый полуинтервал $[a; b)$ покрывающий $E_1$ и $E_2$, тогда $[a; b) \backslash E_2$ очевидно является простым множеством. В таком случае исходную разность можно записать в виде $E_1 \cap ([a; b) \backslash E_2)$, что будет пересечением простых множеств, а значит является полуинтервалом. 
\end{enumerate}
\begin{flushright}
$\blacksquare$
\end{flushright}



