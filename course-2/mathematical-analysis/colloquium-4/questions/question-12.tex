\subsection{Бесконечная дифференцируемость и голоморфность аналитической функции. Нуль аналитической функции и его порядок. Изолированность нуля аналитической функции. Теорема единственности аналитической функции.}

\subsubsection{Бесконечная дифференцируемость и голоморфность аналитической функции.}
\begin{proposal}
	Пусть $f(z)$ - аналитическая в точке $z_0$. Значит, функция $f(z)$ представима в окрестности $z_0$ в виде ряда $\sum_{k=0}^{\infty}c_k(z-z_0)^k$, $|z-z_0| < \delta$. Тогда ряд будет абсолютно сходиться $\forall z: |z - z_0| = \varepsilon < \delta \implies \sum_{k=0}^{\infty} |c_k|\varepsilon^k$. Отсюда следует $\exists A > 0: |c_k|\varepsilon^k \leq A \implies |c_k| \leq \dfrac{A}{\varepsilon^k}$.
	
	Покажем что аналитическая функция дифференцируема. Рассмотрим $|z-z_0| \leq \varepsilon_1 < \varepsilon < \delta$. Берём приращение $0 < |h| < \varepsilon - \varepsilon_1$.
	
	$$\dfrac{(z+h) - f(z)}{h} = \sum_{k=1}^{\infty}c_k \dfrac{(z-z_0+h)^k - (z-z_0)^k}{h}$$
	
	$$\dfrac{(z-z_0+h)^k - (z-z_0)^k}{h} = k(z-z_0)^{k-1} + c_k^2 h (z-z_0)^{k-2} + \dots + c_k^k h^{k-1} = \sum_{k=1}^{\infty} k c_k(z-z_0)^{k-1} + hc_2 + h\sum_{k=3}^{\infty}c_k(c_k^2(z-z_0 + \dots + c_k^k h^{k-2}))$$
	
	При $k \geq 3$: $c_k^2 \varepsilon_{k-2} + c_k^3 \varepsilon_1^{k-3} |h| + \dots + c_k^k|h|^{k-2} \leq c_k^2(\varepsilon_1 + |h|)^{k-2}$.
	
	Теперь возьмём по модулю третью сумму: $|h\sum_{k=3}^{\infty}c_k(c_k^2(z-z_0 + \dots + c_k^k h^{k-2}))|\leq \sum_{k=3}^{\infty} \dfrac{A}{\varepsilon^k} \dfrac{k(k-1)}{2}(\varepsilon_1 + |h|)^{k-2}$; так как $\dfrac{\varepsilon_1 + |h|}{\varepsilon} < 1$, то полученный ряд сходится. Взяв $h \rightarrow 0$, получим $f'(z) = \sum_{k=1}^{\infty} k c_k (z-z_0)^{k-1}$. Мы обосновали что комплексную аналитическую функцию можно почленно дифференцировать. В таком случае, можно заметить, что $f'$ - тоже аналитическая функция, а значит $f''(z) = \sum_{k=2}^{\infty} k(k-1)c_k(z-z_0)^{k-2}$, ч.т.д.
	
	Мы получили, что $f$ бесконечно дифференцируема, а из дифференцируемости вытекает её голоморфность. $f(z)$ имеет производную, это равносильно условиям Коши-Римана, кроме того $f'(z)$ непрерывна, а это даёт голоморфность.
\end{proposal}

\subsubsection{Нуль аналитической функции, его порядок и изолированность.}

Пусть в $z_0$ значение аналитической функции $f(z_0)$ равно 0. В этом случае $z_0$ называется нулём функции $f(z)$. Тогда в разложении в ряд Тейлора будет отсутствовать свободный член $c_0=0$. В случае когда отсутствуют все слагаемые, содержащие $(z - z_0)^i, i < n$, где $n$ - некоторое число, то разложение будет иметь вид $f(z) = \sum_{k=n}^{\infty}c_k(z-z_0)^k$, а сама точка $z_0$ будет называться нулём порядка $n$.

Под изолированностью нуля какой-либо функции подразумевается существование такой окрестности нуля, что в ней отсутствуют другие нули.

\subsubsection{Теорема единственности аналитической функции.}

\begin{theorem*}
	Теорема (прим. техера: подготовительная) единственности аналитической функции.
	
	Если $f$ аналитична в точке $z_0$ и $z_0$ является предельной точкой последовательности нулей функции $f$, т.е. $\exists{z_n}: z_n \rightarrow z_0, f(z_n) = 0 \forall n$, то $f(z)\equiv0$ в некоторой окрестности точки $z_0$.
\end{theorem*}

\begin{proof}
	Так как $f$ непрерывна, то $f(z_0) = 0 \implies$ в разложении $f(z) = \sum_{k=0}^{\infty} c_k(z-z_0)^k$ некоторое количество начальных коэффициентов будет равно нулю: $c_0 = c_1 = \dots = c_n = 0$, т.е. $f(z) = (z-z_0)^n (c_{n+1} + c_{n+2}(z-z_0) + \dots)$. Получили, что $z_0$ - это ноль функции $f$ кратности $n$.
	
	Рассмотрим сумму в скобках. Она задаёт голоморфную функцию $g(z)$. Значит, $g(z)$ - непрерывна, и так как $c_{n+1} \neq 0$, то существует такая окрестность $|z - z_0| < \varepsilon: |g(z)| > 0$. $\implies$ в круге радиуса $|z-z_0| < \varepsilon$ нет дургих нулей функции, т.е. $z_0$ - ноль аналитической функции должен быть изолированным. А это противоречит тому, что у нас $n$ нулей, т.е. такого $n$ не существует, а значит $f(z) = 0$ в некоторой окрестности $z$.
\end{proof}

Отсюда перейдём непосредственно к самой теореме о единственности. 

\begin{theorem*}
	Теорема (прим. техера: основная) единственности аналитической функции.
	
	Если две функции $f_1(z), f_2(z) \in \Sigma D$ совпадают на множестве $\varepsilon$, которое имеет хотя бы одну предельную точку $z_0 \in D$, то $f_1(z) = f_2(z)$ всюду в $D$.
\end{theorem*}

\begin{proof}
	Рассмотрим $f = f_1 - f_2$. Покажем, что $f \equiv 0$ в $D$. Т.е. требуется доказать, что множество $F = {z\in D: f(z) = 0}$, в которое включено $\varepsilon$ совпадёт с $D$. Предельная точка $z_0$ является нулём функции $f$ в силу непрерывности. Из теоремы доказанной ранее получим, что $f \equiv 0$ в некоторой окрестности $z_0$, ибо в противном случае эта точка не могла бы быть предельной для множества нулей $f$. Таким образом, получим, что ядро множества $F$ непусто - оно содержит в себе точку $z_0$. По построению $F$ открыто, но при этом замкнуто относительно области $D$. По ранее доказанной теореме можем сказать, что взяв точку $b \in D$, мы получим предельную точку для $F$, а потому $f(b) \equiv 0$, т.е. $b \in F$. Так как по определению области $D$ связно, то имеем $F = D$. А значит $f \equiv 0$ на всей $D$, и $f_1(z) = f_2(z)$.
\end{proof}
