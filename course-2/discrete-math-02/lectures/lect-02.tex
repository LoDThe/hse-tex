\ProvidesFile{lect-02.tex}[Лекция 2]

\section{Лекция 2}

\subsection{Универсальный алгоритм}

\begin{definition}
    Универсальной вычислимой функцией (у. в. ф.) называется такая вычислимая $U \colon \N^{2} \pto \N$, что для любой вычислимой $f \colon \N \pto \N$ существует $n \in \N$ (\enquote{программа}) такая, что
    $$
        \forall x \in \N \quad U(n, x) \simeq f(x).
    $$
\end{definition}

Пусть $\UU$ --- некоторый алгоритм, который вычисляет у. в. ф. $U$.
Тогда, по существу, $\UU$ --- интерпретатор универсального языка программирования, где программами являются натуральные числа.
Если имеется какая-то функция от двух аргументов $V \colon \N^{2} \pto \N$, то мы можем построить ее график (поверхность в $\N^{3}$).
Зафиксируем ее первый аргумент $n$, и получим функцию $V_{n} \colon \N \pto \N$ такую, что для любых $n$ и $x$ верно $V_{n}(x) \simeq V(n, x)$.
Функция $V_{n}$ называется {\it $n$-ым сечением функции $V$ по первому аргументу}.
Например, если была $V(x, y) = x + y$, то зафиксировав первый аргумент мы получим функцию \enquote{прибавить $x$} от одного аргумента.
В таком случае, можно переписать условие универсальности следующим образом:
\begin{statement}
    Функция $U$ является универсальной в классе вычислимых функций $\iff$ для любой вычислимой $f \colon \N \pto \N$ существует $n \in \N$ такое, что $U_{n} = f$.
\end{statement}

Тогда можно сформулировать следующее свойство:
\begin{property}[алгоритмов]
    Существует универсальная вычислимая функция.
\end{property}
Данное утверждение эквивалентно тому, что на Python можно написать интерпретатор Python.

\begin{definition}
    Функция $W \colon \N^{2} \to \N$ называется универсальной вычислимой тотальной функцией, если
    \begin{enumerate}[nosep]
        \item $W$ вычислима.
        \item Для любой вычислимой тотальной $g \colon \N \to \N$ существует $m$ такое, что $W_{m} = g$.
    \end{enumerate}
\end{definition}

\begin{statement}
    Не существует универсальной вычислимой тотальной функции $W$.
\end{statement}

\begin{proof}
    Рассмотрим функцию $D \colon \N \to \N$, $D(x) = W(x, x)$. Она является вычислимой и тотальной.
    Рассмотрим теперь функцию $g$ такую, что $g(x) = D(x) + 1$. Она также будет вычислимой и тотальной.
    Раз $W$ универсальна, то существует $m$ такое, что $W_{m} = g$, то есть
    $$
        \exists m \colon \forall x ~W(m, x) = g(m).
    $$
    Положим $x = m$, тогда
    $$
        W(m, m) = D(m) + 1 = W(m, m) + 1 \implies 0 = 1.
    $$
     Получили противоречие, значит такой $W$ не существует.
\end{proof}

В случае для частично определенной у. в. ф. равенство $U(m, m) \simeq U(m, m) + 1$ допустимо, так как означает, что на $m$ функция $U$ не определена.
\begin{theorem}
    Существует перечислимое неразрешимое множество.
\end{theorem}

\begin{proof}
    Пусть $U$ --- универсальная вычислимая функция (мы знаем, что такая существует).
    Рассмотрим $K_{U} = \{n \in \N \mid U(n, n)~\text{определена}\} = \dom d_{U}$, где $d_{U}(x) \simeq U(x, x)$ --- диагональ у. в. ф. $U$.
    Ясно, что $d_{U}$ вычислима $\implies$ $K_{U} = \dom d_{U}$ перечислимо.

    Допустим, что $K_{U}$ разрешимо.
    Рассмотрим функцию $r \colon \N \pto \N$ такую, что
    $$
        r(x) = \begin{cases}
            1, & \text{если } x \notin K_{U}, \\
            \text{undefined}, & \text{если } x \in K_{U}.
        \end{cases}
    $$
    Функция $r$ является полухарактеристической функцией $K_{U}^{\complement}$, $r = w_{K_{U}^{\complement}}$.
    $K_{U}$ разрешимо $\implies$ $K_{U}^{\complement}$ разрешимо $\implies$ $K_{U}^{\complement}$ перечислимо $\implies$ $w_{K_{U}^{\complement}}$ вычислима.

    Тогда существует $n \in \N$ такое, что $\forall x~U(n, x) \simeq r(x) \implies$ положим $x = n$ $\implies U(n, n) = r(n)$.
    Рассмотрим несколько случаев:
    \begin{enumerate}
        \item $U(n, n)$ определено $\implies$ $n \in K_{U} \implies r(n)$ не определено $\implies  U(n, n)$ не определено.
        \item $U(n, n)$ не определено $\implies$ $n \in K_{U} \implies r(n) = 1 \implies U(n, n) = 1 \implies U(n, n)$ определено.
    \end{enumerate}
    Ни один из случаев не приводит к чему-то разумному $\implies$ противоречие $\implies$ $K_{U}$ не разрешимо.
\end{proof}

Определение $K_{U}$ можно сформулировать следующим образом: 
\begin{gather}
    K_{U} = \left\{n \in \N \mid U \text{ останавливается на входе } (n, n)\right\}, \\
    \text{или} \\
    K_{U} = \left\{n \in \N \mid \text{программа $n$ на входе $n$ останавливается}\right\}.
\end{gather}

Поэтому вся эта теория называется \enquote{Проблемой самоприменимости}: остановится ли программа, если ей на вход передать ее же.
Оказывается, что эта проблем не разрешима, что мы сейчас и доказали.

Рассмотрим $S_{U} = \left\{(n,  x)  \in \N^{2} \mid U(n, x)\text{ определена}\right\} = \dom U$.
Поскольку $U$ вычислима, то $S_{U}$ перечислимо.
Заметим, что $n \in K_{U} \iff (n, n) \in S_{U}$, но тогда $S_{U}$ разрешимо $\implies$ $K_{U}$ разрешимо $\implies$ противоречие $\implies$ $S_{U}$ не разрешимо.
Это называется \enquote{Проблемой остановки}: остановка программы $n$ на входе $x$.

\subsection{Т-предикаты}

Зафиксируем универсальную вычислимую функцию $U$ и алгоритм $\UU$, ее вычисляющий.
Рассмотрим следующее множество:
$$
    T_{(U)}^{\prime} = \left\{(n, x, y, k) \mid \text{алгоритм $\UU$ на входе $(n, x)$ остановится за $k$ шагов и выведет $y$}\right\}.
$$
То свойство, что алгоритм можно исполнить по шагам отражено в том, что $T^{\prime}$ разрешимо.
Рассмотрим также множество
$$
    T = \left\{(n, x, k) \in \N^{3} \mid \text{алгоритм $\UU$ на входе $(n, x)$ остановится за $k$ шагов}\right\}.
$$
Множество $T$ также разрешимо.

\subsection{Альтернативный взгляд на перечислимые неразрешимые множества}

Пусть $U$ --- универсальная вычислимая функция, и $d(x) \simeq U(x, x)$.

\begin{statement} \label{st::02::03::01}
    Для любой вычислимой $f \colon \N \pto \N$ существует $n$ такое, что $f(n) \simeq d(n)$.
\end{statement}

\begin{proof}
    Поскольку $U$ универсальна, то существует $n$ такое, что $\forall x$ $U(n, x) \simeq f(x)$.

    Положим $n = x \implies U(n, n) \simeq f(n) \simeq d(n)$.
\end{proof}

\begin{definition}
    Пусть дана $f \colon \N \pto \N$.
    Будем говорить, что функция $g$ {\it продолжает} $f$, если $\dom f \subseteq \dom g$ и для любого $x \in \dom f$ верно $f(x) = g(x)$.
\end{definition}

\begin{statement}
    У функции $d$ не существует вычислимого тотального продолжения.
\end{statement}

\begin{proof}
    Пусть $g \colon \N \pto \N$ --- вычислимое тотальное продолжение $d$.
    То есть, для любого $x \in \dom d$ $d(x) = g(x)$.
    Рассмотрим $h \colon \N \to \N$, $h(x) = g(x) + 1$.
    Ясно, что $h$ вычислима и тотальна, однако $h$ всюду отличается от $d$, поскольку
    \begin{align}
        x \in \dom d &\implies h(x) = g(x) + 1 = d(x) + 1 \neq d(x); \\
        x \notin \dom d &\implies h(x) \text{ определена } \cancel{\simeq}~d(x) \text{ не определена}.
    \end{align}
    Так как любая вычислимая функция где-то совпадает с $d$, то $h$ не вычислима.
\end{proof}

\begin{statement}
    Если функция $f$ вычислима, но не имеет вычислимого тотального продолжения, то $\dom f$ перечислимо, но не разрешимо.
\end{statement}

\begin{proof}
    Если $f$ вычислима, то $\dom f$ перечислимо.
    Пусть $\dom f$ разрешимо.
    Тогда рассмотрим тотальную вычислимую функцию
    $$
        g(x) = \begin{cases}
            f(x), & x \in \dom f, \\
            2020, & x \notin \dom f.
        \end{cases}
    $$
    
    Заметим, что $g$ --- это вычислимое тотальное продолжение $f$, но его не существует $\implies$ противоречие.

    Альтернативно, $g(x)$ можно задать следующим образом: $g(x) = \chi_{\dom f}(x) \cdot f(x) + (1 - \chi_{\dom f}(x)) \cdot 2020$.
\end{proof}

\begin{statement}
    Существует вычислимая $f \colon N \pto \{0, 1\}$ такая, что у $f$ нет вычислимого тотального продолжения.
\end{statement}

\begin{proof}
    Определим $f$ следующим образом:
    $$
        f(x) = \begin{cases}
            0, & x \in \dom d \text{ и } d(x) > 0, \\
            1, & x \in \dom d \text{ и } d(x) = 0, \\
            \text{undefined}, & x \notin \dom d.
        \end{cases}
    $$
    Такая $f$ вычислима, поскольку $f \simeq h(d(x))$, где 
    $$
        h(y) = \begin{cases}
            0, & y > 0, \\
            1, & y = 0.
        \end{cases}
    $$
    $h$ является вычислимой функцией $\implies$ $f$ вычислима как композиция вычислимых функций.
    Тогда для любого $x \in \dom d$ \ $f(x) \neq d(x) \implies$ у $f$ нет вычислимого тотального продолжения (см. утверждение \ref{st::02::03::01}).
\end{proof}

\begin{definition}
    Пусть $A, B, C \subseteq \N$.
    Будем говорить, что $C$ {\it отделяет} $A$ от $B$, если $A \subseteq C$ и $B \subseteq C^{\complement}$. 
\end{definition}

\begin{corollary}
    Существуют перечислимые множества $A$ и $B$ такие, что $A \cap B = \varnothing$, но не существует разрешимого $C$ такого, что $C$ отделяет $A$ от $B$.
\end{corollary}

Отсюда следует, что $A$ перечислимо, но не разрешимо, так как иначе разрешимое $A$ отделяло бы $A$ от $B$. 
\begin{proof}
    Пусть $f$ --- функция из предыдущего утверждения, то есть
    $$
        f(x) = \begin{cases}
            0, & x \in \dom d \text{ и } d(x) > 0, \\
            1, & x \in \dom d \text{ и } d(x) = 0, \\
            \text{undefined}, & x \notin \dom d.
        \end{cases}
    $$
    Положим $A = f^{-1}(\{1\})$, $B = f^{-1}(\{0\})$.
    Очевидно, что $A \cap B = \varnothing$.
    $A$ и $B$ перечислимы как прообразы перечислимых множеств $\{1\}$, $\{0\}$ под действием вычислимой функции $f$.
    Положим, что $\exists C$, отделяющее $A$ от $B$.
    Рассмотрим характеристическую функцию множества $C$:
    \begin{align}
        x \in A &\implies \chi_C(x) = 1 = f(x); \\
        x \in B &\implies \chi_C(x) = 0 = f(x).
    \end{align}
    Но тогда для любого $x \in \dom f$ \ $f(x) = \chi_{C}(x)$, то есть $\chi_{C}$ --- тотальное вычислимое продолжение $f$ ${\implies \chi_{C}}$~не~вычислима $\implies$ $C$ не разрешимо.
\end{proof}

\subsection{Главные универсальные вычислимые функции}

\begin{definition}
    Функция $U \colon \N^{2} \pto \N$ называется главной универсальной вычислимой функцией, если
    \begin{enumerate}
        \item $U$ вычислима;
        \item Для любой вычислимой функции $V \colon \N^{2} \pto \N$ существует вычислимая тотальная функция $S \colon \N \to \N$ такая, что
        $$
            \forall x~U(S(n), x) \simeq V(u, x) \iff \forall n~U_{S(n)} = V_{n}.
        $$
    \end{enumerate}
\end{definition}

\begin{statement}
    Если $U$ --- главная универсальная вычислимая функция, то $U$ является у. в. ф.
\end{statement}

\begin{proof}
    Мы хотим, чтобы для любой вычислимой $f \colon \N \pto \N$ существовало $n \in \N$ такое, что $U_{n} = f$.
    Рассмотрим функцию $V$ такую, что $\forall k, x ~ V(k, x) \simeq f(x)$, и $\forall k ~ V_{k} = f$.
    Тогда, по свойству (2) из определения главной у. в. ф., существует вычислимая тотальная функция $S$ такая, что $\forall k~U_{S(k)} = V_{k}$.
    Положив $k$ равным любому числу (например, 2020), получим $U_{S(2020)} = V_{2020} = f \implies n = S(2020)$.
\end{proof}

\begin{statement}
    Существует вычислимая биекция $h \colon \N^{2} \to \N$.
\end{statement}

Введем обозначение для кода пары натуральных чисел $\left< n, m\right> = h(n, m)$.
\begin{statement}
    Существует вычислимые тотальные функции $\pi_{1}$ и $\pi_{2}$ такие, что $\forall n$ $\forall m$ $\pi_{1}(\left< n, m \right>) = n$ и $\pi_{2}(\left< n, m\right>) = m$.
\end{statement}

\begin{proof}
    Без ограничений общности предъявим алгоритм только для $\pi_{1}$.
    \begin{enumerate}[nosep]
        \item Получаем на вход некоторое $z = \left< n, m \right>$ --- код некоторой пары;
        \item Перечисляем все пары $(k, l) \in \N^{2}$ и для каждой проверяем равенство $z$ и $\left< k, l\right>$.
        \item Если да, вернем $k$.
    \end{enumerate}
    Такой алгоритм корректен, потому что любой код, который нам подают на вход корректен в силу сюръективности $h$, а в силу инъективности такая пара --- единственная.
\end{proof}

\begin{theorem}
    Если существует универсальная вычислимая функция $U$, то существует и главная универсальная вычислимая функция.
\end{theorem}

\begin{proof}
    Рассмотрим функцию $W \colon \N^{2} \pto \N$ такую, что 
    $$
        \forall n ~ \forall x ~ W(n, x) \simeq U(\pi_{1}(n), \left<\pi_{2}(n), x\right>).
    $$
    Функция $W$ вычислима как композиция вычислимых функций.
    Покажем, что $W$ является искомой главной универсальной вычислимой функцией.
    Пусть дана вычислимая $V \colon \N^{2} \pto \N$.
    Рассмотрим $V^{\prime} \colon \N \pto \N$ такую, что $V^{\prime} \simeq V(\pi_{1}(x), \pi_{2}(x))$.
    $V^{\prime}$ также вычислима, а значит $\exists m$ такое, что $U_{m} = V^{\prime}$.
    Теперь для любого $n$ положим $S(n) = \left< m, n \right>$, она будет вычислимой и тотальной.
    Проверим, что она подходит:
    \begin{multline}
        \forall n ~ \forall x ~ W(S(n), x) \simeq W(\left< m, n \right>, x) \simeq \\
        \simeq U(\pi_{1}(\left< m, n \right>), \left< \pi_{2}(\left< m, n \right>), x \right> ) \simeq U(m, \left< n, x\right> ) \simeq U_{m}(\left< n, x \right> ) \simeq \\
        \simeq V^{\prime}(\left< n, x\right>) \simeq V(\pi_{1}(\left< n, x \right>), \pi_{2}(\left< n, x \right>)) \simeq V(n, x).
    \end{multline}
    Таким образом, $\forall n ~ W_{S(n)} = V_{n} \implies W$ --- главная у. в. ф.
\end{proof}
