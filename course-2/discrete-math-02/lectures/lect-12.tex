\ProvidesFile{lect-12.tex}[Лекция 12]

\section{Лекция 12}

\subsection{Обобщение логического следования и выполнимости на теории со свободными переменными}

Мы хотим понять, верно ли, что нам достаточно тех предложений, которые содержатся в теории $T$, чтобы выполнялось $\varphi$?
Иначе говоря, мы хотим понять, следует ли из нашего набора аксиом какое-нибудь утверждение.
На прошлой лекции мы доказывали, что в теории порядка из аксиом антирефлексивности и транзитивности следует антисимметричность.
Теперь мы хотим найти способ автоматически (или полуавтоматически) получать все следствия нашей теории.
Хоть у нас и нет разрешающего алгоритма для этой задачи, тем не менее, есть полуразрешающий алгоритм, который позволяет перечислить все следствия нашей теории (механическим путем получить все теоремы).

Как же это сделать?
Прежде всего, нам потребуется сделать следующее обобщение:
\begin{definition}[логическое следование]
    Пусть $\Gamma \subseteq \Fm_{\sigma}$ и $\varphi \in \Fm_{\sigma}$.
    Тогда $\Gamma \models \varphi \iff \forall \MM \forall \pi (\forall \GG \in \Gamma \colon [\GG]_{\MM}(\pi) = 1 \implies [\varphi]_{\MM}(\pi) = 1)$.
    Условие $\forall \GG \in \Gamma \colon [\GG]_{\MM}(\pi) = 1$ равносильно $[\Gamma]_{\MM}(\pi) = 1$.
\end{definition}
Понятно, что если мы возьмем в качестве $\Gamma$ теорию, а в качестве $\pi$ --- предложение, то получится ровно тоже самое, что мы сформулировали на прошлой лекции.

\subsection{Сведение задачи установления логического следования к проверке выполнимости}

\begin{statement}
    $\Gamma \models \varphi \iff \Gamma \cup \{\neg \varphi\}$ не выполнимо.
\end{statement}

Это утверждение несколько похоже на выражение общезначимости через выполнимость.

\begin{proof}
    $\Gamma \cup \{\neg \varphi\}$ не выполнимо $\implies \neg \exists \MM \exists \pi~ ([\Gamma]_{\MM}(\pi) = 1 \land [\neg \varphi]_{\MM}(\pi) = 1) \iff \forall M \forall \pi~ (\neg [\Gamma]_{\MM}(\pi) = 1  \lor \neg [\neg\varphi](\pi) = 1)$.
    Вспомним, что $\neg \alpha \lor \neg \varphi \equiv \alpha \implies \neg \beta$, тогда $\forall M \forall \pi~ (\neg [\Gamma]_{\MM}(\pi) = 1  \lor \neg [\neg\varphi](\pi) = 1) \iff \forall \MM \forall \pi~ ([\Gamma]_{\MM}(\pi) = 1 \implies \neg [\neg\varphi]_{\MM}(\pi) = 1) \iff \forall \MM \forall \pi~ ([\Gamma]_{\MM}(\pi) = 1 \implies [\neg\varphi]_{\MM}(\pi) = 0) \iff \forall \MM \forall \pi~ ([\Gamma]_{\MM}(\pi) = 1 \implies [\varphi]_{\MM}(\pi) = 1) \iff \Gamma \models \varphi$.
\end{proof}

\paragraph{Простыми словами}
Что означает, что из $\Gamma$ следует $\varphi$?
Значит, во всякой структуре, при любой оценке, если верно $\Gamma$, то верно и $\varphi$.
Значит, не существует такой структуры, где $\Gamma$ было бы верно, а $\varphi$ верно не было бы.

\begin{corollary}
    $\Gamma \models \bot \iff \Gamma$ не  выполнима
\end{corollary}

\begin{proof}
    $\Gamma \models \bot \iff \Gamma \cup \{\neg \bot\}$ не выполнимо.
    Но ведь отрицание тождественной лжи --- тождественная истина.
    Но когда выполнима $\Gamma$ и тождественная истина?
    Ну тогда, когда выполнима $\Gamma$, то есть, из $\Gamma$ следует тождественная ложь тогда и только тогда, когда у $\Gamma$ нет модели.
\end{proof}

\paragraph{Вывод}
Проверка логического следования сводится к проверке выполнимости.

\subsection{Сколемизация: теорема о сохранении равновыполнимости}

{\it Тут повторяются все определения и примеры про сколемизацию с прошлой лекции, так что я решил не писать их еще раз.}

\begin{definition}
    $\Gamma$ выполнимо $\iff$ $\exists \MM \exists \pi~ [\Gamma]_{\MM}(\pi) = 1$.
\end{definition}

\begin{definition}
    Множества формул $\Gamma$ и $\Delta$ {\it равновыполнимы} $\iff$ ($\Gamma$ выполнимо $\iff$ $\Delta$ выполнимо).
\end{definition}

\begin{statement}
    Для любой предварённой формулы $\varphi$ такой, что она не содержит фиктивных кванторов, существует сколемовская нормальная форма $\psi$ такая, что $\varphi \twoheadrightarrow_{S} \psi$.
\end{statement}

\begin{proof}
    Применим алгоритм сколемизации и заметим, что кванторов у нас конечное число, поэтому алгоритм точно остановится.
\end{proof}

\begin{theorem}[о сколемизации]
    Пусть $\Gamma \subseteq \Fm$ и $\varphi \to_{S} \psi$.
    Тогда $\Gamma \cup \{\varphi\}$ и $\Gamma \cup \{\psi\}$ равновыполнимы.
\end{theorem}

Что это означает?
Это означает, что у нас было какое-то множество формул (хочется сказать теорий, но в нашем множестве могут быть свободные переменные), мы взяли оттуда одну формулу и подвергли ее сколемизации.
Тогда то, что получилось, равновыполнимо с исходным множеством.

\begin{proof}
    Раз у нас была сколемизация, то $\varphi \eqcirc \forall \vec{x} \exists y \theta$, а $\psi \eqcirc \forall \vec{x}~ \theta(f\vec{x}/y)$, где $f$ \enquote{свежая}, то есть, в частности, $f$ \underline{не встречается} в $\Gamma$.
    Сделаем несколько наблюдений:
    \begin{enumerate}
        \item В формуле $\exists y \theta$ нет кванторов по $\vec{x}$ (иначе в $\varphi$ есть фиктивный квантор) $\implies f\vec{x}$-$y$-$\theta$, ну а раз подстановка корректная, то формула $\theta(f\vec{x}/y) \implies \exists y \theta$ общезначима.
        Поскольку формула общезначима, то, по факту с семинара, формула $\forall \vec{x} \theta(f\vec{x}/y) \implies \forall \vec{x} \exists y \theta$ также общезначима.
        Вспомним про определение $\varphi$ и $\psi$, и заметим, что $\psi \implies \varphi$, то есть, в любой структуре, где верно $\psi$, верно и $\varphi$.
        Тогда если $\exists \MM~ \exists \pi$ такие, что $[\Gamma \cup \{\psi\}]_{\MM}(\pi) = 1$, то $[\Gamma \cup \{\varphi\}]_{\MM}(\pi) = 1$.
        Значит, если $\Gamma \cup \{\psi\}$ выполнима, то $\Gamma \cup \{\varphi\}$ выполнима.
        \item Допустим, что $\Gamma \cup \{ \forall \vec{x} \exists y \theta \}$ выполнима.
        Тогда $\exists \MM$ $\exists \pi$ ($[\Gamma]_{\MM}(\pi) = 1$ и $[\forall \vec{x} \exists y \theta]_{\MM}(\pi) = 1$).
        Заметим, что
        $$
            [\forall \vec{x} \exists y \theta]_{\MM}(\pi) = 1 \iff \forall \vec{a} \in M~ \exists b \in M~ [\theta]_{\MM}(\pi_{x~y}^{\vec{a}~b}) = 1.
        $$
        Теперь рассмотрим $\MM^{\prime} = (\MM, f^{\MM})$ (мы расширили структуру, добавив\footnote{Такая операция называется {\it обогащением} структуры $\MM$.} в нее интерпертацию символа $f$).
        Как мы можем ее расширить?
        Для этого нам нужно определить значение $f^{\MM}(\vec{a})$.
        Положим его равным $b \in M$ такому, что $\forall \vec{a} [\theta]_{\MM}(\pi_{x~y}^{\vec{a}~b}) = 1$, то есть, $\forall \vec{a} \in M^{n}~ [\theta]_{\MM}(\pi_{x~y}^{\vec{a}~f^{\MM}(\vec{a})}) = 1$.
        Такая $f^{\MM}$ существует в силу аксиомы выбора.
        Ну вот мы взяли такую интерпретацию, дальше мы хотим посмотреть, что же будет с $\psi$ и ее значением
        $$
            [\forall \vec{x} \theta (f\vec{x}/y)]_{\MM^{\prime}}(\pi).
        $$
        В старой структуре у нее вообще не было значения, потому что там у нас не интерпретировался символ $f$.
        В новой же структуре
        $$
            [\forall \vec{x} \theta (f\vec{x}/y)]_{\MM^{\prime}}(\pi) = 1 \iff \forall \vec{a} \in M~ [\theta(f\vec{x}/y)]_{\MM^{\prime}}(\pi_{\vec{x}}^{\vec{a}}) = 1.
        $$
        Докажем последнее равенство.
        Мы можем воспользоваться леммой \ref{lemma::correct-substitution} и получится, что
        $$
            [\theta(f\vec{x}/y)]_{\MM^{\prime}}(\pi_{\vec{x}}^{\vec{a}}) = [\theta]_{\MM^{\prime}}\left(\pi_{\vec{x}~y}^{\vec{a}~[f\vec{x}]_{\MM^{\prime}}(\pi_{x}^{\vec{a}})}\right) = [\theta]_{\MM^{\prime}}\left(\pi_{\vec{x}~y}^{\vec{a}~f^{\MM}(\vec{a})}\right) \overset{f\text{ --- свежая}}{=} [\theta]_{\MM}\left(\pi_{x~y}^{\vec{a}~f^{\MM}(\vec{a})}\right) = 1
        $$
        Итак, $[\varphi]_{\MM^{\prime}}(\pi) = 1$.
        А что с $\Gamma$?
        В $\Gamma$ ничего не изменится, поскольку $f$ не встречается в $\Gamma$, поэтому
        $$
            [\Gamma]_{\MM^{\prime}}(\pi) = [\Gamma]_{\MM}(\pi) = 1.
        $$
        Таким образом, $[\Gamma \cup \{ \varphi\}]_{\MM^{\prime}}(\pi) = 1$, то есть $\Gamma \cup \{\psi\}$ выполнима. \qedhere
    \end{enumerate}
\end{proof}

\subsection{Исчисление резолюций}

\subsubsection{Правила резолюции и подстановки}

Введем два {\it правила вывода}.
Что это вообще такое?
Мы не даем общее понятие, но это, своего рода, функции на формулах.

\paragraph{Правило резолюции}
Если есть формулы $\neg A \lor B$ и $A \lor C$, разрешается получить формулу $B \lor C$.
В логике это записывается следующим образом:
$$
    \Resol{\neg A \lor B}{A \lor C}{B \lor C}.
$$
Частный случай:
$$
    \Resol{\neg A}{A}{\bot}.
$$

\begin{statement}[корректность правила резолюции]
    $\forall \MM~ \forall \pi$ если $[\neg A \lor B]_{\MM}(\pi) = 1$ и $[A \lor C]_{\MM}(\pi) = 1$, то $[B \lor C]_{\MM}(\pi) = 1$.
\end{statement}

Другими словами, формула $(((\neg A \lor B) \land (A \lor C)) \implies (B \lor C))$ общезначима.

\begin{proof}
    Пусть $[\neg A \lor B]_{\MM}(\pi) = 1$ и $[A \lor C]_{\MM}(\pi) = 1$, тогда
    \begin{table}[h!]
        \centering
        \begin{tabular}{|l|l|l|l|}
            \hline
            $[A]_{\MM}(\pi)$ & $[B]_{\MM}(\pi)$ & $[C]_{\MM}(\pi)$ & $[\neg A \lor B]_{\MM}(\pi) = 1 \land [A \lor C]_{\MM}(\pi) = 1$ \\ \hline
            0 & 0 & 0 & 0 \\ \hline
            0 & 0 & 1 & 1 \\ \hline
            0 & 1 & 0 & 0 \\ \hline
            0 & 1 & 1 & 1 \\ \hline
            1 & 0 & 0 & 0 \\ \hline
            1 & 0 & 1 & 0 \\ \hline
            1 & 1 & 0 & 1 \\ \hline
            1 & 1 & 1 & 1 \\ \hline
        \end{tabular}
    \end{table}
    Заметим, что на всех наборах, на которых истинны наши формулы, истинна либо $B$, либо $C$, поэтому
    $$
        [\neg A \lor B]_{\MM}(\pi) = 1 \land [A \lor C]_{\MM}(\pi) = 1 \implies [B \lor C](\pi) = 1
    $$
    вне зависимости от $A$.
\end{proof}

\begin{definition}
    {\it Универсальной формулой} называется формула вида $\forall \vec{x} \varphi_{0}$, где $\varphi_{0}$ --- безкванторная.
\end{definition}

\paragraph{Правило подстановки}
Если $A$ универсальная, то из $\forall y A$ разрешается получить формулу $A(t/y)$, где $t$ --- любой терм.

Не требуется, чтобы подстановка была корректной.
То есть, мы можем в универсальную формулу подставить любой терм.
Символически это правило записывают так:
$$
    \Subst{\forall y A}{A(t/y)},
$$
где $A$ --- универсальная.

\begin{statement}[Корректность правила подстановки]
    Пусть $A$ --- универсальная, тогда $\forall \MM~ \forall \pi$ если $[\forall y A]_{\MM}(\pi) = 1$, то $[A(t/y)]_{\MM}(\pi) = 1$.
\end{statement}

\begin{proof}
    $A \eqcirc \forall \vec{x} A_{0}$, где $A_{0}$ --- безкванторная.
    Нам дано, что $[\forall y \forall \vec{x} A_{0}]_{\MM}(\pi) = 1$, это равносильно тому, что $\forall a \in M~\forall \vec{b} \in M^{n}~ [A_{0}](\pi_{y~\vec{x}}^{a~\vec{b}}) = 1$.
    А теперь сделаем следующий фокус: мы же можем переставлять одноименные кванторы местами ($\forall u \forall v \varphi \equiv \forall v \forall u \varphi$), поэтому
    $$
        \forall a \in M~\forall \vec{b} \in M^{n}~ [A_{0}](\pi_{y~\vec{x}}^{a~\vec{b}}) = 1 \iff \forall \vec{b} \in M^{n}~ \forall a \in M~ [A_{0}](\pi_{y~\vec{x}}^{a~\vec{b}}) = 1
    $$
    Мы хотим, чтобы
    $$
        [A(t/y)]_{\MM}(\pi) = 1 \iff [\forall \vec{x}~A_{0}(t/y)]_{\MM}(\pi) = 1.
    $$
    Когда эта штука равна 1?
    Ну тогда и только тогда, когда $\forall \vec{b} \in M^{n}~ [A_{0}(t/y)]_{\MM}(\pi_{\vec{x}}^{\vec{b}}) = 1$, но у нас $A_{0}$ безкванторная, значит совершенно любой терм $t$-$y$-$A_{0}$, потому что там некому связать переменные из терма $t$.
    Значит, по лемме \ref{lemma::correct-substitution},
    $$
        [\forall \vec{x}~A_{0}(t/y)]_{\MM}(\pi) = 1 \iff [A_{0}]_{\MM}\left( \pi_{\vec{x}~y}^{\vec{b}~[t](\pi_{\vec{x}}^{\vec{b}})} \right).
    $$
    Для любого $\vec{b}$ положим $a = [t](\pi_{\vec{x}}^{\vec{b}})$, тогда $\forall \vec{b} \in M^{n}~ [A_{0}](\pi_{\vec{x}~y}^{\vec{b}~[t](\pi_{\vec{x}}^{\vec{b}})}) = 1$ потому что $\pi_{\vec{x}~y}^{\vec{b}~a} = \pi_{y~\vec{x}}^{a~\vec{b}}$ в силу того, что переменные $\vec{x}$ и $y$ различаются.
    Ну все, мы получили, что
    $$
        [A_{0}]_{\MM}\left( \pi_{\vec{x}~y}^{\vec{b}~[t](\pi_{x}^{\vec{b}})} \right) = 1 \iff [A(t/y)]_{\MM}(\pi) = 1.
    $$
\end{proof}

\subsubsection{Выводимость в ИР}

\paragraph{Исчисление резолюций}
Будем говорить неформально, пусть $\Vir{\Gamma}{\varphi}$ (читается как \enquote{$\varphi$ выводится из $\Gamma$ в исчислении резолюций}), это равносильно тому, что $\varphi$ можно получить из формул в $\Gamma$, применяя к ним правила резолюции и правила подстановки конечное число раз.

\paragraph{От автора}
К сожалению, у меня больше нет ни сил, ни времени, чтобы дотехать конспект лекции по дискре, поэтому сорри, дальше нужно только смотреть лекции (продолжение текущей лекции начинается на 1:24:00)

% \subsubsection{Корректность ИР}

% \begin{theorem}
%     Если $\Vir{\Gamma}{\varphi}$, то $\Gamma \models \varphi$.
% \end{theorem}

% \begin{proof}
%     Мы ведь знаем, что само правило резолюций корректно.
% \end{proof}

% \subsubsection{Универсальные дизъюнкты}

% \subsubsection{Применение ИР к теориям произвольного вида}

% \subsubsection{Формулировка теоремы о полноте ИР}