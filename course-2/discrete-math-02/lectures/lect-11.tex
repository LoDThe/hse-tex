\ProvidesFile{lect-11.tex}[Лекция 11]

\section{Лекция 11}

Давайте перейдем теперь к идейному содержанию построенной нами теории.
Зачем нам все это нужно?
Логика, по сути дела, изучает, что делает математика.
В частности, нам хочется понять, что вообще с этой логикой можно делать.
Давайте, например, попробуем поговорить о группах.
Мы можем смотреть на группу как на некоторую структуру
$$
    \GG = (A; =; +^{(2)}, -^{(1)}; 0),
$$
но ведь не всякая интерпретация такой структуры является группой.
Мы хотим, что бы наша бинарная операция была ассоциативна, то есть
$$
    \GG \models \forall x \forall y \forall z ~ (x + y) + z = x + (y + z).
$$
Таким образом мы получили полугруппу.
Какие еще свойства должны выполняться в группе?
Нам нужно, чтобы для каждого элемента существовал обратный к нему, то есть
$$
    \GG \models \forall x~ (x + (-x) = 0).
$$
Будем конструировать абелеву группу, поэтому потребуем еще и коммутативность:
$$
    \GG \models \forall x~ (x + (-x) = 0 \land (-x) + x = 0).
$$
Осталось сказать, что 0 является нейтральным элементом, то есть
$$
    \GG \models \forall x~ (x + 0 = x \land 0 + x = x).
$$

Всего этого достаточно, чтобы $\GG$ была группой, однако, с другой стороны, у нас получилось три формулы.
И вот эти формулы являются аксиомами группы.
Назовем три предложения, которые обозначают наши аксиомы, соответственно $\varphi_{\text{ass}}$, $\varphi_{\text{inv}}$ и $\varphi_{\text{neut}}$ и рассмотрим множество
$$
        T = \{\varphi_{\text{ass}}, \varphi_{\text{inv}}, \varphi_{\text{neut}}\}.
$$
Мы, по сути дела, сказали, что
$$
    \forall \text{ нормальной } \GG \quad (\GG \text{ --- группа } \iff \forall \varphi \in T~ \GG \models \varphi.)
$$
Нормальность значит, что символ равенства обозначает равенство в нашем привычном понимании.
Поскольку любая группа удовлетворяет предложениям из $T$, можно сказать, что $T$ --- теория групп.

Давайте теперь подойдем с другой стороны.
Рассмотрим структуру $\MM = (A; =; <)$.
Когда такая штука является частично упорядоченным множеством?
Можем ли мы написать формулы, которые выполняются тогда и только тогда, когда она является порядком\footnote{тоже самое, что и частично упорядоченное множество.}?
\begin{description}
    \item[Антирефлексивность] $\MM \models \varphi_{\text{antireflex}} \eqcirc \forall x~ \neg (x < x)$;
    \item[Транзитивность] $\MM \models \varphi_{\text{transit}} \eqcirc \forall x \forall y \forall z~ (x < y \land y < z \implies x < z)$.
\end{description}
Утверждается, что любая нормальная $\MM$ является частично упорядоченным множеством $\iff \forall \varphi \in T_{\text{ord}} = \{\varphi_{\text{antireflex}}, \varphi_{\text{transit}}\}~ \MM \models \varphi$.
Что мы видим?
Разные классы математических структур можно задавать просто записав множество предложений, которые выполняются тогда и только тогда, когда наша структура является тем, что мы описываем (группой, порядком).

\subsection{Логическое (семантическое) следствие}

\begin{definition}
    {\it Теория} (в сигнатуре $\sigma$) --- это любое множество предложени в сигнатуре $\sigma$.
\end{definition}

\begin{definition}
    $\MM \models T \iff \forall \varphi \in T~ \MM \models \varphi$.
\end{definition}

\begin{definition}
    Теория $T$ {\it выполнима} (или {\it совместна}) $\iff \exists \MM~ \MM \models \TT$.
\end{definition}

\begin{statement}[антисимметричность порядка]
    Для любой модели $\MM$
    $$
        \MM \models T_{\text{ord}} \implies \MM \models \forall x \forall y~ (x < y \land y < x \implies x = y).
    $$
\end{statement}

\begin{proof}
    Мы знаем, что $\MM \models \forall x~ \neg x < x$ и $\MM \models \forall x \forall y \forall z~ (x < y \land y < z \implies x < z)$.
    Хотим, чтобы $\MM \models \forall x \forall y~ (x < y \land y < x \implies x = y)$.
    Зафиксируем $a, b \in M$, тогда $a <^{\MM} b$ и $b <^{\MM} a$.
    Тогда, по транзитивности, $a <^{\MM} a$, а по антирефлексивности $\neg a <^{\MM} a$.
    Получили противоречие, то есть $\forall a, b \in \MM$
    $$
        \MM \models \neg(a <^{\MM} b \land b <^{\MM} a),
    $$
    тогда $\MM \models (x < y \land y < x \implies x = y)(a, b)$, потому что импликация истинна для произвольных $a$, $b$.
\end{proof}

В некотором смысле, $\varphi_{\text{antisym}}$ следует из $\varphi_{\text{antireflex}}$ и $\varphi_{\text{transit}}$, то есть, следует из $T_{\text{ord}}$.

\begin{definition}
    Предложение $\varphi$ {\it логически следует} из теории $T$, если $\forall \MM~ (M \models T \implies \MM \models \varphi)$.
\end{definition}

\paragraph{Обозначение}
$T \models \varphi$ (из теории $T$ следует $\varphi$)

Что значит логическое следование?
Оно значит, что в любой структуре, где верна теория $T$, верно и предложение $\varphi$.

Вернемся к теории групп.
Выразим, что нейтральный элемент единственный.
Для начала, нам потребуется выразить условие нейтральности элемента:
$$
    \psi \eqcirc \forall y\forall x~ (x + y = x \land y + x = x) \implies y = 0.
$$
Такая формула является фактом из теории групп, но как это доказать?

\begin{statement*}
    $T_{\text{гр}} \models \psi$ ($\iff \forall \text{ нормальной } \MM~(\MM \models T_{\text{гр}} \implies \MM \models \psi)$).
\end{statement*}

\begin{proof}
    Нам дано, что у нас есть нормальная структура $\MM$ и $\MM \models T_{\text{гр}}$, то есть
    \begin{align}
        \MM &\models \varphi_{\text{ass}}, \\
        \MM &\models \varphi_{\text{inv}}, \\
        \MM &\models \varphi_{\text{neut}}.
    \end{align}
\end{proof}

\begin{proof}
    Зафиксируем $b \in M$ и допустим, что $\MM \models (\forall x~ (x + y = x \land y + x = x))(b)$.
    Мы хотим, чтобы $b = 0^{\MM}$.
    Рассуждаем как в обычной математике.
    Нам нужно подставить в качестве $x$ $0^{\MM}$, тогда
    $$
        0^{\MM} + b = 0^{\MM},
    $$
    но из $\varphi_{\text{neut}}$ мы знаем, что $0^{\MM} + z = z$, тогда
    $$
        0^{\MM} + b = 0^{\MM} = b.
    $$
\end{proof}

Давайте напоследок введем еще одну теорию, которая нам понадобится --- теория $\DLO$\footnote{Dense Linear Order.}:
$$
    \DLO = T_{\text{ord}} \cup \{\forall x \forall y (x < y \lor y < x \lor x = y), \forall x \forall y (x < y) \implies \exists z (x < z \land z < y), \forall x \exists y~ x < y, \forall x \exists y~ y < x\}.
$$
Наш $\DLO$ отличается от общепринятого $\DLO$ отсутствием наибольшего и наименьшего элементов.
Для любой нормальной $\MM$ ($\MM \models \DLO \iff \MM$ --- плотный линейный порядок без максимума и минимума).

\paragraph{Примеры}
$(\R, <) \models \DLO$; $(\Q, <) \models \DLO$; $(\R \cap (0, 1), <) \models \DLO$.

\paragraph{Контрпримеры}
\begin{itemize}
    \item $(\N, <) \centernot\models \DLO$ --- присутствует минимум;
    \item $(\Z, <) \centernot\models \DLO$ --- нет максимума и минимума, но порядок не плотный;
    \item $(R \cap [0, 1], <) \centernot\models \DLO$ --- присутствует и минимум, и максимум.
\end{itemize}

\subsection{Теорема компактности (без доказательства)}

До сих пор мы рассматривали примеры, в которых логика нам не давала ничего кроме лишних мучений.
Вот, например, такой вопрос: предположим, что мы знаем, что какая-то формула логически следует из теории (т.е. $T \models \varphi$), всегда ли существует конечное подмножество $T^{\prime} \subseteq T$ такое, то $T^{\prime} \models \varphi$?
Учитывая тот факт, что теория может быть бесконечной, такое утверждение сложно доказать.
Тем не менее, оно верно и называется {\it теоремой о компактности}.
Выглядит оно может быть и не очень презентабельно, но из него следует куча всяких вещей.
Например, мы можем с вами доказать, что не существует теории, которая имеет модели сколь угодно большой конечной мощности, но не имеет бесконечной модели.
Пока мы просто поверим в эту теорему, а потом ее докажем.

\begin{statement}
    Можем ли мы придумать такую теорию $T_{\text{fin}}$ конечных множеств, что для любой нормальной $\MM$ ($\MM \models T_{\text{fin}} \iff M \text{ конечна}$)?
\end{statement}

\subsection{Невозможность аксиоматизации класса конечных нормальных структур}

\subsection{Сколемизация}