\ProvidesFile{lect-07.tex}[Лекция 7]

\section{Лекция 7}

В прошлый раз мы разобрались с формальным определением формул 1-го порядка.
Теперь надо понять, что же эти формулы означают.
Рассмотрим, например, формулу
$$
    \exists x \quad x + y = 3.
$$
Вообще говоря, значение этой формулы зависит от того, что поставлено в соответствие пременной $y$.
Поэтому, значение формулы всегда представляет из себя {\it что-то}, зависящее от значения переменных.
Поэтому, если всем переменным придать значение из какого-нибудь множества, то мы желаем, чтобы значением формулы оказалось либо {\it да}, либо {\it нет}.
То есть, формула, при фиксированных значениях переменных, либо ложна, либо истинна.
Ну а значениями термов, из которых строятся такие формулы (например, $x + y$ или 3) должны быть объекты из некоторого множества (например, числа).
Ну и кроме того, значение формулы зависит от того, что мы понимаем под функциональными и реляционными отношениями.
Например, если мы пишем \enquote{+}, то он может обозначать разные вещи в зависимости от объектов, которым он применяется (сложение матриц, сложение чисел).
Отсюда видно, что нам нужно зафиксировать множество, откуда будут браться значения переменных, функциями на котором мы будем интерпертировать наши функциональные символы, отношения на котором мы будем интерпретировать символами отношений, и элементами которого мы будем интерпертировать константы.
Разумеется, понимание кванторов тоже будет опираться на это множество ($\exists x$ предполагает, что множество значений для $x$ фиксированно).

\subsection{Интерпретация сигнатуры}

Попробуем теперь все тоже самое провести абстрактно.
Допустим, у нас дана некоторая сигнатура 
$$
    \sigma = (\Rel_{\sigma}, \Func_{\sigma}, \Const_{\sigma}),
$$
тогда
\begin{definition}
    Интерпретация сигнатуры $\sigma$ --- это пара $\MM = (M, I_{\MM})$, где $M \neq \varnothing$, а $I_{\MM}$ --- это такое отображение\footnote{Дашков не сказал, откуда и куда, но сказал, что оно страшное :(}, что
    \begin{enumerate}
        \item $\forall R^{(n)} \in \Rel_{\sigma} \quad I_{\MM}(R) \subseteq M^{n}$, то есть $I_{\MM}(R)$ --- это $n$-арное отношение на $M$;
        \item $\forall f^{(m)} \in \Func_{\sigma} \quad I_{\MM}(f) \colon M^{m} \to M$;
        \item $\forall c \in \Const_{\sigma} \quad I_{\MM}(c) \in M$.
    \end{enumerate}
    Множество $M$ называется {\it носителем интерпретации} $\MM$.
\end{definition}
То есть, что делает функция $I$?
Она каждому символу отношения ставит в соответствие отношение соответствующей валентности, каждому символу функциональному ставит в соответствиее функцию, каждому константному символу ставит в соответствие определенный элемент.
То есть, у нас были просто символы, а теперь мы их наделили значением.
Получается, что $I_{\MM}(*)$ --- это {\it интерпретация} символа $*$ в $\MM$.
Принято писать $R^{\MM}, f^{\MM}, c^{\MM}$ вместо, соответственно, $I_{\MM}(R)$, $I_{\MM}(f)$, $I_{\MM}(c)$.
С другой стороны, что такое интерпретация?
Можно сказать, что интерпретация --- это какое-то множество, и на нем заданы (возможно, бесконечные) список отношений, функций и элементов, причем такой, что они поставлены в соответствие символам сигнатуры.
Это значит, что можно было бы за первчиный объект брать саму эту интерпретацию.
На самом деле, это более логично, и в таком случае мы можем интерпретацию назвать $\sigma$-структурой.
Почему структурой?
Вообще, структурой в математике называют множество и какой-то список отношений, функций и элементов на нем, а $\sigma$-структура значит, что она поставлена в соответствие сигнатуре $\sigma$.

Давайте рассмотрим какой-нибудь пример.
Вообще говоря, я уже отмечал в прошлый раз, что нам в логике важно отделить символ от его значения.
Почему это так?
Ведь когда мы работаем с ними в анализе или в теории вероятностей, или еще где-то, мы этого не делаем.
Там значения символов, на самом деле, очень жестко фиксировано, если мы не говорим о переменных.
Ну а в логике нам важно рассматривать такие ситуации, когда одному и тому же символу могут быть приданы разные значения, и смотреть на взаимоотношение таких различных интерпретаций.

\begin{convention}
    Следующие обозначения эквивалентны:
    $$
        (x_{1}, \ldots, x_{n}) \in R^{\MM} \iff R^{\MM}(x_{1}, \ldots, x_{n}). 
    $$
\end{convention}

\paragraph{Пример}
Рассмотрим сигнатуру $\sigma = (\Rel_{\sigma} = \{ Q^{(3)} \}; \Func_{\sigma} = \{ f^{(2)}, \#^{(1)} \}; \Const_{\sigma} = \{ \$ \})$ и две ее интерпретации
$$
    \MM = ( \{ 0, 1 \}; Q^{\MM}; f^{\MM}, \#^{\MM}; \$^{\MM}),
$$
где 
\begin{align}
    &Q^{\MM} \subseteq \{0, 1\}^{3} \colon (x, y, z) \in Q^{\MM} \iff x = y = z \implies Q^{\MM} = \{ (0, 0, 0), (1, 1, 1)\}; \\
    &f^{\MM}(x, y) = x \oplus y, \quad \#^{\MM}(x) = x; \\
    &\$^{\MM} = 1.
\end{align}
Мы получили какую-то интерпретацию $\sigma$.
Какого-то особого смысла в ней нет, но как пример она вполне себе подойдет.

Можно рассмотреть и другую интерпретацию $\NN = (\N; Q^{\NN}; f^{\NN}, \#^{\NN}; \$^{\NN})$:
\begin{align}
    &(x, y, z) \in Q^{\NN} \iff x + y = z; \\
    &f^{\NN}(x, y) = (x \cdot y)^{2}, \quad \#^{\NN}(x) = x + 7; \\
    &\$^{\NN} = 2020.
\end{align}
В ней также нет ничего содержательного, просто она отличается от $\MM$.

\subsection{Оценка переменных}

Все-таки, нашей конечной целью является определение значения формулы и значения терма.
Мы определили интерпертацию, то есть у нас теперь имеются значения наших базовых символов.
А как же нам дать значение формулы и значение терма?
Мы понимаем, что оно зависит от значения переменных.
Как же нам придать значение переменным?

\begin{definition}
    {\it Оценкой переменных} в интерпертации $\MM$ называется любая функция\footnote{такая функция ставит каждой переменной в соответствие элемент из носителя интерпретации} $\pi \colon \Var \to M$.
\end{definition}

Например, можно считать, что всем переменным с четными номерами ставится в соответствие 5, а всем переменным с нечетными номерами ставится в соответствие их номер.
Получится оценка переменных в любой интерпретации, где носителем выступает множество натуральных чисел.

Итак, оценка переменных принимает на вход сам символ (не его значение, до него мы пока даже не дошли, а именно символ) и этому символу ставит в соотвествие элемент нашего носителя.

\subsection{Значение терма в интерпретации при оценке}

\begin{notation*}
    Будем обозначать через $[t]_{\MM}(\pi)$ значение терма $t$ в интерпретации $\MM$ при оценке переменных $\pi$.
\end{notation*}

\begin{definition}
    Определим значение терма $t \in \Term_{\sigma}$ в интерпретации $\MM$ при оценке $\pi$ рекурсивно:
    \begin{enumerate}
        \item $x \in \Var \implies [x]_{\MM}(\pi) = \pi(x)$;
        \item $x \in \Const_{\sigma} \implies [c]_{\MM}(\pi) = c^{\MM}$;
        \item $[f t_{1} \ldots t_{m}]_{\MM}(\pi) = f^{\MM}([t_{1}]_{\MM}(\pi), \ldots, [t_{n}]_{\MM}(\pi))$.
    \end{enumerate}
\end{definition}

\paragraph{Пример}
Давайте возьмем нашу дикую интерпретацию $\NN$ и будем примеры приводить в ней.
Какой у нас здесь есть терм?
Ну, например,
$$
    f(\# x)x.
$$
Какое же у него значение?
Оно зависит, естественно, от оценки.
Давайте ее посчитаем:
$$
    \left[ f(\# x)x \right]_{\NN}(\pi) = f^{\NN}\left( [\# x](\pi), [x](\pi) \right) = f^{\NN}\left( \#^{\NN}([x](\pi)), [x](\pi) \right) = f^{\NN}\left( \#^{\NN}(\pi(x)), \pi(x) \right).
$$
Как видите, все свелось к тому, что нужно знать, что будет поставлено в соответствие $x$ функцией $\pi$.
Допустим, $\pi(x) = 9$, тогда
$$
    \left[ f(\# x)x \right]_{\NN}(\pi) = f^{\NN}\left( \#^{\NN}(\pi(x)), \pi(x) \right) = ((9 + 7) \cdot 9)^{2}.
$$
Здесь $\#^{\NN}(\pi(x)) = \#^{\NN}(9) = 9 + 7$.

Итак, мы определили рекурсивно значение терма, дальше следует одно простое замечание: какая бы ни была оценка, нам совершенно безразлично как она оценит те переменные, которые в нашем терме не встречаются.
Вот это совершенно простое, но фундаментально важное наблюдение.

\begin{lemma} \label{lem::evaluation-of-variables-in-term}
    Если $\forall y \in V(t)$\footnote{см. определение \ref{def::variables-in-term}} $\pi_{1}(y) = \pi_{2}(y)$, то
    $$
        [t]_{\MM}(\pi_{1}) = [t]_{\MM}(\pi_{2}).
    $$
\end{lemma}

\begin{proof}
    Приведем доказательство индукцией по построению терма $t$.
    \begin{enumerate}
        \item Допустим, что $t = x \in \Var$, тогда $x \in V(t)$.
        Имеем $[t](\pi_1) = \pi_{1}(x)$, $[t](\pi_{2}) = \pi_{2}(x)$, но, поскольку $x \in V(t)$, то $\pi_{1}(x) = \pi_{2}(x)$.
        \item Допустим, что $t = c \in \Const_{\sigma}$, тогда $[t](\pi_{1}) = c^{\MM} = [t](\pi_{2})$.
        \item Допустим, что $t = f t_{1} \ldots t_{n}$, тогда, по предположению индукции, $\forall i~(\forall y \in V(t_{i})~\pi_{1}(y) = \pi_{2}(y) \implies [t_{i}](\pi_{1}) = [t_{i}](\pi_{2}))$.
        Мы хотим доказать, что значение большого терма будет одинаковым при любых двух оценках, которые совпадают на всех встречающихся в нем переменных.
        То есть, мы допускаем, что $\forall y \in V(t)~(\pi_{1}(y) = \pi_{2}(y))$.
        Теперь заметим, что $V(t) = V(t_{1}) \cup \ldots \cup V(t_{n})$, то есть множество переменных в терме --- это объединение множеств переменных в каждом вложенном в него терме.
        Итак,
        $$
            [f t_{1} \ldots t_{n}](\pi_{1}) = f^{\MM}([t_{1}](\pi_{1}), \ldots, [t_{n}](\pi_{1})).
        $$
        Любой $y$, встречающийся в терме $t_{i}$, встречается и в терме $t$ (и наоборот).
        Значит, по предположению индукции, $\forall i~ [t_{i}](\pi_{1}) = [t_{i}](\pi_{2})$, значит
        $$
            f^{\MM}([t_{1}](\pi_{1}), \ldots, [t_{n}](\pi_{1})) = f^{\MM}([t_{1}](\pi_{2}), \ldots, [t_{n}](\pi_{2})).
        $$
        Тогда получается, что
        $$
            [f t_{1} \ldots t_{n}](\pi_{1}) = [f t_{1} \ldots t_{n}](\pi_{2}).
        $$ \qedhere
    \end{enumerate}
\end{proof}

Содержательно это утверждение можно сформулировать так: значение терма зависит от значения оценки только тех переменных, которые в нем встречаются.
Теперь мы определим значение формулы.
Как это сделать?
Формула, как известно, строится с помощью термов, символов отношений, логических связок и кванторов.

\subsection{Значение формулы в интерпретации при оценке}

\begin{notation*}
    Будем обозначать через $[\varphi]_{\MM}(\pi) \in \{0, 1\}$ значение формулы $\varphi \in \Fm_{\sigma}$ в интерпретации $\MM$ при оценке переменных $\pi$.
\end{notation*}

\paragraph{Замена оценки переменной}
Пусть нам дана оценка $\pi \colon \Var \to \MM$.
Давайте заведем оценку, которая от нее отличается строго в одной точке (переменной).
Пусть $y \in \Var$ и $m \in M$, определим оценку $(\pi + (y \mapsto m)) \colon \Var \to M$ (альтернативное обозначение --- $\pi_{y}^{m}$) такую, что
$$
    \pi_{y}^{m}(x) = \begin{cases}
        m, & x \eqcirc y, \\
        \pi(x), & x \neq y.
    \end{cases}
$$
Обратим внимание на знак \enquote{$\eqcirc$}, он обозначает равенство не по значению переменной (его мы все еще не ввели), а именно что они синтаксически совпадают (за $x$ скрывается какая-то буква из алфавита $\Var$, за $y$ скрывается какая-то буква из алфавита $\Var$, и если эти буквы совпали, то мы вернем $m$, иначе вернем $\pi(x)$).
Иначе говоря, мы меняем значение функции $\pi$, которая на вход принимает переменные, в одной точке.

\begin{definition}
    Определим значение формулы $\varphi$ в интерпретации $\MM$ при оценке переменных $\pi$ рекурсивно:
    \begin{enumerate}
        \item $[R t_{1} \ldots t_{n}]_{\MM}(\pi) = \begin{cases}
            1, & ([t_{1}](\pi), \ldots, [t_{n}](\pi)) \in R^{\MM}; \\
            0, & \text{иначе};
        \end{cases}$
        \item $[\varphi \land \psi]_{\MM}(\pi) = \text{И}([\varphi]_{\MM}(\pi), [\psi]_{\MM}(\pi))$, $[\varphi \lor \psi]_{\MM}(\pi) = \text{ИЛИ}([\varphi]_{\MM}(\pi), [\psi]_{\MM}(\pi))$, $[\neg\varphi]_{\MM}(\pi) = \text{НЕ}([\varphi]_{\MM}(\pi))$, и так далее;
        \item 
        \begin{align}
            &[\forall x~\varphi]_{\MM}(\pi) = \begin{cases}
                1, & \text{для всех } m \in M : [\varphi](\pi_{x}^{m}) = 1, \\
                0, & \text{иначе}.
            \end{cases} \\
            &[\exists x~\varphi]_{\MM}(\pi) = \begin{cases}
                1, & \text{существует } m \in M : [\varphi](\pi_{x}^{m}) = 1, \\
                0, & \text{иначе}.
            \end{cases}
        \end{align}
    \end{enumerate}
\end{definition}

Обратим внимание на то, что мы кванторы существования и всеобщности определяем через кванторы существования и всеобщности.
C этим ничего не поделаешь, ведь мы не пытаемся объяснить, что такое квантор всеобщности или квантор существования, а просто пытаемся формализовать это, научиться формальное выражение исследовать с помощью нашей неформлаьной логики.

\paragraph{Пример}
Рассмотрим следующую формулу $\varphi$ в интерпретации $\NN$:
$$
    \varphi = \exists y~ \forall y ~ Qxyy.
$$
Это замкнутная формула, то есть она не содержит свободных переменных.
Ее значение не зависит от оценки переменных.
Посчитаем ее оценку при $\pi$:
$$
    [\exists x~ \forall y~ Qxyy](\pi) = 1 \iff \text{существует } a \in \NN~ [\forall y~ Qxyy](\pi_{x}^{a}) = 1,
$$
распишем последнюю оценку:
\begin{multline}
    [\forall y~ Qxyy]\left(\pi_{x}^{a}\right) = 1 \iff \text{для всех } b \in \NN [Qxyy]\left( \pi_{x~y}^{a~b} \right) = 1 \iff [x]\left(\pi_{x~y}^{a~b}\right) + [y]\left(\pi_{x~y}^{a~b}\right) = [y]\left(\pi_{x~y}^{a~b}\right) \iff \\ \iff \pi_{x}^{a}(x) + \pi_{x~y}^{a~b}(y) = \pi_{x~y}^{a~b}(y) \iff a + b = b,~\text{положим } a = 0, \text{ тогда } 0 + b = b \implies \text{формула $\varphi$ истинна}.
\end{multline}

\begin{lemma}
    Если $\forall y \in \FreeVar(\varphi)~\pi_{1}(y) = \pi_{2}(y)$, то
    $$
        [\varphi]_{\MM}(\pi_{1}) = [\varphi]_{\MM}(\pi_{2}).
    $$
\end{lemma}

\begin{proof}
    Приведем доказательство индукцией по построению формулы $\varphi$:
    \begin{enumerate}
        \item $\varphi \eqcirc R t_{1} \ldots t_{n}$, тогда $[\varphi](\pi_{1}) = 1 \iff R^{\MM}([t_{1}](\pi_{1}), \ldots, [t_{n}](\pi_{1}))$.
        Вспомним, что $\FreeVar(R t_{1} \ldots t_{n}) = V(t_{1}) \cup \ldots \cup V(t_{n})$, тогда $V(t_{i}) \subseteq \FreeVar(R t_{1}, \ldots, t_{n})$, тогда для всех переменных, встречающихся в $t_{i}$, наши оценки ведут себя одинаково $\implies$ значения термов при этих оценках одинаковы по лемме \ref{lem::evaluation-of-variables-in-term}.
        
        Тогда $R^{\MM}([t_{1}](\pi_{1}), \ldots, [t_{n}](\pi_{1})) \iff R^{\MM}([t_{1}](\pi_{2}), \ldots, [t_{n}](\pi_{2})) \iff [\varphi](\pi_{2}) = 1$;
        \item $[\theta \land \psi](\pi) = \text{И}([\theta](\pi_{1}), [\psi](\pi_{1}))$.
        Вспомним, что $\FreeVar(\theta \land \psi) = \FreeVar(\theta) \cup \FreeVar{\psi} \implies \FreeVar(\theta) \subseteq \FreeVar(\theta \land \varphi)$, тогда, по предположению индукции, $\text{И}([\theta](\pi_{1}), [\psi](\pi_{2})) = \text{И}([\theta](\pi_{2}), [\psi](\pi_{2})) = [\theta \land \psi](\pi_{2})$.
        Для других логических функций аналогично.
        \item Разберем только квантор всеобщности.
        Допустим $\varphi = \forall z~ \psi$.
        Нам дано, что $\forall y \in \FreeVar(\forall z~ \psi)$ $\pi_{1}(y) = \pi_{2}(y)$, тогда
        $$
            [\forall z~ \psi]_{\MM}(\pi_1) = 1 \iff \forall m \in M~ [\psi](\pi_{1} + (z \mapsto m)) = 1.
        $$
        По предположению индукции, мы знаем, что если у нас две оценки совпадают на всех переменных, свободных в $\psi$, то и значения соответствующие тоже совпадают.
        Поскольку $\FreeVar(\forall z~\psi) = \FreeVar(\psi) \setminus \{z\}$, то $\FreeVar(\psi) \subseteq \FreeVar(\forall z~ \psi) \cup \{z\}$.
        \begin{statement}
            $\forall y \in \FreeVar(\psi)$ $(\pi_{1} + (z \mapsto m))(y) = (\pi_{2} + (z \mapsto m))(y)$.
        \end{statement}
        \begin{proof}
            $y \in \FreeVar(\psi)$, тогда возможны два случая:
            \begin{enumerate}
                \item $y \in \FreeVar(\forall z~\psi) = \FreeVar(\varphi)$, тогда $y \neq z$ потому что $z$ не входит свободно в $\psi$.
                Значит, $(\pi_{1} + (z \mapsto m))(y) = \pi_{1}(y) = \pi_{2}(y) = (\pi_{2} + (z \mapsto m))(y)$.
                \item $y \eqcirc z \implies (\pi_{1} + (z \mapsto m))(y) = m = (\pi_{2} + (z \mapsto m))(y)$. \qedhere
            \end{enumerate}
        \end{proof}
        По предположению индукции, для более простых формул наше утверждение верно.
        Мы знаем, что для более простой формулы $\psi$ верно, что если какие-то две оценки совпадают на свободных в ней переменных, то значения этой более простой формулы при таких оценках обязаны совпасть:
        $$
            [\psi](\pi_{1} + (z \mapsto m)) = [\psi](\pi_{2} + (z \mapsto m)).
        $$
        Получается, что для любого $m \in M$ $[\psi](\pi_{2} + (z \mapsto m)) = 1 \implies [\forall z~\psi](\pi_{2}) = 1$. \qedhere
    \end{enumerate}
\end{proof}

Итак, мы получили очень важную теорему: значение формулы при оценке зависит только от значения этой оценки на переменных, свободно входящих в формулу.
Введем теперь более удобные обозначения, с которыми мы и будем, в основном, работать.

\begin{notation*}
    Зафиксируем набор различных переменных $(x_{1}, \ldots, x_{n}) \in \Var^{n}$.
    Будем писать $\varphi(x_{1}, \ldots, x_{n})$, если $\FreeVar(\varphi) \subseteq \{x_{1}, \ldots, x_{n}\}$.
\end{notation*}

\begin{notation*}
    Пусть $\vec{a} = (a_{1}, \ldots, a_{n}) \in M^{n}$, тогда $\forall \varphi(x_{1}, \ldots, x_{n})$, $\forall \vec{a}$
    $$
        \MM \models \varphi(\vec{a}) \iff \forall\pi~[\varphi]_{\MM}\left(\pi_{x_{1}~x_{2}~\ldots~x_{n}}^{a_{1}~a_{2}~\ldots~a_{n}}\right).
    $$
    Читается это следующим образом: в $\MM$ истинна $\varphi$ на наборе $\vec{a}$.
\end{notation*}

\paragraph{Пример}
Зафиксируем интерпретации $\MM = (\N; =^{\N}; +^{\N}, \cdot^{\N}; 0, 1, 2)$ и $\NN = (\Z; =^{\Z}; +^{\Z}, \cdot^{\Z}; 0, 1, 2)$.
Рассмотрим формулу $\varphi \eqcirc (x + 1 = 0)$, тогда
\begin{align}
    &\MM ~\cancel{\models}~ \varphi(5), \\
    &\MM ~\cancel{\models}~ \varphi(0), \\
    &\NN \models \varphi(-1).
\end{align}
Соответственно, мы можем сказать, что
$$
    \NN \models \exists \varphi~\varphi.
$$
На каком наборе?
А ни на каком, потому что набор может быть пустым, например, $\FreeVar(\exists x~ (x + 1 = 0)) = \varnothing$.

\begin{definition}
    Формула $\varphi(x_{1}, \ldots, x_{n}) \in \Fm_{\sigma}$ {\it выражает отношение} $X \subseteq M^{n}$ в интерпретации $\MM$ сигнатуры $\sigma$ $\iff$
    $$
        \forall \vec{a} \in M^{n}~(\MM \models \varphi(\vec{a}) \iff \vec{a} \in X).
    $$
    В такой ситуации пишут еще, что $\varphi^{\MM} = X$.
\end{definition}
