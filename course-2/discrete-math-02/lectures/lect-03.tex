\ProvidesFile{lect-03.tex}[Лекция 3]

\section{Лекция 3}

\subsection{Отношение m-сводимости}

\begin{definition}
    Пусть $A, B \subseteq \N$.
    Говорят, что $A$ $m$-сводится к $B$ (обозн. $A \mreduce B$) тогда и только тогда, когда существует вычислимая тотальная $f \colon \N \to \N$ такая, что
    $$
        \forall n~ n \in A \iff f(n) \in B.
    $$
    В таком случае можно написать $A \mreduce^{f} B$.
\end{definition}

\begin{lemma} \label{lemma::03::01::01}
    Для любых $A, B, C \subseteq \N$ справедливо следующее:
    \begin{enumerate}
        \item $A \mreduce^{Id} A$;
        \item $A \mreduce^{f} B$ и $B \mreduce^{g} C$ $\implies$ $A \mreduce^{g \circ f} C$;
        \item $A \mreduce^{f} B \implies A^{\complement} \mreduce^{f} B^{\complement}$.
        \item $A \mreduce B$ и $B$ разрешимо $\implies$ $A$ разрешимо.
        \item $A \mreduce B$ и $B$ перечислимо $\implies$ $A$ перечислимо.
        \item Для любого $n \in \N$, $\omega_{A}(n) = 1 \iff \omega_{B}(f(n)) = 1$.
    \end{enumerate}
\end{lemma}

\begin{corollary}
    Если $A$ неразрешимо (неперечислимо), и $A \mreduce B$, то $B$ неразрешимо (неперечислимо).
\end{corollary}
\begin{proof}
    Записать контрапозицию для пунктов (4), (5) леммы \ref{lemma::03::01::01}.
\end{proof}

\begin{statement}
    Если $A$ разрешимо и $\varnothing \neq B \neq \N$, то $A \mreduce B$.
\end{statement}
\begin{proof}
    В силу ограничений на множество $B$, $\exists b \in B$ и $\exists a \in B^{\complement}$.
    Тогда определим $f$ следующим образом:
    $$
        f(n) = \begin{cases}
            b, & n \in A, \\
            a, & n \notin A.
        \end{cases}
    $$
    Функция $f$ является вычислимой $\implies$ $\forall n$ $n \in A \iff f(n) \in B$.
\end{proof}

\begin{corollary}[Отсутствие антисимметричности у отношения $m$-сводимости]
    Пусть $A$ --- множество всех четных чисел, $B$ --- множество всех нечетных чисел.
    Тогда $A \mreduce B$, $B \mreduce A$, однако $A \neq B$.
\end{corollary}

\begin{corollary}[Несводимость к дополнению]
    Существует множество $A$ такое, что $A~\cancel{\leqslant}_{m}~A^{\complement}$.
\end{corollary}
\begin{proof}
    Рассмотрим множество $K$ --- перечислимое, но неразрешимое.
    Тогда $K^{\complement}$ неперечислимо (иначе по теореме Поста $K$ разрешимо).
    Если $K \mreduce K^{\complement}$, то $K^{\complement} \mreduce K^{\complement^{\complement}}$ $\implies$ $K^{\complement} \mreduce K$, то есть неперечислимое множество $m$-сводится к перечислимому, противоречие.
\end{proof}

\begin{statement}[отсутствие универсального множества сводимости]
    $\nexists A$ такое что, $\forall B$ $B \mreduce A$.
\end{statement}
\begin{proof}
    $B \mreduce A \iff$ существует тотальная вычислимая $f$ такая, что $\forall n \in \N$ $n \in B \iff f(n) \in A$.
    Тогда всякое множество $B$ однозначно определяется сводящей функцией:
    $$
        B = \left\{n \in B \mid f(n) \in A\right\}.
    $$
    Поэтому, таких множеств $B$ не больше, чем сводящих функций, но, поскольку все сводящие функции вычислимы, то их не более чем счетно, а подмножеств $\N$ континуум.
\end{proof}
\begin{lemma}
    Пусть $U$ --- главная универсальная вычислимая функция, и 
    $$
        K_{U} = \left\{n \in \N \mid U(n, n) \text{ определено}\right\}.
    $$
    Тогда любое перечислимое множество $A$ $m$-сводится к $K_{U}$.
\end{lemma}
\begin{proof}
    Рассмотрим функцию
    $$
        V\left(n, x\right) \simeq \begin{cases}
            1, & n \in A, \\
            \text{undefined}, & n \notin A.
        \end{cases}
    $$
    То есть, $V\left(n, x\right) \simeq \omega_{A}\left(n\right) \implies V$ вычислима.
    Так как $U$ главная, существует вычислимая тотальная $S: \N \to \N$ такая, что
    $$
        \forall n~U_{S\left(n\right)} = V_{n} \iff \forall n~\forall x~ U\left(S\left(n\right), x\right) \simeq V\left(n, x\right).
    $$
    
    Если $n \in A$, то $V_{n}$ всюду определена $\implies$ $U_{S\left(n\right)}$ всюду определена $\implies$ определено $U_{S\left(n\right)}\left(S\left(n\right)\right) \implies$ определено $U\left(S\left(n\right), S\left(n\right)\right)$ $\implies S\left(n\right) \in K_{U}$.
    
    Если $n \notin A$, то $V_{n}$ нигде не определена $\implies$ $U_{S\left(n\right)}$ нигде не определена $\implies$ не определено $U_{S\left(n\right)}\left(S\left(n\right)\right) \implies S\left(n\right) \notin K_{U}$.

    Таким образом, существует вычислимая тотальная $S$ такая, что $\forall n$ $n \in A \iff S\left(n\right) \in K \iff A \mreduce^{S} K_{U}$.
\end{proof}

\begin{example}
    Существует неперечислимое множество такое, что все его элементы четные.
\end{example}
\begin{proof}
    Рассмотрим множество $X = \left\{2n~:~n \in K^{\complement}\right\}$.
    Заметим, что $\forall n$ $n \in K^{\complement} \iff 2n \in X$, что равносильно $K^{\complement} \mreduce X$.
    С другой стороны, все элементы $X$ четны, и, поскольку $K^{\complement}$ не перечислимо, то и $X$ не перечислимо.
\end{proof}

\begin{example}
    Пусть $U$ --- главная у. в. ф. и пусть $Z = \left\{n \in \N~:~U_{n} \text{ нигде не определена}\right\}$.
    Рассмотрим функцию
    $$
        V\left(n, x\right) = \begin{cases}
            1, & n \in K, \\
            \text{undefined}, & n \notin K;
        \end{cases}
    $$
    Поскольку $K$ перечислимо, то $\omega_{K}$ вычислима, но так как $V\left(n, x\right) \simeq \omega_{K}\left(n\right)$, то $V$ вычислима.

    Так как $U$ --- главная, то существует вычислимая тотальная $S$ такая, что
    $$
        \forall n~ U_{S\left(n\right)} = V_{n}.
    $$
    Мы видим, что если $n \in K \implies$ $V_{n}$ всюду определено.
    Иначе $V_{n}$ нигде не определено.
    То есть $n \in K^{\complement} \iff V_{n}$ нигде не определено $\iff U_{S\left(n\right)}$ нигде не определена $\iff S\left(n\right) \in Z$.
    Тогда $\forall n$ $n \in K^{\complement} \iff S\left(n\right) \in Z$.
    Отсюда $K^{\complement} \mreduce^{S} Z$, но поскольку $K^{\complement}$ не перечислимо, то и $Z$ не перечислимо.

    По определению, $Z^{\complement} = \left\{n \in \N~:~U_{n} \text{ где-то определена}\right\} = \left\{n \in \N~:~\exists x~U\left(n, x\right)\text{ определена}\right\}$.
    Вспомним про $T$-предикат $T\left(n, x, k\right) \iff \UU$ останавливается на входе $\left(n, x\right)$ за $k$ шагов.
    Но такое мгножество $T$ разрешимо, а $Z^{\complement} = \left\{n \in \N~:~\exists x~\exists k~T\left(n, x, k\right)\right\} \implies Z^{\complement}$ перечислимо.
\end{example}

\begin{example}
    Рассмотрим перечислимое множество $Z^{\complement} = \left\{n \in \N~:~U_{n} \text{ где-то определена}\right\} \neq \varnothing$.
    Тогда существует вычислимая тотальная $f: \N \to \N$ такая, что $Z^{\complement} = \range f = \left\{f\left(0\right), f\left(1\right), \ldots\right\}$.
    Рассмотрим $W: \N^{2} \pto \N$ такую, что
    $$
        W\left(m, x\right) = \begin{cases}
            \text{undefined}, & m = 0, \\
            U\left(f\left(m - 1\right), x\right), & m \geqslant 1.
        \end{cases}
    $$
    $W$, очевидно, является вычислимой.
    Докажем, что она универсальна.
    Рассмотрим произвольную вычислимую функцию $g:\N \pto \N$.
    Рассмотрим несколько случаев:
    \begin{enumerate}
        \item $g$ нигде не определена $\implies$ $g = W_{0}$.
        \item $g$ где-то определена $\implies $ $\exists k$ такое, что $U_{k} = g$ (в силу универсальности $U$) $\implies \exists k \in Z^{\prime}:~U_{k} = g \implies \exists m \in \N:~U_{f\left(m\right)} = g$.
        Тогда $g = U_{f\left(m\right)} = W_{m + 1}$.
    \end{enumerate}
    Получается, что $W$ является универсальной вычислимой функцией.
    Введем множество $Z^{\prime} = \left\{m \in \N~:~W_{m}\text{ нигде не определена}\right\}$.
    Поскольку $\forall m$ $W_{m + 1} = U_{f\left(m\right)}$, где $f$ --- где-то определенная функция, поскольку $\range f = Z^{\complement} \neq \varnothing$, а $f\left(m\right)$ --- индексы относительно $U$ где-то определенных функций $\implies$ единственным номером нигде не определенной функции является 0 $\implies Z^{\prime} = \left\{0\right\}$.
    Поскольку $Z^{\complement}$ конечно, то оно разрешимо $\implies$ перечислимо, но $Z$ не перечислимо $\implies$ $W$ не является главной универсальной вычислимой функцией.
\end{example}

\begin{corollary}
    Существуют не главные у. в. ф.
\end{corollary}

\subsection{Рекурсия}

\begin{theorem}[Клини]
    Пусть $U$ --- главная универсальная вычислимая функция.
    Тогда для любой вычислимой тотальной функции $f: \N \to \N$ существует $n \in \N$ такое, что $U_{f\left(n\right)} = U_{n}$.
    То есть,
    $$
        \forall x~ U\left(f\left(n\right), x\right) \simeq U\left(n, x\right).
    $$
\end{theorem}
\begin{proof}
    Рассмотрим функцию $V:~\N^{2} \pto \N$ такую, что
    $$
        \forall k~ \forall x~ V\left(k, x\right) \simeq U\left(U\left(k, k\right), x\right).
    $$
    Такая функция вычислима.
    Поскольку $U$ --- главная, то существует вычислимая тотальная функция $S:~\N \to \N$ такая, что
    \begin{equation} \label{eq::03::03::01}
        \forall k~U_{S\left(k\right)} = V_{k} \iff \forall k~\forall x~U\left(S\left(k\right), x\right) \simeq V\left(k, x\right) \simeq U\left(U\left(k, k\right), x\right)
    \end{equation}
    Рассмотрим функцию $f \circ S$.
    Она вычислима и тотальная как композиция вычислимых тотальных функций.
    Поскольку $U$ --- у. в. ф. $\implies$ $\exists t$ такое, что $U_{t} = f \circ S$.
    Тогда
    \begin{equation} \label{eq::03::03::02}
        \forall x~U\left(t, x\right) \simeq f\left(s\left(x\right)\right).
    \end{equation}
    По формуле \eqref{eq::03::03::01}, 
    \begin{equation}
        \forall x~U\left(S\left(t\right), x\right) \simeq U\left(U\left(t, t\right), x\right) \underset{\eqref{eq::03::03::02}}{\simeq} U\left(f\left(S\left(t\right)\right), x\right).
    \end{equation}
    Получилось, что
    $$
        \forall x~ U\left(S\left(t\right), x\right) \simeq U\left(f\left(S\left(t\right)\right), x\right).
    $$
    Положим $n = S\left(t\right)$, тогда $\forall x~ U\left(n, x\right) \simeq U\left(f\left(n\right), x\right) \iff U_{n} = U_{S\left(n\right)}$.
\end{proof}
