\ProvidesFile{lect-01.tex}[Лекция 1]

\section{Лекция 1}

\subsection{Алгоритм и его неформальное определение}

\begin{enumerate}
    \item Алгоритмов счетно много;
    \item Алгоритм исполняется по шагам;
    \item Алгоритм работает конечно много шагов или зацикливается;
    \item Алгоритм принимает вход и может выдавать что-то на выход\footnote{Вход и выход --- слова конечного алфавита.};
\end{enumerate}

\begin{statement}
    Слов конечного непустого алфавита лишь счетно много.
\end{statement}

Удобно считать, что на вход подаются конечные пары натуральных чисел.
Алгоритм может вычислять частичные функции $f: \N \pto \N$.

\begin{definition}
    Алгоритм $\FF$ вычисляет функцию $f: \N \pto \N$, если для любого $x \in \N$
    \begin{align}
        x \in \dom f &\implies \text{ алгоритм $\FF$ на входе $x$ останавливается за конечное число шагов и выводит $f(x)$; } \\
        x \notin \dom f &\implies \text{ алгоритм $\FF$ на входе $x$ не останавливается ни за какое конечное число шагов. }
    \end{align}
\end{definition}

\begin{definition}
    Функция $f$ называется {\it вычислимой}, если существует алгоритм, который ее вычисляет.
\end{definition}

Рассмотрим следующую тотальную функцию:
$$
    f(x) = \begin{cases}
        1, & \text{если бог есть}, \\
        0, & \text{иначе}.
    \end{cases}
$$
Утверждается, что $f$ --- вычислима.
Действительно, рассмотрим следующие два алгоритма:

\noindent
\begin{minipage}[t]{0.45\textwidth}
    \begin{lstlisting}[frame=single, numbers=left, gobble=8]
        int f(int x) {
            return 0;
        }
    \end{lstlisting}
\end{minipage}
\hfill
\begin{minipage}[t]{0.45\textwidth}
    \begin{lstlisting}[frame=single, numbers=left, gobble=8]
        int f(int x) {
            return 1;
        }
    \end{lstlisting}
\end{minipage}


Какой-то из них вычисляет $f$, однако мы точно не можем сказать какой.

\begin{definition}
    Множество $A \subseteq \N$ называется {\it разрешимым}, если существует алгоритм $\AA$ такой, что $\forall x \in \N$
    \begin{align}
        x \in A &\implies \text{ $\AA$ выводит 1 и останавливается; } \\
        x \notin A &\implies \text{ $\AA$ выводит 0 и останавливается. }
    \end{align}
\end{definition}

\begin{statement}
    $A$ разрешимо $\iff$ вычислима характеристическая функция $\chi_{A}: \N \to \left\{0, 1\right\}$ множества~$A$:
    $$
        \chi_{A}\left(n\right) = \begin{cases}
            1, & n \in A; \\
            0, & n \notin A.
        \end{cases}
    $$
\end{statement}

Характеристическая функция существует у любого подмножества натуральных чисел, но не для всякого подмножества она вычислима.

\begin{statement}
    Существует неразрешимое множество.
\end{statement}
\begin{proof}
    Алгоритмов лишь счетно много, в то время как подмножеств $\N$ несчетно много.
\end{proof}

\begin{corollary}
    Существует невычислимая функция.
\end{corollary}

\begin{statement}
    Если $A$ конечно, то $A$ разрешимо.
\end{statement}
\begin{proof}
    Положим $A = \left\{a_{1}, \ldots, a_{n}\right\}$, тогда следующий алгоритм вычисляет $\chi_{A}$:
    \begin{lstlisting}[escapeinside={(*@}{@*)}, numbers=left, gobble=8]
        int in_A((*@$x$@*)) {
            return ((*@$x$@*) == (*@$a_{1}$@*)) || ((*@$x$@*) == (*@$a_{2}$@*)) || ... || ((*@$x$@*) == (*@$a_{n}$@*));
        }
    \end{lstlisting}
\end{proof}

\begin{statement}
    $A$, $B$ разрешимы $\implies$ разрешимы $A \cup B$, $A \cap B$, $A^{\complement}$, $A \times B$.
\end{statement}
\begin{proof}~
    \begin{align}
        \chi_{A \cap B}(n) &= \chi_{A}\left(n\right) \cdot \chi_{B}\left(n\right); \\
        \chi_{A \cup B}(n) &= \chi_{A}\left(n\right) + \chi_{B}\left(n\right); \\
        \chi_{A^{\complement}}(n) &= 1 - \chi_{A}\left(n\right); \\
        \chi_{A \times B}(n) &= \chi_{A}\left(n\right) \cdot \chi_{B}\left(m\right).
        \qedhere
    \end{align}
\end{proof}

\begin{definition}
    Множество $A \subseteq \N$ называется {\it перечислимым}, если существует алгоритм $\AA$ такой, что, работая на пустом входе, $\AA$ никогда не останавливается, но в процессе работы выводит все элементы множества $A$ и только их.
\end{definition}

\begin{statement}
    Если А разрешимо, то оно перечислимо.
\end{statement}
\begin{proof}~
    \begin{lstlisting}[frame=single, numbers=left, gobble=4]
        n = 0;
        while (1) {
            if (in_A(n)) {
                print(n);
            }
            ++n;
        }
    \end{lstlisting}
\end{proof}

Докажем, что из перечислимости не следует конечность.
Рассмотрим $A = \N$ --- бесконечное множество, $\chi_{A}(n) = \chi_{\N}(n) = 1$ --- вычислима.

\begin{theorem}[Поста]
    $A$ разрешимо $\iff$ $A$ и $A^{\complement}$ перечислимы.
\end{theorem}
\begin{proof}~
    \begin{description}
        \item[$\implies$] $A$ разрешимо $\implies$ $A^{\complement}$ разрешимо $\implies$ $A^{\complement}$ перечислимо, \hspace{5pt} $A$ разрешимо $\implies$ $A$ перечислимо.
        \item[$\impliedby$] Пусть $\AA$ перечисляет $A$, и $\BB$ перечисляет $A^{\complement}$.
        Мы хотим по заданному $n$ посчитать $\chi_{A}\left(n\right)$.
        Поочередно будем делать по шагу алгоритмов $\AA$ и $\BB$.
        Если на каком-то шаге $\AA$ выведет $n$, то $n \in A \implies$ выведем 1 и остановимся.
        Если на каком-то шаге $\BB$ выведет $n$, то $n \notin A \implies$ выведем 0 и остановимся.
        Поскольку $A \cup A^{\complement} = \N$, то одно из этих событий обязательно случится и алгоритм остановится.
        \qedhere
    \end{description}
\end{proof}

\begin{definition}
    Пусть $A \subseteq \N^{k}$, $\left(a_{1}, \ldots, a_{k}\right) \in A$.
    Проекцией множества $A$ на координату $i$ называется
    $$
        \pr^{i} A = \left\{b \in \N \mid \left(a_{1}, \ldots, a_{i - 1}, b, a_{i + 1}, \ldots, a_{k} \in A\right)\right\}.
    $$
\end{definition}

\begin{statement}
    $A$, $B$ перечислимы $\implies$ $A \cup B$, $A \cap B$, $A \times B$, $\pr^{i} A$ перечислимы.
\end{statement}
\begin{proof}
    Пусть $\AA$ перечисляет $A$ и $\BB$ перечисляет $B$.
    \begin{enumerate}
        \item[$\pr^{i} A:$] Запустим $\AA$, и для каждого напечатанного набора $\left(a_{1}, \ldots, a_{k}\right)$ будем брать $a_{i}$.
        \item[$A \cup B:$] Будем поочередно делать шаги перечислителей $\AA$ и $\BB$.
        Все напечатанные кем-то из них числа отправляем на вывод.
        \item[$A \cap B:$] Будем поочередно делать шаги перечислителей $\AA$ и $\BB$.
        Будем добавлять весь вывод $\AA$ в буффер $A^{\prime}$; аналогично для $\BB$ и $B^{\prime}$.
        Пусть $A_{i}^{\prime}$ --- состояние буффера $A^{\prime}$ после $i$-ого шага $\AA$.
        После того, как мы сделали $i$-ый шаг $\AA$ и $i$-ый шаг $\BB$, выведем элементы {\it конечного} множества $A^{\prime}_{i} \cap B^{\prime}_{i}$ и перейдем к $i + 1$ шагу $\AA$ и $\BB$.
        \item[$A \times B:$] Аналогично $A \cap B$, только выводим $A^{\prime}_{i} \times B^{\prime}_{i}$.
        \qedhere
    \end{enumerate}
\end{proof}

\subsection{Равносильные определения перечислимости}

\begin{definition}
    Пусть есть функция $f \colon \N \pto \N$.
    Определим для функции ее график:
    $$
        \Gamma_{f} = \left\{\left(x, y\right) \in \N^{2} \mid f\left(x\right) = y\right\}.
    $$
\end{definition}

\begin{theorem}
    Пусть $f \colon \N \pto \N$.
    Тогда $f$ вычислима $\iff$ $\Gamma_{f}$ перечислим.
\end{theorem}

\begin{proof}~
    \begin{description}
    \item[$\impliedby$]
        Дано $x$ и требуется вывести $f(x)$, если оно существует.
        Запустим перечислитель $\Gamma_f$ и дождемся первой пары вида $(x, y)$.
        Так как $(x, y) \in \Gamma_f$, то $y = f(x)$.
        Выводим $y$ и завершаемся.
        Если же $x$ не принадлежит $\dom f$, то алгоритм будет ждать такой пары бесконечно.

    \item[$\implies$]
        Дано $f$ и вычисляющий её алгоритм $\FF$. Хотим получить перечислитель $\Gamma_f$.
        Запустим перечислитель $\N^3$ и для каждой выведенной тройки $(x, y, k) \in \N^3$ сделаем $k$ шагов алгоритма $\FF$ на входе $x$.
        Если он вывел $y$ и остановился, то $f(x) = y$; напечатаем пару $(x, y)$.

        Допустим $(x, y) \in \Gamma_f$. Тогда $f(x) = y$, а значит существует такое число шагов $k$, что $\FF$ на входе $x$ выведет $y$ за $k$ шагов. Мы когда-нибудь рассмотрим тройку $(x, y, k)$ и выведем $(x, y)$.
        \qedhere
    \end{description}
\end{proof}

\begin{corollary}
    Если $f$ вычислима и $A$ перечислимо, то $f(A)$ перечислимо.
\end{corollary}

\begin{proof}
    $f(A) = \pr^2 \left(\Gamma_f \cap \left(A \times \N\right)\right)$.

    Так как $f$ вычислима, то $\Gamma_f$ перечислим.
    Так как $A$ перечислимо, то $A \times \N$ перечислимо.
    Тогда, $\Gamma_f \cap \left(A \times \N\right)$ перечислимо, а значит перечислимо и $\pr^2\left(\Gamma_f \cap \left(A \times \N\right)\right)$.
    Аналогично $f^{-1}(A) = \pr^{1}\left(\Gamma_f \cap \left(\N \times A\right)\right)$
\end{proof}

\begin{corollary}
    Если $f \colon \N \pto \N$ вычислима, то $\dom f$ и $\range f$ перечислимы.
\end{corollary}

\begin{proof}
    $\dom f = f^{-1}(\N), \quad \range f = f(\N)$.
\end{proof}

\begin{definition}
    $f(x) \simeq g(x)$ (\enquote{совпадает}), если или $f(x)$ и $g(x)$ оба определены и равны, либо оба не определены для любого $x$.
\end{definition}

\begin{definition}
    Если $A \subseteq \N$, то \textit{полухарактеристическая функция} $\omega_A \colon \N \pto \N$ множества $A$:
    $$
        \omega_A(n) \simeq \begin{cases}
            1, &n \in A, \\
            \text{undefined}, &n \notin A.
        \end{cases}
    $$
\end{definition}

\begin{comment*}
    $\dom \omega_A = A$.
\end{comment*}

\begin{statement}
    $A$ перечислимо $\iff$ $\omega_A$ вычислима ($A$ полуразрешимо).
\end{statement}

\begin{proof}~
    \begin{description}
    \item[$\implies$] 
        Дано число $n$, хотим вывести $1$ если оно есть в $A$.
        Запустим алгоритм, перечисляющий $A$. Если он вывел $n$, то выводим $1$ и завершаемся.
        Если чисто $n$ не встречается в $A$, то алгоритм будет работать бесконечно.
    \item[$\impliedby$] 
        $A = \dom \omega_A$.
        \qedhere
    \end{description}
\end{proof}

\begin{statement}
    Если $A$ перечислимо и $A \neq \varnothing$, то существует вычсилимая тотальная $f \colon \N \to \N$, такая что $A = \range f$ (то есть $A = \left\{f(0), f(1), \dots\right\}$).
\end{statement}

\begin{proof}
    Пусть $\AA$ перечисляет $A$. Так как $A \neq \varnothing$, то $\AA$ напечатает какое-то число $a \in A$.
    Пусть впервые $a$ будет напечатано на шаге $k$.
    Тогда положим $f(0) := a$, а $f(n + 1) := $ \enquote{последнее число, которое $\AA$ напечатает за $n + k + 1$ шагов}.
    Очевидно, что $f$ вычислима.
\end{proof}

\begin{corollary}
    Если $A$ перечислимо, то существует вычислимая $f \colon \N \pto \N$, такая что $A = \range f$.
\end{corollary}

\begin{statement}
    Если $A \subseteq \N$ перечислимо, то существует разрешимое $B \subseteq \N^2$, такое что $A = \pr^{1} B$.
\end{statement}

\begin{proof}
    Пусть $\AA$ перечисляет $A$.
    Пусть $B := \left\{(n, k) \in \N^2 \mid \text{алгоритм $\AA$ выводит $n$ на шаге $k$}\right\}$.

    Тогда $\pr^{1} B = \left\{n \in \N \mid \exists k \colon \text{алгоритм $\AA$ выводит $n$ на шаге $k$}\right\} = \left\{n \in \N \mid n \in A\right\} = A$.
\end{proof}

\begin{theorem}
    $\forall A \subseteq \N$ следующие утверждения раносильны:
    \begin{enumerate}[nosep]
    \item $A$ перечислимо;
    \item $A$ полуразрешимо;
    \item $\exists$ вычислимая $f \colon \N \pto \N$, такая что $A = \dom f$;
    \item $\exists$ вычислимая $f \colon \N \pto \N$, такая что $A = \range f$;
    \item $A = \varnothing$ или $\exists$ вычислимая $f \colon \N \to \N$, такая что $A = \range f$;
    \item $\exists$ разрешимое $B \subseteq \N^2$, такое что $A = \pr^{1} B$.
    \end{enumerate}
\end{theorem}
