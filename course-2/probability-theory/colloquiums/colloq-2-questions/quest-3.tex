\ProvidesFile{quest-03.tex}[Билет 3]

\section{Билет 3}

\begin{center}
    \it Вероятностное пространство в общем случае: алгебра и $\sigma$-алгебра подмножеств.
    Примеры $\sigma$-алгебр, $\sigma$-алгебра, порожденная системой подмножеств, борелевская $\sigma$-алгебра.
\end{center}

\sectionbreak
\subsection{Вероятностное пространство в общем случае: алгебра и $\sigma$-алгебра подмножеств}

\begin{definition*}
    Класс $\mathcal{A}_0$ подмножеств пространства $\Omega$ называется {\it алгеброй}, если
    \begin{enumerate}
        \item $\Omega, \emptyset \in \mathcal{A}_0$;
        \item $A \in \mathcal{A}_0 \implies \Omega \setminus A \in \mathcal{A}_0$;
        \item $A, B \in \mathcal{A}_0 \implies A \cap B, A \cup B \in \mathcal{A}_0$.
    \end{enumerate}
\end{definition*}

\begin{definition*}
    Класс $\mathcal{A}$ подмножеств пространства $\Omega$ называется {\it $\sigma$-алгеброй}, если
    \begin{enumerate}
        \item $\Omega, \emptyset \in \mathcal{A}$;
        \item $A \in \mathcal{A} \implies \Omega \setminus A \in \mathcal{A}$;
        \item $A_n \in \mathcal{A},~\forall n\in \mathbb{N} \implies \bigcap\limits_{n = 1}^\infty A_n, \bigcup\limits_{n = 1}^\infty A_n \in \mathcal{A}$.
    \end{enumerate}
\end{definition*}

\noindent Отметим, что в силу формул
\[
    \Omega \setminus \bigcup\limits_{\alpha} A_\alpha = \bigcap\limits_{\alpha}(\Omega \setminus A_\alpha)
\]
и
\[
    \Omega \setminus \bigcap\limits_{\alpha} A_\alpha = \bigcup\limits_{\alpha}(\Omega \setminus A_\alpha),
\]
в пункте 3 каждого определения достаточно проверять включение либо только для объединений, либо только для пересечений.

\sectionbreak
\subsection{Примеры $\sigma$-алгебр, $\sigma$-алгебра, порожденная системой подмножеств, борелевская $\sigma$-алгебра}

\paragraph{Примеры $\sigma$-алгебр}
Множество всех подмножеств $2^{\Omega}$, $\{\emptyset, \Omega\}$, $\{\emptyset, B, \Omega\setminus B, \Omega\}$ являются $\sigma$ алгебрами.

\paragraph{Примеры алгебр}
Множество всех конечных объединений попарно непересекающихся промежутков $(a, b]$ на $\mathbb{R}$ является алгеброй, но не является $\sigma$ алгеброй, поскольку она не содержит одноточечные множества --- пересечения счетного числа полуинтервалов.

\begin{definition*}
    Говорят, что $\sigma$ алгебра {\it порождена набором множеств $S$}, если эта $\sigma$ алгебра является наименьшей по включению среди всех $\sigma$-алгебр, которые содержат данный набор множеств $S$.
    Такую $\sigma$-алгебру обозначают $\sigma(S)$.
\end{definition*}

\begin{definition*}
    $\sigma$-алгебра называется {\it борелевской $\sigma$-алгеброй $\mathcal{B}(\mathbb{R})$} подмножеств прямой $\mathbb{R}$, если она порождена всеми промежутками (отрезками, интервалами, лучами).
\end{definition*}

Несложно показать, что в определении не обязательно в качестве порождающего множества брать все промежутки.
Например, можно ограничиться только отрезками или только интервалами или только лучами $(-\infty, c]$.
Например, проверим, что $\mathcal{B}(\mathbb{R})$ порождена всеми лучами вида $(-\infty, c]$.
Действительно, $(-\infty, c] \in \mathcal{B}(\mathbb{R})$, как счетное объединение промежутков вида $(-n, c]$, поэтому $\sigma(\{(-\infty, c]\}) \subset \mathcal{B}(\mathbb{R})$.
С другой стороны $(a, b] = (-\infty, b] \setminus (-\infty, a]$, отрезки получаются счетным пересечением промежутков вида $(a-\frac{1}{n}, b]$, интервалы получаются счетным объединением промежутков вида $(a, b-\frac{1}{n}]$, а полуинтервалы вида $[a, b)$ получаются объединением уже полученных отрезков вида $[a, b-\frac{1}{n}]$.
Тем самым, все промежутки принадлежат $\sigma(\{(-\infty, c]\})$, а значит имеет место и включение $\mathcal{B}(\mathbb{R})\subset \sigma(\{(-\infty, c]\})$.