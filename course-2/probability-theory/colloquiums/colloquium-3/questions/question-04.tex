\subsection{Характеристические функции: определение и свойства. Вычисление характеристической функции нормальной случайной величины. Производные характеристических функций.}

\subsubsection{Характеристические функции: определение и свойства.}

\begin{definition*}
    Пусть $X$ это случайная величина. Тогда характеристическая функция случайной величины $X$ это
    \begin{equation*}
        \phi_X(t) := \E[e^{itX}] = \E[\cos (t \cdot X)] + i \cdot \E[\sin (t \cdot X)].
    \end{equation*}
\end{definition*}

\begin{theorem*}[Свойства характеристических функций]
    У характеристической функции есть следующие свойства:
    \begin{enumerate}
    \item
        Для любой случайной величины $X$ выполняется $\phi_X(0) = 1$.
        
    \item 
        Для любой случайной величины и любого $t \in \RR$ выполняется $|\phi_X(t)| \leq 1$.

    \item 
        Для чисел $a$ и $b$ выполняется
        \begin{equation*}
            \phi_{aX + b}(t) = e^{itb} \cdot \phi_X(at).
        \end{equation*}

    \item 
        Если $X_n$ это последовательность \textbf{независимых} случайных величин, то
        \begin{equation*}
            \phi_{X_1 + X_2 + \dots + X_n}(t) = \phi_{X_1}(t) \cdot \phi_{X_2}(t) \cdot \dots \cdot \phi_{X_n}(t).
        \end{equation*}
    \end{enumerate}
\end{theorem*} 

\begin{proof}
    Для доказательства будем пользоваться следующей формулой:
    \begin{equation*}
        \phi_X(t) = \E[\cos (t \cdot X)] + i \cdot \E[\sin (t \cdot X)],
    \end{equation*}
    которая следует из формулы Эйлера $e^{ix} = \cos x + i \cdot \sin x$.

    Докажем свойства:
    \begin{enumerate}
    \item
        Для любой случайной величины $X$ выполняется $\phi_X(0) = 1$.

        Проверяется подстановкой:
        \begin{equation*}
            \phi_X(0) = \E[e^{i \cdot 0 \cdot X}] = \E[e^0] = \E[1] = 1.
        \end{equation*}
        
    \item 
        Для любой случайной величины и любого $t \in \RR$ выполняется $|\phi_X(t)| \leq 1$.

        Рассмотрим случайную величину $Y$. Знаем, что ее дисперсия неотрицательна, то есть $\D[Y] = \E[Y^2] - (\E[Y])^2 \geq 0$, откуда следует, что для любой случайной величины $Y$ справедливо $\E[Y^2] \geq (\E[Y])^2$.

        Значение характеристической функции это комплексное число. Квадрат модуля комплексного числа это сумма квадратов его мнимой и действительной частей:
        \begin{equation*}
            |\phi_X(t)|^2 = (\E[\cos (t \cdot X)])^2 + (\E[\sin (t \cdot X)])^2.
        \end{equation*}
        
        
        С помощью знаний о $\E[Y^2] \geq (\E[Y])^2$ оценим квадрат модуля характеристической функции:
        \begin{align*}
            |\phi_X(t)|^2 
            &= (\E[\cos (t \cdot X)])^2 + (\E[\sin (t \cdot X)])^2
            \leq \E[\cos^2(t \cdot X)] + \E[\sin^2 (t \cdot X)]
            = \E[\cos^2 (t \cdot X) + \sin^2 (t \cdot X)]
            = \E[1]
            = 1.
        \end{align*}    

    \item 
        Для чисел $a$ и $b$ выполняется
        \begin{equation*}
            \phi_{aX + b}(t) = e^{itb} \cdot \phi_X(at).
        \end{equation*}

        Заметим, что если $y$ это некоторое число, то $\E[y \cdot X] = y \cdot \E[X]$ по линейности математического ожидания.

        Запишем по определению:
        \begin{align*}
            \phi_{aX + b}(t)
            = \E[e^{it \cdot aX + it \cdot b}]
            = \E[e^{it \cdot aX} \cdot e^{it \cdot b}]
            = e^{it \cdot b} \cdot \E[e^{it \cdot aX}]
            = e^{it \cdot b} \cdot \phi_{aX}(t).
        \end{align*}

    \item 
        Если $X_n$ это последовательность \textbf{независимых} случайных величин, то
        \begin{equation*}
            \phi_{X_1 + X_2 + \dots + X_n}(t) = \phi_{X_1}(t) \cdot \phi_{X_2}(t) \cdot \dots \cdot \phi_{X_n}(t).
        \end{equation*}

        Пусть $Y_n = e^{i \cdot t \cdot X_n}$. Тогда $Y_1, \dots Y_n$ это последовательность независимых случайных величин (в силу независимости $X_n$) и $\E[Y_1 \cdot \dots \cdot Y_n] = \E[Y_1] \cdot \dots \cdot \E[Y_n]$.

        Запишем по определению:
        \begin{equation*}
            \phi_{X_1 + X_2 + \dots + X_n}(t)
            = \E[e^{itX_1 + \dots + itX_n}]
            = \E[e^{itX_1} \cdot \dots \cdot e^{itX_n}]
            = \E[Y_1 \cdot \dots \cdot Y_n]
            = \E[Y_1] \cdot \dots \cdot \E[Y_n]
            = \phi_{X_1}(t) \cdot \dots \cdot \phi_{X_n}(t).
        \end{equation*}
    \end{enumerate}
\end{proof}

\subsubsection{Вычисление характеристической функции нормальной случайной величины.}

Хотим вычислить $\phi_{\xi}(t)$, где $\xi \sim \mathcal{N}(0, 1)$.

Запишем по определению:
\begin{equation*}
    \phi_{\xi}(t) = \E[e^{it \xi}]
    = \dfrac{1}{\sqrt{2 \pi}} \int_{-\infty}^{+\infty} \cos (tx) \exp[-x^2/2] \dd x
    + \dfrac{i}{\sqrt{2 \pi}} \int_{-\infty}^{+\infty} \sin (tx) \exp[-x^2/2] \dd x.
\end{equation*}

Заметим, что второе слагаемое $\dfrac{i}{\sqrt{2 \pi}} \int_{-\infty}^{+\infty} \sin (tx) \exp[-x^2/2] \dd x$ равно нулю, так как это интеграл нечетной функции по симметричному промежутку. Тогда
\begin{equation*}
    \phi_{\xi}(t) 
    = \dfrac{1}{\sqrt{2 \pi}} \int_{-\infty}^{+\infty} \cos (tx) \exp[-x^2/2] \dd x.
\end{equation*}

Возьмем производную по $t$ (считаем, что она берется):
\begin{align*}
    \phi_{\xi}'(t)'
    &= -\dfrac{1}{\sqrt{2 \pi}} \int_{-\infty}^{+\infty} x \cdot \sin (tx) \exp[-x^2/2] \dd x
    = \dfrac{1}{\sqrt{2 \pi}} \int_{-\infty}^{+\infty} \sin (tx) \dd (\exp[-x^2/2]) \\
    &= \dfrac{1}{\sqrt{2 \pi}} \sin (tx) \exp[-x^2/2]\Big|_{-\infty}^{+\infty} - \dfrac{t}{\sqrt{2\pi}} \int_{-\infty}^{+\infty} \cos (tx) \exp[-x^2/2] \dd x \\
    &= 0 - \dfrac{t}{\sqrt{2\pi}} \int_{-\infty}^{+\infty} \cos (tx) \exp[-x^2/2] \dd x
    = -t \cdot \phi_{\xi}(t).
\end{align*}

Пришли к дифференциальному уравнению:
\begin{equation*}
    \phi_{\xi}'(t) = -t \cdot \phi_{\xi}(t) \implies
    \dfrac{\phi_{\xi}'(t)}{\phi_{\xi}(t)} = -t.
\end{equation*}

Интегрируем обе части:
\begin{equation*}
    \int \dfrac{\dd (\phi_{\xi}(t))}{\phi_{\xi}(t)} = \ln |\phi_{\xi}(t)| + C = \int -t \dd t = - \dfrac{t^2}{2}.
\end{equation*}

Теперь берем экспоненту от обеих частей:
\begin{equation*}
    \phi_{\xi}(t) = C' \cdot \exp[-t^2/2],
\end{equation*}
где $C'$ это некоторая константа. 

Про характеристическую функцию мы знаем, что $\phi_{\xi}(0) = 1$. Тогда
\begin{equation*}
    \phi_{\xi}(0) = 1 = C' \cdot \exp[0] = C',
\end{equation*}
откуда находим $C' = 1$.

Тогда характеристическая функция стандартной нормальной величины имеет следующий вид:
\begin{equation*}
    \phi_{\xi}(t) = \exp[-t^2/2].
\end{equation*}

\subsubsection{Производные характеристических функций.}

\begin{theorem*}
    Пусть $X$ это случайная величина с конечным $k$-ым моментом ($\E[|X|^k] < \infty$). Тогда $\phi_X$ $k$ раз дифференцируема и
    \begin{equation*}
        \phi_{X}^{(k)}(0) = i^k \cdot \E[X^k].
    \end{equation*}
\end{theorem*}

\begin{proof}
    Докажем для $k = 1$, для остальных порядков аналогично.

    Мы хотим найти производную:
    \begin{align*}
        \lim_{h_n \to 0} \dfrac{\phi_X(t + h_n) - \phi_X(t)}{h_n}
        = \lim_{h_n \to 0} \dfrac{1}{h_n} \cdot \left(\E[e^{i (t + h_n) X}] - \E[e^{itX}]\right)
        = \lim_{h_n \to 0} E \left[\dfrac{e^{i (t + h_n) X} - e^{itX}}{h_n}\right]
        =: \lim_{h_n \to 0} E[g_n],
    \end{align*}
    то есть обозначили $g_n = \dfrac{e^{i (t + h_n) X} - e^{itX}}{h_n}$.

    Поймем, что мы знаем про функцию $g_n$:
    \begin{itemize}
    \item 
        У нее есть поточечный предел:
        \begin{equation*}
            \lim_{n \to \infty} g_n(X) = \left(e^{itX}\right)'_t = iX e^{itX}.
        \end{equation*}

    \item 
        Надо как-то оценить $|g_n|$.

        Знаем, что модуль комплексной экспоненты равен $1$, то есть $|e^{itX}| = 1$. Тогда
        \begin{equation*}
            |g_n(X)| 
            = \left|\dfrac{e^{itX} \cdot (e^{ih_nX} - 1)}{h_n}\right|
            = |e^{itX}| \cdot \left|\dfrac{e^{ih_nX} - 1}{h_n}\right|
            = \left|\dfrac{e^{ih_nX} - 1}{h_n}\right|
            = \left|\dfrac{e^{ih_nX} - e^{i \cdot 0 \cdot X}}{h_n}\right|
            = \left(e^{itX}\right)'_t(\xi)
            = \left|iX e^{i \xi X}\right|
        \end{equation*}
        для некоторого $\xi \in (0; h_n)$.

        Предпоследний переход выполнен по теореме Лагранжа, которая гласит следующее:
        \begin{equation*}
            \exists \xi \in (a; b): \quad \dfrac{f'(b) - f'(a)}{b - a} = f'(\xi).
        \end{equation*}

        Опять же воспользуемся тем, что модуль комплексной экспоненты равен $1$:
        \begin{equation*}
            |g_n(X)| 
            = \left|iX e^{i \xi X}\right|
            = |i| \cdot |X| \cdot \left|e^{i \xi X}\right|
            = 1 \cdot |X| \cdot 1
            = |X|.
        \end{equation*}
    \end{itemize}

    Мы получили, что
    \begin{itemize}
    \item 
        $|g_n(X)| \leq |X|$ и $\E[|X|] < \infty$ (для этого и нужна конечность моментов);

    \item 
        $g_n(X) \xrightarrow{\text{п. н.}} i \cdot X \cdot e^{itX}$.
    \end{itemize}

    Тогда по теореме Лебега предел ожиданий есть ожидание предела:
    \begin{equation*}
        \lim_{n \to \infty} \E[g_n(X)] = \E[i \cdot X \cdot e^{itX}].
    \end{equation*}

    Возвращаемся в самое начало:
    \begin{equation*}
        \phi_X(t)' = \lim_{n \to \infty} g_n = i \cdot \E[X \cdot e^{itX}].
    \end{equation*}
\end{proof} 