\section{Билет 6}

\begin{center}
    \it
    Проверка гипотез.
    Ошибки 1-го и 2-го рода.
    Уровень значимости и мощность статистического критерия.
    Пример построения критерия с помощью доверительного интеграла.
    Нижняя оценка суммы вероятностей ошибок 1-го и 2-го рода.
\end{center}

\subsection{Проверка гипотез}

Пускай есть выборка $X_1,\ldots,X_n$ с распределением $P_{\theta}$.

\begin{definition*}
Предположения о значениях $\theta$ и называются статистическими гипотезами. 
\end{definition*}
\begin{example}
$H_0\colon\theta\in\Theta_0~-~$ статистическая\ гипотеза
\end{example}
\begin{definition*}
Простая гипотеза (одноточечная гипотеза)$~-~$гипотеза вида $H_{0}\colon\theta=\Theta_0$, где $\Theta=\{\theta_0\}$
\end{definition*}
\begin{definition*}Гипотеза $H_{1}\colon\theta=\Theta_1~-~$альтернативная гипотеза
\end{definition*}
\begin{example}
$H_1\colon\theta=\overline{\Theta_0}~-~$ альтернативная\ гипотеза
\end{example}

Для проверки гипотез, строят критерий на основе критического множества как правило $\subset \mathbb{R}^n$, то есть действуют по такому принципу:

Выделяют в области значения параметров критического множества $K$, так, что $\forall \theta \in\Theta_0,\ P_\theta((X_1\ldots X_n)\in K)~$ --- <<маленькая>>, тогда 
$X_1\ldots X_n\in K\text{ свидетельствует против гипотезы}\ H_0$, то есть если $X=(X_1\ldots X_n)\in K\Rightarrow H_0$ отклоняется, иначе принимается.

Далее $X=(X_1\ldots X_n)$ 
\subsection{Ошибки 1-го и 2-го рода}
Пусть у нас есть критическое множество $K$. При проверке гипотез мы могли совершить две ошибки:
\begin{definition*}
Ошибка первого рода: отклонение верной гипотезы $H_0$, то есть это $\underset{\theta\in\Theta_0}{P_\theta}(X\in K)$. В случае простой гипотезы $P_{\theta_0}(X\in K)$
\end{definition*}
\begin{definition*}
Ошибка второго рода: принятие ложной гипотезы $H_0$, то есть это $\underset{\theta\in\Theta_1}{P_\theta}(X\notin K)$. В случае простой гипотезы $P_{\theta_1}(X\notin K)$
\end{definition*}

\subsection{Уровень значимости и мощность статистического критерия}
\begin{definition*}
Критерий $K$ имеет уровень значимости $\alpha$, если вероятность ошибки первого рода меньше либо равна $\alpha$, то есть $\underset{\theta\in\Theta_0}{P_\theta}(X\in K)\leq\alpha$.
\end{definition*}

\begin{definition*}
Мощность критерия $K$ это величина, равная 1 - вероятность ошибки второго рода, то есть $\\ 1-\underset{\theta\in\Theta_1}{P_\theta}(X\notin K)=\underset{\theta\in\Theta_1}{P_\theta}(X\in K)$. В случае простой гипотезы величина $\beta=P_{\theta_1}(X\in K)~-~$мощность
\end{definition*} 

Если имеются два критерия $K,S$ уровня значимости $\alpha$, то $K$ более мощный, чем $S$ если $$\forall\theta\in\Theta_1\colon P_\theta(X\in K)\geq P_\theta(X\in S)$$
\subsection{Пример построения критерия с помощью доверительного интеграла}
\begin{example}
Пускай $X_1,\ldots,X_n\sim\mathcal{N}(\theta,1)\\ H_0:\theta=\theta_0\\ H_1:\theta=\theta_1$

Ранее при данных условиях мы получили следующий доверительный интервал:
$$P_{\theta_0}\left(\overline{X_n}-\dfrac{Z_{1-\frac{\alpha}{2}}}{\sqrt{n}}\leq \theta_0\leq \overline{X_n}+\dfrac{Z_{1-\frac{\alpha}{2}}}{\sqrt{n}}\right)=P_{\theta_0}\left(\theta_0-\dfrac{Z_{1-\frac{\alpha}{2}}}{\sqrt{n}}\leq \overline{X_n}\leq \theta_0+\dfrac{Z_{1-\frac{\alpha}{2}}}{\sqrt{n}}\right)=1-\alpha$$

Выберем критическое множество $K\colon \left\{X\colon
\overline{X_n}>\theta_0+\dfrac{Z_{1-\frac{\alpha}{2}}}{\sqrt{n}}\right\}\cup\left\{X\colon \overline{X_n}<\theta_0-\dfrac{Z_{1-\frac{\alpha}{2}}}{\sqrt{n}}\right\}\\$
Тогда $P_{\theta_0}(X\in K)=1-(1-\alpha)=\alpha$

Найдем ошибку второго рода: 
        \begin{align}
            P_{\theta_1}(X\notin K)=P_{\theta_1}\left(\theta_0-\dfrac{Z_{1-\frac{\alpha}{2}}}{\sqrt{n}}\leq \overline{X_n}\leq \theta_0+\dfrac{Z_{1-\frac{\alpha}{2}}}{\sqrt{n}}\right)&=
        \\
        =P_{\theta_1}\Bigg(\sqrt{n}\left(\theta_0-\theta_1\right)-Z_{1-\frac{\alpha}{2}}\leq \underbrace{\sqrt{n}\left(\overline{X_n}-\theta_1\right)}_{\sim\mathcal{N}(0,1)}\leq \sqrt{n}\left(\theta_0-\theta_1\right)&+Z_{1-\frac{\alpha}{2}}\Bigg)=
        \\
        =\Phi\left(\sqrt{n}\left(\theta_0-\theta_1\right)+Z_{1-\frac{\alpha}{2}}\right)-\Phi\left(\sqrt{n}\left(\theta_0-\theta_1\right)-Z_{1-\frac{\alpha}{2}}\right)
        \end{align}
Посмотрим, что происходит при $n\rightarrow \infty$:
\begin{enumerate}
    \item $\theta_0>\theta_1\implies P_{\theta_1}(X\notin K)\rightarrow 0$
    \item $\theta_0<\theta_1\implies P_{\theta_1}(X\notin K)\rightarrow 0$
\end{enumerate}
Таким образом, мы получили состоятельный критерий.

\begin{definition*}
	Критерий состоятелен, если с ростом объема выборки его мощность стремится к 1.
\end{definition*}
\end{example}

\subsection{Нижняя оценка суммы вероятностей ошибок 1-го и 2-го рода}
\begin{theorem*}
Пусть у нас есть две гипотезы:
\begin{enumerate}
    \item $H_0:\rho=f_0$
    \item $H_1:\rho=f_1$
\end{enumerate}
Сумма ошибки первого рода и ошибки второго рода больше либо равна $1-\dfrac{1}{2}\int\limits_{\mathbb{R}^n}|f_0-f_1|dx$
\end{theorem*}
\begin{proof}
Найдем сумму ошибок первого и второго рода:
        \begin{align}
            P_{0}(X\in K)+P_{1}(X\notin K) =\int\limits_{K}f_0dx&+\underbrace{\int\limits_{\mathbb{R}^n\setminus K}f_1dx}_{1-\int\limits_K f_1dx}=1+\int\limits_{K}(f_0-f_1)dx\geq
        \\
        \text{Введем множество}&\ S=\{f_0\leq f_1\} 
        \\
        \geq 1+\int\limits_{K\cap S}(f_0-f_1)dx &\geq 1+\int\limits_{S}(f_0-f_1)dx
        \end{align}
Рассмотрим отдельно интеграл $\int\limits_{S}(f_0-f_1)dx$:
\begin{align}
            \int\limits_{S}(f_0-f_1)dx=\int\limits_{S}f_0dx-\int\limits_{S}f_1dx=1-\int\limits_{\mathbb{R}^n\setminus S}f_0\ dx-1+\int\limits_{\mathbb{R}^n\setminus S}f_1dx =\int\limits_{\mathbb{R}^n\setminus S}(f_1-f_0)dx
        \end{align}
В силу того, как мы выбрали множество $S$, можно увидеть, что
\begin{enumerate}
    \item $\int\limits_{S}(f_0-f_1)dx=-\int\limits_{S}|f_0-f_1|dx$
    \item $\int\limits_{\mathbb{R}^n\setminus S}(f_1 -f_0)dx=-\int\limits_{\mathbb{R}^n\setminus S}|f_0-f_1|dx$
\end{enumerate}
Тогда мы получаем, что $$-\int\limits_{\mathbb{R}^n\setminus S}|f_0-f_1|dx=-\int\limits_{S}|f_0-f_1|dx=-\dfrac{1}{2}\int\limits_{\mathbb{R}^n}|f_0-f_1|dx\implies P_{0}(X\in K)+P_{1}(X\notin K)\geq 1-\dfrac{1}{2}\int\limits_{\mathbb{R}^n}|f_0-f_1|dx$$
\end{proof}
