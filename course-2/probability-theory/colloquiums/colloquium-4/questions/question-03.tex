\section{Билет 3}

\begin{center}
    \it
    Информация Фишера и неравенство Рао-Крамера.
    Критерий равенства в неравенстве Рао-Крамера.
\end{center}

\subsection{Информация Фишера и неравенство Рао-Крамера}

\begin{definition*} 
    Информация Фишера $I(\theta) = \EE_{\theta} \left( \frac{\partial }{\partial \theta} L(x, \theta) \right)^2$
\end{definition*} 

Выведение и альтернативные варианты (здесь $\theta_0$ — реальный параметр):
\begin{flalign*}
& L(x, \theta) \darrowwtextover{\text{ф-ла Тейлора}}{0.5em}{\simeq} L(x, \hat{\theta}_n(x)) + 
\frac{1}{2} \underbracket{\frac{\partial^2 }{\partial \theta^2} L(x, \hat{\theta}_n(x) )}_{
    \mathclap{\leq 0 \text{ тк точка максимума}}
} \left( \theta - \hat{\theta}_n(x) \right)^2 \\
& \text{Точка максимума близка к параметру, посмотрим на вторую производную в реальном параметре:} \\
& - \EE_{\theta_0} \frac{\partial^2}{\partial \theta^2} L(x, \theta_0) = 
- \EE_{\theta_0} \frac{\partial }{\partial \theta} 
\left( \frac{\frac{\partial }{\partial \theta} P(x, \theta_0)}{P(x, \theta_0)}  \right) = 
- \EE_{\theta_0} \left( \frac{\frac{\partial^2}{\partial \theta^2} P(x, \theta_0)}{P(x, \theta_0)} - 
\frac{\left( \frac{\partial }{\partial \theta} P(x, \theta_0) \right)^2}{P(x, \theta_0)^2} \right) \oeq \\
& \left[ \text{трюки с производными так как} 
\left( \frac{a}{b} \right)' = \frac{a'}{b} - \frac{ab'}{b^2}, \text{ но здесь } a = b' \implies
\left( \frac{b'}{b} \right)' = \frac{b''}{b} - \frac{(b')^2}{b^2} \right] \\
& \oeq -\int_{\RR^n} \left( \frac{\frac{\partial^2}{\partial \theta^2} P(x, \theta_0)}{P(x, \theta_0)} - 
\frac{\left( \frac{\partial }{\partial \theta} P(x, \theta_0) \right)^2}{P(x, \theta_0)^2} \right) 
P(x, \theta_0) dx =
- \int_{\RR^n} \frac{\partial^2}{\partial \theta^2} P(x, \theta_0) dx + 
\int_{\RR^n} \left( \frac{\frac{\partial }{\partial \theta} P(x, \theta_0)}{P(x, \theta_0)} \right)^2 
P(x, \theta_0) dx
\end{flalign*}

Предположим, что выполнены условия регулярности
\begin{itemize} \itemsep 0em
\item $P(x, \theta)$ дважды непрерывно дифференцируема по $\theta$
\item $P(x, \theta) > 0$ на каком-то множесте иксов (прямая, отрезок, точки в дискретном случае) 
    $\forall \theta$
\item Производную и интеграл можно переставить 
\end{itemize} 

Из этого следует 
\begin{flalign*}
    & \int_{\RR^n} P(x, \theta) dx = 1 \implies 
    \int_{\RR^n} \frac{\partial }{\partial \theta} P(x, \theta) dx = \frac{\partial 1}{\partial \theta} = 0,
    \qquad \int_{\RR^n} \frac{\partial^2}{\partial \theta^2} P(x, \theta) dx = 0
\end{flalign*}

Итак 
\begin{flalign*}
    & - \EE_{\theta_0} \frac{\partial^2}{\partial \theta^2} L(x, \theta_0) = 
    \underbracket{- \int_{\RR^n} \frac{\partial^2}{\partial \theta^2} P(x, \theta_0) dx}_{=0} + 
    \int_{\RR^n} \left( \frac{\frac{\partial }{\partial \theta} P(x, \theta_0)}{P(x, \theta_0)} \right)^2 
    P(x, \theta_0) dx = 
    \EE_{\theta_0} \left( \frac{\partial }{\partial \theta} L(x, \theta_0) \right)^2 = I(\theta)
\end{flalign*}

В дискретном случае меняем интегралы на суммы

Предположения про $P(x, \theta)$ очень натуральны, поэтому их никто не проверяет 

\begin{proposition*} 
Пусть выполнены условия регулярности. Тогда
\begin{equation*}
    I(\theta) = \mathbb{D}_{\theta} \left( \frac{\partial }{\partial \theta} L(x, \theta) \right) = n \cdot i(\theta)
    \text{, где }
    i(\theta) = \EE_{\theta} \left( \frac{\partial }{\partial \theta} \ln \rho_{\theta} (X_1) \right)^2
.\end{equation*}
Здесь $i(\theta)$ --- информация Фишера выборки из одного элемента
\end{proposition*} 
\begin{proof} 
\begin{flalign*}
    & \EE_{\theta} \left( \frac{\partial }{\partial \theta} L(x, \theta) \right) = 
    \int_{\RR^n} \frac{\frac{\partial }{\partial \theta} P(x, \theta)}{P(x, \theta)} P(x, \theta) dx = 
    \int_{\RR^n} \frac{\partial }{\partial \theta} P(x, \theta) dx = 0 \implies 
    \mathbb{D}_{\theta} \left( \frac{\partial }{\partial \theta} L(x, \theta) \right) = 
    \EE_{\theta} \left( \frac{\partial }{\partial \theta} L(x, \theta) \right)^2 \\
    & \frac{\partial }{\partial \theta} L(x, \theta) = 
    \sum_i \underbracket{\frac{\partial}{\partial \theta} \ln \rho_\theta (X_i)}_{
        \mathclap{ \substack{\text{независимые}\\\text{одинаково распределенные}} }
    } \implies 
    \mathbb{D}_\theta \left( \frac{\partial }{\partial \theta} L(x, \theta) \right) = 
    n \mathbb{D}_\theta \left( \frac{\partial }{\partial \theta} \ln \rho_{\theta} (X_1) \right) =
    n \EE_\theta \left( \frac{\partial }{\partial \theta} \ln \rho_{\theta} (X_1) \right)^2 = n \cdot i(\theta)
\end{flalign*}
\end{proof} 

\begin{theorem*} \textbf{Неравенство Рао-Крамера}
Пусть выполняются условия регулярности, а $\theta_n(x)$ — несмещенная оценка функции $\tau(\theta)$ 
(как правило $\tau(\theta) = \theta$, но иногда мы пытаемся оценить не саму $\theta$, 
а какую-то функцию $\theta$), тогда
\[
    \mathbb{D}_\theta (\theta_n(x)) \geq \frac{\left( \tau'(\theta) \right)^2}{I(\theta)}
\]
\end{theorem*} 

\noindent
Неравенство нужно чтобы находить эффективные оценки: там где достигается равенство,
там и оценка эффективна.

\begin{proof} 
\begin{flalign*}
    & \tau(\theta) = \EE_\theta (\theta_n (x) ) = \int_{\RR_n} \theta_n (x) P(x, \theta) dx. \\
    & \tau'(\theta) =
    \int_{\RR^n} \theta_n (x) \frac{\partial }{\partial \theta}  P(x, \theta) dx =
    \int_{\RR^n} \theta_n (x) \frac{\partial }{\partial \theta}  P(x, \theta) dx - 
    \tau(\theta) \overbracket{\int_{\RR^n} \frac{\partial }{\partial \theta} P(x, \theta) dx}^{
        \parbox[t][1sp][b]{3em}{$=0$}
    } =
    \int_{\RR^n} \left( \theta_n (x) - \tau (\theta) \right) 
    \frac{\partial }{\partial \theta}  P(x, \theta) dx = \\
    & = \int_{\RR^n} \left( \theta_n (x) - \tau (\theta) \right) 
    \underbracket{\frac{\frac{\partial }{\partial \theta}  P(x, \theta)}{P(x, \theta)}}_{
        \mathclap{=\frac{\partial }{\partial \theta} \ln P(x, \theta) 
        = \frac{\partial }{\partial \theta} L(x, \theta)}
    } P(x, \theta)  dx = 
    \EE_\theta \left( \left( \theta_n (x) - \tau (\theta) \right) 
    \frac{\partial }{\partial \theta} L(x, \theta) \right) 
    \darrowwtextover{\text{Коши-Буняковский}}{0.5em}{\leq} \\
    & \leq \sqrt{
        \EE_\theta \left( \theta_n (x) - \tau (\theta) \right)^2
    }
    \sqrt{ 
        \EE_\theta \left( \frac{\partial }{\partial \theta} L(x, \theta) \right)^2
    } = 
    \sqrt{ \mathbb{D}_\theta \left( \theta_n (x) \right) }
    \sqrt{ I(\theta) } \\
    & \tau'(\theta) \leq 
    \sqrt{ \mathbb{D}_\theta \left( \theta_n (x) \right) }
    \sqrt{ I(\theta) } \implies
    (\tau'(\theta))^2 \leq 
    \mathbb{D}_\theta \left( \theta_n (x) \right)
    I(\theta).
\end{flalign*}
\end{proof} 

\subsection{Критерий равенства в неравенстве Рао-Крамера}

Равенство достигается когда достигается равенство в Коши-Буняковском, то есть 
\[
    \theta_n(x) - \tau(\theta) = \underbracket{c_n(\theta)}_\text{const}
    \frac{\partial }{\partial \theta} L(x, \theta)
\]

Пример в ситуации бернулли:
\begin{flalign*}
    & \frac{\partial }{\partial \theta} L(x, \theta) = \frac{\sum_i X_i - n \theta}{\theta(1-\theta)} = 
    \underbracket{\frac{n}{\theta(1-\theta)}}_{\text{const}} (\overline{X}_n - \theta) \implies
    \text{ в Рао-Крамере достигается равенство} \implies \bar{X}_n \text{ эфф оценка}
\end{flalign*}

Если (в случае оценки $\theta$, то есть $\tau(\theta) = \theta$) существует несмещенная оценка $\hat{\theta}_n$,
на которой достигается равенство в Рао-Крамере, то это оценка максимального правдоподобия
\begin{flalign*}
    & \hat{\theta}_n(x) - \theta = c_n(\theta) \frac{\partial }{\partial \theta} L(x, \theta) \quad
    \text{Возьмем } \theta = \theta^*(x) \text{ оценка макс правдоподобия} \implies 
    \frac{\partial }{\partial \theta} L(x, \theta^*(x)) = 0 \implies \hat{\theta}_n(x) = \theta^*(x)
\end{flalign*}
