\section{Билет 7}

\begin{center}
    \it
    Теорема Неймана-Пирсона и пример её применения.
\end{center}

\subsection{Теорема Неймана-Пирсона}
Пусть гипотеза $H_0$ утверждает, что плотность выборки -- это $f_0$, а альтернативная
гипотеза $H_1$ утверждает, что плотность выборки -- это $f_1$. \\
Предположим, что
 $\forall \alpha \in [0, 1] \ \exists t := t(\alpha): P_0(f_1(x) \geqslant tf_0(x)) = \alpha$. 

\begin{theorem*}[Неймана-Пирсона]
     В такой постановке наиболее мощный критерий уровня значимости $\alpha$ имеет вид \\
    $K_{t(\alpha)} := \{ f_1(x) \geqslant t(\alpha)f_0(x)\}$.
\end{theorem*}
\begin{proof}
    Пусть $S$ -- тоже критерий уровня значимости $\alpha$: $P_0(X \in S) \leqslant \alpha = P_0(X \in K_{t(\alpha)})$.
    Хотим сравнить $P_1(X \in K_{t(\alpha)}) - P_1(X \in S)$. Хотим, чтобы это было больше либо равно нуля. Это и будет
    означать, что у нас критерий наиболее мощный. \\
    $\displaystyle P_1(X \in K_{t(\alpha)}) - P_1(X \in S) = \int_{K_{t(\alpha)}}f_1dx - \int_{S}f_1dx$ = [
        можем выкинуть пересечение, так как на пересечении эти интегралы просто сократятся
    ] = $\int_{K_{t(\alpha)}\backslash S}f_1dx - \int_{S\backslash K_{t(\alpha)}}f_1dx$. \\
    Заметим, что на $S\backslash K_{t(\alpha)}$ выполнено $f_1 < t(\alpha)f_0$, так как это взято из
    дополнения к $K_{t(\alpha)}$, где по условию выполняется $ f_1(x) \geqslant t(\alpha)f_0(x)$. Поэтому имеем: 
    $\int_{K_{t(\alpha)}\backslash S}f_1dx - \int_{S\backslash K_{t(\alpha)}}f_1dx \geqslant
    t(\alpha) \int_{K_{t(\alpha)}\backslash S}f_0dx - t(\alpha)\int_{S\backslash K_{t(\alpha)}}f_0dx$ = [снова добавим пересечение
    и вынесем $t(\alpha)$] = $t(\alpha) \cdot (\int_{K_{t(\alpha)}}f_0dx - \int_{S}f_0dx) =
    t(\alpha) \cdot (P_0(X \in K_{t(\alpha)}) - P_0(X \in S)) \geqslant 0$ из построения критерия
     $S$ ($P_0(X \in S) \leqslant \alpha = P_0(X \in K_{t(\alpha)})$). \\
     
    Получили: $P_1(X \in K_{t(\alpha)}) - P_1(X \in S) \geqslant 0$, что и требовалось доказать.
\end{proof}

\subsection{Пример применения теоремы Неймана-Пирсона}
\begin{example}
    Пусть у нас выборка из нормального закона $N(\theta, 1)$. Пусть наша гипотеза $H_0$ говорит, что
    $\theta = \theta_0$, а альтернативная гипотеза $H_1$ говорит, что $\theta = \theta_1 > \theta_0$. \\
    
    $f_1(X) = \dfrac{1}{\sqrt{2\pi}^n} \cdot \exp(-\dfrac{1}{2} \sum_{j = 1}^{n} (X_j - \theta_1)^2)$ \\
    $f_0(X) = \dfrac{1}{\sqrt{2\pi}^n} \cdot \exp(-\dfrac{1}{2} \sum_{j = 1}^{n} (X_j - \theta_0)^2)$ \\

    Зададим критерий $K_t$ из теоремы Неймана-Пирсона (ничего в 0 не обращается --
    сразу можем поделить): \\ $K_t = \left\{\dfrac{f_1}{f_0} \geqslant t\right\} =
    \left \{exp \left( \dfrac{1}{2} \sum_{j = 1}^{n}[(X_j - \theta_0)^2 - (X_j - \theta_1)^2] \right) \geqslant t \right \}$ =
    [логарифмируем, расскрываем скобки, умножаем на два]  = $\left \{\sum_{j = 1}^n[2X_j(\theta_1 - \theta_0)] +
     n(\theta_0^2 + \theta_1^2) \geqslant 2\ln t \right \} =
     \left \{(\theta_1 - \theta_0) \overline{X_n} \geqslant \dfrac{\ln t}{n} - \dfrac{(\theta_0^2 + \theta_1^2)}{2} \right \}$
     = [по условию $\theta_1 > \theta_0 \Rightarrow$ поделим] =
     $\left \{\overline{X_n} \geqslant \dfrac{\dfrac{\ln t}{n} - \dfrac{(\theta_0^2 + \theta_1^2)}{2}}{\theta_1 - \theta_0} \right \}$ \\
     
     Таким образом пришли к тому, что $K_t =\left\{\dfrac{f_1}{f_0} \geqslant t\right\}$ равносильно
     множеству $\widetilde{K}_{s} = \left \{\overline{X_n} \geqslant s \right \}$. Равносильно в
     том смысле, что для каждого $t$ мы можем подобрать $s(t)$, что множество $K_t$ совпадает с
     $\widetilde{K}_{s(t)}$. Теперь будем искать критические множества именно в таком виде (для удобства). \\
     
    Должно выполняться: $P_0(X \in K_t) = \alpha \Leftrightarrow P_0(X \in \widetilde{K}_{s(t)}) = \alpha$.
    А что это за вероятности? Это вероятность \\ $P_{\theta_0}(\overline{X_n} \geqslant s) = \alpha$ \\
    То есть, $P_{\theta_0}(\sqrt{n}(\overline{X_n} - \theta_0) \geqslant \sqrt{n}(s - \theta_0)) = \alpha$,
    где $\sqrt{n}(\overline{X_n} - \theta_0) \sim N(0, 1)$, поэтому тут просто написано,\\ что
    $1 - \Phi(\sqrt{n}(s - \theta_0)) = \alpha$. \\
    
    Значит, выбираем квантиль нормального закона уровня $1 - \alpha$:
    $Z_{1 - \alpha} = \sqrt{n}(s - \theta_0) \Rightarrow \\ s = \theta_0 + \dfrac{Z_{1 - \alpha}}{\sqrt{n}}$. Выразили
    $s$.  \\
    
    Таким образом, наше критическое множество $\left \{ \overline{X_n} \geqslant \theta_0 + \dfrac{Z_{1 - \alpha}}{\sqrt{n}}\right \}$. Это
    критерий уровня значимости $\alpha$. \\

    Теперь посчитаем мощность (это же самый мощный критерий): \\
    $P_{\theta_1}\left (\overline{X_n} \geqslant \theta_0 + \dfrac{Z_{1 - \alpha}}{\sqrt{n}} \right) = 
    P_{\theta_1}\left (\sqrt{n}(\overline{X_n} - \theta_1) \geqslant \sqrt{n}(\theta_0
    - \theta_1) + Z_{1 - \alpha} \right) = 1 - \Phi(\sqrt{n}(\theta_0
    - \theta_1) + Z_{1 - \alpha})$. \\
    
    Заметим, что если объём выборки $n$ устремить к бесконечности, то точка, в которой мы берём
    $\Phi$ стремится к минус бесконечности (так как $(\theta_0
    - \theta_1) < 0$ по условию), поэтому мощность стремится к 1.\\
    
    По теореме Неймана-Пирсона выписанная мощность максимальна.
\end{example}
