\subsection{Абсолютная непрерывность математического ожидания.
Теорема Лебега о мажорируемой сходимости.
Подстановка сходящейся по вероятности последовательности случайных величин в непрерывную функцию.
Взаимосвязь сходимостей по вероятности и по распределению.}

\subsubsection{Абсолютная непрерывность математического ожидания.}

\begin{proposal} 
Имеем случайную величину $Y \geq 0$ п.н., матожидание конечно.

Хотим показать $\forall \eps \exists \delta > 0: P(A) \leq \delta \Rightarrow \EE \left( Y I_A \right) < \eps$.
Словами: если множество $A$ маленькое, то матожидание случайной величины, которая много где ноль, тоже маленькое.
\end{proposal} 

\begin{proof} 
Так как $Y \geq 0, \EE Y = \EE Y_{+} = \sup \left\{ \EE U: U \leq Y_{+}, \text{огр.} \right\}$. 

Фиксируем $U$ так, чтобы $\EE U \leq \EE Y_{+} \leq \EE U + \frac{\eps}{2}$
— будем пользоваться тем, что $U$ ограничена, то есть $\exists R \in \operatorname{const} : R \geq U$. 

\[ 
    \EE \left( Y I_A \right) = \EE \left( Y_{+} I_A \right) = 
    \EE \left( \left( Y_{+} - U \right) I_A \right) + \EE \left( U I_A \right) \leq
    \EE \left( Y_{+} - U \right) + R P(A) \leq 
    \frac{\eps}{2} + R \delta
\]

Выбрав $\delta = \frac{\eps}{2 R}$ получаем то что требовали.
\end{proof} 

Абсолютная непрерывность потому что на самом деле можно было взять $|Y|$ и работать с ним, 
но мы решили просто сказать что он неотрицателен удобства ради.

\subsubsection{Теорема Лебега о мажорируемой сходимости.}
\begin{theorem*}
Если
\begin{itemize} 
    \item $X_n \xrightarrow{P} X$ (или $X_n \xrightarrow{\text{п. н.}} X$ потому что следует); 
    \item существует случайная величина $Y$ такая что $|X_n| \leq Y$ п. н.; $|X| \leq Y$ п. н.,
\end{itemize} 

То $\EE X_n \rightarrow \EE X$
\end{theorem*} 

\begin{proof} 
\[
    |\EE X_n - \EE X| \leq \EE |X_n - X| = 
    \EE \left[ |X_n - X| I_{\left\{ |X_n - X| \leq \eps \right\}} \right] + 
    \EE \left[ |X_n - X| I_{\left\{ |X_n - X| > \eps \right\}} \right]
\]

$\EE \left[ |X_n - X| I_{\left\{ |X_n - X| \leq \eps \right\}} \right] \leq \eps$

$\EE \left[ |X_n - X| I_{\left\{ |X_n - X| > \eps \right\}} \right] \leq 
\EE \left[ 2 Y I_{\left\{ |X_n - X| > \eps \right\}} \right]$. Так как $X_n \xrightarrow{P} X \implies$ 
$P \left( |X_n - X| > \eps \right) \rightarrow 0$.

По Абсолютной непрерывности математического ожидания: 
$\forall \eps \exists \delta: P(A) < \delta \implies \EE \left[ Y I_A \right] < \eps$.
Здесь $P(|X_n - X| > \eps) < \delta$, поэтому применимо.

Получается $|\EE X_n - \EE X| \leq \EE |X_n - X| = \mathellipsis \leq 3 \eps$
\end{proof} 

\subsubsection{Подстановка сходящейся по вероятности последовательности случайных величин в непрерывную функцию.}
Это лекция 1, если что

\begin{proposition} 
$X_n \xrightarrow{P} X$, $g: \RR \to \RR$ — непрерывная $\implies g(X_n) \xrightarrow{P} g(X)$
\end{proposition} 

\begin{proof} 
Для фиксированного $R$, $g$ равномерно непрерывна на отрезке $[-R, R]$, то есть 
\[
    \forall \eps > 0 \exists \delta > 0: x, y \in [-R, R], |x - y| < \delta \implies |g(x) - g(y)| < \eps
\]

Рассмотрим множество 
\begin{multline*} 
    \left\{ |g(X_n) - g(X) | \geq \eps \right\} \subseteq \\
    \subseteq \left\{ |g(X_n) - g(X) | \geq \eps, X_n, X \in [-R, R] \right\} \cup
    \left\{ |g(X_n) - g(X) | \geq \eps, X_n \notin [-R, R] \right\} \cup
    \left\{ |g(X_n) - g(X) | \geq \eps, X \notin [-R, R] \right\}
\end{multline*}
\[
    P \left( |g(X_n) - g(X)| \geq \eps \right) \leq P \left( |g(X_n) - g(X) | \geq \eps, X_n, X \in [-R, R] \right) + 
    P (|X_n| > R) + P(|X| > R) 
\]

По условию равномерной непрерывности 
$P \left( |g(X_n) - g(X) | \geq \eps, X_n, X \in [-R, R] \right) \leq P \left( |X_n - X| \geq \delta \right)$ 
— иначе условие выполнялось бы, и разница между образами была бы меньше эпсилона.

Заметим, что $ \left\{ |X_n - X + X| > R \right\} \subseteq 
\left\{ |X_n - X| > \frac{R}{2} \right\} \cup \left\{ |X| > \frac{R}{2} \right\}$, тогда
$P(|X_n| > R) \leq P \left( |X_n - X| > \frac{R}{2} \right) + P \left( |X| > \frac{R}{2} \right)$

Получается 
\[
    P \left( |g(X_n) - g(X) | \geq \eps \right) \leq
    P \left( |X_n - X| \geq \delta \right) + 
    P \left( |X_n - X| > \frac{R}{2} \right) + P \left( |X| > \frac{R}{2} \right) + P ( |X| > R)
\]

Взяв большие $R$ получаем $P \left( |X| > \frac{R}{2} \right) + P \left( |X| > R \right) \to 0$ очев.

Взяв большие $n$ из-за сходимости по вероятности получаем 
$P \left( |X_n - X| \geq \delta \right) + P \left( |X_n - X| > \frac{R}{2} \right) \to 0$.

Получается 
\[ 
    0 \leq \liminf P(| g(X_n) - g(X) | \geq \eps) \leq \limsup P(| g(X_n) - g(X) | \geq \eps) \leq 0
\]
то есть $g(X_n) \xrightarrow{P} g(X)$

\end{proof} 


\subsubsection{Взаимосвязь сходимостей по вероятности и по распределению.}

\begin{corollary} 
$X_n \xrightarrow{P} X \implies X_n \xrightarrow{d} X$
\end{corollary} 

\begin{proof} 
Нужно доказать, что для любой ограниченной непрерывной $g$ верно $\EE g(X_n) \to \EE g(X)$ (по эквивалентному определению сходимости по распределению)

В силу ограниченности имеем $|g(t)| \leq M \forall t \implies g(X_n) \leq M, g(X) \leq M$

По предыдущему пункту $g(X_n) \xrightarrow{P} g(X)$

Введем случайную величину $Y = M$ и применим Лебега (оба условия выполняются), тогда $\EE g(X_n) \to \EE g(X)$, то есть
$X_n \xrightarrow{d} X$

\end{proof} 

