\ProvidesFile{quest-05.tex}[Билет 5]

\section{Билет 5}

\begin{center}
    \it Случайные величины на общих вероятностных пространствах: определение и основные свойства (прообраз борелевских лежит в $\sigma$-алгебре, сумма и произведение случайных величин --- случайная, предел случайных --- случайная)
\end{center}

\sectionbreak
\subsection{Случайные величины на общих вероятностных пространствах}

\begin{definition}
    Пусть задано вероятностное пространство $(\Omega, \mathcal{A}, P)$.
    Функция $X \colon \Omega \to \mathbb{R}$ называется {\it случайной величиной}, если
    для всякого числа $t \in \mathbb{R}$ выполнено
    \[
        X^{-1}((-\infty, t]) = \{\omega \in \Omega \mid X(\omega) \leqslant t\} \in \mathcal{A}.
    \]
\end{definition}

\begin{proposal*}
    Если $X$ случайная величина, то $\{\omega \mid X(\omega) \in B\} = X^{-1}(B) \in \mathcal{A}$ для всякого $B \in \mathcal{B}(\mathbb{R})$.
\end{proposal*}

\begin{proof}
    Напомним следующие соотношения для прообраза функции:
    \[
        X^{-1}\Bigl( \bigcap\limits_{n=1}^\infty A_n \Bigr) = \bigcap\limits_{n=1}^\infty X^{-1}(A_n), \quad X^{-1}\Bigl(\bigcup\limits_{n = 1}^\infty A_n\Bigr) = \bigcup\limits_{n = 1}^\infty X^{-1}(A_n), \quad X^{-1}(\mathbb{R} \setminus B) = \Omega \setminus X^{-1}(B).
    \]
    Рассмотрим систему множеств
    \[
        \mathcal{C} := \{B \subset \mathbb{R} \mid X^{-1}(B) \in \mathcal{A}\}.
    \]
    Эта система образует $\sigma$-алгебру.
    Действительно, $X^{-1}(\emptyset) = \emptyset \in \mathcal{A}$ и
    $X^{-1}(\mathbb{R}) = \Omega \in \mathcal{A}$.
    Если $B \in \mathcal{C}$, то $X^{-1}(\mathbb{R} \setminus B) = \Omega \setminus X^{-1}(B) \in \mathcal{A}$.
    Наконец, если $B_n \in \mathcal{C}$, то
    \[
        X^{-1}\Bigl(\bigcap\limits_{n = 1}^\infty B_n\Bigr) = \bigcap\limits_{n = 1}^\infty X^{-1}(B_n) \in \mathcal{A}.
    \]
    По условию $\sigma$-алгебра $\mathcal{C}$ содержит все лучи вида $(-\infty, t]$.
    Мы знаем, что $\mathcal{B}(\mathbb{R})$ --- наименьшая по включению $\sigma$ алгебра, содержащая все лучи такого вида, поэтому $\mathcal{B}(\mathbb{R}) \subset \mathcal{C}$, что и требовалось.
\end{proof}

\begin{comment*}
    Т. к. $\{ X^2 \leqslant t \} = \{-\sqrt{t} \leqslant x \leqslant \sqrt{t}\}$ (при $t \geq 0$) и отрезок $[-\sqrt{t}, \sqrt{t}]$ --- борелевское множество, получаем, что $X^2$ --- также случайная величина.
    Можно проверить, что для случайной величины $X$ и для любой \enquote{разумной} функции $f\colon \mathbb{R}\to \mathbb{R}$
    (например, если $f$ непрерывная), $f(X)$ также будет случайной величиной.
\end{comment*}

\begin{proposal*}
    Пусть $X, Y$ --- случайные величины.
    Тогда случайными величинами будут $\alpha X + \beta Y$, $X \cdot Y$.
\end{proposal*}

\begin{proof}
    Ясно, что $\alpha X$ и $\beta Y$ --- случайные величины.
    Проверим, что $X + Y$ --- случайная величина:
    \[
        \{X+Y > t\} = \{ X > t - Y\} = \bigcup\limits_{r_n \in \mathbb{Q}}(\{\omega \mid X(\omega) > r_n\} \cap \{\omega \mid r_n > t - Y(\omega)\}) \in \mathcal{A}.
    \]
    В последнем переходе мы воспользовались тем, что $\mathbb{Q}$ всюду плотно в $\mathbb{R}$, поэтому между любыми двумя вещественными числами есть рациональное число.
    Поэтому и $\{X + Y \leqslant t\} \in \mathcal{A}$, а значит $X + Y$ --- случайная величина.
    Для произведения заметим, что $X \cdot Y = \frac{1}{2}\bigl((X + Y)^2 - X^2 - Y^2\bigr)$, и утверждение следует из уже доказанных.
\end{proof}

\begin{proposal*}
    Пусть $X_n$ --- случайные величины и для всякого $\omega$ существует предел $\lim_{n \to \infty} X_n(\omega) = X(\omega)$.
    Тогда $X$ является случайной величиной.
\end{proposal*}

\begin{proof}
    Рассмотрим множество $\{\omega \colon ~ X(\omega) \leqslant t\}$. Заметим, что $X(\omega) \leqslant t$ тогда и только тогда,
    когда для каждого натурального числа $k$ найдётся такой номер $N$, что для всех $n > N$ верно неравенство $X_n(\omega)
    \leqslant t + \frac{1}{k}$. На языке теории множеств эту формулу фразу можно записать так
    \[
        \{\omega \colon ~ X(\omega) \leqslant t\} = \bigcap_k \bigcup_N \bigcap_{n > N} \{\omega \colon ~ X(\omega) \leqslant t + \frac{1}{k}\}
    \]
    Остаётся заметить, что $\{\omega \colon ~ X(\omega) \leqslant t + \frac{1}{k}\} \in \mathcal{A}$
\end{proof}

Таким образом, со случайными величинами можно выполнять арифметические операции и переходить к пределу.