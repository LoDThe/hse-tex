\ProvidesFile{quest-08.tex}[Билет 8]

\section{Билет 8}

\begin{center}
    \it Совместное распределение случайных величин, корректность определения.
    Функция совместного распределения и четыре ее основных свойства.
    Формулировка теоремы об однозначности задания совместного распределения функцией совместного распределения и о существовании распределения с заданой функцией совместного распределения (т.е. с функцией, удовлетворяющей четырем свойствам).
    Неоднозначность задания совместного распределения распределениями компонент (пример).
\end{center}

\sectionbreak
\subsection{Совместное распределение случайных величин, корректность определения}

\begin{definition}
    Пусть $X$ и $Y$ --- случайные величины.
    {\it Совместным распределением} случайных величин $X, Y$ называется вероятностная мера $\mu_{X, Y}$ на $\mathcal{B}(\mathbb{R}^2)$,
    определяемая следующим образом:
    \[
        \mu_{X, Y}(B) = P(\{\omega \mid (X(\omega), Y(\omega)) \in B\}).
    \]
\end{definition}

\begin{proposal*}
    Определение выше корректно в том смысле, что для $B \in \mathcal{B}(\mathbb{R}^2)$
    \[
        \{\omega \mid (X(\omega), Y(\omega))\in B\} \in \mathcal{A}.
    \]
\end{proposal*}

\begin{proof}
    Рассмотрим отображение $g \colon \Omega \to \mathbb{R}^2$, $g(\omega) = (X(\omega), Y(\omega))$.
    Аналогично тому, как мы уже делали, проверяется, что система множеств
    \[
        \mathcal{C} := \{B \subset \mathbb{R}^2 \mid g^{-1}(B) \in \mathcal{A}\}
    \]
    является $\sigma$-алгеброй.
    Заметим, что параллелепипеды $[a, b] \times [c, d] \in \mathcal{C}$, т.к.
    \[
        g^{-1}([a, b] \times [c, d]) = \{\omega \mid X(\omega) \in [a, b], Y(\omega) \in [c, d]\} = \{\omega \mid X(\omega) \in [a, b]\} \cap \{\omega \mid Y(\omega) \in[c, d]\}.
    \]
    Тем самым, $\mathcal{C}$ --- некотрая $\sigma$-алгебра, содержащая все параллелепипеды, а $\mathcal{B}(\mathbb{R}^2)$ --- это наименьшая по включению $\sigma$-алгебра, содержащая все параллелепипеды.
\end{proof}

\sectionbreak
\subsection{Функция совместного распределения и четыре ее основных свойства.}

\begin{definition}
Функцию
\[
    F_{X, Y}(x, y) = P(\{\omega \mid X(\omega) \leqslant x,~ Y(\omega) \leqslant y\}) = \mu_{X, Y}((-\infty, x] \times (-\infty, y]).
\]
называют {\it функцией совместного распределения} случайных величин $X$ и $Y$ или функцией распределения случайного вектора $(X, Y)$.
\end{definition}

\begin{proposal*}
    Функция $F$ совместного распределения пары случайных величин удовлетворяет следующим свойствам:
    \begin{enumerate}
        \item $F \colon \mathbb{R}^2 \to [0, 1]$ и $F(b, d) - F(a, d) - F(b, c) + F(a, c) \geqslant 0$ для всякого прямоугольника $(a, b] \times (c, d]$;
        \item $F$ непрерывна справа по совокупности переменных;
        \item $\lim\limits_{(x, y) \to (u, v)} F(x, y) = 0$ если хотя бы одна из переменных $u$ или $v$ равна $-\infty$;
        \item $\lim\limits_{(x, y) \to (+\infty, +\infty)} F(x, y) = 1$.
    \end{enumerate}
\end{proposal*}
\begin{proof}
    Доказательство повторяет рассуждения одномерного случая.
    Например, докажем $(2)$.
    Заметим, что
    \[
        \bigcap\limits_{k \in \mathbb{N}}\{\omega \mid X(\omega) \leqslant x + \tfrac{1}{k}, Y(\omega) \leqslant y + \tfrac{1}{k}\} =
    \{\omega \mid X(\omega) \leqslant x, Y(\omega) \leqslant y\}.
    \]
    Поэтому $P(X \leqslant x + \tfrac{1}{k}, Y \leqslant y + \tfrac{1}{k}) \to P(X \leqslant x, Y \leqslant y)$ и для каждого $\varepsilon$ найдется такое $k$, что
    \[
        P(X \leqslant x + \tfrac{1}{k}, Y \leqslant y + \tfrac{1}{k}) < P(X \leqslant x, Y \leqslant y) + \varepsilon.
    \]
    Если теперь $x_n \to x$, $x_n \geqslant x$, $y_n \to y$, $y_n \geqslant y$, то для произвольного $k$ найдется номер $n_0$, начиная с которого выполняется
    \[
        x \leqslant x_n < x + \tfrac{1}{k},~ y \leqslant y_n < y + \tfrac{1}{k}.
    \]
    Поэтому при $n > n_0$
    \[
        P(X \leqslant x, Y \leqslant y) \leqslant P(X \leqslant x_n, Y \leqslant y_n) \leqslant P(X \leqslant x + \tfrac{1}{k}, Y \leqslant y + \tfrac{1}{k}) < P(X \leqslant x, Y \leqslant y) + \varepsilon.
    \]
    Утверждения $(3)$ и $(4)$ обосновываются аналогично.
\end{proof}

\sectionbreak
\subsection{Формулировка теоремы об однозначности задания совместного распределения функцией совместного распределения и о существовании распределения с заданой функцией совместного распределения}

\begin{theorem}
Совместное распределение пары случайных величин $\mu_{X, Y}$ однозначно задается функцией совместного распределения $F_{X, Y}$.
Кроме того, для всякой функции $F$, удовлетворяющей свойствам $(1), (2), (3), (4),$ существует вероятностное пространство $(\Omega, \mathcal{A}, P)$ и пара случайных величин $X, Y$ с функцией совместного распределения $F$.
\end{theorem}

\sectionbreak
\subsection{Неоднозначность задания совместного распределения распределениями компонент}

\paragraph{Пример}
Пусть в квадрате $[0, 1] \times [0, 1]$ случайно выбирается точка $(x, y)$.
Случайные величины $X(x, y) = x$ и $Y(x, y) = y$ имеют равномерное распределение на $[0, 1]$ и их совместное распределение является равномерным на $[0, 1]\times [0, 1]$, т. е. вероятность попадания в множество $B$ равна площади этого множества.
Будем теперь выбирать точку $(x, y)$ случайным образом на диагонали квадрата $[0, 1] \times [0, 1]$, а случайные величины останутся прежними.
Для всякого отрезка $[a, b] \subset [0, 1]$ вероятность того, что $(x, y) \in [a, b] \times \mathbb{R}$ равна вероятности попасть в отрезок длины $(b - a)\sqrt{2}$ при бросании точки на отрезок длины $\sqrt{2}$, т. е. равна $b - a$. Таким образом, $X$ и $Y$ опять имеют равномерное распределение на $[0, 1]$, но совместное распределение у них совсем другое.