\ProvidesFile{quest-02.tex}[Билет 2]

\section{Билет 2}

\begin{center}
    \it Модель Эрдеша-Реньи случайного графа.
    Теорема о надежности сети.
\end{center}

\sectionbreak
\subsection{Модель Эрдеша-Реньи случайного графа}

\begin{definition*}[модель Эрдеша-Реньи]
    Пусть $V_n$ --- конечное множество $\{1, 2, \ldots, n\}$, элементы которого мы называем {\it вершинами}.
    Будем проводить между двумя различными вершинами ребро (только одно) с вероятностью $p$ независимо от остальных пар вершин.
    Получающийся граф будем называть {\it случайным графом в модели Эрдеша-Реньи}.
\end{definition*}

Множество элементарных исходов $\Omega$ состоит из $C_n^2$ ребер.
Событием называется любое подмножество ребер в клике на $n$ вершинах $E \subseteq \Omega$.
Вероятность $E$ задается формулой
\[
    P(E) = p^{\abs{E}}(1 - p)^{C_n^2 - \abs{E}}.
\]

\sectionbreak
\subsection{Теорема о надежности сети}

\begin{theorem*}[о надежности сети в общем случае]
    Если $p = \frac{c \ln n}{n}$, то при $c > 1$ вероятность того, что граф связен, стремится к $1$ (граф почти всегда связен), а при $c < 1$ вероятность того, что граф связен, стремится к $0$ (граф почти всегда не связен).
\end{theorem*}

\begin{theorem*}[о надежности сети в частном случае]
    Если $p = \frac{c \ln n}{n}$ и $c > 2$, то граф почти всегда связен.
\end{theorem*}

\begin{proof}
    Пусть случайная величина $X_n$ --- число компонент связности в случайном графе $G$, если граф не является связным, и $X_n = 0$ в случае связности $G$.
    Нам надо доказать, что $P(X_n > 1) \to 0$ при $n \to \infty$.
    По неравенству Чебышева
    \[
        P(X_n > 1) \leqslant \E X_n.
    \]
    Следовательно, достаточно доказать стремление к нулю $\E X_n$.
    Пусть $K_1, \ldots, K_{C_n^k}$ --- все $k$-элементные подмножества $V_n$.
    Через $X_{n, k, i}$ обозначим случайную величину, которая равна единице в случае, когда $K_i$ является компонентой связности, и равна нулю в случае, когда это не так.
    Ясно, что
    \[
        \E X_n = \sum_{k = 1}^{n - 1} \sum_{i = 1}^{C_n^k} \E X_{n, k, i}.
    \]
    Заметим, что $\E X_{n, k, i} = P(X_{n, k, i} = 1)$, а эта вероятность оценивается\footnote{Мы говорим, что любая компонента связности на $k$ вершинах никак не соединена с оставшимися $n - k$ вершинами, однако не все графы, для которых это верно, являются компонентами связности.} через вероятность того, что вершины из множества $K_i$ не соединены ребрами с вершинами из $V_n \setminus K_i$.
    Пусть $q = 1-p$, тогда имеет место оценка\footnote{Выбрали вершину (всего $k$ штук) и удалили ребра из нее в оставшиеся $n - k$ вершин.}
    \[
        \E X_n \leqslant \sum_{k = 1}^{n - 1} C_n^k q^{k(n - k)}.
    \]
    Эта сумма симметрична и, удваивая ее, можно считать, что суммирование идет по $k \leqslant \frac{n}{2}$.
    При таких $k$ имеет место неравенство $k (n - k) \geqslant k (n - \frac{n}{2}) = \frac{kn}{2}$.
    Добавим и вычтем $1 + q^{n^{2} / 2}$, чтобы можно было свернуть по формуле бинома Ньютона
    $$
    \sum_{k = 1}^{n - 1}C_n^k q^{k(n - k)} \leqslant 2 \sum_{k = 1}^{n - 1} C_n^k (q^{n / 2})^k = 2(1 + q^{n / 2})^n - 2 - 2q^{n^2 / 2}.
    $$
    По условию, $q = 1 - p = \frac{2a \ln n}{n}$, где $a > 1$.
    Имеем
    $$
    q^{n / 2} = e^{2^{-1}n \ln(1 - \frac{2a \ln n}{n})} = e^{-a \ln n + \beta_n} = \frac{1}{n^a}e^{\beta_n}, \quad \beta_n \to 0.
    $$
    Следовательно,
    $$
    (1 + q^{n/2})^n = \Bigl( 1 + \frac{1}{n^a}e^{\beta_n} \Bigr)^n \to 1,
    $$
    и
    $$
    2(1 + q^{n / 2})^n - 2 - 2q^{n^2 / 2} \to 0.
    $$
    Таким образом, $\E X_n \to 0$ и теорема доказана.
\end{proof}