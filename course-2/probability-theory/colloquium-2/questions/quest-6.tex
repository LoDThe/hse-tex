\ProvidesFile{quest-06.tex}[Билет 6]

\section{Билет 6}

\begin{center}
    \it Распределение случайной величины и функция распределения.
    Три основных свойства функции распределения.
    Формулировка теоремы о продолжении счетно аддитивной функции множества с алгебры на $\sigma$-алгебру, порожденную исходной алгеброй.
    Идея построения меры Лебега равномерного распределения на отрезке с борелевской $\sigma$-алгебой.
    Формулировка теоремы об однозначности задания распределения функцией распределения и о существовании распределения с заданой функцией распределения (т.е. с функцией, удовлетворяющей трем свойствам).
    Идея доказательства.
\end{center}

\sectionbreak
\subsection{Распределение случайной величины и функция распределения}

\begin{definition}
    {\it Распределением} случайной величины $X$ называется вероятностная мера $\mu_{X}$ на $\mathcal{B}(\mathbb{R})$, определяемая равенством
    \[
        \mu_{X}(B) = P(\{\omega \mid X(\omega) \in B\}) = P(X^{-1}(B)).
    \]
\end{definition}

Обратим внимание, что, как и в дискретном случае, распределение случайной величины это мера на значениях случайной величины, т. е. мера $\mu_{X}$ показывает с какой вероятностью принимаются те или иные значения $X$.

\begin{definition}
Функция
\[
    F_{X}(t) = \mu_{X}((-\infty, t]) = P(\{\omega \mid X(\omega) \leqslant t\}).
\]
называется {\it функцией распределения} случайной величины $X$.
\end{definition}

Из определения $F_X$ следует, что $P(a < X \leqslant b) = \mu_X((a,b]) = F(b) - F(a)$.

\sectionbreak
\subsection{Три основных свойства функции распределения}

\begin{proposal*}
    Функция $F_X$ удовлетворяет следующим свойствам:
    \begin{enumerate}
        \item $F_X\colon \mathbb{R}\to [0, 1]$ не убывает;
        \item $F_X$ непрерывна справа;
        \item $\lim\limits_{t \to -\infty} F_X(t) = 0$ и $\lim\limits_{t \to +\infty} F_X(t) = 1$.
    \end{enumerate}
\end{proposal*}

\begin{proof}
    Т.к. $\{\omega \mid X(\omega) \leqslant t\} \subset \{\omega \mid X(\omega) \leqslant s\}$ при $t \leqslant s$, то получаем cвойство 1.

    \noindent Обоснуем пункт 2.
    Пусть $t_n \to t$, $t_n \geqslant t$.
    Заметим, что
    \[
        \{X \leqslant t\} = \bigcap\limits_{k = 1}^\infty\{X \leqslant t + \tfrac{1}{k}\}.
    \]
    В силу непрерывности вероятностной меры $P$ получаем
    \[
        \lim_{k \to \infty}F_X(t + \tfrac{1}{k}) = \lim_{k \to \infty}P(X \leqslant t + \tfrac{1}{k}) = P(X \leqslant t) = F_X(t).
    \]
    Значит для каждого $\varepsilon > 0$ найдется $k$, для которого
    \[
        F_X(t) \leqslant F_X(t + \tfrac{1}{k}) \leqslant F_X(t) + \varepsilon.
    \]
    Т.к. $t_n \to t$, $t_n \geqslant t$, то найдется номер $n_0$, начиная с которого $t \leqslant t_n < t + \frac{1}{k}$.
    В силу монотонности $F_x(t) \leqslant F_X(t_n) \leqslant F_X(t + \frac{1}{k}) \leqslant F_X(t) + \varepsilon$
    при $n\geqslant n_0$.
    Это и означает, что $\lim\limits_{n \to \infty}F_X(t_n) = F(t)$.

    Свойство 3 обосновывается аналогично.
\end{proof}

\sectionbreak
\subsection{Формулировка теоремы о продолжении счетно аддитивной функции множества с алгебры на $\sigma$-алгебру, порожденную исходной алгеброй}

\begin{theorem}[б/д]
    Пусть $\mathcal{A}_0$ есть некоторая алгебра подмножеств пространства $\Omega$ и пусть $P_0 \colon \mathcal{A}_0 \to [0, 1]$ {\it счетно аддитивная} функция множества на алгебре $\mathcal{A}_0$.
    Тогда существует единственная вероятностная мера $P$ на $\sigma(\mathcal{A}_0)$, продолжающая функцию $P_0$, т.е. $P(A) = P_0(A)$ для произвольного множества $A\in\mathcal{A}_0$.
\end{theorem}

\sectionbreak
\subsection{Идея построения меры Лебега равномерного распределения на отрезке с борелевской $\sigma$-алгебой}

Мера Лебега --- обычная длина, т. е. $\lambda([a, b]) = b - a$.

\paragraph{Схема построения меры Лебега}
Рассмотрим алгебру $\mathcal{A}_0$ конечных объединений попарно непересекающихся промежутков вида $(a, b] \subset [0, 1]$ и возможно одноточечного множества $\{0\}$.
Для множества $A = \bigsqcup\limits_{j = 1}^m (a_j, b_j]$ с попарно непересекающимися $(a_j, b_j]$ зададим меру Лебега равенством
\[
    \lambda(A) := \sum\limits_{j = 1}^{m}(b_j - a_j).
\]
Нетрудно проверить, что это корректно определенная аддитивная функция множества на $\mathcal{A}_0$.
Если теперь проверить, что она оказывается счетно аддитивной на этой алгебре (что верно), то по теореме о продолжении меры существует единственная вероятностная мера на $\mathcal{B}([0, 1])$,
совпадающая с $\lambda$ на $\mathcal{A}_0$.

\sectionbreak
\subsection{Формулировка теоремы об однозначности задания распределения функцией распределения и о существовании распределения с заданой функцией распределения}

\begin{theorem}
    Распределение $\mu_X$ однозначно определяется функцией распределения $F_X$.
    Кроме того, если задана функция $F$, удовлетворяющая свойствам 1, 2, 3, то существует вероятностное пространство $(\Omega, \mathcal{A}, P)$ и случайная величина $X$ с функцией распределения $F$.
\end{theorem}

Эта теорема позволяет говорить о распределении случайной величины без уточнения, на каком вероятностном пространстве задана случайная величина и как именно она задана.

\paragraph{Идея доказательства}
Наметим основные идеи доказательства.
Первая часть является прямым следствием теоремы о продолжении меры.
Пусть $A = \bigcup_{j = 1}^n (a_j, b_j]$, причем $(a_j, b_j] \cap (a_k, b_k] = \emptyset$ при $j \neq k$.
Тогда
\[
    \mu_X(A) = \sum\limits_j F_X(b_j) - F_X(a_j).
\]
Кроме того, множества $A$ указанного вида образуют алгебру $\mathcal{A}_0$ подмножеств $\mathbb{R}$.
Поэтому, если есть две случайные величины с одной и той же функцией распределения, то по теореме о продолжении меры (часть о единственности продолжения) их распределения также совпадают на всех множествах из $\sigma(\mathcal{A}_0) = \mathbb{B}(\mathbb{R})$.

Доказательство второй части аналогично рассуждению о построении меры Лебега.
Будем строить вероятностную меру $P$ на $\Omega = \mathbb{R}$ с $\mathcal{A} = \mathcal{B}(\mathbb{R})$.
Рассмотрим алгебру $\mathcal{A}_0$ множеств вида $A = \bigcup\limits_{j = 1}^n (a_j, b_j]$, где $(a_j, b_j] \cap (a_k, b_k] = \emptyset$ при $j \neq k$.
Для такого множества $A$ положим $P(A) := \sum\limits_{j = 1}^{n} F(b_j) - F(a_j)$.
Нетрудно видеть, что корректно определена (т.е. для разных представлений $A$ равенство дает одно и тоже число) аддитивная функция множества на алгебре $\mathcal{A}_0$.
Если теперь суметь проверить счетную аддитивность $P$ на $\mathcal{A}_0$, то $P$ продолжается до счетно аддитивной меры на $\sigma(\mathcal{A}_0) = \mathcal{B}(\mathbb{R})$.
Если теперь рассмотреть случайную величину $X(\omega)=\omega$, то $F_X(t) = F(t)$ при $t\in \mathbb{R}$ в силу того, что $P((-\infty, t])$, являясь продолжением, стовпадает с $F(t)-F(-\infty) = F(t)$.