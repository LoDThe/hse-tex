\ProvidesFile{quest-04.tex}[Билет 4]

\section{Билет 4}

\begin{center}
    \it Аддитивные и счетно аддитивные функции множества на алгебрах и $\sigma$-алгебрах.
    Вероятностная мера и определение вероятностного пространства.
    Эквивалентность счетной аддитивности и непрерывности в нуле для неотрицательной аддитивной функции множества на алгебре.
    Свойства непрерывности вероятностной меры.
\end{center}

\sectionbreak
\subsection{Аддитивные и счетно аддитивные функции множества на алгебрах и $\sigma$-алгебрах}

\begin{definition}
    Пусть $\mathcal{A}_0$ --- алгебра множеств.
    Функция $P \colon \mathcal{A}_0 \to [0, 1]$ называется {\it аддитивной}, если для произвольных $A, B \in \mathcal{A}_0$, $A \cap B = \emptyset$ выполнено
    \[
        P(A\cup B) = P(A) + P(B).
    \]
\end{definition}

\begin{definition}
    Функция $P \colon \mathcal{A}_0 \to [0, 1]$ называется {\it счетно аддитивной}, если для всякого не более чем счетного набора попарно непересекающихся событий $A_n \in \mathcal{A}_{0}$, для которых $\bigcup\limits_{n = 1}^\infty A_n \in \mathcal{A}_0$
    выполняется
    \[
        P\left( \bigcup\limits_{n = 1}^\infty A_n \right) = \sum_{n = 1}^\infty P(A_n).
    \]
\end{definition}

\sectionbreak
\subsection{Вероятностная мера и определение вероятностного пространства}

\begin{definition}
    Пусть $\mathcal{A}$ -- $\sigma$ алгебра. Функция $P\colon\mathcal{A}\to[0, 1]$ называется {\it вероятностной мерой} на $\mathcal{A}$, если $P(\Omega) = 1$ и $P$ --- счетно аддитивна на $\mathcal{A}$.
\end{definition}

\begin{definition}
    Пусть $\mathcal{A}$ --- $\sigma$-алгебра подмножеств $\Omega$, тогда тройку $(\Omega, \mathcal{A}, P)$ называют {\it вероятностным пространством}.
\end{definition}

\sectionbreak
\subsection{Эквивалентность счетной аддитивности и непрерывности в нуле для неотрицательной аддитивной функции множества на алгебре}

\begin{proposal*}
    Пусть $P \colon \mathcal{A}_0 \to [0, 1]$ --- аддитивная функция множества на алгебре $\mathcal{A}_0$.
    Функция $P$ счетно аддитивна на $\mathcal{A}_0$ тогда и только тогда, когда для произвольного набора $A_n \in \mathcal{A}_{0}$, $A_{n + 1}\subset A_n$, $\bigcap\limits_{n = 1}^\infty A_n = \emptyset$ выполнено
    \[
        \lim\limits_{n \to \infty}P(A_n) = 0.
    \]
\end{proposal*}

\begin{proof}~
    \begin{description}
        \item[$\implies$] Пусть $P$ счетно аддитивна на $\mathcal{A}_0$.
        Рассмотрим множества $C_n = A_n \setminus A_{n + 1}$.
        Тогда
        \[
            A_1 = \bigcup\limits_{n = 1}^\infty C_n, \ldots, A_{N + 1} = \bigcup\limits_{k = N + 1}^{\infty} C_k,
        \]
        и
        \[
            P(A_1) = \sum\limits_{n = 1}^N P(C_n) + P(A_{N + 1}).
        \]
        Если $P$ счетно аддитивна, то $\sum\limits_{n = 1}^N P(C_n) \to P(A_1)$, а $P(A_{N + 1}) \to 0$.
        \item[$\impliedby$] Пусть $C_n \in \mathcal{A}_0$ --- набор попарно непересекающихся множеств, причем известно, что $\bigcup\limits_{n = 1}^\infty C_n = A_1 \in \mathcal{A}_0$.
        Пусть
        \[
            A_{N + 1} = \bigcup\limits_{k = N + 1}^{\infty} C_k,
        \]
        тогда $A_{N + 1} \subset A_N$, причем $\bigcap\limits_{N = 1}^\infty A_N = \emptyset$.
        Если $P(A_{N+1})\to 0$, то $P(A_1)=\sum\limits_{n=1}^N P(C_n) + P(A_{N+1})$ и переходя к пределу, получаем
        \[
            P(A_1) = \sum\limits_{n = 1}^\infty P(C_n). \qedhere
        \]
    \end{description}
\end{proof}

\sectionbreak
\subsection{Свойства непрерывности вероятностной меры}

\begin{corollary*}[непрерывность вероятностной меры]
    Пусть $(\Omega, \mathcal{A}, P)$ --- вероятностное пространство.
    Тогда
    \begin{enumerate}
        \item Если $A_n \in \mathcal{A}$, $A_{n + 1} \subset A_n$ и $A = \bigcap\limits_{n = 1}^\infty A_n$, то $\lim\limits_{n \to \infty} P(A_n) = P(A)$;
        \begin{proof}
            Рассмотрим $A_n' = A_n \setminus A$. Очевидно, $\bigcap\limits_{i=1}^{\infty} A_i' = \varnothing \implies \lim\limits_{n \to \infty} P(A_n') = 0$. В то же время $P(A_n) = P(A_n' \sqcup A) = P(A_n') + P(A)$. Значит, $\lim\limits_{n \to \infty} P(A_n) = P(A)$
        \end{proof}
        \item Если $A_n \in \mathcal{A}$, $A_n \subset A_{n + 1}$ и $A = \bigcup\limits_{n = 1}^\infty A_n$, то $\lim\limits_{n \to \infty} P(A_n) = P(A)$.
        \begin{proof}
            Рассмотрим $A_n' = \Omega \setminus A_n$. Тогда $\bigcap\limits_{i=1}^{\infty} A_n' = \Omega \setminus A$. По п. 1:
            \[
                1 - P(A) = P(\Omega \setminus A) = P\left(\bigcap\limits_{i=1}^{\infty} A_n'\right) = \lim\limits_{n \to \infty} P(A_n') = \lim\limits_{n \to \infty}P(\Omega \setminus A_n) = 1 - \lim\limits_{n \to \infty}P(A_n)
            \]
        \end{proof}
    \end{enumerate}
\end{corollary*}

В частности,
\[
    P \Bigl(\bigcup\limits_{n = 1}^\infty A_n\Bigr) = \lim\limits_{N \to \infty} P\Bigl(\bigcup\limits_{n = 1}^N A_n\Bigr) \leqslant \sum\limits_{n = 1}^{\infty} P(A_n).
\]