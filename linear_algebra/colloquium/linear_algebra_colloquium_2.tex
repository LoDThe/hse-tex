\documentclass[a4paper]{article}
\usepackage{header}


\newcommand\enumtocitem[3]{\item\textbf{#1}\addtocounter{#2}{1}\addcontentsline{toc}{#2}{\protect{\numberline{#3}} #1}}
\newcommand\defitem[1]{\enumtocitem{#1}{subsection}{\thesubsection}}
\newcommand\proofitem[1]{\enumtocitem{#1}{subsubsection}{\thesubsubsection}}

\newlist{colloq}{enumerate}{1}
\setlist[colloq]{label=\textbf{\arabic*.}}


\title{\HugeЛинейная алгебра, Коллоквиум \uppercase\expandafter{\romannumeral 2\relax}}
\author{
    Бобень Вячеслав \\
    \href{https://teleg.run/darkkeks}{@darkkeks},
    \href{https://github.com/LoDThe/hse-tex}{GitHub} \\
    Благодарность выражается Левину Александру (\href{https://teleg.run/azerty1234567890}{@azerty1234567890}) \\
    и Милько Андрею (\href{https://teleg.run/andrew_milko}{@andrew\_milko}) за видеозаписи лекций.
}

\usepackage[yyyymmdd,hhmmss]{datetime}
\settimeformat{xxivtime}
\renewcommand{\dateseparator}{.}
\date{Дата изменения: \today \ в \currenttime \\ 2019 --- 2020}

\begin{document}
    \maketitle

    \epigraph{
        ``К коллоку можете даже не готовиться''.
    }{\rightline{{\rm --- Роман Сергеевич Авдеев}}}

    \tableofcontents

    \newpage

    \section{Определения и формулировки}

    \begin{colloq}

    \defitem{Сумма двух подпространств векторного пространства}

        Пусть $V$ -- векторное пространство над $F$.

        $U, W \subseteq V$ -- подпространства.

        \begin{definition}
            \textit{Суммой} подпространств $U$, $W$ называется множество
            \begin{equation*}
                U + W := \{u + w \mid v \in U, w \in W\}
            .\end{equation*}
        \end{definition}


    \defitem{Теорема о связи размерности суммы двух подпространств с размерностью их пересечения}

        \begin{theorem}
            $\dim (U \cap W) + \dim (U + W) = \dim U + \dim W$.
        \end{theorem}

        \begin{example}
            Всякие две плоскости в $\RR^3$ (содержащие 0) имеют общую прямую.

            Здесь $V = \RR^3$, $\dim U = 2$, $\dim W = 2$.

            При этом $\dim (U + W) \leq 3$.

            Тогда, $\dim (U \cap W) = \dim U + \dim W - \dim (U + W) \geq 2 + 2 - 3 = 1$.
        \end{example}


    \defitem{Сумма нескольких подпространств векторного пространства}

        Пусть $U_1, \dots U_k \subseteq V$ -- подпространства.

        \begin{definition}
            \textit{Суммой} подпространств $U_1, \dots U_k$ называется множество
            \begin{equation*}
                U_1 + \dots + U_k = \{u_1 + \dots + u_k \mid u_i \in U_i\}
            .\end{equation*}
        \end{definition}

        \begin{comment}
            $\dim (U_1 + \dots + U_k) \leq \dim U_1 + \dots + \dim U_k$.
        \end{comment}


    \defitem{Линейная независимость нескольких подпространств векторного пространства}

        \begin{definition}
            Подпространства $U_1, \dots, U_k$ называются \textit{линейно независимыми}, если $\forall u_1 \in U_1, \dots, u_k \in U_k$ из условия $u_1 + \dots + u_k = 0$ следует $u_1 = \dots = u_k = 0$.
        \end{definition}

        \begin{example}
            Если $\dim U_i = 1$ и $U_i = \langle u_i \rangle \ \forall i$, то $U_1, \dots, U_k$ линейно независимы $\iff$ $u_1, \dots, u_k$ линейно независимы.
        \end{example}


    \defitem{Разложение векторного пространства в прямую сумму подпространств}

        \begin{definition}
            Говорят, что векторное пространство $V$ разлагается в \textit{прямую сумму} $U_1, \dots, U_k$, если
            \begin{enumerate}
            \item $V = U_1 + \dots + U_k$,
            \item $U_1, \dots, U_k$ линейно независимы.
            \end{enumerate}

            Обозначение: $V = U_1 \oplus U_2 \oplus \dots \oplus U_k$.
        \end{definition}

        \begin{example}
            Если $e_1, \dots, e_n$ -- базис $V$, то $V = \langle e_1 \rangle \oplus \langle e_2 \rangle \oplus \dots \oplus \langle e_n \rangle$
        \end{example}


    \defitem{При каких условиях на подпространства $U_1$, $U_2$ векторного пространства $V$ имеет место разложение $V = U_1 \oplus U_2$?}

        \begin{math}
            V = U_1 \oplus U_2 \iff \begin{cases}
                V = U_1 + U_2, \\
                U_1 \cap U_2 = 0,
            \end{cases} \iff \begin{cases}
                \dim V = \dim U_1 + \dim U_2, \\
                U_1 \cap U_2 = 0.
            \end{cases}
        \end{math}


    \defitem{Проекция вектора на подпространство вдоль дополнительного подпространства}

        \begin{comment}
            $V = U_1 \oplus U_2 \implies \forall v \in V \ \exists! u_1 \in U_1, u_2 \in U_2$, такие что $v = u_1 + u_2$.

            Тогда, $u_1$ называется проекцией вектора $v$ на $U_1$ вдоль $U_2$.
            
            Так же, $u_2$ называется проекцией вектора $v$ на $U_2$ вдоль $U_1$.
        \end{comment}


    \defitem{Матрица перехода от одного базиса векторного пространства к другому}

        Пусть $(e_1, \dots, e_n)$ и $(e'_1, \dots, e'_n)$ --- два базиса в $V$,
        \begin{equation*}
            (e'_1, \dots, e'_n) = (e_1, \dots, e_n) \cdot C
        ,\end{equation*}
        при этом $\det C \neq 0$.

        \begin{definition}
            Матрица $C$ называется \textit{матрицей перехода} от базиса $(e_1, \dots, e_n)$ к базису $(e'_1, \dots, e'_n)$.
        \end{definition}

        \begin{comment}
            Матрица перехода от $(e'_1, \dots, e'_n)$ к $(e_1, \dots, e_n)$ --- это $C^{-1}$.
        \end{comment}


    \defitem{Формула преобразования координат вектора при замене базиса}
        
        Пусть $C$ --- матрица перехода от базиса $\E = (e_1, \dots, e_n)$ к базису $\E' = (e'_1, \dots, e'_n)$, $v \in V$, тогда
        \begin{align*}
            v &= x_1 e_1 + \dots + x_n e_n \\
            v &= x'_1 e'_1 + \dots + x'_n e'_n
        .\end{align*}

        \begin{proposal}
            \begin{equation*}
                \begin{pmatrix} x_1 \\ \dots \\ x_n \end{pmatrix} 
                = C \cdot \begin{pmatrix} x'_1 \\ \dots \\ x'_n \end{pmatrix}
            .\end{equation*}
        \end{proposal}


    \defitem{Линейное отображение векторных пространств, его простейшие свойства}

        Пусть $V$, $W$ --- векторные пространства над $F$.

        \begin{definition}
            Отображение $\phi \colon V \to W$ называется \textit{линейным}, если
            \begin{enumerate}
            \item \label{lec16:def_1_1}$\phi(v_1 + v_2) = \phi(v_1) + \phi(v_2)$,
            \item \label{lec16:def_1_2} $\phi(\lambda v) = \lambda \phi(v)$.
            \end{enumerate}

            $\forall v_1, v_2, v \in V$, $\forall \lambda \in F$.
        \end{definition}

        \paragraph{Простейшие свойства}~

        \begin{enumerate}
        \item $\phi(\overrightarrow{0_V}) = \overrightarrow{0_W}$.
        \item $\phi(-v) = -\phi(v)$.
        \end{enumerate}


    \defitem{Изоморфизм векторных пространств, изоморфные векторные пространства}

        \begin{definition}
            Отображение $\phi \colon V \to W$ называется \textit{изоморфизмом} если оно линейно и биективно.

            Обозначение: $\phi \colon V \MapsTo W$.
        \end{definition}

        \begin{definition}
            Два векторных пространства $V$, $W$ называются \textit{изоморфными}, если существует изоморфизм \\ ${\phi\colon V \MapsTo W}$.

            Обозначается: $V \simeq W$ (либо $V \cong W$).
        \end{definition}


    \defitem{Какими свойствами обладает отношение изоморфности на множестве всех векторных пространств?}

        \begin{theorem}
            Отношение изоморфности является отношением эквивалентности на множестве всех векторных пространств над фиксированным полем $F$.
        \end{theorem}


    \defitem{Критерий изоморфности двух конечномерных векторных пространств}

        \begin{theorem}
            Пусть $V$, $W$ --- два конечномерных векторных пространства над $F$.

            Тогда, $V \simeq W \iff \dim V = \dim W$.
        \end{theorem}


    \defitem{Матрица линейного отображения}

        Пусть $V, W$ --- векторные пространства над $F$.

        $\E = (e_1, \dots, e_n)$ --- базис $V$,

        $\F = (f_1, \dots, f_m)$ --- базис $W$.

        \bigskip
        Пусть $\phi\colon V \to W$ --- линейное отображение.

        $\forall j = 1, \dots, n$

        $\phi(e_j) = a_{1j} f_1 + a_{2j} f_2 + \dots + a_{mj} f_m = (f_1, \dots, f_m) \begin{pmatrix} a_{1j} \\ a_{2j} \\ \dots \\ a_{mj} \end{pmatrix}$.

        Тогда, $(\phi(e_1), \dots, \phi(e_n)) = (f_1, \dots, f_m) \cdot A$, где $A = (a_{ij}) \in \text{Mat}_{m \times n} (F)$.

        \begin{definition}
            $A$ называется матрицей линейного отображения $\phi$ в базисах $\E$ и $\F$.

            Обозначение: $A = A(\phi, \E, \F)$.
        \end{definition}

        В $j$-м столбце матрицы $A$ стоят координаты вектора $\phi(e_j)$ в базисе $\F$.


    \defitem{Связь между координатами вектора и его образа при линейном отображении}

        \begin{proposal}
            Пусть $\phi \colon V \to W$ --- линейное отображение,

            $\E = (e_1, \dots, e_n)$ --- базис $V$,

            $\F = (f_1, \dots, f_m)$ --- базис $W$,

            $A = A(\phi, \E, \F)$.

            \begin{math}
                \begin{aligned}
                    v \in V \implies &v = x_1 e_1 + \dots + x_n e_n, \\
                    &\phi(v) = y_1 f_1 + \dots + y_m f_m.
                \end{aligned}
            \end{math}

            Тогда,
            \begin{equation*}
                \begin{pmatrix} y_1 \\ \vdots \\ y_m \end{pmatrix} = A \begin{pmatrix} x_1 \\ \vdots \\ x_n \end{pmatrix}
            .\end{equation*}
        \end{proposal}


    \defitem{Формула изменения матрицы линейного отображения при замене базисов}

        Пусть $\E'$ --- другой базис в $V$, $\F'$ --- другой базис в $W$.

        $\E' = \E \cdot C_{\in M_n}$,

        $\F' = \F \cdot D_{\in M_m}$.

        $A = A(\phi, \E, \F)$,

        $A' = A(\phi, \E', \F')$.

        \begin{proposal}
            $A' = D^{-1} A C$.
        \end{proposal}


    \defitem{Сумма двух линейных отображений и её матрица. Произведение линейного отображения на скаляр и его матрица}

        Пусть $\phi, \psi \in \hom(V, W)$, $\lambda \in F$.

        \begin{definition}~
            \begin{enumerate}
            \item \textit{Суммой} линейных отображений $\phi$ и $\psi$ называется линейное отображение $\phi + \psi \in \hom(V, W)$, \\ такое что ${(\phi + \psi)(v) := \phi(v) + \psi(v)}$.
            \item Произведение $\phi$ на $\lambda$ --- это линейное отображение $\lambda \phi \in \hom(V, W)$, такое что $(\lambda\phi)(v) := \lambda \phi(v)$.
            \end{enumerate}
        \end{definition}


        Зафиксируем базисы $\E = (e_1, \dots, e_n)$ в $V$ и $\F = (f_1, \dots, f_m)$ в $W$.

        \begin{proposal}~
            \begin{enumerate}
            \item
                \begin{math}
                    \begin{aligned}[t]
                        \phi, \psi \in \hom(V, W), \
                        &A_\phi = A(\phi, \E, \F)& \\
                        &A_\psi = A(\psi, \E, \F)& \\
                        &A_{\phi + \psi} = A(\phi + \psi, \E, \F)& \implies A_{\phi + \psi} = A_\phi + A_\psi
                    \end{aligned}
                \end{math}

            \item
                \begin{math}
                    \begin{aligned}[t]
                        \lambda \in F, \phi \in \hom(V, W), \
                        &A_\phi = A(\phi, \E, \F)& \\
                        &A_{\lambda \phi} = A(\lambda \phi, \E, \F)& \implies A_{\lambda \phi} = \lambda A_\phi
                    \end{aligned}
                \end{math}
            \end{enumerate}
        \end{proposal}


    \defitem{Композиция двух линейных отображений и её матрица}

        Пусть $U \xrightarrow{\psi} V \xrightarrow{\phi} W$ --- цепочка линейных отображений, а $\phi \circ \psi : U \to W$ --- их композиция,

        $\E = (e_1, \dots, e_n)$ --- базис $V$,

        $\F = (f_1, \dots, f_m)$ --- базис $W$,

        $\G = (g_1, \dots, g_k)$ --- базис $U$.

        $A_{\phi} = A(\phi, \E, \F)$,

        $A_\psi = A(\psi, \G, \E)$,

        $A_{\phi \circ \psi} = A(\phi \circ \psi, \G, \F)$.

        Тогда, $A_{\phi \circ \psi} = A_\phi \cdot A_\psi$.


    \defitem{Ядро и образ линейного отображения. Являются ли они подпространствами в соответствующих векторных пространствах?}

        Пусть $\phi \colon V \to W$.

        \begin{definition}
            \textit{Ядро} линейного отображения $\phi$ --- это $\ker \phi := \{v \in V \mid \phi(v) = 0\} \subseteq V$.

            \textit{Образ} линейного отображения $\phi$ --- это $\Im \phi := \phi(V) \subseteq W$.
        \end{definition}

        \begin{proposal}~
            \begin{enumerate}[nosep]
            \item Ядро --- подпространство в $V$.
            \item Образ --- подпространство в $W$.
            \end{enumerate}
        \end{proposal}


    \defitem{Критерий инъективности линейного отображения в терминах его ядра}

        Пусть $V, W$ --- векторные пространства над $F$, 

        $\phi \colon V \to W$ --- линейное отображение.

        \begin{proposal}~
            \begin{enumerate}[label=(\alph*)]
            \item $\phi$ инъективно $\iff \ker \phi = \{0\}$,
            \item $\phi$ сюръективно $\iff \Im \phi = W$.
            \end{enumerate}
        \end{proposal}


    \defitem{Связь между рангом матрицы линейного отображения и размерностью его образа}

        Пусть 
        \begin{math}
            \begin{aligned}[t]
                \E &= (e_1, \dots, e_n) \text{ --- базис $V$}, \\
                \F &= (f_1, \dots, f_m) \text{ --- базис $W$}, \\
                A &= A(\phi, \E, \F).
            \end{aligned}
        \end{math}

        \begin{theorem}
            $\rk A = \dim \Im \phi$.
        \end{theorem}

        \begin{comment}
            Число $\dim \Im \phi$ называется \textit{рангом} линейного отображения $\phi$, обозначается $\rk \phi$.
        \end{comment}

        \begin{corollary}
            $\rk A$ не зависит от выбора пары базисов $\E$ и $\F$.
        \end{corollary}


    \defitem{Каким свойством обладает набор векторов, дополняющих базис ядра линейного отображения до базиса всего пространства?}

        \begin{proposal}
            Пусть $e_1, \dots, e_k$ --- базис $\ker \phi$ и векторы $e_{k + 1}, \dots, e_n$ дополняют его до базиса всего $V$.

            Тогда, $\phi(e_{k + 1}), \dots, \phi(e_n)$ образуют базис в $\Im \phi$.
        \end{proposal}


    \defitem{Теорема о связи размерностей ядра и образа линейного отображения}

        \begin{theorem}
            $\dim \Im \phi + \dim \ker \phi = \dim V$.
        \end{theorem}


    \defitem{К какому простейшему виду можно привести матрицу линейного отображения путём замены базисов?}

        \begin{proposal}
            Пусть $\rk \phi = r$. Тогда существует базис $\E$ в $V$ и базис $\F$ в $W$, такие что
            \begin{equation*}
                A(\phi, \E, \F) = \left(
                    \begin{array}{c|c}
                        E & 0 \\
                        \hline
                        0 & 0
                    \end{array}
                \right) = \bordermatrix{    
                    &   & r &   &   &   & n - r &   \cr
                    & 1 & 0 & 0 & \dots & 0 & 0 & 0 \cr
                  \hspace{0.7cm} r & 0 & \ddots & 0 & \dots & 0 & 0 & 0 \cr
                    & 0 & 0 & 1 & \dots & 0 & 0 & 0 \cr
                    & \vdots & \vdots & \vdots & \ddots & \vdots & \vdots & \vdots \cr
                    & 0 & 0 & 0 & \dots & 0 & 0 & 0 \cr
              m - r & 0 & 0 & 0 & \dots & 0 & 0 & 0 \cr
                    & 0 & 0 & 0 & \dots & 0 & 0 & 0
                }
            .\end{equation*}
        \end{proposal}


    \defitem{Линейная функция на векторном пространстве}

        \begin{definition}
            \textit{Линейной функцией} (или \textit{линейной формой}, или \textit{линейным функционалом}) на $V$ называется всякое линейное отображение $\alpha \colon V \to F$.
        \end{definition}

        \begin{designation}
            $V^{*} := \hom(V, F)$ --- множество всех линейных функций на $V$.
        \end{designation}


    \defitem{Сопряжённое (двойственное) векторное пространство и его размерность}

        Из общей теории линейных отображений:
        \begin{enumerate}
        \item $V^{*}$ --- векторное пространство (оно называется \textit{сопряженным} или \textit{двойственным}).
        \item Если $\E = (e_1, \dots, e_n)$ --- фиксированный базис в $V$, то есть изоморфизм $V^{*} \simeq \text{Mat}_{1 \times n}(F)$ (а это ни что иное, как строки длины $n$).

            $\alpha \to (\alpha_1, \dots, \alpha_n)$

            $v = x_1 e_1 + \dots + x_n e_n$

            $\alpha(v) = (\alpha_1, \dots, \alpha_n) \begin{pmatrix} x_1 \\ \dots \\ x_n \end{pmatrix} = \alpha_1 x_1 + \dots + \alpha_n x_n$.

            $\alpha_i = \alpha(e_i)$ --- коэффициенты линейной функции $\alpha$ в базисе $\E$.
        \end{enumerate}

        \begin{corollary}
            $\dim V^{*} = \dim V$ ($\implies V^{*} \simeq V$).
        \end{corollary}


    \defitem{Базис сопряжённого пространства, двойственный к данному базису исходного векторного пространства}

        При $i = 1, \dots, n$ рассмотрим линейную функцию $\epsilon_i \in V^{*}$, соответствующую строке $(0 \dots 1 \dots 0)$. Тогда $\epsilon_1, \dots, \epsilon_n$ --- базис $V^{*}$, он однозначно определяется условием $\epsilon_i(e_j) = \delta_{ij} = \begin{cases}
            1, &i = j, \\
            0, &i \neq j.
        \end{cases}$. ($\delta_{ij}$ --- символ Кронекера)


        \begin{definition}
            Базис $(\epsilon_1, \dots, \epsilon_n)$ пространства $V^{*}$, определенный условием выше, называется базисом, \textit{двойственным} (сопряженным) к базису $\E$.

            Удобная запись условия:
            \begin{equation*}
                \begin{pmatrix} \epsilon_1 \\ \dots \\ \epsilon_n \end{pmatrix} (e_1, \dots, e_n) = E
            .\end{equation*}
        \end{definition}


    \defitem{Билинейная форма на векторном пространстве}

        Пусть $V$ --- векторное пространство над $F$.

        \begin{definition}
            \textit{Билинейная форма} на $V$ --- это отображение $\beta \colon V \times V \to F$, линейное по каждому аргументу.
        \end{definition}

        \paragraph{Линейность по 1-му аргументу}
        \begin{itemize}[nosep]
        \item $\beta(x_1 + x_2, y) = \beta(x_1, y) + \beta(x_2, y) \quad \forall x_1, x_2, y \in V$,
        \item $\beta(\lambda x, y) = \lambda\beta(x, y) \quad \forall x, y \in V, \ \lambda \in F$.
        \end{itemize}

        \paragraph{Линейность по 2-му аргументу}
        \begin{itemize}[nosep]
        \item $\beta(x, y_1 + y_2) = \beta(x, y_1) + \beta(x, y_2) \quad \forall x, y_1, y_2 \in V$,
        \item $\beta(x, \lambda y) = \lambda\beta(x, y) \quad \forall x, y \in V, \ \lambda \in F$.
        \end{itemize}


    \defitem{Матрица билинейной формы}

        Считаем, что $\dim V = n < \infty$.

        Пусть $\E = (e_1, \dots, e_n)$ --- базис $V$.

        \begin{definition}
            Матрицей билинейной формы $\beta$ в базисе $\E$ называется такая матрица $B \in M_n$, что $b_{ij} = \beta(e_i, e_j)$.

            Обозначение: $B(\beta, \E)$.
        \end{definition}


    \defitem{Формула для вычисления значений билинейной формы в координатах}

        Пусть
        \begin{math}
            \begin{aligned}[t]
                x &= x_1 e_1 + \dots + x_n e_n, \\
                y &= y_1 e_1 + \dots + y_n e_n.
            \end{aligned}
        \end{math}

        Тогда,
        \begin{align*}
            \beta(x, y)
            &= (x_1, \dots, x_n) B \begin{pmatrix} y_1 \\ \dots \\ y_n \end{pmatrix}
        .\end{align*}


    \defitem{Формула изменения матрицы билинейной формы при замене базисов}

        $B = B(\beta, \E)$.

        Пусть $\E' = (e'_1, \dots e'_n)$ --- другой базис $V$.

        $\E' = \E \cdot C$.

        $B' := B(\beta, \E')$.

        \begin{proposal}
            $B' = C^{T} B C$.
        \end{proposal}


    \defitem{Симметричная билинейная форма. Критерий симметричности билинейной формы в терминах её матрицы}

        \begin{definition}
            Билинейная форма $\beta$ называется \textit{симметричной}, если $\beta(x, y) = \beta(y, x) \ \forall x, y \in V$.
        \end{definition}

        Пусть $\E$ --- произвольный базис $V$.

        \begin{proposal}
            $\beta$ симметрична $\iff B = B^{T}$.
        \end{proposal}


    \defitem{Квадратичная форма}

        Пусть $\beta \colon V \times V \to F$ --- билинейная форма на $V$.

        \begin{definition}
            Отображение $Q_\beta \colon V \to F$, $Q_\beta(x) = \beta(x, x)$, называется \textit{квадратичной формой}, ассоциированной с билинейной формой $\beta$.
        \end{definition}

        Пусть $\E$ --- базис $V$, $x = x_1 e_1 + \dots x_n e_n$, $B = B(\beta, \E)$.

        Тогда,
        \begin{equation*}
            Q_\beta(x) = (x_1 \dots x_n) B \begin{pmatrix} x_1 \\ \dots \\ x_n \end{pmatrix} = \sum_{i = 1}^{n} \sum_{j = 1}^{n} b_{ij} x_i x_j = \sum_{i = 1}^{n} b_{ii} x_i^2 + \sum_{1 \leq i < j \leq n}^{n} (b_{ij} + b_{ji}) x_i x_j
        .\end{equation*}


    \defitem{Соответствие между симметричными билинейными формами и квадратичными формами}

        \begin{proposal}
            Пусть в поле $F$ выполнено условие $1 + 1 \neq 0$ (то есть $2 \neq 0$). Тогда отображение $\beta \mapsto Q_\beta$ является биекцией между симметричными билинейными формами на $V$ и квадратичным формами на $V$.
        \end{proposal}


    \defitem{Симметризация билинейной формы}

        Билинейная форма $\sigma(x, y) = \frac{1}{2} \left(\beta(x, y) + \beta(y, x)\right)$ называется \textit{симметризацией} билинейной формы $\beta$.

        Если $B$ и $S$ --- матрицы билинейных форм $\beta$ и $\sigma$ в некотором базисе, то $S = \frac{1}{2} (B + B^{T})$.


    \defitem{Поляризация квадратичной формы}

        Симметричная билинейная форма $\beta(x, y) = \frac{1}{2} \left[Q(x + y) - Q(x) - Q(y)\right]$ называется \textit{поляризацией} квадратичной формы $Q$.


    \defitem{Матрица квадратичной формы}

        \begin{definition}
            Матрицей квадратичной формы $Q$ в базисе $\E$ называется матрица соответствующей симметричной билинейной формы (поляризации) в базисе $\E$.

            Обозначение: $B(Q, \E)$.
        \end{definition}

        \begin{example}
            Пусть $Q(x_1, x_2) = x_1^2 + x_1 x_2 + x_2^2$.

            Если $\E$ --- стандартный базис, то $B(Q, \E) = \begin{pmatrix} 1 & \frac{1}{2} \\ \frac{1}{2} & 1 \end{pmatrix}$.
        \end{example}


    \defitem{Канонический вид квадратичной формы}

        \begin{definition}
            Квадратичная форма $Q$ имеет в базисе $\E$ \textit{канонический вид}, если $B(Q, \E)$ диагональна.

            Если $B(Q, \E) = \diag(b_1, b_2, \dots, b_n)$, то $Q(x_1, \dots, x_n) = b_1 x_1^2 + b_2 x_2^2 + \dots + b_n x_n^2$.
        \end{definition}


    \defitem{Нормальный вид квадратичной формы над $\RR$}

        \begin{definition}
            Квадратичная форма над $\RR$ имеет \textit{нормальный вид} в базисе $\E$, если в этом базисе
            \begin{equation*}
                Q(x_1, \dots, x_n) = \epsilon_1 x_1^2 + \dots + \epsilon_n x_n^2
            ,\end{equation*}
            где $\epsilon_i \in \{-1, 0, 1\}$.
        \end{definition}


    \defitem{Индексы инерции квадратичной формы над $\RR$}

        Пусть $F = \RR$.

        Пусть $Q \colon V \to \RR$ --- квадратичная форма.

        Можно привести к нормальному виду
        \begin{equation*}
            Q(x_1, \dots, x_n) = x_1^2 + \dots + x_s^2 - x_{s + 1}^2 - \dots - x_{s + t}^2
        .\end{equation*}

        Здесь
        \begin{math}
            \begin{aligned}[t]
                &i_+ := s \text{ --- положительный индекс инерции квадратичной формы $Q$}, \\
                &i_- := t \text{ --- отрицательный индекс инерции квадратичной формы $Q$}.
            \end{aligned}
        \end{math}


    \defitem{Закон инерции для квадратичной формы над $\RR$}

        \begin{theorem}
            Числа $i_+$ и $i_-$ не зависят от базиса в котором $Q$ принимает нормальный вид.
        \end{theorem}


    \defitem{Положительно/неотрицательно определённая квадратичная форма над $\RR$}

    \defitem{Отрицательно/неположительно определённая квадратичная форма над $\RR$}

    \defitem{Неопределённая квадратичная форма над $\RR$}

        \begin{definition}
            Квадратичная форма $Q$ над $\RR$ называется
        \end{definition}
        \begin{table}[H]
        {\renewcommand{\arraystretch}{1.7}
            \begin{tabular}{c|c|c|c|c}
                Термин & Обозначение & Условие & Нормальный вид & Индексы инерции \\ \hline
                Положительно определённой & $Q > 0$ & $Q(x) > 0 \ \forall x \neq 0$ & $x_1^2 + \dots + x_n^2$ & $i_+ = n, i_- = 0$ \\ \hline
                Отрицательно определённой & $Q < 0$ & $Q(x) < 0 \ \forall x \neq 0$ & $-x_1^2 - \dots - x_n^2$ & $i_+ = 0, i_- = n$ \\ \hline
                Неотрицательно определённой & $Q \geq 0$ & $Q(x) \geq 0 \ \forall x$ & $x_1^2 + \dots + x_k^2, \ k \leq n$ & $i_+ = k, i_- = 0$ \\ \hline
                Неположительно определённой & $Q \leq 0$ & $Q(x) \leq 0 \ \forall x$ & $-x_1^2 - \dots - x_k^2, \ k \leq n$ & $i_+ = 0, i_- = k$ \\ \hline
                Неопределённой & --- & 
                \begin{math}
                    \begin{aligned}
                        \exists x : Q(x) > 0 \\
                        \exists y : Q(y) < 0
                    \end{aligned}
                \end{math} &
                \begin{math}
                    \begin{gathered}
                        x_1^2 + \dots + x_s ^2 - x_{s + 1}^2 - x_{s + t}^2 \\
                        s, t \geq 1
                    \end{gathered}
                \end{math} & $i_+ = s, i_- = t$
                \\ \hline
            \end{tabular}       
        }
        \end{table}


    \defitem{Способ нахождения индексов инерции квадратичной формы над $\RR$, вытекающий из метода Якоби}

        Пусть $Q \colon V \to \RR$ --- квадратичная форма,

        $\E = (e_1, \dots, e_n)$ --- базис,

        $B = B(Q, \E)$, 

        $\delta_k$ --- $k$-й угловой минор матрицы $B$.


        \begin{corollary}[из метода Якоби]
            Пусть $\delta_k \neq 0 \ \forall k$. Тогда:

            Число $i_+$ равно количеству \underline{сохранений знака} в последовательности $1, \delta_1, \dots, \delta_n$.

            Число $i_-$ равно количеству \underline{перемен знака} в последовательности $1, \delta_1, \dots, \delta_n$.
        \end{corollary}


    \defitem{Критерий Сильвестра положительной определённости квадратичной формы над $\RR$}

        Пусть 
        \begin{math}
            \begin{aligned}[t]
                &V \text{ --- векторное пространство над $\RR$, $\dim V = n$}, \\
                &\E = (e_1, \dots, e_n) \text{ --- базис $V$}, \\
                &B = B(Q, \E), \\
                &B_k \text{ --- левый верхний $k \times k$ блок}, \\
                &\delta_k = \det B_k.
            \end{aligned}
        \end{math}

        \begin{theorem}[Критерий Сильвестра положительной определенности]
            \begin{equation*}
                Q > 0 \iff \delta_k > 0 \ \forall k = 1 \dots n
            .\end{equation*}
        \end{theorem}


    \defitem{Критерий отрицательной определённости квадратичной формы над $\RR$}

        \begin{corollary}
            \begin{equation*}
                Q < 0 \iff \delta_k \begin{cases}
                    > 0 & \text{при } k \divby 2, \\
                    < 0 & \text{при } k \!\!\not\;\divby 2.
                \end{cases}
            \end{equation*}
        \end{corollary}


    \defitem{Евклидово пространство}

        \begin{definition}
            \textit{Евклидово пространство} --- это векторное пространство $\EE$ над $\RR$, на котором задано \textit{скалярное произведение}, то есть такое отображение $(\bigcdot, \bigcdot)\colon \EE \times \EE \to \RR$, что
            \begin{enumerate}[nosep]
                \item $(\bigcdot, \bigcdot)$ --- симметричная билинейная форма,
                \item Квадратичная форма $(x, x)$ положительно определённая.
            \end{enumerate}
        \end{definition}


    \defitem{Длина вектора в евклидовом пространстве}

        \begin{definition}
            \textit{Длина} вектора $x \in \EE$ --- это $|x| := \sqrt{(x, x)}$.

            Свойство: 
            $|x| \geq 0$, причем $|x| = 0 \iff x = 0$.
        \end{definition}

        \begin{example}
            Если $\EE = \RR^n$ со стандартным скалярным произведением, то $|x| = \sqrt{x_1^2 + \dots + x_n^2}$.
        \end{example}


    \defitem{Неравенство Коши--Буняковского}

        \begin{proposal}[неравенство Коши-Буняковского]
            $\forall x, y \in \EE$ верно $|(x, y)| \leq |x| \cdot |y|$, причём равенство $\iff$ ${x, y}$ пропорциональны.
        \end{proposal}

        \begin{example}
            Пусть $\EE = \RR^n$ со стандартным скалярным произведением, тогда
            \begin{equation*}
                |x_1 y_1 + \dots + x_n y_n| \leq \sqrt{x_1^2 + \dots + x_n^2} \cdot \sqrt{y_1^2 + \dots + y_n^2}
            .\end{equation*}
        \end{example}


    \defitem{Угол между ненулевыми векторами евклидова пространства}

        Пусть $x, y \in \EE \setminus \{0\}$, тогда $-1 \leq \dfrac{(x, y)}{|x| \cdot |y|} \leq 1$.

        \begin{definition}
            Угол между ненулевыми векторами $x, y \in \EE$, это такой $\alpha \in [0, \pi]$, что $\cos \alpha = \dfrac{(x, y)}{|x| \cdot |y|}$.

            Тогда $(x, y) = |x| |y| \cos \alpha$.
        \end{definition}


    \defitem{Матрица Грама системы векторов евклидова пространства}

        Пусть $v_1, \dots, v_k$ --- произвольная система векторов.

        \begin{definition}
            \textit{Матрица Грама} этой системы --- это
            \begin{equation*}
                G(v_1, \dots, v_k) = \begin{pmatrix}
                    (v_1, v_1) & (v_1, v_2) & \dots & (v_1, v_k) \\
                    (v_2, v_1) & (v_2, v_2) & \dots & (v_2, v_k) \\
                    \vdots & \vdots & \ddots & \vdots \\
                    (v_k, v_1) & (v_k, v_2) & \dots & (v_k, v_k)
                \end{pmatrix}
            .\end{equation*}
        \end{definition}

        \begin{example}
            $\EE = \RR^n$ со стандартным скалярным произведением.

            $a_1, \dots, a_k \in \RR^n \leadsto A := (a_1, \dots, a_k) \in \text{Mat}_{n \times k}(\RR)$.

            Тогда, $G(a_1, \dots, a_k) = A^T \cdot A$.
        \end{example}


    \defitem{Свойства определителя матрицы Грама}

        \begin{proposal}
            $\forall v_1, \dots, v_k \in \EE \implies \det G(v_1, \dots, v_k) \geq 0$.

            Более того, $\det G(v_1, \dots, v_k) > 0 \iff v_1, \dots, v_k$ линейно независимы. 
        \end{proposal}


    \defitem{Ортогональное дополнение подмножества евклидова пространства}

        \begin{definition}
            \textit{Ортогональное дополнение} множества $S \subseteq \EE$ --- это множество $S^{\perp} := \{x \in \EE \mid (x, y) = 0 \ \forall y \in S\}$.
        \end{definition}


    \defitem{Чему равна размерность ортогонального дополнения к подпространству евклидова пространства?}

        Пусть $\dim \EE = n$, $S \subseteq \EE$ --- подпространство.

        Тогда, $\dim S^{\perp} = n - \dim S$.


    \defitem{Каким свойством обладают подпространство евклидова пространства и его ортогональное дополнение?}

        Считаем, что $\dim \EE = n < \infty$.

        \begin{proposal}
            Пусть $S \subseteq \EE$ --- подпространство.
            Тогда:
            \begin{enumerate}
            \item $\dim S^{\perp} = n - \dim S$.
            \item $\EE = S \oplus S^{\perp}$.
            \item $(S^{\perp})^{\perp} = S$.
            \end{enumerate}
        \end{proposal}


    \defitem{Ортогональная проекция вектора на подпространство}

        $\Downarrow$


    \defitem{Ортогональная составляющая вектора относительно подпространства}

        $S$ --- подпространство $ \implies \EE = S \oplus S^{\perp}$

        $\forall v \in \EE \exists! x \in S, y \in S^{\perp}$, такие что $x + y = v$.

        \begin{definition}~
            \begin{enumerate}
            \item 
                $x$ называется \textit{ортогональной проекцией} вектора $v$ на подпространство $S$.

                Обозначение: $x = \pr_S v$.

            \item
                $y$ называется \textit{ортогональной составляющей} вектора $v$ относительно подпространства $S$.

                Обозначение: $y = \ort_S v$.
            \end{enumerate}
        \end{definition}


    \defitem{Формула для ортогональной проекции вектора на подпространство в $\RR^n$, заданное своим базисом}
            
        Пусть $\EE = \RR^n$ со стандартным скалярным произведением.

        $S \subseteq \EE$ --- подпространство, $a_1, \dots, a_k$ --- базис $S$.

        Пусть $A := (a_1, \dots, a_k) \in \text{Mat}_{n \times k}(\RR)$, $A^{(i)} = a_i$.

        \begin{proposal}
            $\forall v \in \RR^n \quad \pr_S v = A (A^{T} A)^{-1} A^{T} v$.
        \end{proposal}

    \defitem{Ортогональная система векторов евклидова пространства. Ортогональный базис}

        $\Downarrow$


    \defitem{Ортонормированная система векторов евклидова пространства. Ортонормированный базис}

        \begin{definition}
            Система ненулевых векторов $v_1, \dots, v_k$ называется
            \begin{enumerate}[nosep]
                \item \textit{ортогональной}, если $(v_i, v_j) = 0 \ \forall i \neq j$ (то есть $G(v_1, \dots, v_k)$ диагональна),
                \item \textit{ортонормированной}, если $(v_i, v_j) = 0 \ \forall i \neq j$ и $(v_i, v_i) = 1$ ($\iff |v_i| = 1$).
                    То есть $G(v_1, \dots, v_k) = E$.
            \end{enumerate}
        \end{definition}

        \begin{comment}
            Всякая ортогональная (и в частности ортонормированная) система векторов автоматически линейно независима.
            \begin{equation*}
                \det G(v_1, \dots, v_k) = |v_1|^2 \cdot |v_2|^2 \dotsm |v_k|^2 \neq 0
            .\end{equation*}
        \end{comment}

        \begin{definition}
            Базис пространства называется \textit{ортогональным} (соответственно \textit{ортонормированным}), если он является ортогональной (ортонормированной) системой векторов.
        \end{definition}


    \defitem{Описание всех ортонормированных базисов евклидова пространства в терминах одного такого базиса и матриц перехода}

        Пусть $\E = (e_1, \dots, e_n)$ --- ортонормированный базис в $E$.

        Пусть $\E' = (e'_1, \dots, e'_n)$ --- какой-то другой базис.

        $(e'_1, \dots, e'_n) = (e_1, \dots, e_n) \cdot C$, $C \in M_n^{0}(\RR)$.

        \begin{proposal}
            $\E'$ --- ортонормированный базис $\iff C^{T} \cdot C = E$.
        \end{proposal}


    \defitem{Ортогональная матрица}

        \begin{definition}
            Матрица $C \in M_n(\RR)$ называется \textit{ортогональной} если $C^{T} C = E$.
        \end{definition}

        \begin{comment}
            $C^{T} C = E \iff C C^{T} = E \iff C^{-1} = C^{T}$.
        \end{comment}

        \begin{properties}~
            \begin{enumerate}
            \item $C^{T} C = E \implies $ система столбцов $C^{(1)}, \dots, C^{(n)}$ --- это ортонормированный базис в $\RR^n$,
            \item $C C^{T} = E \implies $ система строк $C_{(1)}, \dots, C_{(n)}$ --- это тоже ортонормированный базис в $\RR^n$,
            \end{enumerate}
            В частности, $|c_{ij}| \leq 1$.
            \begin{enumerate}[resume]
            \item $\det C = \pm 1$.
            \end{enumerate}
        \end{properties}

        \begin{example}
            $n = 2$.
            Ортогональный матрицы:
            \begin{equation}
                \begin{gathered}
                    \begin{pmatrix} 
                        \cos \phi & -\sin \phi \\
                        \sin \phi & \cos \phi
                    \end{pmatrix} \\
                    \det = 1
                \end{gathered}
                \hspace{1cm}
                \begin{gathered}
                    \begin{pmatrix} 
                        \cos \phi & \sin \phi \\
                        \sin \phi & -\cos \phi
                    \end{pmatrix} \\
                    \det = -1
                \end{gathered}
            .\end{equation}
        \end{example}


    \defitem{Формула для координат вектора в ортогональном и ортонормированном базисах евклидова пространства}

        Пусть $\EE$ --- евклидово пространство, $(e_1, \dots, e_n)$ --- ортогональный базис.

        $v \in \EE$.

        \begin{proposal}
            $v = \dfrac{(v, e_1)}{(e_1, e_1)}e_1 + \dfrac{(v, e_2)}{(e_2, e_2)}e_2 + \dots + \dfrac{(v, e_n)}{(e_n, e_n)}e_n$.

            В частности, если $e_1, \dots, e_n$ ортонормирован, то $v = (v, e_1)e_1 + \dots + (v, e_n) e_n$.
        \end{proposal}


    \defitem{Формула для ортогональной проекции вектора на подпространство в терминах его ортогонального базиса}
    
        Пусть $S \subseteq \EE$ --- подпространство.

        $e_1, \dots, e_k$ --- ортогональный базис в $S$.

        \begin{proposal}
            $\forall v \in \EE \quad \pr_S v = \sum_{i = 1}^{k} \dfrac{(v, e_i)}{(e_i, e_i)} e_i$.

            В частности, если $e_1, \dots, e_k$ ортонормирован, то $\pr_S v = \sum_{i = 1}^{k} (v, e_i) e_i$.
        \end{proposal}


    \defitem{Метод ортогонализации Грама--Шмидта}

        Как построить ортогональный (ортонормированный) базис в $\EE$?

        Если $f_1, \dots, f_n$ --- ортогональный базис, то $\left(\frac{f_1}{|f_1|}, \dots, \frac{f_n}{|f_n|}\right)$ --- ортонормированный базис.

        Тогда, достаточно построить ортогональный базис.

        Пусть $e_1, \dots, e_k$ --- линейно независимая система векторов.

        $i$-й угловой минор в $G(e_1, \dots, e_k)$ --- это $\det G(e_1, \dots, e_k) > 0$.

        \bigskip
        Следовательно, применим метод Якоби:

        $\exists!$ система векторов $f_1, \dots, f_k$, такая что 
        \begin{align*}
        &f_1 = e_1, \\
        &f_2 \in e_2 + \left< e_1 \right>, \\
        &f_3 \in e_3 + \left< e_1, e_2 \right>, \\
        &\dots, \\
        &f_k \in e_k + \left< e_1, \dots, e_{k - 1} \right>
        \end{align*}


        И выполнены следующие свойства $\forall i = 1, \dots, k$:

        \begin{enumerate}[start=0,nosep]
        \item $f_1, \dots, f_k$ ортогональны.
        \item $\left< e_1, \dots, e_i \right> = \left< f_1, \dots, f_i \right>$.
        \item $f_i = e_i - \sum_{j = 1}^{i - 1} \dfrac{(e_i, f_j)}{(f_j, f_j)} f_j$ \customlabel{lec23:star}{($\star$)}.
        \item $\det G(e_1, \dots, e_i) = \det G(f_1, \dots, f_i)$ \customlabel{lec23:heart}{$(\heartsuit)$}.
        \end{enumerate}

        \bigskip
        Построение базиса $f_1, \dots, f_k$ по формулам \ref{lec23:star} называется методом (процессом) ортогонализации Грамма-Шмидта.


    \defitem{Теорема Пифагора в евклидовом пространстве}

        \begin{theorem}
            Пусть $x, y \in \EE$, $(x, y) = 0$. Тогда $|x + y|^2 = |x|^2 + |y|^2$.
        \end{theorem}


    \defitem{Расстояние между векторами евклидова пространства}

        \begin{definition}
            \textit{Расстояние} между векторами $x, y \in \EE$ --- это $\rho(x, y) = |x - y|$.
        \end{definition}


    \defitem{Неравенство треугольника в евклидовом пространстве}

        \begin{proposal}
            $\forall a, b, c \in \EE \implies \rho(a, b) + \rho(b, c) \geq \rho(a, c)$.
        \end{proposal}


    \defitem{Теорема о расстоянии между вектором и подпространством в терминах ортогональной составляющей}

        \begin{theorem}
            Пусть $x \in \EE$, $S \subseteq \EE$ --- подпространство. Тогда, $\rho(x, S) = \left|\ort_S x\right|$, причем $\pr_S x$ --- это ближайший к $x$ вектор из $S$.
        \end{theorem}


    \defitem{Псевдорешение несовместной системы линейных уравнений}

        СЛУ $Ax = b$, $A \in \text{Mat}_{m \times n}(\RR)$, $x \in \RR^n$, $b \in \RR^m$.
        \begin{equation*}
            x_0 \text{ --- решение системы} \iff Ax_0 = b \iff Ax_0 - b = 0 \iff |Ax_0 - b| = 0 \iff \rho(Ax_0, b) = 0
        .\end{equation*}

        Если СЛУ несовместна, то $x_0$ называется \textit{псевдорешением}, если $\rho(Ax_0, b)$ минимально.

        \begin{equation*}
            \rho(Ax_0, b) = \min_{x \in R^n} \rho(Ax, b)
        .\end{equation*}

        $x_0$ --- решение задачи оптимизации $\rho(Ax, b) \xrightarrow[x \in \RR^n]{} \min$.

        Если столбцы $A^{(1)}, \dots, A^{(n)}$ линейно независимы, то псевдорешение единственно и может быть найдено по формуле $x_0 = (A^{T} A)^{-1} A^{T} b$.


    \defitem{Формула для расстояния от вектора до подпространства в терминах матриц Грама}

        Пусть $\EE$ --- евклидово пространство, $\dim \EE = n < \infty$.

        $S \subseteq \EE$ --- подпространство, $e_1, \dots, e_k$ --- базис в $S$.

        \begin{theorem}
            $\forall x \in \EE \quad \rho(x, S)^2 = \dfrac{\det G(e_1, \dots, e_k, x)}{\det G(e_1, \dots, e_k)}$.
        \end{theorem}


    \defitem{$k$-мерный параллелепипед и его объём}

        \begin{definition}
            \textit{$k$-мерный параллелепипед}, натянутый на векторы $a_1, \dots, a_k$, это множество
            \begin{equation*}
                P(a_1, \dots, a_k) := \left\{ \sum_{i = 1}^{k} x_i a_i \bigg| 0 \leq x_i < 1 \right\}
            .\end{equation*}

            Основание: $P(a_1, \dots, a_{k - 1})$.

            Высота: $h := \left| \ort_{\left< a_1, \dots, a_{k - 1} \right>} a_k \right|$.
        \end{definition}

        \begin{definition}
            \textit{$k$-мерный объем} $k$-мерного параллелепипеда $P(a_1, \dots, a_k)$ --- это величина $\vol P(a_1, \dots, a_k)$, определяемая индуктивно:

            \begin{description}
            \item[$k = 1$] $\implies \vol P(a_1) := |a_1|$.
            \item[$k > 1$] $\implies \vol P(a_1, \dots, a_k) := \vol P(a_1, \dots, a_{k - 1}) \cdot h$.
            \end{description}
        \end{definition}


    \defitem{Формула для объёма $k$-мерного параллелепипеда в $n$-мерном евклидовом пространстве}

        \begin{theorem}
            $\vol P(a_1, \dots, a_k)^2 = \det G(a_1, \dots, a_k)$.
        \end{theorem}


    \defitem{Формула для объёма $n$-мерного параллелепипеда в $n$-мерном евклидовом пространстве в терминах координат в ортонормированном базисе}

        Пусть $(e_1, \dots, e_n)$ --- ортонормированный базис в $\EE$,

        $(a_1, \dots, a_n) = (e_1, \dots, e_n) \cdot A$, $A \in M_n(\RR)$.

        \begin{proposal}
            $\vol P(a_1, \dots, a_n) = \left|\det A\right|$.
        \end{proposal}
    

    \defitem{В каком случае два базиса евклидова пространства называются одинаково ориентированными?}

        Пусть $\E = (e_1, \dots, e_n)$ и $\E' = (e'_1, \dots, e'_n)$ --- два базиса в $\EE$.

        $(e'_1, \dots, e'_n) = (e_1, \dots, e_n) \cdot C$, $C \in M_n^0(\RR)$.

        \begin{definition}
            Говорят, что $\E$ и $\E'$ одинаково ориентированы, если $\det C > 0$.
        \end{definition}


    \defitem{Ориентированный объём $n$-мерного параллелепипеда в $n$-мерном евклидовом пространстве}

        Фиксируем ориентацию в $\EE$.

        Фиксируем положительно ориентированный ортонормированный базис $\E = (e_1, \dots, e_n)$ в $\EE$.

        Пусть $a_1, \dots, a_n \in \EE$, $(a_1, \dots, a_n) = (e_1, \dots, e_n) \cdot A$.

        \begin{definition}
            \textit{Ориентированным объемом} параллелепипеда $P(a_1, \dots, a_n)$ называется величина
            \begin{equation*}
                \Vol (a_1, \dots, a_n) = \det A
            .\end{equation*}
        \end{definition}


    \defitem{Свойства ориентированного объёма $n$-мерного параллелепипеда в $n$-мерном евклидовом пространстве}

        \begin{enumerate}[nosep]
        \item $\Vol(a_1, \dots, a_n)$ линеен по каждому аргументу.
        \item $\Vol(a_1, \dots, a_n)$ кососимметрична (то есть меняет знак при перестановке любых двух аргументов).
        \item $\Vol(a_1, \dots, a_n) > 0 \iff (a_1, \dots, a_n)$ --- положительно ориентированный базис в $\EE$.
        \item $\Vol(a_1, \dots, a_n) < 0 \iff (a_1, \dots, a_n)$ --- отрицательно ориентированный базис в $\EE$.
        \item $\Vol(a_1, \dots, a_n) = 0 \iff (a_1, \dots, a_n)$ линейно зависимы.
        \end{enumerate}


    \defitem{Связь векторного произведения со скалярным и ориентированным объёмом}

        Если $v = [a, b]$, то $(v, x) = \Vol(a, b, x) \quad \forall x \in \RR^3$. 


    \defitem{Формула для вычисления векторного произведения в терминах координат в положительно ориентированном ортонормированном базисе}

        Если $\E = (e_1, e_2, e_3)$ --- положительно ориентированный базис и
        \begin{math}
            \ \begin{aligned}[t]
                a &= a_1 e_1 + a_2 e_2 + a_3 e_3 \\
                b &= b_1 e_1 + b_2 e_2 + b_3 e_3
            \end{aligned},
        \end{math}
        то 
        \begin{equation}
            \tag{$\star$}
            \label{lec25:v}
            [a, b] = \begin{vmatrix}
                e_1 & e_2 & e_3 \\
                a_1 & a_2 & a_3 \\
                b_1 & b_2 & b_3
            \end{vmatrix}
            := \begin{vmatrix} 
                a_2 & a_3 \\
                b_2 & b_3
            \end{vmatrix} e_1 - \begin{vmatrix} 
                a_1 & a_3 \\
                b_1 & b_3
            \end{vmatrix} e_2 + \begin{vmatrix} 
                a_1 & a_2 \\
                b_1 & b_2
            \end{vmatrix} e_3
        .\end{equation}


    \defitem{Смешанное произведение векторов трёхмерного евклидова пространства}

        \begin{definition}
            $\forall a, b, c \in \EE$ число $(a, b, c) := ([a, b], c)$ называется \textit{смешанным произведением} векторов $a, b, c$.
        \end{definition}

        \begin{comment}
            $(a, b, c) = \Vol(a, b, c)$.
        \end{comment}


    \defitem{Формула для вычисления смешанного произведения в терминах координат в положительно ориентированном ортонормированном базисе}

        Если $e_1, e_2, e_3$ --- положительно ориентированный ортонормированный базис, то
        \begin{equation*}
            \left.\begin{aligned}
                a &= a_1 e_1 + a_2 e_2 + a_3 e_3 \\
                b &= b_1 e_1 + b_2 e_2 + b_3 e_3 \\
                c &= c_1 e_1 + c_2 e_2 + c_3 e_3
            \end{aligned} \right| \implies (a, b, c) = \begin{vmatrix} 
                a_1 & a_2 & a_3 \\
                b_1 & b_2 & b_3 \\
                c_1 & c_2 & c_3
            \end{vmatrix}
        \end{equation*}


    \defitem{Критерий компланарности трёх векторов трёхмерного евклидова пространства}

        $a, b, c$ компланарны $\iff (a, b, c) = 0$.


    \defitem{Критерий коллинеарности двух векторов трёхмерного евклидова пространства}

        \begin{proposal}
            $a, b \in \EE$ коллинеарны $\iff [a, b] = 0$.
        \end{proposal}


    \defitem{Геометрические свойства векторного произведения}

        \begin{proposal}~
            \begin{enumerate}[nosep]
            \item $[a, b] \perp \left< a, b \right>$.
            \item $\left|[a, b]\right| = \vol P(a, b)$.
            \item $\Vol(a, b, [a, b]) \geq 0$.
            \end{enumerate}
        \end{proposal}


    \defitem{Линейное многообразие. Характеризация линейных многообразий как сдвигов подпространств}

        \begin{definition}
            \textit{Линейное многообразие} в $\RR^n$ --- это множество решений некоторой совместной СЛУ.
        \end{definition}

        Пусть $Ax = b$ --- СЛУ, $\varnothing \neq L \subseteq \RR^n$ --- множество решений, $x_z \in L$ --- частное решение.

        Было: Лемма: $L = x_z + S$, где $S$ --- множество решений ОСЛУ $Ax = 0$.

        \begin{proposal}
            Множество $L \subseteq \RR^n$ является линейны многообразием $\iff L = v_0 + S$ для некоторых $v_0 \in \RR^n$ и подпространства $S \subseteq \RR^n$. 
        \end{proposal}


    \defitem{Критерий равенства двух линейных многообразий. Направляющее подпространство и размерность линейного многообразия}

        \begin{proposal}
            Пусть $L_1 = v_1 + S_1$ и $L_2 = v_2 + S_2$ --- два линейных многообразия в $\RR^n$. Тогда,
            \begin{equation*}
                L_1 = L_2 \iff \begin{cases}
                    S_1 = S_2 \ (= S) \\
                    v_1 - v_2 \in S
                \end{cases}
            .\end{equation*}
        \end{proposal}


        Если $L$ --- линейное многообразие, то $L = v_0 + S$, где $S$ определено однозначно.

        \begin{definition}
            $S$ называется \textit{направляющим подпространством} линейного многообразия $L$.
        \end{definition}

        \begin{definition}
            \textit{Размерностью} линейного многообразия называется размерность его направляющего подпространства.
        \end{definition}


    \defitem{Теорема о плоскости, проходящей через $k + 1$ точку в $\RR^n$}

        \begin{theorem}~
            \begin{enumerate}[label=\alph*)]
            \item Через любые $k + 1$ точек в $\RR^n$ проходит плоскость размерности $\leq k$.
            \item Если это точки не лежат в плоскости размерности $<k$, то через них проходит ровно одна плоскость размерности $k$.
            \end{enumerate}
        \end{theorem}

        \begin{corollary}~
            \begin{enumerate}
            \item Через любые две различные точки проходит ровно одна прямая.
            \item Через любые три точки, не лежащие на одной прямой, проходит ровно одна плоскость.
            \end{enumerate}
        \end{corollary}


    \defitem{Три способа задания прямой в $\RR^2$. Уравнение прямой в $\RR^2$, проходящей через две различные точки}

        \begin{figure}[h]
            \centering
            \def\svgwidth{10cm}
            \input{lines_r2.pdf_tex}
        \end{figure}


        \begin{enumerate}
            \item $Ax + By = C \quad\quad (A, B) \neq (0, 0)$ --- нормаль.
            \item векторное уравнение $(v - v_0, n) = 0$, где $n$ --- нормаль.
            \item параметрическое уравнение $v = v_0 + at$, где $a$ --- направляющий вектор.

                \begin{math}
                    \begin{cases}
                        x = x_0 + a_1 t, \\
                        y = y_0 + a_2 t.
                    \end{cases} \quad
                    \begin{gathered}
                        a = (a_1, a_2) \\
                        v_0 = (x_0, y_0)
                    \end{gathered}
                \end{math}
        \end{enumerate}


        \begin{equation*}
            \begin{vmatrix} 
                x - x_0 & y - y_0 \\
                x_1 - x_0 & y_1 - y_0
            \end{vmatrix} = 0 \hspace{1cm}
            \frac{x - x_0}{x_1 - x_0} = \frac{y - y_0}{y_1 - y_0} \hspace{1cm}
            \begin{gathered}
                x_1 = x_0 \implies x = x_0, \\
                y_1 = y_0 \implies y = y_0.
            \end{gathered}
        \end{equation*}


    \defitem{Три способа задания плоскости в $\RR^3$}

        \begin{figure}[h]
            \centering
            \def\svgwidth{10cm}
            \input{plane_with_normal.pdf_tex}
        \end{figure}

        \begin{enumerate}
            \item $Ax + By + Cz = D \quad\quad(A, B, C) \neq (0, 0, 0)$ --- нормаль.
            \item векторное уравнение $(v - v_0, n) = 0$.
            \item параметрическое уравнение $v = v_0 + at + bs$, где $a, b$ --- направляющие векторы (базис в направляющем подпространстве).
        \end{enumerate}


    \defitem{Уравнение плоскости в $\RR^3$, проходящей через три точки, не лежащие на одной прямой}

        \begin{equation*}
            \begin{vmatrix} 
                x - x_0 & y - y_0 & z - z_0 \\
                x_1 - x_0 & y_1 - y_0 & z_1 - z_0 \\
                x_2 - x_0 & y_2 - y_0 & z_2 - z_0
            \end{vmatrix} = 0
        .\end{equation*}


    \defitem{Три способа задания прямой в $\RR^3$}

        \begin{figure}[h]
            \centering
            \def\svgwidth{10cm}
            \input{line_r3.pdf_tex}
        \end{figure}

        \begin{enumerate}
        \item 
            \begin{math}
                \begin{cases}
                    A_1 x + B_1 y + C_1 z = D_1, \\
                    A_2 x + B_2 y + C_2 z = D_2
                \end{cases} \hspace{1cm} 
                \rk \begin{pmatrix} A_1 & B_1 & C_1 \\ A_2 & B_2 & C_2 \end{pmatrix} = 2
            \end{math}

        \item векторное уравнение $[v - v_0, a] = 0$, где $v - v_0$ --- точка, $a$ --- направляющий вектор.
        \item параметрическое уравнение $v = v_0 + at$. 

            \begin{math}
                \begin{gathered}
                    v_0 = (x_0, y_0, z_0) \\
                    a = (a_1, a_2, a_3)
                \end{gathered}
                \quad \longrightarrow \quad
                \begin{cases}
                    x = x_0 + a_1 t, \\
                    y = y_0 + a_2 t, \\
                    z = z_0 + a_3 t.
                \end{cases}
                \quad \iff \quad
                \colorboxed{red}{\dfrac{x - x_0}{a_1} = \dfrac{y - y_0}{a_2} = \dfrac{z - z_0}{a_3}}
                \text{ --- каноническое уравнение прямой}
            \end{math}

            Если, например $a_1 = 0$, то пишут 
            \begin{math}
                \begin{cases}
                    \displaystyle
                    \frac{y - y_0}{a_2} = \frac{z - z_0}{a_3} \\
                    x = x_0
                \end{cases}
            \end{math}
        \end{enumerate}


    \defitem{Уравнения прямой в $\RR^3$, проходящей через две различные точки}

        Уравнение прямой, проходящей через точки $(x_0, y_0, z_0)$ и $(x_1, y_1, z_1)$:
        \begin{equation*}
            \frac{x - x_0}{x_1 - x_0} = \frac{y - y_0}{y_1 - y_0} = \frac{z - z_0}{z_1 - z_0}
        .\end{equation*}


    \defitem{Случаи взаимного расположения двух прямых в $\RR^3$}

        \begin{figure}[h]
            \centering
            \def\svgheight{5cm}
            \input{two_lines.pdf_tex}
        \end{figure}


        \begin{enumerate}[nosep]
            \item \label{lec26:l3} Совпадают.
            \item \label{lec26:l4} Параллельны.
            \item \label{lec26:l5} Пересекаются в точке.
            \item Скрещиваются.
        \end{enumerate}

        \medskip
        $\hyperref[lec26:l3]{1)}$ или $\hyperref[lec26:l4]{2)} \iff [a_1, a_2] = \overrightarrow{0}$.

        $\hyperref[lec26:l3]{1)}, \ \hyperref[lec26:l4]{2)}$ или $\hyperref[lec26:l5]{3)} \iff (a_1, a_2, v_1 - v_2) = 0$.
    

    \defitem{Формула для расстояния от точки до прямой в $\RR^3$}

        \begin{figure}[h]
            \centering
            \def\svgheight{5cm}
            \input{point_to_line.pdf_tex}
        \end{figure}

        Расстояние от точки $v$ до прямой $l = v_0 + at$:
        \begin{equation*}
            \displaystyle
            \rho(v, l) = \frac{\left|[v - v_0, a]\right|}{|a|}
        \end{equation*}
    

    \defitem{Формула для расстояния от точки до плоскости в $\RR^3$}

        \begin{figure}[h]
            \centering
            \def\svgheight{5cm}
            \input{point_to_plane.pdf_tex}
        \end{figure}

        Расстояние от точки $v$ до плоскости $P$ с направляющей нормалью $n$ и направляющим подпространством $S$ ($S = n^{\perp}$):
        \begin{equation*}
            \displaystyle
            \rho(v, P) = \frac{|(v - v_0, n)|}{|n|}.
        \end{equation*}

    

    \defitem{Формула для расстояния между двумя скрещивающимися прямыми в $\RR^3$}

        \begin{figure}[h]
            \centering
            \def\svgheight{5cm}
            \input{skew_lines.pdf_tex}
        \end{figure}

        Расстояние между двумя скрещивающимися прямыми $l_1 = v_1 + a_1 t$ и $l_2 = v_2 + a_2 t$:
        \begin{equation*}
            \begin{gathered}
                P_1 = v_1 + \left< a_1, a_2 \right> \\
                P_2 = v_2 + \left< a_1, a_2 \right> \\
                \rho(l_1, l_2) = \rho(p_1, p_2)
            \end{gathered}
            \hspace{2cm}
            \rho(l_1, l_2) = \frac{\left|(a_1, a_2, v_1 - v_2)\right|}{\left|[a_1, a_2]\right|}
        \end{equation*}


    \defitem{Линейный оператор}

        Пусть $V$ --- векторное пространство над $F$, $\dim V = n$.

        \begin{definition}
        \textit{Линейным оператором} (или \textit{линейным преобразованием}) на/в $V$ называется всякое линейное отображение $\phi \colon V \to V$ (то есть из $V$ \underline{\underline{в себя}}).
        \end{definition}

        $\L(V) := \hom(V, V)$ --- все линейные операторы на/в $V$.


    \defitem{Матрица линейного оператора}

        Пусть $\phi \in \L(V)$, $\E = (e_1, \dots, e_n)$ --- базис $V$.

        Тогда, $(\phi(e_1), \dots, \phi(e_n)) = (e_1, \dots, e_n) \cdot A$, $\quad A \in M_{n}(F)$.

        $A$ называется матрицей линейного оператора в базисе $\E$.

        Обозначение: $A(\phi, \E)$.

        В столбце $A^{(j)}$ записаны координаты вектора $\phi(e_j)$ в базисе $\E$.


    \defitem{Связь между координатами вектора и его образа при действии линейного оператора}

        Пусть $\phi \in \L(V)$, $\E = (e_1, \dots, e_n)$ --- базис $V$, $A = A(\phi, \E)$, 

        \begin{math}
            \begin{cases}
                v = x_1 e_1 + \dots + x_n e_n \\
                \phi(v) = y_1 e_1 + \dots + y_n e_n
            \end{cases} \implies \begin{pmatrix} y_1 \\ \dots \\ y_n \end{pmatrix} = A \cdot \begin{pmatrix} x_1 \\ \dots \\ x_n \end{pmatrix}
        \end{math}


    \defitem{Формула изменения матрицы линейного оператора при переходе к другому базису}

        Пусть $\E'$ --- другой базис $V$, $\E' = \E \cdot C$, $C \in M_n^{0}(F)$

        $A = A(\phi, \E)$, $A' = A(\phi, \E') \implies A' = C^{-1} A C$.


    \defitem{Подобные матрицы}

        \begin{definition}
            Матрицы $A, A' \in M_n$ называются \textit{подобными}, если $\exists C \in M_n^{0}(F)$, такая что $A' = C^{-1} A C$.
        \end{definition}


    \defitem{Подпространство, инвариантное относительно линейного оператора}

        \begin{definition}
            Подпространство $U \subseteq V$ называется \textit{инвариантным относительно $\phi$} (или $\phi$-\textit{инвариантным}), \\если $\phi(U) \subseteq U$ (то есть $\phi(u) \in U \ \forall u \in U$).    
        \end{definition}


    \defitem{Вид матрицы линейного оператора в базисе, дополняющем базис инвариантного подпространства}

        % TODO


    \defitem{Вид матрицы линейного оператора в базисе, согласованном с разложением пространства в прямую сумму двух инвариантных подпространств}

        %TODO


    \defitem{Собственный вектор линейного оператора}

        Вектор $v \in V$ называется \textit{собственным} для $\phi$, если $v \neq 0$ и $\phi(v) = \lambda v$ для некоторого $\lambda \in F$.


    \defitem{Собственное значение линейного оператора}

        Элемент $\lambda \in F$ называется \textit{собственным значением} для $\phi$, если $\exists v \in V$, такой что $v \neq 0$ и $\phi(v) = \lambda v$.


    \defitem{Спектр линейного оператора}

        Множество всех собственных значений линейного оператора называется его \textit{спектром} и обозначается $\spec \phi$.


    \defitem{Диагонализуемый линейный оператор}

        \begin{definition}
            Линейный оператор $\phi$ называется \textit{диагонализуемым}, если существует базис в $V$, в котором матрица линейного оператора $\phi$ диагональна.
        \end{definition}


    \defitem{Критерий диагонализуемости линейного оператора в терминах собственных векторов}

        \begin{proposal}
            Линейный оператор $\phi$ диагонализуем $\iff$ в $V$ есть базис из собственных векторов.
        \end{proposal}


    \defitem{Собственное подпространство линейного операторё}

        Пусть $\phi \in \L(V)$, $\lambda \in F$.

        $V_\lambda(\phi) := \{v \in V \mid \phi(v) = \lambda v\}$.

        \begin{definition}
            $\lambda \in \spec \phi \implies V_\lambda(\phi)$ называется \textit{собственным подпространством} линейного оператора $\phi$, отвечающим собственному значению $\lambda$.
        \end{definition}


    \defitem{Характеристический многочлен линейного оператора}

        \begin{definition}
            Многочлен $\chi_\phi(t) := (-1)^n \det(\phi - t \cdot \mathrm{Id}) \in F[t]$ называется \textit{характеристическим многочленом} линейного оператора $\phi$.
        \end{definition}


    \defitem{Связь спектра линейного оператора с его характеристическим многочленом}

        \begin{corollary}
            $\lambda \in \spec \phi \iff \chi_\phi(\lambda) = 0$, то есть $\lambda$ --- корень характеристического многочлена.
        \end{corollary}

        \begin{corollary}
            $|\spec \phi| \leq n$.   
        \end{corollary}


    \defitem{Алгебраическая кратность собственного значения линейного оператора}

        Пусть $\lambda \in \spec \phi$.

        Пусть $a_\lambda = a_\lambda(\phi) := $ кратность $\lambda$ как корня многочлена $\chi_\phi(t)$. То есть $\chi_\phi(t) \divby (t - \lambda)^{a_\lambda}$ и $\chi_\phi(t) \!\!\not\;\divby (t - \lambda)^{a_\lambda + 1}$.

        \begin{definition}
            $a_\lambda$ называется \textit{алгебраической кратностью} собственного значения $\lambda$.
        \end{definition}


    \defitem{Геометрическая кратность собственного значения линейного оператора}

        \begin{definition}
            Число $g_\lambda = g_\lambda(\phi) := \dim V_\lambda(\phi)$ называется \textit{геометрической кратностью} собственного значения $\lambda$.
        \end{definition}


    \defitem{Связь между алгебраической и геометрической кратностями собственного значения линейного оператора}

        \begin{proposal}
            $g_\lambda \leq a_\lambda \ \forall \lambda \in \spec \phi$.
        \end{proposal}


    \defitem{Критерий диагонализуемости линейного оператора в терминах его характеристического многочлена и кратностей его собственных значений}

        \begin{theorem}{(критерий диагонализуемости)} $\phi$ диагонализуемо $\iff$ выполнены одновременно следующие условия:
            \begin{enumerate}
                \item $\chi_\phi(t)$ разлагается на линейные множители.
                \item если $\chi_\phi(t) = (t - \lambda_1)^{k_1} \cdot \ldots \cdot (t - \lambda_s)^{k_s}$, то $g_{\lambda_i} = a_{\lambda_i} \ \forall i$. (то есть $\lambda_i \neq \lambda_j$ при $i \neq j$)
            \end{enumerate}
        \end{theorem}


    \defitem{Линейное отображение евклидовых пространств, сопряжённое к данному}



    \defitem{Линейный оператор в евклидовом пространстве, сопряжённый к данному}



    \defitem{Самосопряжјнный линейный оператор в евклидовом пространстве}



    \defitem{Теорема о каноническом виде самосопряжённого линейного оператора}



    \defitem{Каким свойством обладают собственные подпространства самосопряжённого линейного оператора, отвечающие попарно различным собственным значениям}



    \defitem{Приведение квадратичной формы к главным осям}



    \defitem{Ортогональный линейный оператор}



    \defitem{Теорема о пяти эквивалентных условиях, определяющих ортогональный линейный оператор}



    \defitem{Теорема о каноническом виде ортогонального линейного оператора}



    \defitem{Классификация ортогональных линейных операторов в трёхмерном евклидовом пространстве}


    \end{colloq}


    \newpage
    \section{Вопросы на доказательство}

    \subsection{Подпространства}
    \begin{colloq}
        \proofitem{Теорема о связи размерности суммы двух подпространств с размерностью их пересечения}

            \begin{theorem}
                $\dim (U \cap W) + \dim (U + W) = \dim U + \dim W$.
            \end{theorem}

            \begin{example}
                Всякие две плоскости в $\RR^3$ (содержащие 0) имеют общую прямую.

                Здесь $V = \RR^3$, $\dim U = 2$, $\dim W = 2$.

                При этом $\dim (U + W) \leq 3$.

                Тогда, $\dim (U \cap W) = \dim U + \dim W - \dim (U + W) \geq 2 + 2 - 3 = 1$.
            \end{example}

            \begin{proof}
                Пусть $\dim (U \cap W) = p$, $\dim U = q$, $\dim W = r$.

                Пусть $a = \{a_1, \dots, a_p\}$ -- базис в $U \cap W$.

                Тогда, $a$ можно дополнить до базиса в $U$ и в $W$:

                $b = \{b_1, \dots, b_{q - p}\}$ -- такая система, что $a \cup b$ -- базис в $U$.

                $c = \{c_1, \dots, c_{r - p}\}$ -- такая система, что $a \cup c$ -- базис в $W$.

                \bigskip
                Докажем, что $a \cup b \cup c$ -- базис в $U + W$.
                \begin{enumerate}
                \item 
                    $\langle a \cup b \cup c \rangle = U + W$:

                    $v \in U + W \implies \exists u \in U, w \in W$, такие что $v = u + w$.

                    $u \in U = \langle a \cup b \rangle \subseteq \langle a \cup b \cup c \rangle$.

                    $w \in W = \langle a \cup c \rangle \subseteq \langle a \cup b \cup c \rangle$.

                    Значит, $v \in \langle a \cup b \cup c \rangle$.

                \item
                    $a \cup b \cup c$ линейно независимо.

                    Пусть $\underbrace{\alpha_1 a_1 + \dots + \alpha_p a_p}_x + \underbrace{\beta_1 b_1 + \dots + \beta_{q - p} b_{q - p}}_y + \underbrace{\gamma_1 c_1 + \dots + \gamma_{r - p} c_{r - p}}_z = 0$, где $\alpha_i, \beta_j, \gamma_k \in F$.

                    Тогда, $z = -\underset{\in U}{x} - \underset{\in U}{y} \in U$.

                    Но, $z \in W$, значит $z \in U \cap W$. 

                    То есть $z = \lambda_1 a_1 + \dots + \lambda_p a_p$, $\lambda_i \in F$.

                    Тогда, $\lambda_1 a_1 + \dots + \lambda_p a_p - \gamma_1 c_1 - \dots - \gamma_{r - p} c_{r - p} = 0$

                    Так как $a \cup c$ линейно независимо, то $\lambda_1 = \dots = \lambda_p = \gamma_1 = \dots = \gamma_{r - p} = 0$ и $z = 0$.

                    Следовательно, $x + y = 0$, то есть $\alpha_1 a_1 + \dots \alpha_p a_p + \beta_1 b_1 + \dots + \beta_{q - p} b_{q - p} = 0$.

                    Так как $a \cup b$ линейно независимо, то $\alpha_1 = \dots = \alpha_p = \beta_1 = \dots = \beta_{q - p} = 0$.

                    Получаем, что $a \cup b \cup c$ линейно независимо.
                \end{enumerate}

                Итог: $a \cup b \cup c$ -- базис в $U + W$.
                \begin{align*}
                    \dim (U + W) 
                    &= |a| + |b| + |c| \\
                    &= p + q - p + r - p \\
                    &= q + r - p \\
                    &= \dim U + \dim W - \dim (U \cap W)
                .\qedhere\end{align*}.
            \end{proof}

        \proofitem{Теорема о пяти эквивалентных условиях, определяющих линейно независимый набор подпространств векторного пространства}

            \begin{theorem}
                Следующие условия эквивалентны:
                \begin{enumerate}[label=(\arabic*)]
                \item \label{lec15:eq_thm_1} $U_1, \dots, U_k$ линейно независимы.
                \item \label{lec15:eq_thm_2} всякий $u \in U_1 + \dots + U_k$ единственным образом представим в виде $u = u_1 + \dots + u_k$, где $u_i \in U_i$.
                \item \label{lec15:eq_thm_3} Если $\E_i$ --- базис в $U_i \ \forall i$, то $\underbrace{\E_1 \sqcup \E_2 \sqcup \dots \sqcup \E_k}_\text{объединение мультимножеств}$ --- базис в $U_1 + \dots + U_k$.
                \item \label{lec15:eq_thm_4} $\dim (U_1 + \dots + U_k) = \dim U_1 + \dots + \dim U_k$.
                \item \label{lec15:eq_thm_5} $\forall i = 1, \dots, k \quad U_i \cap (U_1 + \dots + U_{i - 1} + U_{i + 1} + \dots + U_k) = 0$.
                \end{enumerate}
            \end{theorem}

            \begin{example}
                Если $\E_1 = \{e_1, e_2\}, \E_2 = \{e_2, e_3\}$, то
                \begin{itemize}[nosep]
                \item $\E_1 \cup \E_2 = \{e_1, e_2, e_3\}$ -- 3 элемента,
                \item $\E_1 \sqcup \E_2 = \{e_1, e_2, e_2, e_3\}$ -- 4 элемента.
                \end{itemize}
            \end{example}

            \begin{proof}
                Пусть $\widehat{U}_i = U_1 + \dots + U_{i - 1} + U_{i + 1} + \dots + U_k$.

                \begin{description}
                \item[\ref{lec15:eq_thm_1}$\implies$\ref{lec15:eq_thm_2}]
                    Пусть $u_1 + \dots + u_k = u'_1 + \dots + u'_k$, где $u_i, u'_i \in U_i$.

                    Тогда, $\underset{\in U_1}{(u_1 - u'_1)} + \underset{\in U_2}{(u_2 - u'_2)} + \dots + \underset{\in U_k}{(u_k - u'_k)} = 0 \implies u_i - u'_i = \dots = u_k - u'_k = 0$.

                    То есть, $u_1 = u'_1, \dots, u_k = u'_k$.

                \item[\ref{lec15:eq_thm_2}$\implies$\ref{lec15:eq_thm_3}]
                    Пусть $u \in U_1 + \dots + U_k$ -- произвольный.

                    $u$ единственным образом представим в виде $u = u_1 + \dots + u_k$, где $u_i \in U_i$,

                    $u_i$ единственным образом представим в виде линейной комбинации векторов из $\E_i$.
             
                    Следовательно, $u$ единственным образом представим в виде линейной комбинации векторов из $\E_1 \sqcup \dots \sqcup \E_k$.

                    То есть, $\E_1 \sqcup \dots \sqcup \E_k$ --- базис в $U_1 + \dots + U_k$.

                \item[\ref{lec15:eq_thm_3}$\implies$\ref{lec15:eq_thm_4}]
                    Очевидно. 

                \item[\ref{lec15:eq_thm_4}$\implies$\ref{lec15:eq_thm_5}]
                    \begin{align*}
                        \dim (U_i \cap \widehat{U}_i) 
                        &= \dim U_i + \dim \widehat{U}_i - \dim (U_1 + \dots + U_k) \\
                        &\leq \dim U_i + (\dim U_1 + \dots + \dim U_{i - 1} + \dim U_{i + 1} + \dots + \dim U_k) - (\dim U_1 + \dots + \dim U_k) \\
                        &= 0
                    .\end{align*}

                \item[\ref{lec15:eq_thm_5}$\implies$\ref{lec15:eq_thm_1}]
                    $u_1 + \dots + u_k = 0$, где $u_i \in U_i$.

                    Тогда, $\underset{\in U_i}{u_i} = \underbrace{-u_1 - \dots - u_{i - 1} - u_{i + 1} - \dots - u_k}_{\in \widehat{U}_i}$

                    Следовательно, $u_i \in U_i \cap \widehat{U}_i = 0 \implies u_i = 0$.
                    \qedhere
                \end{description}
            \end{proof}

            \begin{corollary}
                Пусть $k = 2$, тогда

                $U_1, U_2$ линейно независимы $\iff$ $U_1 \cap U_2 = 0$.
            \end{corollary}

    \end{colloq}

    \subsection{Линейные отображения}
    \begin{colloq}
        \proofitem{Свойства отношения изоморфности на множестве всех векторных пространств}

            \begin{proposal}
                \label{lec16:prop_1}
                Если $\phi \colon V \to W$ --- изоморфизм, то $\phi^{-1}$ --- тоже изоморфизм. 
            \end{proposal}

            \begin{proof}
                Биективность есть, так как $\phi^{-1}$ --- обратное отображение.
                Проверим линейность:
                \begin{enumerate}[label=\arabic*)]
                \item $w_1, w_2 \in W \implies w_1 = \phi(\phi^{-1}(w_1))$, $w_2 = \phi(\phi^{-1}(w_2))$
                    \begin{align*}
                        \phi^{-1}(w_1 + w_2) &= \phi^{-1}\Big(\underbrace{\phi\left(\phi^{-1}(w_1)\right)}_{w_1} + \underbrace{\phi\left(\phi^{-1}(w_2)\right)}_{w_2}\Big) \\
                        &= \underbrace{\phi^{-1}\big(\phi}_{Id}\left(\phi^{-1}(w_1) + \phi^{-1}(w_2)\right)\!\big)\\
                        &= \phi^{-1}(w_1) + \phi^{-1}(w_2)
                    .\end{align*}

                \item 
                    \begin{align*}
                        \phi^{-1}(\lambda \cdot w_1) &= \phi^{-1}\left(\lambda \cdot \phi\left(\phi^{-1}\left(w_1\right)\right)\right) \\
                        &= \underbrace{\phi^{-1}\big(\phi}_{Id}\left(\lambda \cdot \phi^{-1}(w_1)\right)\!\big) \\
                        &= \lambda \phi^{-1}(w_1)
                    .\qedhere\end{align*}
                \end{enumerate}
            \end{proof}


            \bigskip
            Пусть $U \xrightarrow{\psi} V \xrightarrow{\phi} W$, тогда $\phi \circ \psi : U \to W$ --- композиция.

            \begin{proposal}~
                \label{lec16:prop_2}
                \begin{enumerate}[nosep]
                \item Если $\phi$, $\psi$ линейны, то $\phi \circ \psi$ тоже линейна.
                \item Если $\phi$, $\psi$ --- изоморфизмы, то $\phi \circ \psi$ --- тоже изоморфизм.
                \end{enumerate}
            \end{proposal}

            \begin{proof}~
                \begin{enumerate}
                \item \label{lec16:comp_1}
                    \begin{enumerate}[label=(\arabic*)]
                        \item $(\phi \circ \psi)(u_1 + u_2) = \phi(\psi(u_1 + u_2)) = \phi(\psi(u_1) + \psi(u_2)) = \phi(\psi(u_1)) + \phi(\psi(u_2)) = (\phi \circ \psi)(u_1) + (\phi \circ \psi)(u_2)$.
                        \item $(\phi \circ \psi)(\lambda u) = \phi(\psi(\lambda u)) = \phi(\lambda \psi(u)) = \lambda \phi(\psi(u)) = \lambda(\phi \circ \psi)(u)$.
                    \end{enumerate}
                \item из \ref{lec16:comp_1} следует, что $(\phi \circ \psi)$ линейно, но при этом биективно как композиция двух биекций.
                    \qedhere
                \end{enumerate}
            \end{proof}

            \bigskip
            \begin{theorem}
                Отношение изоморфности является отношением эквивалентности на множестве всех векторных пространств над фиксированным полем $F$.
            \end{theorem}

            \begin{proof}~
                \begin{enumerate}
                \item Рефлексивность: $Id: V \MapsTo V$.
                \item Симметричность: $V \simeq W \implies W \simeq V$ следует из \hyperref[lec16:prop_1]{Предложения 1}.
                \item Транзитивность: $U \simeq V$, $V \simeq W \implies U \simeq W$ следует из \hyperref[lec16:prop_2]{Предложения 2}.
                    \qedhere
                \end{enumerate}
            \end{proof}


        \proofitem{Критерий изоморфности двух конечномерных векторных пространств}

            \begin{theorem}
                Пусть $V$, $W$ --- два конечномерных векторных пространства над $F$.

                Тогда, $V \simeq W \iff \dim V = \dim W$.
            \end{theorem}

            \begin{lemma}
                \label{lec16:lemma_1}
                $\dim V = n \implies V \simeq F^n$.
            \end{lemma}

            \begin{proof}
                Фиксируем базис $(e_1, \dots, e_n)$ в $V$.

                Тогда, отображение $\phi \colon V \to F^n$ --- изоморфизм.
                \begin{equation*}
                    v = x_1 e_1 + \dots + x_n e_n \implies \phi(v) := \begin{pmatrix} x_1 \\ \vdots \\ x_n \end{pmatrix}.
                \end{equation*}
            \end{proof}

            \begin{lemma}
                \label{lec16:lemma_2}
                Пусть $\phi \colon V \MapsTo W$ и $e_1, \dots, e_n$ --- базис $V$, тогда $\phi(e_1), \dots, \phi(e_n)$ --- базис $W$. 
            \end{lemma}

            \begin{proof}
                Пусть $w \in W$. Тогда $\exists x_1, \dots, x_n \in F$, такие что $\phi^{-1}(w) = x_1 e_1 + \dots + x_n e_n$.

                Тогда, $w = \phi\left(\phi^{-1}(w)\right) = \phi(x_1 e_1 + \dots + x_n e_n) = x_1 \phi(e_1) + \dots + x_n \phi(e_n) \implies W = \langle \phi(e_1), \dots, \phi(e_n) \rangle$.

                Теперь докажем линейную независимость:

                Пусть $\alpha_1 \phi(e_1) + \dots + \alpha_n \phi(e_n) = \overrightarrow{0}$.

                Тогда, $\phi(\alpha_1 e_1 + \dots + \alpha_n e_n) = \overrightarrow{0}$.

                Применяя $\phi^{-1}$ получаем, $\alpha_1 e_1 + \dots + \alpha_n e_n = \phi^{-1}(\overrightarrow{0}) = \overrightarrow{0}$. Значит, $\alpha_1 = \dots = \alpha_n = 0$.

                Итог: $\phi(e_1), \dots, \phi(e_n)$ --- базис в $V$.
            \end{proof}

            \begin{proof}[Доказательство теоремы]~
                \begin{description}
                \item[$\impliedby$] Пусть $\dim V= \dim W = n$. Тогда по \hyperref[lec16:lemma_1]{лемме 1} $V \simeq F^n, W \simeq F^n$, значит $V \simeq W$.
                \item[$\implies$] Пусть $V \simeq W$. Фиксируем изоморфизм $\phi \colon V \MapsTo W$.

                    Тогда по \hyperref[lec16:lemma_2]{лемме 2} получаем, что $\phi(e_1), \dots, \phi(e_n)$ --- базис $W$, а значит $\dim V = n = \dim W$.
                    \qedhere
                \end{description}
            \end{proof}


        \proofitem{Существование и единственность линейного отображения с заданными образами базисных векторов}
        \label{2_2_3}

            Пусть $V$, $W$ --- векторные пространства над $F$ и $(e_1, \dots, e_n)$ --- фиксированный базис в $V$.

            \begin{proposal}~
                \label{lec16:propopop}
                \begin{enumerate}
                \item Если $\phi \colon V \to W$ --- линейное отображение, то $\phi$ однозначно определяется векторами $\phi(e_1), \dots, \phi(e_n)$,
                \item $\forall w_1, \dots, w_n \in W \ \exists!$ линейное отображение $\phi$, такое что, $\phi(e_1) = w_1, \dots, \phi(e_n) = w_n$.
                \end{enumerate}
            \end{proposal}

            \begin{proof}~
                \begin{enumerate}
                \item \label{lec16:kekek}
                    $v \in V \implies v = x_1 e_1 + \dots + x_n e_n \implies \phi(v) = x_1 \phi(e_1) + \dots + x_n \phi(e_n)$.
                \item 
                    Зададим $\phi \colon V \to W$ формулой $\phi(x_1 e_1 + \dots + x_n e_n) = x_1 w_1 + \dots + x_n w_n$.

                    Тогда $\phi$ --- линейное отображение из $V$ в $W$ (упражнение).

                    Единственность следует из \ref{lec16:kekek}.
                    \qedhere
                \end{enumerate}
            \end{proof}


        \proofitem{Связь между координатами вектора и его образа при линейном отображении}

            \begin{proposal}
                Пусть $\phi \colon V \to W$ --- линейное отображение,

                $\E = (e_1, \dots, e_n)$ --- базис $V$,

                $\F = (f_1, \dots, f_m)$ --- базис $W$,

                $A = A(\phi, \E, \F)$.

                \begin{math}
                    \begin{aligned}
                        v \in V \implies &v = x_1 e_1 + \dots + x_n e_n, \\
                        &\phi(v) = y_1 f_1 + \dots + y_m f_m.
                    \end{aligned}
                \end{math}

                Тогда,
                \begin{equation*}
                    \begin{pmatrix} y_1 \\ \vdots \\ y_m \end{pmatrix} = A \begin{pmatrix} x_1 \\ \vdots \\ x_n \end{pmatrix}
                .\end{equation*}
            \end{proposal}

            \begin{proof}
                $v = (e_1, \dots, e_m) \begin{pmatrix} x_1 \\ \vdots \\ x_n \end{pmatrix}$.

                Значит, 
                \begin{align*}
                    \phi(v) &= (\phi(e_1), \dots, \phi(e_n)) \begin{pmatrix} x_1 \\ \vdots \\ x_n \end{pmatrix} = (f_1, \dots, f_m) \cdot A \cdot \begin{pmatrix} x_1 \\ \vdots \\ x_n \end{pmatrix}
                .\end{align*}

                При этом, 
                \begin{align*}
                    \phi(v) = (f_1, \dots, f_m) \begin{pmatrix} y_1 \\ \vdots \\ y_m \end{pmatrix}
                .\end{align*}

                Так как $f_1, \dots, f_m$ линейно независимы, то
                \begin{equation*}
                    A \begin{pmatrix} x_1 \\ \vdots \\ x_n \end{pmatrix} = \begin{pmatrix} y_1 \\ \vdots \\ y_m \end{pmatrix}
                .\end{equation*}
            \end{proof}


        \proofitem{Формула изменения матрицы линейного отображения при замене базисов}

            Пусть теперь $\E'$ --- другой базис в $V$, $\F'$ --- другой базис в $W$.

            $\E' = \E \cdot C_{\in M_n}$,

            $\F' = \F \cdot D_{\in M_m}$.

            $A = A(\phi, \E, \F)$,

            $A' = A(\phi, \E', \F')$.

            \begin{proposal}
                $A' = D^{-1} A C$.
            \end{proposal}

            \begin{proof}
                \begin{equation*}
                    (e'_1, \dots e'_n) = (e_1, \dots, e_n) \cdot C
                .\end{equation*}
                
                Применим $\phi$, 
                \begin{align*}
                    (\phi(e'_1), \dots, \phi(e'_n)) &= (\phi(e_1), \dots, \phi(e_n)) \cdot C = (f_1, \dots, f_m) \cdot A \cdot C
                \end{align*}
                
                При этом,
                \begin{align*}
                    (\phi(e'_1), \dots, \phi(e'_n)) &= (f'_1, \dots, f'_m) \cdot A' = (f_1, \dots, f_m) \cdot D \cdot A'
                .\end{align*}

                Отсюда,
                \begin{equation*}
                    A \cdot C = D \cdot A' \implies A' = D^{-1} \cdot A \cdot C
                .\qedhere\end{equation*}
            \end{proof}


        \proofitem{Изоморфизм $\hom(V,W) \MapsTo \text{Mat}_{m \times n}(F)$ при фиксированных базисах $V$ и $W$}

            \begin{corollary}
                При фиксированном $\E$ и $\F$ отображение $\phi \mapsto A(\phi, \E, \F)$ является изоморфизмом между $\hom(V, W)$ и $\text{Mat}_{m \times n}(F)$.
            \end{corollary}

            Зафиксируем базисы $\E = (e_1, \dots, e_n)$ в $V$ и $\F = (f_1, \dots, f_m)$ в $W$.

            \paragraph{Линейность.}
            \begin{enumerate}
            \item
                \begin{math}
                    \begin{aligned}[t]
                        \phi, \psi \in \hom(V, W), \
                        &A_\phi = A(\phi, \E, \F)& \\
                        &A_\psi = A(\psi, \E, \F)& \\
                        &A_{\phi + \psi} = A(\phi + \psi, \E, \F)& \implies A_{\phi + \psi} = A_\phi + A_\psi
                    \end{aligned}
                \end{math}

            \item
                \begin{math}
                    \begin{aligned}[t]
                        \lambda \in F, \phi \in \hom(V, W), \
                        &A_\phi = A(\phi, \E, \F)& \\
                        &A_{\lambda \phi} = A(\lambda \phi, \E, \F)& \implies A_{\lambda \phi} = \lambda A_\phi
                    \end{aligned}
                \end{math}
            \end{enumerate}

            \begin{proof}~
                \begin{enumerate}
                \item 
                    \begin{align*}
                        (f_1, \dots, f_m) \cdot A_{\phi + \psi}
                        &= ((\phi + \psi)(e_1), \dots, (\phi + \psi)(e_n)) \\
                        &= (\phi(e_1), \dots, \phi(e_n)) + (\psi(e_1), \dots, \psi(e_n)) \\
                        &= (f_1, \dots, f_m) A_\phi + (f_1, \dots, f_m) A_\psi \\
                        &= (f_1, \dots, f_m) (A_\phi + A_\psi)
                    .\end{align*}

                    Следовательно, $A_{\phi + \psi} = A_\phi + A_\psi$.

                \item 
                    \begin{align*}
                        (f_1, \dots, f_m) \cdot A_{\lambda \phi}
                        &= ((\lambda \phi)(e_1), \dots, (\lambda \phi)(e_n)) \\
                        &= ((\phi)(e_1), \dots, (\phi)(e_n)) \lambda \\
                        &= (f_1, \dots, f_m) A_{\phi} \lambda
                    .\end{align*}
                    Следовательно, $A_{\lambda \phi} = \lambda A_{\phi}$.
                    \qedhere
                \end{enumerate}
            \end{proof}

            \paragraph{Биективность.}

            По определению, в $j$-м столбце матрицы $A$ стоят координаты вектора $\phi(e_j)$ в базисе $\F$.

            А значит, из пункта \hyperref[2_2_3]{2.3} следует, что при фиксированных базисах $\E$ и $\F$ отображение $\phi \mapsto A(\phi, \E, \F)$ является биекцией между $\hom(V, W)$ и $\text{Mat}_{m \times n}(F)$.


        \proofitem{Матрица композиции двух линейных отображений}

            Пусть $U \xrightarrow{\psi} V \xrightarrow{\phi} W$ --- цепочка линейных отображений, а $\phi \circ \psi : U \to W$ --- их композиция,

            $\E = (e_1, \dots, e_n)$ --- базис $V$,

            $\F = (f_1, \dots, f_m)$ --- базис $W$,

            $\G = (g_1, \dots, g_k)$ --- базис $U$.

            $A_{\phi} = A(\phi, \E, \F)$,

            $A_\psi = A(\psi, \G, \E)$,

            $A_{\phi \circ \psi} = A(\phi \circ \psi, \G, \F)$.

            Тогда, $A_{\phi \circ \psi} = A_\phi \cdot A_\psi$.

            \begin{proof}
                $(\psi(g_1), \dots, \psi(g_k)) = (e_1, \dots, e_n) A_\psi$.
                Тогда применяя $\phi$,
                \begin{equation*}
                    \left(\phi\left(\psi(g_1)\right), \dots, \phi\left(\psi(g_k)\right)\right) = \left(\phi(e_1), \dots, \phi(e_n)\right) A_\psi = (f_1, \dots, f_m) A_\phi A_\psi
                .\end{equation*}
                С другой стороны,
                \begin{equation*}
                    \left(\phi\left(\psi(g_1)\right), \dots, \phi\left(\psi(g_k)\right)\right) = (f_1, \dots, f_m) A_{\phi \circ \psi}
                .\end{equation*}

                Значит, $A_\phi \cdot A_\psi = A_{\phi \circ \psi}$.
            \end{proof}


        \proofitem{Утверждение о том, что ядро и образ --- подпространства в соответствующих векторных пространствах}

            \begin{proposal}~
                \begin{enumerate}
                \item Ядро --- подпространство в $V$.
                \item Образ --- подпространство в $W$.
                \end{enumerate}
            \end{proposal}

            \begin{proof}~
                \begin{enumerate}
                \item 
                    \begin{enumerate}
                    \item $\phi(0_V) = 0_W$,
                    \item $v_1, v_2 \in \ker \phi \implies \phi(v_1 + v_2) = \phi(v_1) + \phi(v_2) = 0 + 0 = 0 \implies v_1 + v_2 \in \ker \phi$,
                    \item $\lambda \in F, v \in V \implies \phi(\lambda v) = \lambda \phi(v) = \lambda 0 = 0 \implies \lambda v \in \ker \phi$.
                    \end{enumerate}

                \item
                    \begin{enumerate}
                    \item $0_W = \phi(W) \in \Im \phi$,
                    \item $w_1, w_2 \in \Im \phi \implies \exists v_1, v_2 : w_1 = \phi(v_1), w_2 = \phi(v_2) \implies w_1 + w_2 = \phi(v_1) + \phi(v_2) = \phi(v_1 + v_2) \in \Im \phi$,
                    \item $\phi \in F, w \in \Im \phi \implies \exists v \in V : w = \phi(v) \implies \lambda w = \lambda \phi(v) = \phi(\lambda v) \in \Im \phi$.
                        \qedhere
                    \end{enumerate}
                \end{enumerate}
            \end{proof}


        \proofitem{Связь между рангом матрицы линейного отображения и размерностью его образа}

            Пусть $U \subseteq V$ --- подпространство, $u_1, \dots, u_k$ --- базис в $U$.

            \begin{lemma}
                Тогда, $\phi(U) = \left< \phi(u_1), \dots, \phi(u_k) \right>$.
                В частности, $\dim \phi(U) \leq \dim U$ и $\dim \Im \phi \leq \dim V$.
            \end{lemma}

            \begin{proof}
                $u \in U \implies u = \alpha_1 u_1 + \dots + \alpha_k u_k$, $\alpha_i \in F$, тогда
                \begin{equation*}
                    \phi(u) = \alpha_1 \phi(u_1) + \dots + \alpha_k \phi(u_k) \in \left< \phi(u_1), \dots, \phi(u_k) \right>
                .\qedhere\end{equation*}
            \end{proof}

            \bigskip
            Пусть 
            \begin{math}
                \begin{aligned}[t]
                    \E &= (e_1, \dots, e_n) \text{ --- базис $V$}, \\
                    \F &= (f_1, \dots, f_m) \text{ --- базис $W$}, \\
                    A &= A(\phi, \E, \F).
                \end{aligned}
            \end{math}

            \begin{theorem}
                $\rk A = \dim \Im \phi$.
            \end{theorem}

            \begin{proof}
                По лемме, $\Im \phi = \left< \phi(e_1), \dots, \phi(e_n) \right>$. Поэтому, $\dim \Im \phi = \rk \{\phi(e_1), \dots, \phi(e_n)\}$.

                Так как $j$-й столбец матрицы $A$ составлен из координат вектора $\phi(e_j)$ в базисе $\F$, то
                \begin{equation*}
                    \alpha_1 \phi(e_1) + \dots + \alpha_n \phi(e_n) = 0 \iff \alpha_1 A^{(1)} + \dots + \alpha_n A^{(n)} = 0
                .\end{equation*}

                Значит, $\dim \Im \phi = \rk \{\phi(e_1), \dots, \phi(e_n)\} = \rk \{A^{(1)}, \dots, A^{(n)}\} = \rk A$.
            \end{proof}

            \begin{comment}
                Число $\dim \Im \phi$ называется \textit{рангом} линейного отображения $\phi$, обозначается $\rk \phi$.
            \end{comment}


        \proofitem{Лемма о дополнении базиса ядра линейного отображения до базиса всего пространства}
            \label{lemma_o_dopolnenii_bazisa_jadra}

            \begin{proposal}
                Пусть $e_1, \dots, e_k$ --- базис $\ker \phi$ и векторы $e_{k + 1}, \dots, e_n$ дополняют его до базиса всего $V$.

                Тогда, $\phi(e_{k + 1}), \dots, \phi(e_n)$ образуют базис в $\Im \phi$.
            \end{proposal}

            \begin{proof}
                $\Im \phi = \left< \phi(e_1), \dots, \phi(e_k), \phi(e_{k + 1}), \dots, \phi(e_n) \right> = \left< \phi(e_{k + 1}), \dots, \phi(e_n) \right>$. (так как $\phi(e_1) = \dots = \phi(e_k) = 0$).

                Осталось показать, что $\phi(e_{k + 1}), \dots, \phi(e_n)$ линейно независимы.

                Пусть $\alpha_{k + 1} \phi(e_{k + 1}) + \dots + \alpha_n \phi(e_n) = 0$, где $\alpha_i \in F$.

                Тогда $\phi(\alpha_{k + 1} e_{k + 1} + \dots \alpha_n e_n) = 0 \implies \alpha_{k + 1} e_{k + 1} + \dots + \alpha_n e_n \in \ker \phi$.

                Но тогда $\alpha_{k + 1} e_{k + 1} + \dots \alpha_n e_n = \beta_1 e_1 + \dots + \beta_k e_k$, где $\beta_j \in F$.

                Так как $(e_1, \dots, e_n)$ --- базис $V$, то $\alpha_i = \beta_j = 0 \ \forall i, j$.
            \end{proof}


        \proofitem{Теорема о связи размерностей ядра и образа линейного отображения, приведение матрицы линейного отображения к диагональному виду с единицами и нулями на диагонали}

            \begin{theorem}
                $\dim \Im \phi + \dim \ker \phi = \dim V$.
            \end{theorem}

            \begin{proof}
                Вытекает из \hyperref[lemma_o_dopolnenii_bazisa_jadra]{предыдущего предложения} так как в его доказательстве:

                $\dim V = n$,

                $\dim \ker \phi = k$,

                $\dim \Im \phi = n - k$.
            \end{proof}

            \begin{proposal}
                Пусть $\rk \phi = r$. Тогда существует базис $\E$ в $V$ и базис $\F$ в $W$, такие что
                \begin{equation*}
                    A(\phi, \E, \F) = \left(
                        \begin{array}{c|c}
                            E & 0 \\
                            \hline
                            0 & 0
                        \end{array}
                    \right) = \bordermatrix{    
                        &   & r &   &   &   & n - r &   \cr
                        & 1 & 0 & 0 & \dots & 0 & 0 & 0 \cr
                      \hspace{0.7cm} r & 0 & \ddots & 0 & \dots & 0 & 0 & 0 \cr
                        & 0 & 0 & 1 & \dots & 0 & 0 & 0 \cr
                        & \vdots & \vdots & \vdots & \ddots & \vdots & \vdots & \vdots \cr
                        & 0 & 0 & 0 & \dots & 0 & 0 & 0 \cr
                  m - r & 0 & 0 & 0 & \dots & 0 & 0 & 0 \cr
                        & 0 & 0 & 0 & \dots & 0 & 0 & 0
                    }
                .\end{equation*}
            \end{proposal}

            \begin{proof}
                Пусть $e_{r + 1}, \dots, e_n$ --- базис $\ker \phi$. Дополним его векторами $e_1, \dots, e_r$ до базиса всего $V$.

                Положим $f_1 = \phi(e_1)$, $\dots$, $f_r = \phi(e_r)$, тогда $(f_1, \dots, f_r)$ --- базис $\Im \phi$.

                Дополним $f_1, \dots, f_r$ до базиса $f_1, \dots, f_m$ всего $W$.

                Тогда, $\E = (e_1, \dots, e_n)$ и $\F = (f_1, \dots, f_m)$ --- искомые базисы.
            \end{proof}

    \end{colloq}

    \subsection{Линейные, билинейные и квадратичные формы}
    \begin{colloq}
        \proofitem{Свойство базиса сопряжённого векторного пространства}

            $\epsilon_i (x_1 e_1 + \dots + x_n e_n) = x_i$, поэтому $\epsilon_i$ называется $i$-й координатной функцией в базисе $\E$.

            \begin{proposal}
                Всякий базис пространства $V^*$ двойствен некоторому базису пространства $V$. 
            \end{proposal}

            \begin{proof}
                Пусть $\epsilon = \begin{pmatrix} \epsilon_1 \\ \dots \\ \epsilon_n \end{pmatrix}$ --- базис пространства $V^*$.
                Фиксируем какой-то базис $\E' = (e'_1, \dots, e'_n)$ пространства $V$, и пусть $\epsilon' = \begin{pmatrix} \epsilon'_1 \\ \dots \\ \epsilon'_n \end{pmatrix}$ --- соответствующий ему двойственный базис $V^*$.

                Тогда, 
                $\begin{pmatrix} \epsilon_1 \\ \dots \\ \epsilon_n \end{pmatrix} = C \cdot \begin{pmatrix} \epsilon'_1 \\ \dots \\ \epsilon'_n \end{pmatrix}$
                для некоторой матрицы $C \in M^0_n(F)$.

                \bigskip
                Положим $(e_1, \dots, e_n) = (e'_1, \dots, e'_n) \cdot C^{-1}$.

                Тогда, 
                \begin{equation*}
                    \begin{pmatrix} \epsilon_1 \\ \dots \\ \epsilon_n \end{pmatrix} (e_1, \dots, e_n) = C \begin{pmatrix} \epsilon'_1 \\ \dots \\ \epsilon'_n \end{pmatrix} (e'_1, \dots, e'_n) C^{-1} = C E C^{-1} = E
                .\end{equation*}

                Значит, $\epsilon$ двойствен к $\E$.
            \end{proof}

            \begin{exercise}
                $\E$ определён однозначно.
            \end{exercise}


        \proofitem{Формула для вычисления значений билинейной формы в координатах}

            Пусть
            \begin{math}
                \begin{aligned}[t]
                    x &= x_1 e_1 + \dots + x_n e_n, \\
                    y &= y_1 e_1 + \dots + y_n e_n.
                \end{aligned}
            \end{math}

            Тогда,
            \begin{align*}
                \beta(x, y)
                &= \beta\left(\sum_{i = 1}^{n} x_i e_i, \ \sum_{j = 1}^{n} y_j e_j\right)
                = \sum_{i = 1}^{n} x_i \cdot \beta\left(e_i, \ \sum_{j = 1}^{n} y_j e_j\right) \\
                &= \sum_{i = 1}^{n} \sum_{j = 1}^{n} x_i y_j \underbracket{\beta(e_i, e_j)}_{\beta_{ij}}
                = \sum_{i = 1}^{n} \sum_{j = 1}^{n} x_i \beta_{ij} y_j  \\
                &= (x_1, \dots, x_n) B \begin{pmatrix} y_1 \\ \dots \\ y_n \end{pmatrix}
            .\end{align*}


        \proofitem{Существование и единственность билинейной формы с заданной матрицей}

            \begin{proposal}
                Пусть $\E$ --- фиксированный базис $V$.
                \begin{enumerate}
                \item Всякая билинейная форма $\beta$ на $V$ однозначно определяется матрицей $B(\beta, \E)$.
                \item $\forall B \in M_n(F) \ \exists!$ билинейная форма $\beta$ на $V$, такая что $B(\beta, \E) = B$.
                \end{enumerate}
            \end{proposal}

            \begin{proof}~
                \begin{enumerate}
                \item Следует из формулы выше.
                \item Единственность следует из формулы выше. Докажем существование:

                    Определим $\beta$ по формуле выше.

                    Тогда $\beta$ --- билинейная форма на $V$ (упражнение).

                    \begin{equation*}
                        \beta(e_i, e_j) = \bordermatrix{
                              &   &   &   & i &   &   &  \cr
                              & 0 & \dots & 0 & 1 & 0 & \dots & 0
                        } \cdot B \cdot \begin{pmatrix} 
                            0 \\ \dots \\ 0 \\ 1 \\ 0 \\ \dots \\ 0
                        \end{pmatrix} \ j = \beta_{ij}
                    .\end{equation*}

                    Действительно, $B(\beta, \E) = B$.
                    \qedhere
                \end{enumerate}
            \end{proof}


        \proofitem{Формула изменения матрицы билинейной формы при переходе к другому базису}

            $B = B(\beta, \E)$.

            Пусть $\E' = (e'_1, \dots e'_n)$ --- другой базис $V$.

            $\E' = \E \cdot C$.

            $B' := B(\beta, \E')$.

            \begin{proposal}
                $B' = C^{T} B C$.
            \end{proposal}

            \begin{proof}
                \begin{equation*}
                    x = x_1 e_1 + \dots + x_n e_n = x'_1 e'_1 + \dots + x'_n e'_n
                ,\end{equation*}

                \begin{equation*}
                    y = y_1 e_1 + \dots + y_n e_n = y'_1 e'_1 + \dots + y'_n e'_n.
                ,\end{equation*}

                \begin{equation*}
                    \begin{pmatrix} x_1 \\ \dots \\ x_n \end{pmatrix} = C \cdot \begin{pmatrix} x'_1 \\ \dots \\ x'_n \end{pmatrix} \quad \begin{pmatrix} y_1 \\ \dots \\ y_n \end{pmatrix} = C \cdot \begin{pmatrix} y'_1 \\ \dots \\ y'_n \end{pmatrix}
                .\end{equation*}

                Тогда,
                \begin{align*}
                    \beta(x, y) &= (x_1 \dots x_n) B \begin{pmatrix} y_1 \\ \dots \\ y_n \end{pmatrix} = (x'_1 \dots x'_n) C^{T} B C \begin{pmatrix} y'_1 \\ \dots \\ y'_n \end{pmatrix} \\
                    \beta(x, y) &= (x'_1 \dots x'_n) B' \begin{pmatrix} y'_1 \\ \dots \\ y'_n \end{pmatrix}
                .\end{align*}

                Получаем, что $B' = C^{T} B C$.
            \end{proof}


        \proofitem{Критерий симметричности билинейной формы в терминах её матрицы в каком-либо базисе}

            Пусть $\E$ --- произвольный базис $V$.

            \begin{proposal}
                $\beta$ симметрична $\iff B = B^{T}$.
            \end{proposal}

            \begin{proof}~
                \begin{description}
                \item[$\implies$] $b_{ji} = \beta(e_j, e_i) = \beta(e_i, e_j) = b_{ij}$, 
                \item[$\impliedby$] 
                    \begin{math}
                        \begin{aligned}[t]
                            x &= x_1 e_1 + \dots + x_n e_n, \\
                            y &= y_1 e_1 + \dots + y_n e_n,
                        \end{aligned}
                    \end{math}

                    Тогда, 
                    \begin{equation*}
                        \beta(y, x) = (y_1 \dots y_n) B \begin{pmatrix} x_1 \\ \dots \\ x_n \end{pmatrix} = (x_1 \dots x_n) B^{T} \begin{pmatrix} y_1 \\ \dots \\ y_n \end{pmatrix} = (x_1 \dots x_n) B \begin{pmatrix} y_1 \\ \dots \\ y_n \end{pmatrix} = \beta(x, y)
                    .\end{equation*}
                \end{description}
            \end{proof}


        \proofitem{Соответствие между симметричными билинейными формами и квадратичными формами}

            \begin{proposal}
                Пусть в поле $F$ выполнено условие $1 + 1 \neq 0$ (то есть $2 \neq 0$). Тогда отображение $\beta \mapsto Q_\beta$ является биекцией между симметричными билинейными формами на $V$ и квадратичным формами на $V$.
            \end{proposal}


            \begin{proof}~
                \begin{description}
                \item[Сюръективность] $Q$ --- квадратичная форма $ \implies Q = Q_\beta$ для некоторой билинейной формы на $V$.

                    То есть $Q(x) = \beta(x, x) \ \forall x \in V$.

                    Положим $\sigma(x, y) = \frac{1}{2} \left[\beta(x, y) + \beta(y, x)\right]$, тогда $\sigma$ --- симметричная билинейная форма.
                    \begin{equation*}
                        \sigma(x, x) = \frac{1}{2}\left[\beta(x, x) + \beta(x, x)\right] = \beta(x, x)
                    .\end{equation*}

                \item[Инъективность] $\beta$ --- симметричная билинейная форма на $V$.

                    $Q_\beta(x + y) = \beta(x + y, x + y) = \underbrace{\beta(x, x)}_{Q_\beta(x)} + \underbrace{\beta(x, y) + \beta(y, x)}_{\text{равны между собой}} + \underbrace{\beta(y, y)}_{Q_\beta(y)} \implies \beta(x, y) = \frac{1}{2} \left[Q_\beta(x + y) - Q_\beta(x) - Q_\beta(y)\right]$.
                \end{description}
            \end{proof}


        \proofitem{Метод Лагранжа приведения квадратичной формы к каноническому виду}
        \proofitem{Метод Якоби приведения квадратичной формы к каноническому виду}
        \proofitem{Существование нормального вида для квадратичной формы над $\RR$}

            \begin{corollary}[из метода Лагранжа]
                Для всякой квадратичной формы $Q$ над $\RR$ существует базис, в котором $Q$ имеет нормальный вид.
            \end{corollary}

            \begin{proof}
                Из теоремы знаем, что существует базис $\E$, в котором $Q$ имеет канонический вид
                \begin{equation*}
                    Q(x_1, \dots, x_n) = b_1 x_1^2 + \dots + b_n x_n^2
                .\end{equation*}

                Делаем замену
                \begin{equation*}
                    x_i = \begin{cases}
                        \frac{x'_i}{\sqrt{|b_i|}}, &b_i \neq 0 \\
                        x_i, &b_i = 0
                    \end{cases}
                .\end{equation*}

                Тогда в новых координатах $Q(x'_1, \dots, x'_n) = \epsilon_1 x'^2_1 + \dots + \epsilon_n x'^2_n$,

                \begin{equation*}
                    \epsilon_i = \sgn b_i = \begin{cases}
                        1, &b_i > 0 \\
                        0, &b_i = 0 \\
                        -1, &b_i < 0
                    \end{cases}
                .\end{equation*}
            \end{proof}


        \proofitem{Закон инерции}

            Пусть $F = \RR$.

            Пусть $Q \colon V \to \RR$ --- квадратичная форма.

            Можно привести к нормальному виду
            \begin{equation*}
                Q(x_1, \dots, x_n) = x_1^2 + \dots + x_s^2 - x_{s + 1}^2 - \dots - x_{s + t}^2
            .\end{equation*}


            \begin{theorem}
                Числа $i_+$ и $i_-$ не зависят от базиса в котором $Q$ принимает нормальный вид.
            \end{theorem}

            \begin{proof}
                $s + t = \rk Q$, то есть не зависит от выбора базиса. Следовательно, достаточно показать, что число $s$ определено однозначно.

                Пусть $\E = (e_1, \dots, e_n)$ --- базис, в котором $Q$ принимает нормальный вид
                \begin{equation*}
                    Q = x_1^2 + \dots + x_s ^2 - x_{s + 1}^2 - \dots - x_{s + t}^2
                .\end{equation*}

                Пусть $\E' = (e'_1, \dots, e'_n)$ --- другой базис, в котором $Q$ принимает нормальный вид
                \begin{equation*}
                    Q = x'^2_1 + \dots + x'^2_{s'} - x'^2_{s' + 1} - \dots - x'^2_{s' + t'}
                .\end{equation*}

                Предположим, что $s \neq s'$, можно считать что $s > s'$.

                Положим 
                \begin{math}
                    \begin{aligned}[t]
                        &L := \left< e_1, \dots, e_s \right>, \ \dim L = s, \\
                        &L' := \left< e'_{s' + 1}, \dots, e'_n \right>, \ \dim L' = n - s'.
                    \end{aligned}
                \end{math}

                \bigskip
                Так как $L + L' \subseteq V$, то $\dim \left(L + L'\right) \leq n$.

                Тогда, $\dim \left(L \cap L'\right) \geq s + (n - s') - n = s - s' > 0$.

                \bigskip
                Значит, $\exists$ вектор $v \in L \cap L'$, такой что $v \neq 0$.

                Теперь: 
                \begin{enumerate}[nosep]
                    \item Так как $v \in L$, то $Q(v) > 0$,
                    \item Так как $v \in L'$, то $Q(v) \leq 0$.
                \end{enumerate}
             
                Противоречие.
            \end{proof}


        \proofitem{Следствие метода Якоби о нахождении индексов инерции квадратичной формы над $\RR$}

            Пусть $Q \colon V \to \RR$ --- квадратичная форма,

            $\E = (e_1, \dots, e_n)$ --- базис,

            $B = B(Q, \E)$, 

            $\delta_k$ --- $k$-й угловой минор матрицы $B$.


            \begin{corollary}[из метода Якоби]
                Пусть $\delta_k \neq 0 \ \forall k$. Тогда:

                Число $i_+$ равно количеству \underline{сохранений знака} в последовательности $1, \delta_1, \dots, \delta_n$.

                Число $i_-$ равно количеству \underline{перемен знака} в последовательности $1, \delta_1, \dots, \delta_n$.
            \end{corollary}

            \begin{proof}
                Метод Якоби $ \implies \exists $ базис, в котором $Q$ принимает канонический вид
                \begin{equation*}
                    Q = \delta_1 x_1^2 + \frac{\delta_2}{\delta_1} x_2^2 + \dots + \frac{\delta_n}{\delta_{n - 1}} x_{n}^2
                .\end{equation*}

                Здесь, знак отношения $\frac{\delta_i}{\delta_{i - 1}}$ соответствует смене либо сохранению знака в рассматриваемой последовательности.

                По закону инерции, количества знаков $+$ и $-$ не изменяются от выбора базиса.
            \end{proof}


        \proofitem{Критерий Сильвестра положительной определённости квадратичной формы, критерий отрицательной определёности квадратичной формы}

            Пусть 
            \begin{math}
                \begin{aligned}[t]
                    &V \text{ --- векторное пространство над $\RR$, $\dim V = n$}, \\
                    &\E = (e_1, \dots, e_n) \text{ --- базис $V$}, \\
                    &B = B(Q, \E), \\
                    &B_k \text{ --- левый верхний $k \times k$ блок}, \\
                    &\delta_k = \det B_k.
                \end{aligned}
            \end{math}

            \begin{theorem}[Критерий Сильвестра положительной определенности]
                \begin{equation*}
                    Q > 0 \iff \delta_k > 0 \ \forall k = 1 \dots n
                .\end{equation*}
            \end{theorem}

            \begin{proof}~
                \begin{description}
                \item[$\impliedby$] По следствию из метода Якоби, $i_+ = n$, то есть $Q > 0$.
                \item[$\implies$] $Q > 0 \implies \exists C \in M_n^0(\RR)$, такая что $C^TBC = E$.

                    Тогда, $\det C^T \cdot \underset{= \delta_n}{\det B} \cdot \det C = 1$. Отсюда, $\delta_n = \frac{1}{(\det C)^2} > 0$.

                    Теперь, для любого $k$ ограничение $Q$ на $\left< e_1, \dots, e_k \right>$ тоже положительно определённо, а его матрица в базисе $e_1, \dots, e_k$ равна $B_k$. Следовательно, $\det B_k > 0$.
                    \qedhere
                \end{description}
            \end{proof}

    \end{colloq}

    \subsection{Евклидовы пространства}
    \begin{colloq}
        \proofitem{Неравенство Коши--Буняковского}

            \begin{proposal}[неравенство Коши-Буняковского]
                $\forall x, y \in \EE$ верно $|(x, y)| \leq |x| \cdot |y|$, причём равенство $\iff$ $x$, $y$ пропорциональны.
            \end{proposal}

            \begin{proof}
                Случаи:
                \begin{enumerate}
                \item $x, y$ пропорциональны. Тогда, можно считать, что $y = \lambda x$, $\lambda \in \RR$.

                    $|(x, y)| = |(x, \lambda x)| = |\lambda| |(x, x)| = |\lambda| |x|^2 = |x| \cdot |\lambda x| = |x| \cdot |y|$.

                \item $x, y$ не пропорциональны. Тогда $x, y$ линейно независимы.

                    Значит они образуют базис в $\left< x, y \right>$.

                    Получаем
                    \begin{equation*}
                        \begin{vmatrix} 
                            (x, x) & (x, y) \\
                            (y, x) & (y, y)
                        \end{vmatrix} > 0 \quad \text{(критерий Сильвестра)}
                    .\end{equation*}

                    Отсюда, $(x, x) \cdot (y, y) - (x, y)^2 > 0 \implies (x, y)^2 < |x|^2 \cdot |y|^2$.
                \end{enumerate}
            \end{proof}

            \begin{example}
                Пусть $\EE = \RR^n$ со стандартным скалярным произведением, тогда
                \begin{equation*}
                    |x_1 y_1 + \dots + x_n y_n| \leq \sqrt{x_1^2 + \dots + x_n^2} \cdot \sqrt{y_1^2 + \dots + y_n^2}
                .\end{equation*}
            \end{example}


        \proofitem{Свойства определителя матрицы Грама системы векторов евклидова пространства}

            \begin{proposal}
                $\forall v_1, \dots, v_k \in \EE \implies \det G(v_1, \dots, v_k) \geq 0$.

                Более того, $\det G(v_1, \dots, v_k) > 0 \iff v_1, \dots, v_k$ линейно независимы. 
            \end{proposal}

            \begin{proof}
                Пусть $G := G(v_1, \dots, v_k)$.
                Случаи:
                \begin{enumerate}
                \item $v_1, \dots, v_k$ линейно независимы. Тогда, $G$ --- матрица билинейной формы $(\bigcdot, \bigcdot) \Big|_{\left< v_1, \dots, v_k \right>}$ в базисе $v_1, \dots, v_k$ подпространства $\left< v_1, \dots, v_k \right>$, а значит $\det G > 0$ по критерию Сильвестра.
                
                \item $v_1, \dots, v_k$ линейно зависимы. Тогда, $\exists (\alpha_1, \dots, \alpha_k) \in \RR^k \setminus \{0\}$, такие что $\alpha_1 v_1 + \dots + \alpha_k v_k = 0$.

                    А значит,  $\forall i = 1, \dots, k \implies \alpha_1 (v_1, v_i) + \dots + \alpha_k (v_k, v_i) = 0$.
                    
                    Отсюда, $a_1 G_{(1)} + \dots + \alpha_k G_{(k)} = 0 \implies$ строки в $G$ линейно зависимы $\implies \det G = 0$.
                    \qedhere
                \end{enumerate}
            \end{proof}


        \proofitem{Свойства ортогонального дополнения к подпространству в евклидовом пространстве: размерность, разложение в прямую сумму, ортогональное дополнение к ортогональному дополнению}

            Далее считаем, что $\dim \EE = n < \infty$.

            \begin{proposal}
                Пусть $S \subseteq \EE$ --- подпространство.
                Тогда:
                \begin{enumerate}
                \item $\dim S^{\perp} = n - \dim S$.
                \item $\EE = S \oplus S^{\perp}$.
                \item $(S^{\perp})^{\perp} = S$.
                \end{enumerate}
            \end{proposal}

            \begin{proof}~
                \begin{enumerate}
                \item 
                    Пусть $\dim S = k$ и $e_1, \dots, e_k$ --- базис $S$.
                    
                    Дополним $e_1, \dots, e_k$ до базиса $e_1, \dots, e_n$ всего $\EE$.

                    Тогда, $\forall x = x_1 e_1 + \dots + x_n e_n \in \EE$.

                    \begin{align*}
                        x \in S^{\perp} &\iff (x, e_i) = 0 \ \forall i = 1, \dots, k \\
                                        &\iff \begin{cases}
                                            (e_1, e_1) x_1 + \dots + (e_n, e_1) x_n = 0 \\
                                            (e_1, e_2) x_1 + \dots + (e_n, e_2) x_n = 0 \\
                                            \dots \\
                                            (e_1, e_k) x_1 + \dots + (e_n, e_k) x_n = 0
                                        \end{cases}
                    .\end{align*}

                    Это ОСЛУ с матрицей $G \in \text{Mat}_{k \times n}(\RR)$, причём левый $k \times k$ блок в $G$ --- это $\underbrace{G(e_1, \dots, e_k)}_{\det \neq 0}$.

                    Это означает, что $\rk G = k$.

                    Следовательно, пространство решений этой ОСЛУ имеет размерность $n - k$.

                    Отсюда, $\dim S^{\perp} = n - k = n - \dim S$.

                \item
                    \begin{enumerate}
                    \item $\dim S + \dim S^{\perp} = k + (n - k) = n = \dim E$.
                    \item $v \in S \cap S^{\perp} \implies (v, v) = 0 \implies v = 0 \implies S \cap S^{\perp} = \{0\}$.
                    \end{enumerate}

                    А значит, $E = S \oplus S^{\perp}$.

                \item
                    Заметим, что $S \subseteq (S^{\perp})^{\perp}$ (по определению).

                    $\dim (S^{\perp})^{\perp} = n - \dim S^{\perp} = n - (n - \dim S) = \dim S$.

                    Следовательно, $S = (S^{\perp})^{\perp}$.
                    \qedhere
                \end{enumerate}
            \end{proof}


        \proofitem{Формула для ортогональной проекции вектора на подпространство в $\RR^n$ в терминах его произвольного базиса}

            Пусть $\EE = \RR^n$ со стандартным скалярным произведением.

            $S \subseteq \EE$ --- подпространство, $a_1, \dots, a_k$ --- базис $S$.

            Пусть $A := (a_1, \dots, a_k) \in \text{Mat}_{n \times k}(\RR)$, $A^{(i)} = a_i$.

            \begin{proposal}
                $\forall v \in \RR^n \quad \pr_S v = A (A^{T} A)^{-1} A^{T} v$.
            \end{proposal}

            \begin{proof}
                Корректность: $A^{T} A = G(a_1, \dots, a_k) \in M_k^{0}(\RR)$.

                Положим $x := \pr_S v$, $y := \ort_S v$.

                Так как $x \in S$, $x = A \cdot \begin{pmatrix} \alpha_1 \\ \dots \\ \alpha_k \end{pmatrix}$, $\alpha_i \in \RR$.

                $y \in S^{\perp} \implies A^T y = 0$.

                \begin{align*}
                A (A^{T} A)^{-1} A^{T} v
                &= A(A^{T}A)^{-1}A^{T} (x + y) \\
                &= A\lefteqn{\underbracket{\phantom{(A^{T}A)^{-1}A^{T} A}}_E} (A^{T} A)^{-1} A^{T} \overbracket{A \begin{pmatrix} \alpha_1 \\ \dots \\ \alpha_k \end{pmatrix}}^{x} + A(A^{T} A)^{-1} \underbracket{A^{T} y}_{0} \\
                &= A \begin{pmatrix} \alpha_1 \\ \dots \\ \alpha_k \end{pmatrix} = x = \pr_S v
                .\end{align*}
            \end{proof}


        \proofitem{Существование ортонормированного базиса в евклидовом пространстве, дополнение ортогональной (ортонормированной) системы векторов до ортогонального (ортонормированного) базиса}

            \begin{theorem}
                Во всяком конечномерном евклидовом пространстве существует ортонормированный базис.
            \end{theorem}

            \begin{proof}
                Следует из теоремы о приведении квадратичной формы $(x, x)$ к нормальному виду (который будет $E$ в силу положительной определённости).
            \end{proof}

            \begin{corollary}
                Всякую ортогональную (ортонормированную) систему векторов можно дополнить до ортогонального (ортонормированного) базиса.
            \end{corollary}

            \begin{proof}
                Пусть $e_1, \dots, e_k$ --- данная система. 

                Пусть $e_{k + 1}, \dots, e_{n}$ --- это ортогональный (ортонормированный) базис в $\left< e_1, .., e_k \right>^{\perp}$.

                Тогда $e_1, \dots, e_n$ --- искомый базис.
            \end{proof}


        \proofitem{Описание всех ортонормированных базисов в терминах одного и матриц перехода}

            Пусть $\E = (e_1, \dots, e_n)$ --- ортонормированный базис в $E$.

            Пусть $\E' = (e'_1, \dots, e'_n)$ --- какой-то другой базис.

            $(e'_1, \dots, e'_n) = (e_1, \dots, e_n) \cdot C$, $C \in M_n^{0}(\RR)$.

            \begin{proposal}
                $\E'$ --- ортонормированный базис $\iff C^{T} \cdot C = E$.
            \end{proposal}

            \begin{proof}
                $G(e'_1, \dots, e'_n) = C^{T} \underbracket{G(e_1, \dots, e_n)}_E C = C^{T} C$.

                $\E'$ ортонормированный $\iff G(e'_1, \dots, e'_n) = E \iff C^{T} C = E$.
            \end{proof}


        \proofitem{Формула для координат вектора в ортогональном (ортонормированном) базисе. Формула для ортогональной проекции вектора на подпространство в терминах его ортогонального (ортонормированного) базиса}

            Пусть $\EE$ --- евклидово пространство, $(e_1, \dots, e_n)$ --- ортогональный базис, $v \in \EE$.

            \begin{proposal}
                $v = \dfrac{(v, e_1)}{(e_1, e_1)}e_1 + \dfrac{(v, e_2)}{(e_2, e_2)}e_2 + \dots + \dfrac{(v, e_n)}{(e_n, e_n)}e_n$.

                В частности, если $e_1, \dots, e_n$ ортонормирован, то $v = (v, e_1)e_1 + \dots + (v, e_n) e_n$.
            \end{proposal}

            \begin{proof}
                $v = \lambda_1 e_1 + \lambda_2 e_2 + \dots \lambda_n e_n$.

                $\forall i = 1, \dots, n \quad (v, e_i) = \lambda_1 (e_1, e_i) + \dots + \lambda_n (e_n, e_i)$.

                Так как базис ортогонален, то $(v, e_i) = \lambda_i (e_i, e_i) \implies \lambda_i = \dfrac{(v, e_i)}{(e_i, e_i)}$.
            \end{proof}

            \bigskip
            Пусть $S \subseteq \EE$ --- подпространство.

            $e_1, \dots, e_k$ --- ортогональный базис в $S$.

            \begin{proposal}
                $\forall v \in \EE \quad \pr_S v = \sum_{i = 1}^{k} \dfrac{(v, e_i)}{(e_i, e_i)} e_i$.

                В частности, если $e_1, \dots, e_k$ ортонормирован, то $\pr_S v = \sum_{i = 1}^{k} (v, e_i) e_i$.
            \end{proposal}

            \begin{proof}
                Пусть $e_{k + 1}, \dots, e_n$ --- ортогональный базис в $S^{\perp}$. Тогда $e_1, \dots, e_n$ --- ортогональный базис в $\EE$.

                \begin{equation*}
                    v = \underbrace{\sum_{i = 1}^{k} \dfrac{(v, e_i)}{(e_i, e_i)} e_i}_{\in S} + \underbrace{\sum_{i = k + 1}^{n} \dfrac{(v, e_i)}{(e_i, e_i)} e_i}_{\in S^{\perp}}
                .\end{equation*}

                Отсюда,
                \begin{equation*}
                    \pr_S v = \sum_{i = 1}^{k} \dfrac{(v, e_i)}{(e_i, e_i)}
                .\qedhere\end{equation*}
            \end{proof}


        \proofitem{Теорема Пифагора и неравенство треугольника в евклидовом пространстве}

            \begin{theorem}
                Пусть $x, y \in \EE$, $(x, y) = 0$. Тогда $|x + y|^2 = |x|^2 + |y|^2$.
            \end{theorem}

            \begin{proof}
                \begin{equation*}
                    |x + y|^2 = (x + y, x + y) = \underbrace{(x, x)}_{|x|^2} + \underbrace{(x, y)}_{0} + \underbrace{(y, x)}_{0} + \underbrace{(y, y)}_{|y|^2} = |x|^2 + |y|^2
                .\qedhere\end{equation*}
            \end{proof}


        \proofitem{Теорема о расстоянии между вектором и подпространством в терминах ортогональной составляющей}

            \begin{theorem}
                Пусть $x \in \EE$, $S \subseteq \EE$ --- подпространство. Тогда, $\rho(x, S) = \left|\ort_S x\right|$, причем $\pr_S x$ --- это ближайший к $x$ вектор из $S$.
            \end{theorem}

            \begin{proof}
                Положим $y = \pr_S x$, $z = \ort_S x$. Тогда, $x = y + z$.
                Для любого $y' \in S$, $y' \neq 0$ имеем
                \begin{equation*}
                    \rho(x, y + y')^2 = |x - y - y'|^2 = |z - y'|^2 = |z|^2 + |y'|^2 > |z|^2 = |x - y|^2 = \rho(x, y)^2
                .\qedhere\end{equation*}
            \end{proof}


        \proofitem{Метод наименьших квадратов для несовместных систем линейных уравнений: постановка задачи и её решение. Единственность псевдорешения и явная формула для него в случае линейной независимости столбцов матрицы коэффициентов}

            СЛУ $Ax = b$, $A \in \text{Mat}_{m \times n}(\RR)$, $x \in \RR^n$, $b \in \RR^m$.
            \begin{equation*}
                x_0 \text{ --- решение системы} \iff Ax_0 = b \iff Ax_0 - b = 0 \iff |Ax_0 - b| = 0 \iff \rho(Ax_0, b) = 0
            .\end{equation*}

            Если СЛУ несовместна, то $x_0$ называется \textit{псевдорешением}, если $\rho(Ax_0, b)$ минимально.

            \begin{equation*}
                \rho(Ax_0, b) = \min_{x \in R^n} \rho(Ax, b)
            .\end{equation*}

            $x_0$ --- решение задачи оптимизации $\rho(Ax, b) \xrightarrow[x \in \RR^n]{} \min$.

            Пусть $S \subseteq \RR^n$ --- подпространство натянутое на столбцы матрицы $A$.

            $S = \left< A^{(1)}, \dots, A^{(n)} \right>$.

            Положим $c := \pr_S b$.

            \begin{proposal}~
                \begin{enumerate}
                \item $x_0$ --- псевдорешение $Ax = b \iff x_0$ --- решение для $Ax = c$.  
                \item Если столбцы $A^{(1)}, \dots, A^{(n)}$ линейно независимы, то псевдорешение единственно и может быть найдено по формуле $x_0 = (A^{T} A)^{-1} A^{T} b$.
                \end{enumerate}
            \end{proposal}

            \begin{proof}~
                \begin{enumerate}
                    \item
                        \begin{equation*}
                            \forall x \in \RR^n \quad Ax = x_1 A^{(1)} + \dots + x_n A^{(n)} \implies \{Ax \mid x \in \RR^n\} = S \implies \min_{x \in \RR^n} \rho(Ax, b) = \rho(S, b)
                        .\end{equation*}
                        По теореме о расстоянии от вектора до подпространства минимум достигается при $Ax = c = \pr_S b$.

                    \item
                        Так как $A^{(1)}, \dots, A^{(n)}$ линейно независимы, то $c$ единственным образом представим в виде линейной комбинации этих столбцов.

                        Следовательно, $x_0$ единственно.

                        Знаем, что $A \underbrace{(A^{T} A)^{-1} A^{T} b} = c$. Значит, $x_0 = (A^{T} A)^{-1} A^{T} b$.
                        \qedhere
                \end{enumerate}
            \end{proof}


        \proofitem{Формула для расстояния между вектором и подпространством в терминах матриц Грама}

            Пусть $\EE$ --- евклидово пространство, $\dim \EE = n < \infty$.

            $S \subseteq \EE$ --- подпространство, $e_1, \dots, e_k$ --- базис в $S$.

            \begin{theorem}
                $\forall x \in \EE \quad \rho(x, S)^2 = \dfrac{\det G(e_1, \dots, e_k, x)}{\det G(e_1, \dots, e_k)}$.
            \end{theorem}

            \begin{proof}
                Пусть $z := \ort_S x$, тогда $\rho(x, S)^2 = |z|^2$.

                \begin{enumerate}
                \item $x \in S \implies \rho(x, S) = 0$:

                    так как $e_1, \dots, e_k$ линейно независимы, то $\det G(e_1, \dots, e_k, x) = 0$.

                \item $x \not\in S$.

                    Ортогонализация Грама-Шмидта: $e_1, \dots, e_k, x \leadsto f_1, \dots, f_k, z$.

                    По свойству \ref{lec23:heart} получаем
                    \begin{equation*}
                        \dfrac{\det G(e_1, \dots, e_k, x)}{\det G(e_1, \dots, e_k)} = \dfrac{\det G(f_1, \dots, f_k, z)}{\det G(f_1, \dots, f_k)} = \frac{|f_1|^2 \dots |f_k|^2 |z|^2}{|f_1|^2 \dots |f_k|^2} = |z|^2 = \rho(x, S)^2
                    .\end{equation*}
                \end{enumerate}
            \end{proof}


        \proofitem{Две формулы для объёма $k$-мерного параллелепипеда в евклидовом пространстве}

            \begin{theorem}
                $\vol P(a_1, \dots, a_k)^2 = \det G(a_1, \dots, a_k)$.
            \end{theorem}

            \begin{proof}
                Индукция по $k$:

                \begin{description}
                \item[$k = 1:$] $|a_1|^2 = (a_1, a_1)$ --- верно.
                \item[$k > 1:$] $\vol P(a_1, \dots, a_k)^2 = \vol P(a_1, \dots, a_{k - 1})^2 \cdot h^2 = \det G(a_1, \dots, a_{k - 1}) \cdot h^2 = (\star)$.

                    Если $a_1, \dots, a_{k - 1}$ линейно независимы, то $h^2 = \dfrac{\det G(a_1, \dots, a_k)}{\det G(a_1, \dots, a_{k - 1})}$.
                    Тогда, $(\star) = \det G(a_1, \dots, a_k)$.

                    Если же $a_1, \dots, a_{k - 1}$ линейно зависимы, то $\det G(a_1, \dots, a_{k - 1}) = 0 \implies (\star) = 0$. Но $a_1, \dots, a_k$ тоже линейно зависимы, а значит $\det G(a_1, \dots, a_k) = 0$.
                    \qedhere
                \end{description}
            \end{proof}

            \bigskip
            Пусть $(e_1, \dots, e_n)$ --- ортонормированный базис в $\EE$,

            $(a_1, \dots, a_n) = (e_1, \dots, e_n) \cdot A$, $A \in M_n(\RR)$.

            \begin{proposal}
                $\vol P(a_1, \dots, a_n) = \left|\det A\right|$.
            \end{proposal}

            \begin{proof}
                \begin{equation*}
                    G(a_1, \dots, a_n) = A^{T} \cdot A \implies \vol P(a_1, \dots, a_n)^2 = \det (A^{T} A) = \left(\det A\right)^2
                    \qedhere
                .\end{equation*}
            \end{proof}

    \end{colloq}

    \subsection{Элементы аналитической геометрии и линейные многообразия}
    \begin{colloq}
        \proofitem{Теорема о векторном произведении и формуле для него в координатах в положительно ориентированном ортонормированном базисе}

            \begin{lemma}
                Пусть $v_1, v_2 \in \EE$. Тогда, $(v_1, x) = (v_2, x) \ \forall x \in \EE \implies v_1 = v_2$.
            \end{lemma}

            \begin{proof}
                Имеем $(v_1 - v_2, x) = 0 \ \forall x \in \EE$.
                Тогда, $v_1 - v_2 \in \EE^{\perp} = \{0\} \implies v_1 - v_2 = 0 \implies v_1 = v_2$.
            \end{proof}

            \begin{theorem}
                \label{lec25:t}
                Пусть $a, b \in \RR^3$. Тогда
                \begin{enumerate}
                \item $\exists! v \in \EE$, такой что $(v, x) = \Vol(a, b, x) \quad \forall x \in \RR^3$.
                \item Если $\E = (e_1, e_2, e_3)$ --- положительно ориентированный ортонормированный базис и
                    \begin{math}
                        \ \begin{aligned}[t]
                            a &= a_1 e_1 + a_2 e_2 + a_3 e_3 \\
                            b &= b_1 e_1 + b_2 e_2 + b_3 e_3
                        \end{aligned},
                    \end{math}
                    то 
                    \begin{equation}
                        \tag{$\star$}
                        \label{lec25:v}
                        v = \begin{vmatrix}
                            e_1 & e_2 & e_3 \\
                            a_1 & a_2 & a_3 \\
                            b_1 & b_2 & b_3
                        \end{vmatrix}
                        := \begin{vmatrix} 
                            a_2 & a_3 \\
                            b_2 & b_3
                        \end{vmatrix} e_1 - \begin{vmatrix} 
                            a_1 & a_3 \\
                            b_1 & b_3
                        \end{vmatrix} e_2 + \begin{vmatrix} 
                            a_1 & a_2 \\
                            b_1 & b_2
                        \end{vmatrix} e_3
                    .\end{equation}
                \end{enumerate}
            \end{theorem}

            \begin{proof}~
                \begin{description}
                \item[Единственность] если $v'$ --- другой такой вектор, то $(v, x) = (v', x) \ \forall x \in \RR^3$, а значит $v' = v$ по лемме.
                \item[Существование] Покажем, что $v$, заданный формулой \eqref{lec25:v} подойдёт.
                    \begin{align*}
                        x = x_1 e_1 + x_2 e_2 + x_3 e_3 \implies (v, x) &= \begin{vmatrix} 
                            a_2 & a_3 \\
                            b_2 & b_3
                        \end{vmatrix} x_1 - \begin{vmatrix} 
                            a_1 & a_3 \\
                            b_1 & b_3
                        \end{vmatrix} x_2 + \begin{vmatrix} 
                            a_1 & a_2 \\
                            b_1 & b_2
                            \end{vmatrix} x_3 \\ &= \begin{vmatrix} 
                            x_1 & x_2 & x_3 \\
                            a_1 & a_2 & a_3 \\
                            b_1 & b_2 & b_3
                        \end{vmatrix} = \begin{vmatrix} 
                            a_1 & a_2 & a_3 \\
                            b_1 & b_2 & b_3 \\
                            x_1 & x_2 & x_3
                        \end{vmatrix} = \Vol(a, b, x)
                    .\end{align*}
                \end{description}
            \end{proof}


        \proofitem{Критерий коллинеарности двух векторов трёхмерного евклидова пространства}

            \begin{proposal}
                $a, b \in \EE$ коллинеарны $\iff [a, b] = 0$.
            \end{proposal}

            \begin{proof}~
                \begin{description}
                    \item[$\implies$] 
                        \begin{equation*}
                            (a, b, x) = 0 \ \forall x \implies ([a, b], x) = 0 \ \forall x \implies [a, b] = 0
                        .\end{equation*}

                    \item[$\impliedby$]
                        \begin{equation*}
                            [a, b] = 0 \implies ([a, b], x) = 0 \ \forall x \implies (a, b, x) = 0 \ \forall x \in \RR^3
                        .\end{equation*}

                        Если $a$, $b$ линейно независимы, то можно взять $x$, который дополняет их до базиса в $\RR^3$.

                        Тогда, $(a, b, x) \neq 0$ --- противоречие. Значит $a$, $b$ линейно зависимы $\implies$ коллинеарны.
                        \qedhere
                \end{description}
            \end{proof}


        \proofitem{Геометрические свойства векторного произведения}

            \begin{proposal}~
                \begin{enumerate}[nosep]
                \item $[a, b] \perp \left< a, b \right>$.
                \item $\left|[a, b]\right| = \vol P(a, b)$.
                \item $\Vol(a, b, [a, b]) \geq 0$.
                \end{enumerate}
            \end{proposal}

            \begin{proof}~
                \begin{enumerate}
                \item $([a, b], a) = (a, b, a) = 0 = (a, b, b) = ([a, b], b)$.
                \item Если $a$, $b$ коллинеарны, то обе части равны 0.

                    Пусть $[a, b] \neq 0$.
                    \begin{equation*}
                        \left|[a, b]\right|^2 = ([a, b], [a, b]) = (a, b, [a, b]) = (\#) > 0
                    .\end{equation*}
                    \begin{equation*}
                        [a, b] \perp \left< a, b \right> \implies (\#) = \vol P(a, b, [a, b]) = \Vol(a, b, [a, b]) = \vol P(a, b,) \cdot \left|[a, b]\right|
                    .\end{equation*}

                    Сокращая на $|[a, b]| \neq 0$, получаем требуемое.

                \item
                    $\Vol (a, b, [a, b]) = ([a, b], [a, b]) \geq 0$.
                    \qedhere
                \end{enumerate}
            \end{proof}


        \proofitem{Антикоммутативность и билинейность векторного произведения}

            \begin{proposal}~
                \begin{enumerate}
                \item $[a, b] = -[b, a] \quad \forall a, b$ (антикоммутативность).
                \item $[\bigcdot, \bigcdot]$ билинейно (то есть линейно по каждому аргументу).
                \end{enumerate}
            \end{proposal}

            \begin{proof}~
                \begin{enumerate}
                \item 
                    $([a, b], x) = (a, b, x) = -(b, a, x) = -([b, a], x) = (-[b, a], x) \quad \forall x \in \RR^3 \implies [a, b] = -[b, a]$

                \item
                    Пусть $u = [\lambda_1 a_1 + \lambda_2 a_2, b]$, $v = \lambda_1 [a_1, b] + \lambda_2 [a_2, b]$.
                    Тогда $\forall x \in \RR^3$:
                    \begin{align*}
                        (u, x) &= (\lambda_1 a_1 + \lambda_2 a_2, b, x) \\
                               &= \lambda_1 (a_1, b, x) + \lambda_2 (a_2, b, x) \\
                               &= \lambda_1 ([a_1, b], x) + \lambda_2([a_2, b], x) \\
                               &= (\lambda_1[a_1, b] + \lambda_2[a_2, b], x) = (v, x)
                    .\end{align*}

                    Значит $u = v$. Аналогично линейность по второму аргументу.
                    \qedhere
                \end{enumerate}
            \end{proof}


        \proofitem{Линейные многообразия как сдвиги подпространств}

            Пусть $Ax = b$ --- СЛУ, $\varnothing \neq L \subseteq \RR^n$ --- множество решений, $x_z \in L$ --- частное решение.

            Было: Лемма: $L = x_z + S$, где $S$ --- множество решений ОСЛУ $Ax = 0$.

            \begin{proposal}
                Множество $L \subseteq \RR^n$ является линейным многообразием $\iff L = v_0 + S$ для некоторых $v_0 \in \RR^n$ и подпространства $S \subseteq \RR^n$. 
            \end{proposal}

            \begin{proof}~
                \begin{description}
                \item[$\implies$] Из леммы.
                \item[$\impliedby$] $L = v_0 + S$. Значит существует ОСЛУ $Ax = 0$, для которой $S$ является множеством решений. Тогда, $L$ --- множество решений СЛУ $Ax = Av_0$ (по лемме).
                    \qedhere
                \end{description}
            \end{proof}


        \proofitem{Критерий равенства двух линейных многообразий}

            \begin{proposal}
                Пусть $L_1 = v_1 + S_1$ и $L_2 = v_2 + S_2$ --- два линейных многообразия в $\RR^n$. Тогда,
                \begin{equation*}
                    L_1 = L_2 \iff \begin{cases}
                        S_1 = S_2 \ (= S) \\
                        v_1 - v_2 \in S
                    \end{cases}
                .\end{equation*}
            \end{proposal}

            \begin{proof}~
                \begin{description}
                \item[$\impliedby$] 
                    $L_1 = v_1 + S_1 = v_1 + S_2 = v_2 + (v_1 - v_2) + S = v_2 + S = L_2$.
                \item[$\implies$]
                    $v_1 = v_1 + 0 \in L_1 = L_2 = v_2 + S_2 \implies v_1 - v_2 \in S_2$,

                    $v \in S_1 \implies v + v_1 \in L_1 = L_2 = v_2 + S_2 \implies v \in (v_2 - v_1) + S_2 = S_2 \implies S_1 \subseteq S_2$.

                    Аналогично, $v_1 - v_2 \in S_1$ и $S_2 \subseteq S_1$.
                    \qedhere
                \end{description}
            \end{proof}


        \proofitem{Теорема о плоскости, проходящей через $k+1$ точку в $\RR^n$}

            \begin{theorem}~
                \begin{enumerate}[label=\alph*)]
                \item Через любые $k + 1$ точек в $\RR^n$ проходит плоскость размерности $\leq k$.
                \item Если это точки не лежат в плоскости размерности $<k$, то через них проходит ровно одна плоскость размерности $k$.
                \end{enumerate}
            \end{theorem}

            \begin{proof}~
                \begin{enumerate}[label=\alph*)]
                    \item Пусть $v_0, v_1, \dots, v_k$ --- данные точки. Тогда через них проходит плоскость $P = v_0 + \left< v_1 - v_0, \dots, v_k - v_0 \right>$.

                        Ясно, что $\dim P \leq k$.

                    \item Из условия следует, что $\dim P = k \implies v_1 - v_0, \dots, v_k - v_0$ линейно независимы.

                        Если $P' = v_0 + S$ --- другая плоскость размерности $k$, содержащая $v_0, \dots, v_k$, то $v_1 - v_0, \dots, v_k - v_0 \in S \implies S = \left< v_1 - v_0, \dots, v_k - v_0 \right> \implies P' = P$.
                        \qedhere
                \end{enumerate}
            \end{proof}

    \end{colloq}


    \subsection{Линейные операторы}
    \begin{colloq}
        \proofitem{Критерий обратимости линейного оператора в терминах его ядра, образа и определителя}

            Пусть $\phi \in \L(V)$.

            \begin{proposal}
                Следующие условия эквивалентны:
                \begin{enumerate}[nosep]
                \item $\ker \phi = \{0\}$.
                \item $\Im \phi = V$.
                \item $\phi$ обратима (то есть $\phi$ --- изоморфизм $V$ на себя).
                \item $\det \phi \neq 0$.
                \end{enumerate}
            \end{proposal}

            \begin{proof}~
                \begin{description}
                    \item[$1) \iff 2)$] так как $\dim V = \dim \ker \phi + \dim \Im \phi$. 
                    \item[$1) \& 2) \iff 3)$]
                    \item[$2) \iff 4)$] $\Im \phi = V \iff \rk \phi = \dim V \iff \det \phi = 0$.
                \end{description}
            \end{proof}


        \proofitem{Критерий диагонализуемости линейного оператора в терминах собственных векторов}

            \begin{proposal}
                Линейный оператор $\phi$ диагонализуем $\iff$ в $V$ есть базис из собственных векторов.
            \end{proposal}

            \begin{proof}
                Пусть $\E = (e_1, \dots, e_n)$ --- базис $V$.

                \begin{equation*}
                    A(\phi, \E) = \begin{pmatrix}
                        \lambda_1 & 0 & \dots & 0 \\
                        0 & \lambda_2 & \dots & 0 \\
                        \vdots & \vdots & \ddots & \vdots \\
                        0 & 0 & \dots & \lambda_n
                    \end{pmatrix} \iff \phi(e_i) = \lambda_i e_i \ \forall i = 1, \dots, n \iff \text{все $e_i$ --- собственные векторы для $\phi$}
                \end{equation*}
            \end{proof}


        \proofitem{Связь спектра линейного оператора с его характеристическим многочленом}

            Если $\E$ --- какой-либо базис $V$ и $A = (a_{ij}) = A(\phi, \E)$, то
            \begin{equation*}
                \chi_\phi(t) = (-1)^n \det (A - tE) = (-1)^n \begin{vmatrix} 
                    a_{11} - t & a_{12} & a_{13} & \dots & a_{1n} \\ 
                    a_{21} & a_{22} - t & a_{23} & \dots & a_{2n} \\
                    a_{31} & a_{32} & a_{33} - t & \dots & a_{3n} \\
                    \vdots & \vdots & \vdots & \ddots & \vdots \\
                    a_{n1} & a_{n2} & a_{n3} & \dots & a_{nn} - t
                \end{vmatrix}
            \end{equation*}

            $\chi_\phi(t) = t^n + c_{n - 1}t^{n - 1} + \dots + c_1 t + c_0$, где $c_{n - 1} = -\tr \phi$, $c_0 = (-1)^n \det \phi$.

            \begin{corollary}
                $\lambda \in \spec \phi \iff \chi_\phi(\lambda) = 0$, то есть $\lambda$ --- корень характеристического многочлена.
            \end{corollary}

            \begin{corollary}
                $|\spec \phi| \leq n$.   
            \end{corollary}


        \proofitem{Связь между алгебраической и геометрической кратностями собственного значения линейного оператора}

            \begin{proposal}
                $g_\lambda \leq a_\lambda \ \forall \lambda \in \spec \phi$.
            \end{proposal}

            \begin{proof}
                Выберем в $V_\lambda(f)$ базис $e_1, \dots, e_{g_\lambda}$ и дополним его до базиса $(e_1, \dots, e_n) = \E$ всего $V$. Тогда $A(\phi, \E)$ имеет вид
                \begin{equation*}
                    \resizebox{7cm}{!}{
                        \begin{math}
                            \begin{blockarray}{ccccccccccc}
                                \begin{block}{(ccccc|ccccc)c}
                                    \lambda & 0 & 0 & \dots & 0 & & & & & \\
                                    0 & \lambda & 0 & \dots & 0 & & & & & \\
                                    0 & 0 & \lambda & \dots & 0 & & & \scaleobj{2}{B} & & & g_{\lambda}\\
                                    \vdots & \vdots & \vdots & \ddots & \vdots & & \\
                                    0 & 0 & 0 & \dots & \lambda & & \\
                                    \cline{1-10}
                                    & & & & & & & & & & \\
                                    & & & & & & & & & & \\
                                    & & \scaleobj{2}{0} & & & & & \scaleobj{2}{C} & & & n - g_{\lambda} \\
                                    & & & & & & & & & & \\
                                    & & & & & & & & & & \\
                                \end{block}
                                & & g_{\lambda} & & & & & n - g_{\lambda}            
                            \end{blockarray}
                        \end{math}
                    }
                \end{equation*}

                Следовательно, 
                \begin{align*}
                    \chi_\phi(t) 
                    &= (-1)^{n} \cdot \det 
                    \resizebox{4cm}{!}{
                        \begin{math}
                            \begin{blockarray}{(ccc|c)}
                                \lambda - t & \dots & 0 & \\
                                \vdots & \ddots & \vdots & \scaleobj{2}{B} \\
                                0 & \dots & \lambda - t & \\
                                \cline{1-4}
                                & & & \\
                                & \scaleobj{2}{0} & & \scaleobj{2}{C - t E} \\
                                & & & \\
                            \end{blockarray}
                        \end{math}
                    } \\
                    &= (-1)^n (\lambda - t)^{g_\lambda} \cdot \underbracket{\det (C - tE)}_{\text{некий многочлен}} \divby \ (t - \lambda)^{g_{\lambda}} \implies a_\lambda \geq g_\lambda
                .\end{align*}
            \end{proof}


        \proofitem{Линейная независимость собственных подпространств линейного оператора, отвечающих попарно различным собственным значениям}

            \begin{proposal}
                Пусть $ \{\lambda_1, \dots, \lambda_s\} \subseteq \spec \phi$, $\lambda_i \neq \lambda_j$ при $i \neq j$. Тогда собственные подпространства $V_{\lambda_1}(\phi), \dots, V_{\lambda_s}(\phi)$ линейно независимы.
            \end{proposal}

            \begin{proof}
                Индукция по $s$.
                \begin{description}
                \item[База] $s = 1$ --- ясно.
                \item[Шаг] Пусть для $< s$ доказано, докажем для $s$.

                    Возьмем $v_i \in V_{\lambda_i}(\phi) \ \forall i = 1, \dots, s$ и предположим, что $v_1 + \dots + v_s = 0$ ($\star$).

                    Тогда 
                    \begin{math}
                        \begin{aligned}[t]
                            &\phi(v_1 + \dots + v_s) = \phi(0) = 0 \implies \\ 
                            &\phi(v_1) + \dots + \phi(v_s) = 0 \implies \\
                            &\lambda_1 v_1 + \dots + \lambda_s v_s = 0.
                        \end{aligned}
                    \end{math}

                    Вычтем отсюда $(\star) \cdot \lambda_s$:

                    \begin{math}
                        \quad \underset{\neq 0}{(\lambda_1 - \lambda_s)} v_1 + \dots + \underset{\neq 0}{(\lambda_{s - 1} - \lambda_s)} v_{s - 1} = 0.
                    \end{math}

                    По предположению индукции получаем $v_1 = \dots = v_{s - 1} = 0$, а значит и $v_s = 0$.
                    \qedhere
                \end{description}
            \end{proof}


        \proofitem{Диагонализуемость линейного оператора, у которого число корней характеристического многочлена равно размерности пространства}

            \begin{corollary}
                Если $\chi_\phi(t)$ имеет ровно $n$ различных корней, то $\phi$ диагонализуем.
            \end{corollary}

            \begin{proof}
                Пусть $\lambda_1, \dots, \lambda_n$ --- все корни многочлена $\chi_\phi(t)$. 

                Тогда $\forall i = 1, \dots, n \ \dim V_{\lambda_i}(\phi) = 1$. Для каждого $i$ выберем $e_i \in V_{\lambda_i}(\phi) \setminus \{0\}$.

                Тогда $e_1, \dots, e_n$ линейно независимы по предложению, а значит $(e_1, \dots, e_n)$ --- базис из собственных векторов.

                Следовательно, $\phi$ диагонализуем.
            \end{proof}


        \proofitem{Критерий диагонализуемости линейного оператора в терминах его характеристического многочлена и кратностей собственных значений}

            Пусть $V$ --- векторное пространство над $F$, $\dim V = n$, $\phi \in \L(V)$ --- линейный оператор.

            \begin{theorem}{(критерий диагонализуемости)}
                $\phi$ диагонализуемо $\iff$ выполняются одновременно следующие 2 условия:
                \begin{enumerate}
                \item $\chi_\phi(t)$ разлагается на линейные множители.
                \item $\forall \lambda \in \spec \phi \quad g_\lambda = a_\lambda$.
                \end{enumerate}
            \end{theorem}

            \begin{proof}~
                \begin{description}
                \item[$\implies$] $\phi$ диагонализуемо $\implies \exists$ базис $\E = (e_1, \dots, e_n)$, такой что $\chi_\phi(t)$ разлагается на линейные множители:
                    \begin{equation*}
                        A(\phi, \E) = \begin{pmatrix} 
                            \mu_1 & 0 & \dots & 0 \\
                            0 & \mu_2 & \dots & 0 \\
                            \vdots & \vdots & \ddots & \vdots \\
                            0 & 0 & \dots & \mu_n
                        \end{pmatrix} \implies \chi_\phi(t) = (-1)^n \begin{vmatrix} 
                            \mu_1 - t & 0 & \dots & 0 \\
                            0 & \mu_2 - t & \dots & 0 \\
                            \vdots & \vdots & \ddots & \vdots \\
                            0 & 0 & \dots & \mu_n - t
                        \end{vmatrix} = (t - \mu_1) \cdot \ldots \cdot (t - \mu_n)
                    .\end{equation*}

                    Перепишем $\chi_\phi(t)$ в виде $\chi_\phi(t) = (t - \lambda_1)^{k_1} \cdot \ldots \cdot (t - \lambda_s)^{k_s}$, где $ \{\mu_1, \dots, \mu_n\} = \{\lambda_1, \dots, \lambda_s\}, \quad \lambda_i \neq \lambda_j$ при $i \neq j$.

                    $\forall i = 1, \dots, s$ имеем $V_{\lambda_i}(\phi) \supseteq \left< e_j \mid \mu_j = \lambda_i \right> \implies \dim V_{\lambda_i}(\phi) \geq k_i$, то есть $g_{\lambda_i} \geq a_{\lambda_i}$.

                    Но мы знаем, что $g_{\lambda_i} \leq a_{\lambda_i}$. Следовательно, $a_{\lambda_i} = g_{\lambda_i}$.

                \item[$\impliedby$] Пусть $\chi_\phi(t) = (t - \lambda_1)^{k_1} \cdot \ldots \cdot (t - \lambda_s)^{k_s}$, $\lambda_i \neq \lambda_j$ при $i \neq j$.

                    Так как подпространства $V_{\lambda_1}(\phi), \dots, V_{\lambda_s}(\phi)$ линейно независимы, то 
                    \begin{equation*}
                        \dim(V_{\lambda_1}(\phi) + \dots + V_{\lambda_s}(\phi)) = \dim V_{\lambda_1}(\phi) + \dots + \dim V_{\lambda_s}(\phi) = k_1 + \dots + k_s = n = \dim V
                    .\end{equation*}
                    Следовательно, $V = V_{\lambda_1}(\phi) \oplus \dots \oplus V_{\lambda_s}(\phi)$.

                    Если $\E_i$ --- базис в $V_{\lambda_i}(\phi)$, то $\E = \E_1 \sqcup \dots \sqcup \E_s$ --- базис всего $V$, состоящий из собственных векторов, а значит $\phi$ диагонализуем.
                    \qedhere
                \end{description}
            \end{proof}


        \proofitem{Существование собственного вектора у линейного оператора в векторном пространстве над $\CC$. Существование одномерного или двумерного инвариантного подпространства для линейного оператора в векторном пространстве над $\RR$}


            \begin{theorem}
                $F = \RR \implies \forall \phi \in L(V) \ \exists $ либо $1$-мерное, либо $2$-мерное $\phi$-инвариантное подпространство.
            \end{theorem}

            \begin{proof}
                Если $\chi_\phi(t)$ имеет действительные корни, то в $V$ есть собственный вектор $ \implies $ 1-мерное $\phi$-инвариантное подпространство.

                Пусть $\chi_\phi(t)$ не имеет корней в $\RR$. Возьмем какой-нибудь комплексный корень $\lambda + i \mu$, $\mu \neq 0$.

                Фиксируем базис $\E$ в $V$ и положим $A = A(\phi, \E)$. Для $\lambda + i \mu$ у матрицы $A$ существует комплексный собственный вектор, то есть такое $u, v \in \RR^n$, что
                \begin{equation*}
                    A(u + iv) = (\lambda + i \mu) (u + i v) \implies Au + iAv = \lambda u - \mu v + i (\lambda v + \mu u) \implies \begin{cases}
                        Au &= \lambda u - \mu v \\
                        Av &= \lambda v + \mu u
                    \end{cases}
                .\end{equation*}
                
                Значит, векторы в $V$ с координатами $u, v$ порождают $\phi$-инвариантное подпространство $U \subseteq V$ размерности $ \leq 2$.
            \end{proof}

            \begin{exercise}
                $\dim U = 2$.
            \end{exercise}

    \end{colloq}


    \subsection{Линейные отображения и операторы в евклидовых пространствах}
    \begin{colloq}
        \proofitem{Сопряжённое линейное отображение: определение, существование и единственность. Матрица сопряжённого отображения в паре произвольных и паре ортонормированных базисов}

            Пусть 
            \begin{math}
                \begin{aligned}[t]
                    &\EE \text{ --- евклидово пространство со скалярным произведением } (\bigcdot, \bigcdot), \quad \dim \EE = n, \\
                    &\EE \text{ --- другое евклидово пространство со скалярным произведением } (\bigcdot, \bigcdot)', \quad \dim \EE' = m, \\
                    &\phi \colon \EE \to \EE'.
                \end{aligned}
            \end{math}

            \begin{definition}
                Линейное отображение $\psi \colon \EE' \to \EE$ называется \textit{сопряженным} к $\phi$, если 
                \begin{equation*}
                    \label{lec29:def}
                    \tag{$\star$}
                    (\phi(x), y)'  = (x, \psi(y)) \quad \forall x \in \EE, y \in \EE'
                .\end{equation*}

                Обозначение: $\phi^*$.
            \end{definition}

            \begin{proposal}~
                \begin{enumerate}
                \item $\psi$ существует и единственно.
                \item Если $\E$ --- базис $\EE$, $\F$ --- базис $\EE'$, 
                    \begin{math}
                        \begin{aligned}
                            G &= G(e_1, \dots, e_n) \\
                            G' &= G(f_1, \dots, f_m)
                        \end{aligned}
                    \end{math} и 
                    \begin{math}
                        \begin{aligned}
                            A_\phi &= A(\phi, \E, \F) \\
                            A_\psi &= (A, \psi, \F, \E),
                        \end{aligned}
                    \end{math} то $A_\psi = G^{-1} A_\phi^{T} G'$.

                    В частности, если $\E$ и $\F$ ортонормированы, то $A_\psi = A_\phi^{T}$.
                \end{enumerate}
            \end{proposal}

            \begin{proof}
                $x = x_1 e_1 + \dots + x_n e_n \in \EE$, $y = y_1 f_1 + \dots + y_m f_m \in \EE'$.

                \begin{equation*}
                    (\phi(x), y)' = \left(A_\phi \begin{pmatrix} x_1 \\ \dots \\ x_n \end{pmatrix} \right)^{T} \cdot G' \cdot \begin{pmatrix} y_1 \\ \dots \\ y_m \end{pmatrix} = (x_1 \dots x_n) \cdot A_\phi^{T} \cdot G' \cdot \begin{pmatrix} y_1 \\ \dots \\ y_m \end{pmatrix}
                .\end{equation*}

                \begin{equation*}
                    (x, \psi(y)) = (x_1 \dots x_n) \cdot G \cdot A_\psi \cdot \begin{pmatrix} y_1 \\ \dots \\ y_m \end{pmatrix}
                .\end{equation*}

                Так как $\forall B \in \text{Mat}_{m \times n} \quad b_{ij} = \underset{i}{(0 \dots 0 \ 1 \ 0 \dots 0)} \cdot B \cdot \underset{j}{(0 \dots 0 \ 1 \ 0 \dots 0)}^{T}$, то ${\eqref{lec29:def} \iff A_{\phi}^{T} G' = G A_\psi \iff A_\psi = G^{-1} A^{T}_\phi G'}$.

                Отсюда следуют сразу оба утверждения.
            \end{proof}


        \proofitem{Инвариантность ортогонального дополнения к подпространству, инвариантному относительно самосопряжённого линейного оператора}

            \begin{proposal}
                $\phi = \phi^{*}$, $U \subseteq \EE$ --- $\phi$-инвариантное подпространство, тогда $U^{\perp}$ --- тоже $\phi$-инвариантное подпространство.
            \end{proposal}

            \begin{proof}
                $\phi(U) \subseteq U$, хотим $\phi(U^{\perp}) \subseteq U^{\perp}$. 

                $\quad \forall x \in U^{\perp} \quad \forall y \in U \quad (\phi(x), y) = (x, \phi(y)) = 0 \implies \phi(x) \in U^{\perp}$.
            \end{proof}


        \proofitem{Существование собственного вектора для самосопряжённого линейного оператора}
        \proofitem{Существование ортонормированного базиса из собственных векторов для самосопряжённого линейного оператора}
        \proofitem{Приведение квадратичной формы к главным осям}
        \proofitem{Теорема о пяти эквивалентных условиях, определяющих ортогональный линейный оператор}
        \proofitem{Инвариантность ортогонального дополнения к подпространству, инвариантному относительно ортогонального линейного оператора}
        \proofitem{Теорема о каноническом виде ортогонального линейного оператора}
    \end{colloq}

\end{document}
